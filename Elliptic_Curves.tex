\documentclass{article}
%build with recipe latexmk
\usepackage[utf8]{inputenc}
\usepackage[T1]{fontenc}
\usepackage{textcomp}
\usepackage{fancyhdr}
\pagestyle{fancy}

\usepackage{tcolorbox}
\tcbuselibrary{theorems}
\usepackage{babel}
\usepackage{enumerate}
\usepackage{amsmath, amssymb, amsthm}
%\usepackage{a4wide}
\usepackage{float}
\usepackage{tikz-cd}
\usepackage{tikz}
\usepackage{graphicx}
\usepackage{caption}
\usepackage{wrapfig}
\usepackage{setspace}
\setstretch{1.1}
\usepackage{color}
\usepackage{hyperref}
\hypersetup{
    colorlinks=true, %set true if you want colored links
    linktoc=all,     %set to all if you want both sections and subsections linked
    linkcolor=black,  %choose some color if you want links to stand out
}

\theoremstyle{definition}
\newtheorem{theorem}{Theorem}[section]
\newtheorem{lemma}[theorem]{Lemma}
\newtheorem{cor}[theorem]{Corollary}
\newtheorem{prop}[theorem]{Proposition}
\newtheorem{example}{Example}[section]
\newtheorem{defn}{Definition}[section]

\title{Part III - Elliptic Curves
    \\ \large
    Lectured by Tom Fisher 
}
 
\author{Artur Avameri}
\date{}
 
\setcounter{section}{-1}
 
\begin{document}
\maketitle
\tableofcontents
\newpage
 
\section{Introduction}

\marginpar{19 Jan 2024, Lecture 1}


The best books for the course include \textit{The arithmetic of elliptic curves} by Silverman, Springer 1996, and \textit{Lectures on elliptic curves} by Cassels, CUP 1991. 

\section{Fermat's Method of Infinite Descent}

A right--angled triangle $\Delta$ has $a^2+b^2=c^2$ and $\text{area}(\Delta) = \frac{1}{2}ab$.
\begin{defn}
    $\Delta$ is \textbf{rational} if $a,b,c \in \mathbb{Q}$. $\Delta$ is \textbf{primitive} if $a,b,c \in \mathbb{Z}$ are coprime.
\end{defn}
Note that a primitive triangle has pairwise coprime side lengths because $a^2+b^2=c^2$.
\begin{lemma}\label{lemma1.1}
    Every primitive triangle is of the form $(u^2-v^2, 2uv, u^2+v^2)$ for some integers $u > v > 0$.
\end{lemma}
\begin{proof}
    WLOG let $a,b,c$ be odd, even, odd. Then $\left(\frac{b}{2}\right)^2 = \frac{c+a}{2}\frac{c-a}{2}$, where we note that the RHS is a product of positive coprime integers. By unique factorization, $\frac{c+a}{2} = u^2, \frac{c-a}{2}=v^2$ for $u,v \in \mathbb{Z}$. This gives the desired result.
\end{proof}
\begin{defn}
    $D \in \mathbb{Q}_{>0}$ is a \textbf{congruent} number if there exists a rational triangle $\Delta$ with $\text{area}(\Delta)=D$.
\end{defn}
Note that it suffices to consider $D \in \mathbb{Z}_{> 0}$ squarefree.
\begin{example}
    $D=5,6$ are congruent.
\end{example} 
\begin{lemma}\label{lemma1.2}
    $D \in \mathbb{Q}_{>0}$ is congruent $\iff Dy^2 = x^3-x$ for some ${x,y \in \mathbb{Q}, y \neq 0}$.
\end{lemma}
\begin{proof}
    Lemma \ref{lemma1.1} shows that $D$ congruent $\implies Dw^2 = uv(u^2-v^2)$ for some $u,v,w \in \mathbb{Q}, w \neq 0$. This implication also obviously goes the other way. To finish, divide through by $w^4$ and take $x = \frac{u}{v}, y = \frac{w}{v^2}$.
\end{proof}
Fermat showed that $1$ is not a congruent number.
\begin{theorem}\label{theorem1.3}
    There is no solution to $w^2 = uv(u+v)(u-v)$ for $u,v,w \in \mathbb{Z}, w \neq 0$.
\end{theorem}
\begin{proof}
    WLOG assume $u,v$ are coprime and that $u,w > 0$. If $v<0$, then replace $(u,v,w)$ by $(-v,u,w)$. If $u,v$ are both odd, then replace $(u,v,w)$ by $\left(\frac{u+v}{2},\frac{u-v}{2},\frac{w}{2}\right)$. Then $u,v,u+v,u-v$ are pairwise coprime positive integers with their product a square, so by unique factorization in $\mathbb{Z}$, $u=a^2, v = b^2, u+v = c^2, u-v = d^2$ for $a,b,c,d \in \mathbb{Z}$. 
    \vspace{1mm}
     
    Since $u \not\equiv v \pmod{2}$, both $c$ and $d$ are odd. Then $\left(\frac{c+d}{2}\right)^2 + \left(\frac{c-d}{2}\right)^2 = \frac{c^2+d^2}{2} = u = a^2$. This gives a primitive triangle with area $\frac{c^2-d^2}{8} = \frac{v}{4} = \left(\frac{b^2}{2}\right)$. Let $w_1 = \frac{b}{2}$, then by Lemma \ref{lemma1.1}, $w_1^2 = u_1v_1(u_1+v_1)(u_1-v_1)$ for some $u_1, v_1 \in \mathbb{Z}$. Hence we have a new solution to our original question, with $4w_1^2 = b^2 = v \mid w^2 \implies w_1 \le \frac{w}{2}$, so we're done by infinite descent.
\end{proof}
\vspace{1mm}
 
\textbf{A variant for polynomials.} In the above, $K$ is a field with $\text{char }K \neq 2$. Let $\overline{K}$ be the algebraic closure of $K$ and consider for this whole section $K$ with $\text{char }K \neq 2$.

\begin{lemma}\label{lemma1.4}
    Let $u,v \in K[t]$ be coprime. If $\alpha u + \beta v$ is a square for 4 distinct $(\alpha : \beta) \in \mathbb{P}^1$, then $u, v \in K$.
\end{lemma}
\begin{proof}
    WLOG let $K = \overline{K}$ by extending if necessary. Changing coordinates on $\mathbb{P}^1$ (i.e. multiplying by a $2 \times 2$ invertible matrix), we may assume that the points $(\alpha : \beta)$ are $(1 : 0)$, $(0 : 1)$, $(1: -1)$, $(1: - \lambda)$ for $\lambda \in K \setminus \{0,1\}$. Since our field is algebraically closed, let $\mu = \sqrt{\lambda}$. Then $u = a^2, v = b^2, u-v = (a+b)(a-b), u - \lambda v = (a + \mu b)(a - \mu b)$.
    \vspace{1mm}
     
    Unique factorization in $K[t]$ implies that $a+b, a-b, a+ \mu b, a- \mu b$ are squares (since the necessary terms are coprime up to units, i.e. constants). But $\max(\text{deg}(a), \text{deg}(b)) \le \frac{1}{2}\max(\text{deg}(u),\text{deg}(v))$, so by Fermat's method of infinite descent, $u, v \in K$.
\end{proof}
\begin{defn}
    \begin{enumerate}[(i)]
        \item An \textbf{elliptic curve} $E/K$ is the projective closure of the plane affine curve $y^2 = f(x)$ (this is called a Weierstrass equation) where $f \in K[x]$ is a monic cubic polynomial with distinct roots in $\overline{K}$.
        \item For $L/K$ any field extension, $E(L) = \{(x,y) \in L^2 \mid y^2 = f(x)\} \cup \{0\}$ (the point at infinity in the projective closure), it turns out that $E(L)$ is naturally an abelian group.  
    \end{enumerate}
\end{defn}
In this course, we study $E(K)$ for $K$ a finite field, local field, number field.
\vspace{1mm}
 
Lemma \ref{lemma1.2} and Theorem \ref{theorem1.3} show that if $E : y^2 = x^3-x$, then $E(\mathbb{Q}) = \{0, (0,0), (\pm 1, 0)\}$.

\begin{cor}\label{cor1.5}
    Let $E/K$ be an elliptic curve. Then $E(K(t)) = E(K)$.
\end{cor}
\begin{proof}
    WLOG $K = \overline{K}$. By a change of coordinates, we may assume $y^2 = x(x-1)(x-\lambda)$ for some $\lambda \in K\setminus \{0,1\}$. Suppose $(x,y) \in E(K(t))$. Write $x = \frac{u}{v}$ for $u,v \in K(t)$ coprime. Then $w^2 = uv(u-v)(u-\lambda v)$ for some $w \in K[t]$. Unique factorization in $K[t]$ shows that $u,v, u-v, u- \lambda v$ are all squares, so by Lemma \ref{lemma1.4}, $u, v \in K$, so $x, y \in K$.
\end{proof}

\section{Some remarks on algebraic curves}

\marginpar{22 Jan 2024, Lecture 2}

In this section, work over an algebraically closed field $K = \overline{K}$.

\begin{defn}\label{defn2.1}
    A plane curve $C = \{f(x,y) = 0\} \subset \mathbb{A}^2$ (for $f \in K[x,y]$ irreducible) is \textbf{rational} if it has a rational parametrization, i.e. $\exists \phi, \psi \in K(t)$ such that
    \begin{enumerate}[(i)]
        \item The map $\mathbb{A}^1 \to \mathbb{A}^2$ by $t \mapsto (\phi(t), \psi(t))$ is injective on $\mathbb{A}^1\setminus \{\text{finite set}\}$.
        \item $f(\phi(t),\psi(t))=0$ in $K(t)$.
    \end{enumerate}
\end{defn}
\begin{example}
    \begin{enumerate}[(a)]
        \item Any nonsingular conic is rational. For example, for $x^2+y^2=1$, take a line with slope $t$ through $(-1,0)$ (the anchor) and solve to get the rational parametrization $(x,y) = \left(\frac{1-t^2}{1+t^2},\frac{2t}{1+t^2}\right)$.
        \item Any singular plane cubic is rational, for example $y^2=x^3$ giving $(x,y) = (t^2, t^3)$ with the anchor at the singularity $(0,0)$ and $y^2 = x^2(x+1)$ with the parametrization to be computed on Ex. Sheet 1 (anchor still at $(0,0)$).
        \item Corollary \ref{cor1.5} shows that elliptic curves are not rational.
    \end{enumerate}
\end{example}
\textbf{Remark.}
The genus $g(C) \in \mathbb{Z}_{\ge 0}$ is an invariant of a smooth projective curve $C$. If $K=\mathbb{C}$, then $g(C)$ is the genus of the Riemann surface. A smooth plane curve $C \subset \mathbb{P}^2$ of degree $d$ has genus $g(C) = \frac{(d-1)(d-2)}{2}$.
\begin{prop}
    (Here we still assume $K = \overline{K}$). Let $C$ be a smooth projective curve.
    \begin{itemize}
        \item $C$ is rational (see Definition \ref{defn2.1})$\iff$ $g(C)=0$.
        \item $C$ is an elliptic curve $\iff$ $g(C)=1$.
    \end{itemize}
\end{prop}
\begin{proof}
    \begin{enumerate}[(i)]
        \item Omitted.
        \item $(\implies)$: Check $C$ is a smooth plane curve in $\mathbb{P}^2$ (see Ex. Sheet 1) and use the above remark.
        \vspace{1mm}
         
        $(\impliedby)$: We will see this later.
    \end{enumerate}
\end{proof}

\vspace{1mm}
 
\textbf{Order of vanishing.} Let $C$ be an algebraic curve with function field $K(C)$ and let $P \in C$ be a smooth point. Write $\text{ord}_P(f)$ for the order of vanishing of $f \in K(C)$ at $P$ (which is negative if $f$ has a pole at $P$).
\vspace{1mm}
 
\textbf{Fact.} $\text{ord}_P : K(C)^\times \to \mathbb{Z}$ is a discrete valuation, i.e. $\text{ord}_P(f_1f_2) = \text{ord}_P(f_1) + \text{ord}_P(f_2)$ and $\text{ord}_P(f_1+f_2)\ge \min(\text{ord}_P(f_1),\text{ord}_P(f_2))$. 

\begin{defn}
    We say $t \in K(C)^\times$ is a \textbf{uniformizer} at $P$ if $\text{ord}_P(t)=1$.
\end{defn}
\begin{example}
    $C =\{g = 0\} \subset \mathbb{A}^2$ for $g \in K[x,y]$. Then $K(C) = \text{Frac}\left(\frac{K[x,y]}{(g)}\right)$. Write $g = g_0+g_1(x,y) + g_2(x,y) + \ldots$ for $g_i$ homogeneous of degree $i$. Suppose $P = (0,0)$ is a smooth point, e.g. $g_0=0$ and let $g_1(x,y)=\alpha x + \beta y$ with $\alpha,\beta$ not both zero ($\alpha x + \beta y =0$ gives a tangent to the curve at $P$). Let $\gamma,\delta \in K$ and consider also the line $\gamma x + \delta y$ through $P$. Then it is a fact that $\gamma x + \delta y \in K(C)$ is a uniformizer at $P$ if and only if $\alpha \delta - \beta \gamma \neq 0$. 
\end{example}
\begin{example}\label{ex2.3}
    Consider $\{y^2 = x(x-1)(x-\lambda)\} \subset \mathbb{A}^2$ for $\lambda \neq 0,1$ and consider its projective closure by taking $x =\frac{X}{Z}, y = \frac{Y}{Z}$ to get $\{Y^2Z = X(X-Z)(X-\lambda Z)\} \subset \mathbb{P}^2$. This has only one point at infinity, $P = (0 : 1 : 0)$. Our aim is to compute $\text{ord}_P(x)$ and $\text{ord}_P(y)$.
    \vspace{1mm}
     
    For this, put $t = \frac{X}{Y}, w =\frac{Z}{Y}$, so $w \stackrel{(\dagger)}{=}  t(t-w)(t-\lambda w) $. Now $P$ is the point $(t,w) = (0,0)$, which is a smooth point with $\text{ord}_P(t) = \text{ord}_P(t-w) = \text{ord}_P(t-\lambda w) = 1$, so $(\dagger)$ gives $\text{ord}_P(w) = 3$. We now find
    \begin{align*}
        &\text{ord}_P(x) = \text{ord}_P\left(\frac{X}{Z}\right) = \text{ord}_P\left(\frac{t}{w}\right) = 1-3 = -2\\
        &\text{ord}_P(y) = \text{ord}_P\left(\frac{Y}{Z}\right) = \text{ord}_P\left(\frac{1}{w}\right) = -3.
    \end{align*}
\end{example}

\textbf{Riemann--Roch space.} Let $C$ be a smooth projective curve. 
\begin{defn}
    A \textbf{divisor} is a formal sum of points on $C$, say $D = \sum_{P \in C}^{} n_P P$ where $n_P \in \mathbb{Z}$ and $n_P = 0$ for all but finitely many $P \in C$. We say $\text{deg }D= \sum_{P \in C}^{} n_P$.
    \vspace{1mm}
     
    $D$ is \textbf{effective} (written $D\ge 0$) if $n_P \ge 0 ~\forall P \in C$. If $f \in K(C)^\times$, then $\text{div}(f) = \sum_{P \in C}^{} \text{ord}_P(f)P$. The Riemann--Roch space of $D \in \text{Div}(C)$ is  
    \[
        \mathcal{L}(D) = \{f \in K(C)^\times \mid \text{div}(f) + D \ge 0\} \cup \{0\},
    \]
    i.e. the $K$--vector space of rational functions on $C$ with ''poles no worse than specified by $D$''.
\end{defn}
We quote Riemann--Roch for surfaces of genus 1: We have
\begin{align*}
    \text{dim }\mathcal{L}(D) = \begin{cases}
        \text{deg }D &\text{if deg }D>0\\
        0 \text{ or }1 &\text{if deg }D=0\\
        0 &\text{if deg }D<0.
    \end{cases}
\end{align*}
\begin{example}
    We revisit Example \ref{ex2.3}. We have $\mathcal{L}(2P) = \langle 1,x \rangle$ and $\mathcal{L}(3P) = \langle 1,x,y \rangle$.
\end{example}

\end{document}
 