\documentclass{article}
%build with recipe latexmk
\usepackage[utf8]{inputenc}
\usepackage[T1]{fontenc}
\usepackage{textcomp}
\usepackage{fancyhdr}
\pagestyle{fancy}

\usepackage{tcolorbox}
\tcbuselibrary{theorems}
\usepackage{babel}
\usepackage{enumerate}
\usepackage{amsmath, amssymb, amsthm}
%\usepackage{a4wide}
\usepackage{float}
\usepackage{tikz-cd}
\usepackage{tikz}
\usepackage{graphicx}
\usepackage{caption}
\usepackage{wrapfig}
\usepackage{setspace}
\setstretch{1.1}
\usepackage{color}
\usepackage{hyperref}
\hypersetup{
    colorlinks=true, %set true if you want colored links
    linktoc=all,     %set to all if you want both sections and subsections linked
    linkcolor=black,  %choose some color if you want links to stand out
}

\theoremstyle{definition}
\newtheorem{theorem}{Theorem}[section]
\newtheorem{lemma}[theorem]{Lemma}
\newtheorem{cor}[theorem]{Corollary}
\newtheorem{prop}[theorem]{Proposition}
\newtheorem{example}{Example}[section]
\newtheorem{defn}{Definition}[section]

\title{Part III - Elliptic Curves
    \\ \large
    Lectured by Tom Fisher 
}
 
\author{Artur Avameri}
\date{}
 
\setcounter{section}{-1}
 
\begin{document}
\maketitle
\tableofcontents
\newpage
 
\section{Introduction}

The best books for the course include \textit{The arithmetic of elliptic curves} by Silverman, Springer 1996, and \textit{Lectures on elliptic curves} by Cassels, CUP 1991. 

\section{Fermat's Method of Infinite Descent}

A right--angled triangle $\Delta$ has $a^2+b^2=c^2$ and $\text{area}(\Delta) = \frac{1}{2}ab$.
\begin{defn}
    $\Delta$ is \textbf{rational} if $a,b,c \in \mathbb{Q}$. $\Delta$ is \textbf{primitive} if $a,b,c \in \mathbb{Z}$ are coprime.
\end{defn}
Note that a primitive triangle has pairwise coprime side lengths because $a^2+b^2=c^2$.
\begin{lemma}\label{lemma1.1}
    Every primitive triangle is of the form $(u^2-v^2, 2uv, u^2+v^2)$ for some integers $u > v > 0$.
\end{lemma}
\begin{proof}
    WLOG let $a,b,c$ be odd, even, odd. Then $\left(\frac{b}{2}\right)^2 = \frac{c+a}{2}\frac{c-a}{2}$, where we note that the RHS is a product of positive coprime integers. By unique factorization, $\frac{c+a}{2} = u^2, \frac{c-a}{2}=v^2$ for $u,v \in \mathbb{Z}$. This gives the desired result.
\end{proof}
\begin{defn}
    $D \in \mathbb{Q}_{>0}$ is a \textbf{congruent} number if there exists a rational triangle $\Delta$ with $\text{area}(\Delta)=D$.
\end{defn}
Note that it suffices to consider $D \in \mathbb{Z}_{> 0}$ squarefree.
\begin{example}
    $D=5,6$ are congruent.
\end{example} 
\begin{lemma}\label{lemma1.2}
    $D \in \mathbb{Q}_{>0}$ is congruent $\iff Dy^2 = x^3-x$ for some ${x,y \in \mathbb{Q}, y \neq 0}$.
\end{lemma}
\begin{proof}
    Lemma \ref{lemma1.1} shows that $D$ congruent $\implies Dw^2 = uv(u^2-v^2)$ for some $u,v,w \in \mathbb{Q}, w \neq 0$. This implication also obviously goes the other way. To finish, divide through by $w^4$ and take $x = \frac{u}{v}, y = \frac{w}{v^2}$.
\end{proof}
Fermat showed that $1$ is not a congruent number.
\begin{theorem}\label{theorem1.3}
    There is no solution to $w^2 = uv(u+v)(u-v)$ for $u,v,w \in \mathbb{Z}, w \neq 0$.
\end{theorem}
\begin{proof}
    WLOG assume $u,v$ are coprime and that $u,w > 0$. If $v<0$, then replace $(u,v,w)$ by $(-v,u,w)$. If $u,v$ are both odd, then replace $(u,v,w)$ by $\left(\frac{u+v}{2},\frac{u-v}{2},\frac{w}{2}\right)$. Then $u,v,u+v,u-v$ are pairwise coprime positive integers with their product a square, so by unique factorization in $\mathbb{Z}$, $u=a^2, v = b^2, u+v = c^2, u-v = d^2$ for $a,b,c,d \in \mathbb{Z}$. 
    \vspace{1mm}
     
    Since $u \not\equiv v \pmod{2}$, both $c$ and $d$ are odd. Then $\left(\frac{c+d}{2}\right)^2 + \left(\frac{c-d}{2}\right)^2 = \frac{c^2+d^2}{2} = u = a^2$. This gives a primitive triangle with area $\frac{c^2-d^2}{8} = \frac{v}{4} = \left(\frac{b^2}{2}\right)$. Let $w_1 = \frac{b}{2}$, then by Lemma \ref{lemma1.1}, $w_1^2 = u_1v_1(u_1+v_1)(u_1-v_1)$ for some $u_1, v_1 \in \mathbb{Z}$. Hence we have a new solution to our original question, with $4w_1^2 = b^2 = v \mid w^2 \implies w_1 \le \frac{w}{2}$, so we're done by infinite descent.
\end{proof}
\vspace{1mm}
 
\textbf{A variant for polynomials.} In the above, $K$ is a field with $\text{char }K \neq 2$. Let $\overline{K}$ be the algebraic closure of $K$.

\begin{lemma}\label{lemma1.4}
    Let $u,v \in K[t]$ be coprime. If $\alpha u + \beta v$ is a square for 4 distinct $(\alpha : \beta) \in \mathbb{P}^1$, then $u, v \in K$.
\end{lemma}
\begin{proof}
    WLOG let $K = \overline{K}$ by extending if necessary. Changing coordinates on $\mathbb{P}^1$ (i.e. multiplying by a $2 \times 2$ invertible matrix), we may assume that the points $(\alpha : \beta)$ are $(1 : 0)$, $(0 : 1)$, $(1: -1)$, $(1: - \lambda)$ for $\lambda \in K \setminus \{0,1\}$. Since our field is algebraically closed, let $\mu = \sqrt{\lambda}$. Then $u = a^2, v = b^2, u-v = (a+b)(a-b), u - \lambda v = (a + \mu b)(a - \mu b)$.
    \vspace{1mm}
     
    Unique factorization in $K[t]$ implies that $a+b, a-b, a+ \mu b, a- \mu b$ are squares (since the necessary terms are coprime up to units, i.e. constants). But $\max(\text{deg}(a), \text{deg}(b)) \le \frac{1}{2}\max(\text{deg}(u),\text{deg}(v))$, so by Fermat's method of infinite descent, $u, v \in K$.
\end{proof}
\begin{defn}
    \begin{enumerate}[(i)]
        \item An \textbf{elliptic curve} $E/K$ is the projective closure of the plane affine curve $y^2 = f(x)$ (this is called a Weierstrass equation) where $f \in K[x]$ is a monic cubic polynomial with distinct roots in $\overline{K}$.
        \item For $L/K$ any field extension, $E(L) = \{(x,y) \in L^2 \mid y^2 = f(x)\} \cup \{0\}$ (the point at infinity in the projective closure), it turns out that $E(L)$ is naturally an abelian group.  
    \end{enumerate}
\end{defn}
In this course, we study $E(K)$ for $K$ a finite field, local field, number field.
\vspace{1mm}
 
Lemma \ref{lemma1.2} and Theorem \ref{theorem1.3} show that if $E : y^2 = x^3-x$, then $E(\mathbb{Q}) = \{0, (0,0), (\pm 1, 0)\}$.

\begin{cor}
    Let $E/K$ be an elliptic curve. Then $E(K(t)) = E(K)$.
\end{cor}
\begin{proof}
    WLOG $K = \overline{K}$. By a change of coordinates, we may assume $y^2 = x(x-1)(x-\lambda)$ for some $\lambda \in K\setminus \{0,1\}$. Suppose $(x,y) \in E(K(t))$. Write $x = \frac{u}{v}$ for $u,v \in K(t)$ coprime. Then $w^2 = uv(u-v)(u-\lambda v)$ for some $w \in K[t]$. Unique factorization in $K[t]$ shows that $u,v, u-v, u- \lambda v$ are all squares, so by Lemma \ref{lemma1.4}, $u, v \in K$, so $x, y \in K$.
\end{proof}

\end{document}
 