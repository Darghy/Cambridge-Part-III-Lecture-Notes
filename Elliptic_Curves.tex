\documentclass{article}
%build with recipe latexmk
\usepackage[utf8]{inputenc}
\usepackage[T1]{fontenc}
\usepackage{textcomp}
\usepackage{fancyhdr}
\pagestyle{fancy}

\usepackage{tcolorbox}
\tcbuselibrary{theorems}
\usepackage{babel}
\usepackage{enumerate}
\usepackage{amsmath, amssymb, amsthm}
%\usepackage{a4wide}
\usepackage{float}
\usepackage{tikz-cd}
\usepackage{tikz}
\usetikzlibrary{matrix, arrows}
\usepackage{graphicx}
\usepackage{stmaryrd}
\usepackage{caption}
\usepackage{wrapfig}
\usepackage{setspace}
\setstretch{1.1}
\usepackage{color}
\usepackage{hyperref}
\hypersetup{
    colorlinks=true, %set true if you want colored links
    linktoc=all,     %set to all if you want both sections and subsections linked
    linkcolor=black,  %choose some color if you want links to stand out
}
\usepackage[OT2,T1]{fontenc}
\DeclareSymbolFont{cyrletters}{OT2}{wncyr}{m}{n}
\DeclareMathSymbol{\Sha}{\mathalpha}{cyrletters}{"58}

\theoremstyle{definition}
\newtheorem{theorem}{Theorem}[section]
\newtheorem{lemma}[theorem]{Lemma}
\newtheorem{cor}[theorem]{Corollary}
\newtheorem{prop}[theorem]{Proposition}
\newtheorem{example}{Example}[section]
\newtheorem{defn}{Definition}[section]

\title{Part III - Elliptic Curves
    \\ \large
    Lectured by Tom Fisher 
}
 
\author{Artur Avameri}
\date{}
 
\setcounter{section}{-1}
 
\begin{document}
\maketitle
\tableofcontents
\newpage
 
\section{Introduction}

\marginpar{19 Jan 2024, Lecture 1}


The best books for the course include \textit{The arithmetic of elliptic curves} by Silverman, Springer 1996, and \textit{Lectures on elliptic curves} by Cassels, CUP 1991. 

\section{Fermat's Method of Infinite Descent}

A right--angled triangle $\Delta$ has $a^2+b^2=c^2$ and $\text{area}(\Delta) = \frac{1}{2}ab$.
\begin{defn}
    $\Delta$ is \textbf{rational} if $a,b,c \in \mathbb{Q}$. $\Delta$ is \textbf{primitive} if $a,b,c \in \mathbb{Z}$ are coprime.
\end{defn}
Note that a primitive triangle has pairwise coprime side lengths because $a^2+b^2=c^2$.
\begin{lemma}\label{lemma1.1}
    Every primitive triangle is of the form $(u^2-v^2, 2uv, u^2+v^2)$ for some integers $u > v > 0$.
\end{lemma}
\begin{proof}
    WLOG let $a,b,c$ be odd, even, odd. Then $\left(\frac{b}{2}\right)^2 = \frac{c+a}{2}\frac{c-a}{2}$, where we note that the RHS is a product of positive coprime integers. By unique factorization, $\frac{c+a}{2} = u^2, \frac{c-a}{2}=v^2$ for $u,v \in \mathbb{Z}$. This gives the desired result.
\end{proof}
\begin{defn}
    $D \in \mathbb{Q}_{>0}$ is a \textbf{congruent} number if there exists a rational triangle $\Delta$ with $\text{area}(\Delta)=D$.
\end{defn}
Note that it suffices to consider $D \in \mathbb{Z}_{> 0}$ squarefree.
\begin{example}
    $D=5,6$ are congruent.
\end{example} 
\begin{lemma}\label{lemma1.2}
    $D \in \mathbb{Q}_{>0}$ is congruent $\iff Dy^2 = x^3-x$ for some ${x,y \in \mathbb{Q}, y \neq 0}$.
\end{lemma}
\begin{proof}
    Lemma \ref{lemma1.1} shows that $D$ congruent $\implies Dw^2 = uv(u^2-v^2)$ for some $u,v,w \in \mathbb{Q}, w \neq 0$. This implication also obviously goes the other way. To finish, divide through by $w^4$ and take $x = \frac{u}{v}, y = \frac{w}{v^2}$.
\end{proof}
Fermat showed that $1$ is not a congruent number.
\begin{theorem}\label{theorem1.3}
    There is no solution to $w^2 = uv(u+v)(u-v)$ for $u,v,w \in \mathbb{Z}, w \neq 0$.
\end{theorem}
\begin{proof}
    WLOG assume $u,v$ are coprime and that $u,w > 0$. If $v<0$, then replace $(u,v,w)$ by $(-v,u,w)$. If $u,v$ are both odd, then replace $(u,v,w)$ by $\left(\frac{u+v}{2},\frac{u-v}{2},\frac{w}{2}\right)$. Then $u,v,u+v,u-v$ are pairwise coprime positive integers with their product a square, so by unique factorization in $\mathbb{Z}$, $u=a^2, v = b^2, u+v = c^2, u-v = d^2$ for $a,b,c,d \in \mathbb{Z}$. 
    \vspace{1mm}
     
    Since $u \not\equiv v \pmod{2}$, both $c$ and $d$ are odd. Then $\left(\frac{c+d}{2}\right)^2 + \left(\frac{c-d}{2}\right)^2 = \frac{c^2+d^2}{2} = u = a^2$. This gives a primitive triangle with area $\frac{c^2-d^2}{8} = \frac{v}{4} = \left(\frac{b^2}{2}\right)$. Let $w_1 = \frac{b}{2}$, then by Lemma \ref{lemma1.1}, $w_1^2 = u_1v_1(u_1+v_1)(u_1-v_1)$ for some $u_1, v_1 \in \mathbb{Z}$. Hence we have a new solution to our original question, with $4w_1^2 = b^2 = v \mid w^2 \implies w_1 \le \frac{w}{2}$, so we're done by infinite descent.
\end{proof}
\vspace{1mm}
 
\textbf{A variant for polynomials.} In the above, $K$ is a field with $\text{char }K \neq 2$. Let $\overline{K}$ be the algebraic closure of $K$ and consider for this whole section $K$ with $\text{char }K \neq 2$.

\begin{lemma}\label{lemma1.4}
    Let $u,v \in K[t]$ be coprime. If $\alpha u + \beta v$ is a square for 4 distinct $(\alpha : \beta) \in \mathbb{P}^1$, then $u, v \in K$.
\end{lemma}
\begin{proof}
    WLOG let $K = \overline{K}$ by extending if necessary. Changing coordinates on $\mathbb{P}^1$ (i.e. multiplying by a $2 \times 2$ invertible matrix), we may assume that the points $(\alpha : \beta)$ are $(1 : 0)$, $(0 : 1)$, $(1: -1)$, $(1: - \lambda)$ for $\lambda \in K \setminus \{0,1\}$. Since our field is algebraically closed, let $\mu = \sqrt{\lambda}$. Then $u = a^2, v = b^2, u-v = (a+b)(a-b), u - \lambda v = (a + \mu b)(a - \mu b)$.
    \vspace{1mm}
     
    Unique factorization in $K[t]$ implies that $a+b, a-b, a+ \mu b, a- \mu b$ are squares (since the necessary terms are coprime up to units, i.e. constants). But $\max(\text{deg}(a), \text{deg}(b)) \le \frac{1}{2}\max(\text{deg}(u),\text{deg}(v))$, so by Fermat's method of infinite descent, $u, v \in K$.
\end{proof}
\begin{defn}
    \begin{enumerate}[(i)]
        \item An \textbf{elliptic curve} $E/K$ is the projective closure of the plane affine curve $y^2 = f(x)$ (this is called a Weierstrass equation) where $f \in K[x]$ is a monic cubic polynomial with distinct roots in $\overline{K}$.
        \item For $L/K$ any field extension, $E(L) = \{(x,y) \in L^2 \mid y^2 = f(x)\} \cup \{0\}$ (the point at infinity in the projective closure), it turns out that $E(L)$ is naturally an abelian group.  
    \end{enumerate}
\end{defn}
In this course, we study $E(K)$ for $K$ a finite field, local field, number field.
\vspace{1mm}
 
Lemma \ref{lemma1.2} and Theorem \ref{theorem1.3} show that if $E : y^2 = x^3-x$, then $E(\mathbb{Q}) = \{0, (0,0), (\pm 1, 0)\}$.

\begin{cor}\label{cor1.5}
    Let $E/K$ be an elliptic curve. Then $E(K(t)) = E(K)$.
\end{cor}
\begin{proof}
    WLOG $K = \overline{K}$. By a change of coordinates, we may assume $y^2 = x(x-1)(x-\lambda)$ for some $\lambda \in K\setminus \{0,1\}$. Suppose $(x,y) \in E(K(t))$. Write $x = \frac{u}{v}$ for $u,v \in K(t)$ coprime. Then $w^2 = uv(u-v)(u-\lambda v)$ for some $w \in K[t]$. Unique factorization in $K[t]$ shows that $u,v, u-v, u- \lambda v$ are all squares, so by Lemma \ref{lemma1.4}, $u, v \in K$, so $x, y \in K$.
\end{proof}

\section{Some remarks on algebraic curves}

\marginpar{22 Jan 2024, Lecture 2}

In this section, work over an algebraically closed field $K = \overline{K}$.

\begin{defn}\label{defn2.1}
    A plane curve $C = \{f(x,y) = 0\} \subset \mathbb{A}^2$ (for $f \in K[x,y]$ irreducible) is \textbf{rational} if it has a rational parametrization, i.e. $\exists \phi, \psi \in K(t)$ such that
    \begin{enumerate}[(i)]
        \item The map $\mathbb{A}^1 \to \mathbb{A}^2$ by $t \mapsto (\phi(t), \psi(t))$ is injective on $\mathbb{A}^1\setminus \{\text{finite set}\}$.
        \item $f(\phi(t),\psi(t))=0$ in $K(t)$.
    \end{enumerate}
\end{defn}
\begin{example}
    \begin{enumerate}[(a)]
        \item Any nonsingular conic is rational. For example, for $x^2+y^2=1$, take a line with slope $t$ through $(-1,0)$ (the anchor) and solve to get the rational parametrization $(x,y) = \left(\frac{1-t^2}{1+t^2},\frac{2t}{1+t^2}\right)$.
        \item Any singular plane cubic is rational, for example $y^2=x^3$ giving $(x,y) = (t^2, t^3)$ with the anchor at the singularity $(0,0)$ and $y^2 = x^2(x+1)$ with the parametrization to be computed on Ex. Sheet 1 (anchor still at $(0,0)$).
        \item Corollary \ref{cor1.5} shows that elliptic curves are not rational.
    \end{enumerate}
\end{example}
\textbf{Remark.}
The genus $g(C) \in \mathbb{Z}_{\ge 0}$ is an invariant of a smooth projective curve $C$. If $K=\mathbb{C}$, then $g(C)$ is the genus of the Riemann surface. A smooth plane curve $C \subset \mathbb{P}^2$ of degree $d$ has genus $g(C) = \frac{(d-1)(d-2)}{2}$.
\begin{prop}
    (Here we still assume $K = \overline{K}$). Let $C$ be a smooth projective curve.
    \begin{itemize}
        \item $C$ is rational (see Definition \ref{defn2.1})$\iff$ $g(C)=0$.
        \item $C$ is an elliptic curve $\iff$ $g(C)=1$.
    \end{itemize}
\end{prop}
\begin{proof}
    \begin{enumerate}[(i)]
        \item Omitted.
        \item $(\implies)$: Check $C$ is a smooth plane curve in $\mathbb{P}^2$ (see Ex. Sheet 1) and use the above remark.
        \vspace{1mm}
         
        $(\impliedby)$: We will see this later.
    \end{enumerate}
\end{proof}

\vspace{1mm}
 
\textbf{Order of vanishing.} Let $C$ be an algebraic curve with function field $K(C)$ and let $P \in C$ be a smooth point. Write $\text{ord}_P(f)$ for the order of vanishing of $f \in K(C)$ at $P$ (which is negative if $f$ has a pole at $P$).
\vspace{1mm}
 
\textbf{Fact.} $\text{ord}_P : K(C)^\times \to \mathbb{Z}$ is a discrete valuation, i.e. $\text{ord}_P(f_1f_2) = \text{ord}_P(f_1) + \text{ord}_P(f_2)$ and $\text{ord}_P(f_1+f_2)\ge \min(\text{ord}_P(f_1),\text{ord}_P(f_2))$. 

\begin{defn}
    We say $t \in K(C)^\times$ is a \textbf{uniformizer} at $P$ if $\text{ord}_P(t)=1$.
\end{defn}
\begin{example}
    $C =\{g = 0\} \subset \mathbb{A}^2$ for $g \in K[x,y]$. Then $K(C) = \text{Frac}\left(\frac{K[x,y]}{(g)}\right)$. Write $g = g_0+g_1(x,y) + g_2(x,y) + \ldots$ for $g_i$ homogeneous of degree $i$. Suppose $P = (0,0)$ is a smooth point, e.g. $g_0=0$ and let $g_1(x,y)=\alpha x + \beta y$ with $\alpha,\beta$ not both zero ($\alpha x + \beta y =0$ gives a tangent to the curve at $P$). Let $\gamma,\delta \in K$ and consider also the line $\gamma x + \delta y$ through $P$. Then it is a fact that $\gamma x + \delta y \in K(C)$ is a uniformizer at $P$ if and only if $\alpha \delta - \beta \gamma \neq 0$. 
\end{example}
\begin{example}\label{ex2.3}
    Consider $\{y^2 = x(x-1)(x-\lambda)\} \subset \mathbb{A}^2$ for $\lambda \neq 0,1$ and consider its projective closure by taking $x =\frac{X}{Z}, y = \frac{Y}{Z}$ to get $\{Y^2Z = X(X-Z)(X-\lambda Z)\} \subset \mathbb{P}^2$. This has only one point at infinity, $P = (0 : 1 : 0)$. Our aim is to compute $\text{ord}_P(x)$ and $\text{ord}_P(y)$.
    \vspace{1mm}
     
    For this, put $t = \frac{X}{Y}, w =\frac{Z}{Y}$, so $w \stackrel{(\dagger)}{=}  t(t-w)(t-\lambda w) $. Now $P$ is the point $(t,w) = (0,0)$, which is a smooth point with $\text{ord}_P(t) = \text{ord}_P(t-w) = \text{ord}_P(t-\lambda w) = 1$, so $(\dagger)$ gives $\text{ord}_P(w) = 3$. We now find
    \begin{align*}
        &\text{ord}_P(x) = \text{ord}_P\left(\frac{X}{Z}\right) = \text{ord}_P\left(\frac{t}{w}\right) = 1-3 = -2\\
        &\text{ord}_P(y) = \text{ord}_P\left(\frac{Y}{Z}\right) = \text{ord}_P\left(\frac{1}{w}\right) = -3.
    \end{align*}
\end{example}

\textbf{Riemann--Roch space.} Let $C$ be a smooth projective curve. 
\begin{defn}
    A \textbf{divisor} is a formal sum of points on $C$, say $D = \sum_{P \in C}^{} n_P P$ where $n_P \in \mathbb{Z}$ and $n_P = 0$ for all but finitely many $P \in C$. We say $\text{deg }D= \sum_{P \in C}^{} n_P$.
    \vspace{1mm}
     
    $D$ is \textbf{effective} (written $D\ge 0$) if $n_P \ge 0 ~\forall P \in C$. If $f \in K(C)^\times$, then $\text{div}(f) = \sum_{P \in C}^{} \text{ord}_P(f)P$. The Riemann--Roch space of $D \in \text{Div}(C)$ is  
    \[
        \mathcal{L}(D) = \{f \in K(C)^\times \mid \text{div}(f) + D \ge 0\} \cup \{0\},
    \]
    i.e. the $K$--vector space of rational functions on $C$ with ''poles no worse than specified by $D$'' (i.e. every coefficient of $\text{div}(f)+D$ is nonnegative).
\end{defn}
We quote Riemann--Roch for surfaces of genus 1: We have
\begin{align*}
    \text{dim }\mathcal{L}(D) = \begin{cases}
        \text{deg }D &\text{if deg }D>0\\
        0 \text{ or }1 &\text{if deg }D=0\\
        0 &\text{if deg }D<0.
    \end{cases}
\end{align*}
\begin{example}
    We revisit Example \ref{ex2.3}. We have $\mathcal{L}(2P) = \langle 1,x \rangle$ and $\mathcal{L}(3P) = \langle 1,x,y \rangle$.
\end{example}

\marginpar{24 Jan 2024, Lecture 3}

We still have $\text{char }K \neq 2$ and $\overline{K} = K$.

\begin{prop}\label{prop2.2}
    Let $C \subset \mathbb{P}^2$ be a smooth plane cubic and let $P \in C$ be a point of inflection. Then we may change coordinates such that $C : Y^2Z = X(X-Z)(X-\lambda Z)$ and $P = (0:1:0)$ (for some $\lambda \neq 0,1$).
\end{prop}
\begin{proof}
    First change coordinates such that $P = (0:1:0)$. Then change coordinates such that the tangent line becomes $T_p C = \{Z = 0\}$. Say $C = \{F(X,Y,Z) = 0\} \subset  \mathbb{P}^2$. A point on the tangent line is of the form $(t:1:0)$ and since $P \in C$ is a point of inflection, we get $F(t,1,0) = \text{const}\cdot t^3$, i.e. $F$ has no terms $X^2Y, XY^2$ or $Y^3$.
    \vspace{1mm}
     
    Hence $F = \langle Y^2Z, XYZ, YZ^2, X^3, X^2Z, XZ^2, Z^3 \rangle$. Notably, $Y^2Z$ has a nonzero coefficient, otherwise $P \in C$ would be singular, a contradition to $C$ being smooth. The coefficient of $X^3$ is nonzero as well, otherwise $Z \mid F$. We are free to rescale $X,Y,Z,F$, so WLOG $C$ is defined by \[
    Y^2Z + a_1 XYZ + a_3 YZ^2 = X^3 + a_2 X^2 Z + a_4 XZ^2 + a_6 Z^3.
    \]
    Substituting $Y \mapsto Y - \frac{1}{2}a_1 X - \frac{1}{2}a_3 Z$, we may assume $a_1 = a_3 = 0$. This gives \[
    C : Y^2Z = Z^3 f \left(\frac{X}{Z}\right)
    \]
    for a monic cubic polynomial $f$. Since $C$ is smooth, $f$ has distinct roots, WLOG $0,1,\lambda$, so $C: Y^2Z = X(X-Z)(X-\lambda Z)$.
\end{proof}
The form $Y^2Z + a_1 XYZ + a_3 YZ^2 = X^3 + a_2 X^2 Z + a_4 XZ^2 + a_6 Z^3$ is the Weierstrass form. The form $ Y^2Z = X(X-Z)(X-\lambda Z)$ is the Legendre form.

\textbf{Remark.} It can be shown that the points of inflection of a plane curve $C = \{F(X_1,X_2,X_3) = 0\} \subset \mathbb{P}^2$ are given by solving the Hessian:
\[
\begin{cases}
    &\det H = \det\left(\frac{\partial^2 F}{\partial X_i \partial X_j} \right) = 0\\
    &F(X_1,X_2,X_3) = 0.
\end{cases}
\]
\subsection{The degree of a morphism}
Let $\phi : C_1 \to C_2$ be a nonconstant morphism of smooth projective curves. Then $\phi^* : K(C_2) \to K(C_1)$ by $f \mapsto f \circ \phi$, giving an injective map $\phi^* K(C_2)$ to $K(C_1)$.

\begin{defn}
    The \textbf{degree} of $\phi$ is $\text{deg }\phi = [K(C_1) : \phi^* K(C_2)]$.
    \vspace{1mm}
     
    We say $\phi$ is \textbf{separable} if $K(C_1)/\phi^* K(C_2)$ is a separable field extension.
\end{defn}

Suppose $P \in C_1, Q \in C_2$ and $\phi : P \mapsto Q$. Let $t \in K(C_2)$ be a uniformizer at $Q$.
\begin{defn}
    $e_{\phi}(P) = \text{ord}_P(\phi^* t)$, which is always $\ge 1$ and independent of $t$.
\end{defn}
\begin{theorem}\label{theorem2.3}
    Let $\phi : C_1 \to C_2$ be a nonconstant morphism of smooth projective curves. Then \[
    \sum_{P \in \phi^{-1}(Q)}^{} e_{\phi}(P) = \deg \phi ~\forall  Q \in C_2.
    \]
    Moreover, if $\phi$ is separable, then $e_{\phi}(P)=1$ for all but finitely many $P \in C_1$.
\end{theorem}
We don't prove this.
\vspace{1mm}
 
In particular, this shows that:
\begin{enumerate}[(i)]
    \item $\phi$ is surjective (very important here that we're in $\overline{K}$).
    \item $|\phi^{-1}(Q)|\le \deg \phi$.
    \item If $\phi$ is separable, then equality holds in (ii) for all but finitely many points $Q \in C_2$.
\end{enumerate}
\textbf{Important remark.} Let $C$ be an algebraic curve. A rational map is given by 
\begin{align*}
    C &\to \mathbb{P}^n \\
    \phi &\mapsto (f_0,f_1,\ldots,f_n)
\end{align*}
where $f_0,\ldots,f_n \in K(C)$ are not all zero. Then we have a fact: If $C$ is smooth, then $\phi$ is a morphism. This saves us a lot of time (we can go from a rational map to a morphism immediately).

\section{Weierstrass equations}

We now drop the assumption that $\overline{K} = K$, but we will still assume that $K$ is perfect.

\begin{defn}
    An \textbf{elliptic curve} $E/K$ is a smooth projective curve of genus 1 defined over $K$ with a specified $K$--rational point $O = 0_E$.
\end{defn}
\begin{example}
    $\{X^3 + pY^3 + p^2Z^3 = 0\} \subset \mathbb{P}^2$ is not an elliptic curve over $\mathbb{Q}$, since it has no $\mathbb{Q}$--rational point.
\end{example}
\begin{theorem}
    Every elliptic curve $E$ is isomorphic over $K$ to a curve in Weierstrass form via an isomorphism taking $0_E$ to $(0:1:0)$.
\end{theorem}
\textbf{Remark.} Proposition \ref{prop2.2} treated the special case where $E$ is a smooth plane cubic and $0_E$ is a point of inflection.
\vspace{1mm}
 
\textbf{Fact.} If $D \in \text{Div}(E)$ is defined over $K$, then $\mathcal{L}(D)$ has a basis in $K(E)$ (not just in $\overline{K}(E)$). Here $D$ is defined over $K$ if it is fixed by $\text{Gal}(\overline{K}/K)$ (this is unimportant for us and we just write it down to be rigorous).

\begin{proof}
    $\mathcal{L}(2\cdot 0_E) \subset \mathcal{L}(3\cdot 0_E)$. Pick bases $1,x$ and $1,x,y$. Note $\text{ord}_{0_E}(x) = -2$ and $\text{ord}_{0_E}(y) = -3$ (else $x,y$ don't give a basis). The 7 elements $1,x,y,x^2,xy,x^3,y^2$ lie in the 6--dimensional vector space $\mathcal{L}(60_E)$ (as they have at most a sixth order pole), so they must satisfy a linear dependence relation. 

    \vspace{1mm}
     
    Leaving out $x^3$ or $y^2$ leaves us with 6 elements, all with different order poles, giving a basis for $\mathcal{L}(60_E)$. Hence the coefficients of $x^3$ and $y^2$ are nonzero, so by rescaling $x,y$ (if necessary) we get \[
    E': y^2 + a_1xy + a_2y = x^3 + a_2 x^2 + a_4 x + a_6
    \]
    for some $a_i \in K$. Let $E'$ be the curve defined by this equation (or rather its projective closure). There is a morphism $\phi: E \to E' \subset \mathbb{P}^2$ by $P \mapsto (x(P):y(P):1) = \left(\frac{x}{y}(P): 1: \frac{1}{y}(P)\right)$. (Since $E$ is smooth, we know that this rational map is a morphism). Hence $0_E \mapsto (0:1:0)$.
    \vspace{1mm}
     
    We have $E \stackrel{x}{\to} \mathbb{P}^1$ by $x \mapsto (x:1)$ (and similarly for $y$), so 
    \begin{align*}
        &[K(E):K(x)] = \deg(E \stackrel{x}{\to} \mathbb{P}^1) = \text{ord}_{0_E}\left(\frac{1}{x}\right) = 2\\
        &[K(E):K(y)] = \deg(E \stackrel{y}{\to} \mathbb{P}^1) = \text{ord}_{0_E}\left(\frac{1}{y}\right) = 3.
    \end{align*}
    This gives an inclusion of fields $K(x) \le K(E)$ of degree 2, $K(y) \le K(E)$ of degree 3, while $K(x), K(y) \le K(x,y) \le  K(E)$, so tower law gives $[K(E):K(x,y)] = 1 \implies K(E)=K(x,y)=\phi^* K(E') \implies \deg \phi = 1$. (draw a picture!). This gives us an inverse that is a rational map, which we want to show is a morphism. For this, we just need to show that $E'$ is smooth.
    \vspace{1mm}
     
    If $E'$ were singular, then $E$ and $E'$ are rational, a contradiction. So $E'$ is smooth and hence $\phi^{-1}$ is a morphism, so $\phi$ is an isomorphism.
\end{proof}
\begin{prop}\label{prop3.2}
    Let $E,E'$ be elliptic curves over $K$ in Weierstrass form. Then $E \cong E'$ over $K \iff$ the equations are related by a change of variables 
    \begin{align*}
    x &= u^2x' + r \\
    y &= u^3y' + u^2sx' + t
    \end{align*}
    for $r,s,t,u \in K$ with $u \neq 0$. 
\end{prop}
\begin{proof}
    $\mathcal{L}(2\cdot 0_E) = \langle 1, x \rangle = \langle 1, x' \rangle \implies x = \lambda x' + r$ for some ${\lambda, r \in K, \lambda \neq 0}$. Similarly $\mathcal{L}(3\cdot 0_E) = \langle 1,x,y \rangle = \langle 1,x',y' \rangle \implies y = \mu y' + \sigma x' + t$ for some $\mu, \sigma, t \in K, \mu \neq 0$. 
    \vspace{1mm}
     
    Looking at the coefficients of $x^3$ and $y^2$ tells us that $\lambda^3 = \mu^2$, so $\lambda = u^2, \mu = u^3$ for some $u \in K^\times$. Put $s = \frac{\sigma}{u^2}$ to conclude.
\end{proof}

A Weierstrass equation defines an elliptic curve $\iff$ it defines a smooth curve $\iff \Delta(a_1,\ldots,a_6) \neq 0$, where $\Delta \in \mathbb{Z}[a_1,\ldots,a_6]$ is a certain polynomial.

\vspace{1mm}
 
If $\text{char }K \neq 2,3$, we may reduce to the case $E: y^2 = x^3 + ax + b$. In this case, the discriminant is $\Delta = -16(4a^3 + 27b^2)$.
\begin{cor}
    Assume $\text{char }K \neq 2,3$. Elliptic curves 
    \begin{align*}
        &E : y^2 = x^3 + ax + b \\
        &E' : y^2 = x^3 + a'x + b'
    \end{align*}
    are isomorphic over $K \iff \begin{cases}
        a' = u^4 a \\
        b' = u^6 b
    \end{cases}$ for some $u \in K^\times$.
\end{cor}
\begin{proof}
    $E, E'$ are related by a substitution as in Proposition \ref{prop3.2} with $r=s=t=0$.
\end{proof}
\begin{defn}
    The \textbf{$j$--invariant} is $j(E) = \frac{1728(4a^3)}{4a^3+27b^2}$.
\end{defn}
\begin{cor}
    $E \cong E' \implies j(E) \cong j(E')$ and the converse holds if $K = \overline{K}$.
\end{cor}
\begin{proof}
    $E \cong E' \iff \begin{cases}
        a' &= u^4 a \\
        b' &= u^6 b
    \end{cases}$ for some $u \in K^\times \implies (a^3:b^2) = ((a')^3:(b')^2) \iff j(E) = j(E')$. The middle step is reversible if $K=\overline{K}$.
\end{proof}
\section{The Group Law}
Let $E \subset \mathbb{P}^2$ be a smooth plane cubic with $0_E \in E(K)$ (not immediately assumed to be in Weierstrass form). $E$ meets any line in 3 points, counted with multiplicity.
\vspace{1mm}
 
For $P,Q \in E$, let $S$ be the $3^{\text{rd}}$ point of intersction of $PQ$ with $E$ and then let $R$ be the $3^{\text{rd}}$ intersection of $0_ES$ with $E$. We define $P \oplus Q = R$. (Later we drop the circle and just write $+$). If $P=Q$, instead take the tangent line at $P$, i.e. $T_PE$, etc. This is the ''chord and tangent process''.
\begin{theorem}
    $(E, \oplus)$ is an abelian group.
\end{theorem}
\textbf{Remark.} Here $E$ means $E(\overline{K})$ since we haven't specified a field yet.
\begin{proof}
    \begin{enumerate}[(i)]
        \item $\oplus$ is commutative trivially.
        \item $0_E$ is the identity, since the line through $0_EP$ meets $S$ for the $3^{\text{rd}}$ time at $S$ and then $SP$ meets $E$ for the $3^{\text{rd}}$ time at $0_E$ (drawing a picture makes this obvious).
        \item Inverses: Let $S$ be the $3^{\text{rd}}$ intersection of $T_{0_E}$ with $E$ and $Q$ the $3^{\text{rd}}$ intersection of $PS$ with $E$. Then $P \oplus Q = 0_E$.
        \item Associativity is much harder. We have some setup:
    \end{enumerate}
    \begin{defn}
        $D_1,D_2 \in \text{Div}(E)$ are \textbf{linearly equivalent} if $\exists f \in K(E)^\times$ such that $\text{div}(f) = D_1 - D_2$. Write $D_1 \sim D_2$ and $[D] = \{D' \mid D' \sim D\}$.
    \end{defn}
    \begin{defn}
        The \textbf{Picard group} is $\text{Pic}(E) = \text{Div}(E)/\sim$. Also define $\text{Pic}^0(E) = \text{Div}^0(E)/\sim$ where $\text{Div}^0(E) = \{D \in \text{Div}(E) \mid \text{deg}(D) = 0\}$.
    \end{defn}
    We define $\psi: E \to \text{Pic}^0(E)$ by $P \mapsto [(P)-(0_E)]$.
    \begin{prop}
        \begin{enumerate}[(i)]
            \item $\psi(P \oplus Q) = \psi(P) + \psi(Q)$.
            \item $\psi$ is a bijection.
        \end{enumerate}
    \end{prop}
    \begin{proof}
        \begin{enumerate}[(i)]
            \item WLOG let the lines $PQ$ and $0_ES$ be given by $l=0$ and $m=0$. Then \[
            \text{div}\left(\frac{l}{m}\right) = (P)+(S)+(Q)-(0_E)-(S)-(R),
            \]
            hence $(P)+(Q) \sim (P \oplus Q) + (0_E)$, so $(P\oplus Q)- (0_E) \sim (P)-(0_E) + (Q) - (0_E)$, so $\psi(P \oplus Q) = \psi(P) + \psi(Q)$.
            \item Injectivity: Suppose $\psi(P) = \psi(Q)$ for $P \neq Q$. Then $\exists f \in \overline{K}(E)^\times$ such that $\text{div}(f) = (P)-(0_E)-(Q)+(0_E) = (P) - (Q) \implies E \stackrel{f}{\to} \mathbb{P}^1$ has degree $1$ (for example since evaluation at 0 on the affine line gives that $P$ has one root and $Q$ has one pole), so $E \cong \mathbb{P}^1$, a contradiction.
            \vspace{1mm}
             
            Surjectivity: Let $[D] \in \text{Pic}^0(E)$. Then $D + (0_E)$ has degree 1, so by Riemann--Roch, $\text{dim}\mathcal{L}(D+(0_E)) = 1$, so $\exists 0\neq f \in \overline{K}(E)$ such that $\text{div}(f) + D + (0_E) \ge 0$, but $\text{div}(f) + D + (0_E)$ has degree 1, so $\text{div}(f) + D + (0_E) = (P)$ for some $P \in E \implies (P) - (0_E) \sim D \implies  \psi(P) = [D]$. 
        \end{enumerate}
    \end{proof}
    We conclude that $\psi$ identifies $(E,\oplus)$ with $(\text{Pic}^0(E), +)$, so $\oplus$ is associative.
\end{proof}
\marginpar{29 Jan 2024, Lecture 5}

\textbf{Formulae for $E$ in Weierstrass form.}
Let $E : y^2 + a_1xy + a_3 y  = x^3 +a_2x^2 + a_4 x + a_6$. Choose two points $P_1 = (x_1, y_1)$ and $P_2 = (x_2, y_2)$ on it. Let the line through $P_1$ and $P_2$ be given by $y = \lambda x + \nu$ and let it meet $E$ again at $P' = (x', y')$. We want to find $P_1 \oplus P_2 = P_3 = (x_3, y_3) = \ominus P'$ for $\ominus P$ the reflection of $P$ across the $x$--axis. We easily compute $\ominus P_1 = (x_1, -(a_1x +a_3)-y_1)$. 
\vspace{1mm}
 
Substituting $y = \lambda x + \nu$ into our equation for $E$ and looking at the coefficient of $x^2$ gives $\lambda^2 + a_1 \lambda - a_2 = x_1 + x_2 + x' = x_1 + x_2 + x_3$, so $x_3 = \lambda^2 +a_1 \lambda - a_2 - x_1 - x_2$. For $y_3$ we find $$y_3 = -(a_1x' + a_3) - y' = -(a_1x_3+a_3) - (\lambda x_3  + \nu) = - (\lambda+a_1)x_3 - a_3 - \nu.$$
It remains to find formulas for $\lambda$ and $\nu$.
\begin{itemize}
    \item Case 1. $x_1 = x_2$, but $P_1 \neq P_2$. Then $P_1 \oplus P_2 = 0_E$.
    \item Case 2. $x_1 \neq x_2$. Then $\lambda = \frac{y_2-y_1}{x_2-x_1}$ and $\nu = y_1 - \lambda x_1 = \frac{x_2y_1-x_1y_2}{x_2-x_1}$.
    \item Case 3. $P_1 = P_2$. In this case, compute the equation for the tangent line to get $\lambda, \nu$ as rational expressions in $x_1,x_2,y_1,y_2$.
\end{itemize}
\begin{cor}
    $E(K)$ is an abelian group.
\end{cor}
\begin{proof}
    $E(K)$ is a subgroup of $(E, \oplus)$.
    \begin{itemize}
        \item It has identity $0_E$ by definition.
        \item We have closure and inverses through the formulae above.
        \item Associativity and commutativity is inherited.
    \end{itemize}
\end{proof}
\begin{theorem}\label{theorem4.4}
    Elliptic curves are group varieties, i.e. 
    \begin{align*}
        [-1]&:E \to E, P \mapsto \ominus P\\
        \oplus&: E \times E\to E, (P,Q) \mapsto P \oplus Q
    \end{align*}
    are morphisms of algebraic varieties.
\end{theorem}
\begin{proof}
    By the above formulae, $[-1]:E \to E$ is a rational map, i.e. a morphism by our important remark.
    \vspace{1mm}
     
    For $\oplus$, note by the above formulae that $\oplus : E \times E \to E$ is a rational map regular on 
    \begin{align*}
        U = \{(P,Q) \in E \times E \mid 0_E \not\in \{P,Q, P \oplus Q, P \ominus Q  \}\}.
    \end{align*}
    For $P \in E$, let $\tau_P : E \to E$ be the ''translation by $P$'' map, given by $X \mapsto P \oplus X$. $\tau_P$ is a rational map, hence a morphism. Now for $A, B \in E$, we factor $\oplus$ as 
    \begin{align*}
        E \times E \stackrel{\tau_{\ominus A} \times \tau_{\ominus B}}{\to} E \times E \stackrel{\oplus}{\to} E \stackrel{\tau_{A\oplus B}}{\to} E.  
    \end{align*}
    This shows $\oplus$ is regular on $(\tau_{A} \times \tau_B)(U)$, so $\oplus$ is regular on $E \times E$.
\end{proof}
\textbf{Statement of results.}
The following isomorphisms in (i), (ii), (iv) respect the relevant topologies.
\begin{enumerate}[(i)]
    \item $K=\mathbb{C}$. Then $E(\mathbb{C}) \cong \mathbb{C}/\Lambda \cong \mathbb{R}/\mathbb{Z} \times \mathbb{R}/\mathbb{Z}$ for $\Lambda$ a lattice.
    \item $K=\mathbb{R}$. Then \[
    E(\mathbb{R}) = \begin{cases}
        \mathbb{Z}/2\mathbb{Z} \times \mathbb{R}/\mathbb{Z}& \text{ if }\Delta>0\\
        \mathbb{R}/\mathbb{Z} &\text{ if }\Delta<0.
    \end{cases}
    \] 
    \item $K = \mathbb{F}_q$. Then $\left| \left|E(\mathbb{F}_q)\right| - (q+1)\right| \le 2\sqrt{q}$. This is Hasse's Theorem.
    \item For a local field $[K:\mathbb{Q}_p] <\infty$ with ring of integers $\mathcal{O}_K$, $E(K)$ has a subgroup of finite index isomorphic to $(\mathcal{O}_K,+)$.
    \item For a number field $[K:\mathbb{Q}]<\infty$, $E(K)$ is a finitely generated abelian group (this is the Mordell--Weil Theorem).
    Basic group theory says that if $A$ is a finitely generated abelian group, then $A \cong (\text{finite subgroup})\times \mathbb{Z}^r$. Here $r$ is called the rank of $A$. The proof of Mordell--Weil gives an upper bound for $\text{rank }E(K)$, but there is no known algorithm to compute the rank in all cases.
\end{enumerate}
\textbf{Brief remarks on the case $K=\mathbb{C}$.} Let $\Lambda = \{a \omega_1 + b \omega_2 \mid a,b \in \mathbb{Z}\}$ where $\omega_1,\omega_2$ are a basis for $\mathbb{C}$ as an $\mathbb{R}$--vector space. Then meromorphic functions on the Riemann surface $\mathbb{C}/\Lambda$ correspond bijectively with $\Lambda$--invariant meromorphic functions in $\mathbb{C}$. The function field of $\mathbb{C}/\Lambda$ is generated by $\wp(z)$ and $\wp'(z)$, where 
\begin{align*}
    &\wp(z) = \frac{1}{z^2}+\sum_{\lambda \in \Lambda \setminus \{0\}}^{} \left(\frac{1}{(z-\lambda)^2} - \frac{1}{\lambda^2}\right)\\
    &\wp'(z) = -2\sum_{\lambda \in \Lambda}^{} \frac{1}{(z-\lambda)^3}.
\end{align*}
These satisfy $\wp'(z)^2 = 4\wp(z)^3 - g_2\wp(z) - g_3$ for some constants $g_2,g_3 \in \mathbb{C}$ depending on $\Lambda$. One shows $\mathbb{C}/\Lambda \cong E(\mathbb{C})$, where $E: y^2 = 4x^3 - g_2x -g_3$ which is an isomorphism on both groups (via $z \mapsto (\wp(z),\wp'(z))$) and on Riemann surfaces. We have the following result:
\begin{theorem}[Uniformization theorem]
    Every elliptic curve over $\mathbb{C}$ arises in this way.
\end{theorem}
\begin{defn}
    For $n \in \mathbb{Z}$, let $[n]: E \to E$ be given by $P \mapsto \underbrace{P\oplus P \oplus \ldots \oplus P}_{n\text{ copies}}$ if $n>0$ and $[-n] = [-1] \circ [n]$.
\end{defn}
\begin{defn}
    The \textbf{$n$--torsion subgroup} of $E$ is \[
    E[n] = \text{ker}(E \stackrel{[n]}{\longrightarrow} E).
    \]
\end{defn}
If $K=\mathbb{C}$, then $E(\mathbb{C}) \cong \mathbb{C}/\Lambda$, so $E[n] \cong (\mathbb{Z}/n\mathbb{Z})^2$ and $\text{deg}[n] = n^2$. Call these results (1) and (2). We will show that (2) holds over any field $K = \overline{K}$ and (1) holds if $\text{char }K \nmid n$. We sometimes abuse notation and write $E[n] = E[n](\overline{K})$.
\marginpar{31 Jan 2024, Lecture 6}
\begin{lemma}\label{lemma4.6}
    Assume $\text{char }K \neq 2$ and $E: y^2=f(x)=(x-e_1)(x-e_2)(x-e_3)$ (with $e_i \in \overline{K}$). Then $E[2] = \{0, (e_1,0), (e_2,0), (e_3,0)\} \cong (\mathbb{Z}/2\mathbb{Z})^\times$.
\end{lemma}
\begin{proof}
    Let $P = (x,y) \in E$. Then $2[P] = 0 \iff P = -P \iff (x,y) = (x,-y) \iff y=0$.
\end{proof}
\section{Isogenies}
Let $E_1,E_2$ be elliptic curves.
\begin{defn}
    \begin{enumerate}[(i)]
        \item An \textbf{isogeny} $\phi : E_1 \to E_2$ is a nonconstant morphism with $\phi(0_{E_1}) = 0_{E_2}$.
        \item We say $E_1$ and $E_2$ are \textbf{isogenous} if there is an isogeny between them.
    \end{enumerate}
\end{defn}
In (i), nonconstant is equivalent to surjective on $\overline{K}$--points. See Theorem \ref{theorem2.3}.
\begin{defn}
    $\text{Hom}(E_1,E_2) = \{\text{isogenies }E_1 \to E_2\} \cup \{0\}$ (the constant map at $0_E$). This is an abelian group under $(\phi + \psi)(P) := \phi(P) \oplus \psi(P)$.
\end{defn}
If $E_1 \stackrel{\phi}{\to} E_2 \stackrel{\psi}{\to} E_3$ are isogenies, then $\psi \circ \phi$ is an isogeny. By tower law, $\text{deg}(\psi \circ \phi) = \text{deg}(\psi)\text{deg}(\phi)$.
\begin{prop}
    If $0 \neq n \in \mathbb{Z}$, then $[n] : E \to E$ is an isogeny.
\end{prop}
\begin{proof}
    $[n]$ is a morphism by Theorem \ref{theorem4.4}. We need to show $[n] \neq [0]$. Assume $\text{char }K \neq 2$.
    \begin{itemize}
        \item Case $n=2$. Lemma \ref{lemma4.6} implies that $E[2] \neq E$, so $[2] \neq 0$.
        \item Case $n$ odd. Lemma \ref{lemma4.6} implies that $\exists 0 \neq T \in E[2]$. Then $nT = T \neq 0$, so $[n] \neq [0]$.
    \end{itemize}
    Now use $[mn] = [m]\circ[n]$ to conclude.
    \vspace{1mm}
     
    If $\text{char }K=2$, then we can replace Lemma \ref{lemma4.6} with an explicit lemma about 3--torsion points.
\end{proof}
\begin{cor}
    $\text{Hom}(E_1,E_2)$ is a torsion--free $\mathbb{Z}$--module.
\end{cor}
\begin{theorem}\label{theorem5.3}
    Let $\phi : E_1 \to E_2$ be an isogeny. Then \[
    \phi(P+Q) = \phi(P)+\phi(Q) ~\forall P,Q \in E.
    \]
\end{theorem}
\begin{proof}[Sketch proof.]
    $\phi$ induces a map $\phi_* : \text{Div}^0(E_1) \to \text{Div}^0(E_2)$ by $\sum_{P \in E_1}^{} n_P P \mapsto \sum_{p \in E_1}^{} n_P \phi(P)$. Recall $\phi^* : K(E_2) \hookrightarrow K(E_1)$.
    \vspace{1mm}
     
    \textbf{Fact.} If $f \in K(E_1)$, then $\text{div}(N_{K(E_1)/K(E_2)}f) = \phi^*(\text{div }f)$. So $\phi_*$ sends principal divisors to principal divisors. Since $\phi(0_{E_1}) = 0_{E_2}$, the following diagram commutes: $$\begin{tikzcd}[]
        E_1 & E_2\\
        \text{Pic}^0(E_1) & \text{Pic}^0(E_2)
        \arrow["\phi", from=1-1, to=1-2]
        \arrow["g", from=1-2, to=2-2]
        \arrow["f", from=1-1, to=2-1]
        \arrow["\phi_*", from=2-1, to=2-2]
    \end{tikzcd}$$ (with $f(P) = [(P)-(0_{E_1})], g(Q) = [(Q)-(0_{E_2})]$). Since $\phi_*$ is a group homomorphism, $\phi$ is a group homomorphism.
\end{proof}
\begin{lemma}\label{lemma5.4}
    Let $\phi : E_1 \to E_2$ be an isogeny. Then there exists a morphism $\xi$ making the following diagram commute:
    $$\begin{tikzcd}[]
        E_1 & E_2\\
        \mathbb{P}^1 & \mathbb{P}^1
        \arrow["\phi", from=1-1, to=1-2]
        \arrow["x_2", from=1-2, to=2-2]
        \arrow["x_1 ", from=1-1, to=2-1]
        \arrow["\xi", from=2-1, to=2-2]
    \end{tikzcd}$$
    with $x_i$ the $x$--coordinate in a Weierstrass equation for $E_i$. Moreover, if $\xi(t)=\frac{r(t)}{s(t)}$ with $r,s \in K[t]$ coprime, then $\text{deg}(\phi) = \text{deg}(\xi) = \max(\text{deg}(r),\text{deg}(s))$.
\end{lemma}
\begin{proof}
    For $i=1,2$, $K(E_i)/K(x_i)$ is a degree 2 Galois extension with Galois group generated by $[-1]^*$. By Theorem \ref{theorem5.3}, $\phi \circ [-1] = [-1] \circ \phi$, so if $f \in K(x_2)$, then $[-1]^*(\phi^*f) = \phi^*([-1]^* f) = \phi^* f$ and hence $\phi^* f \in K(x_1)$. Hence we find
    $$\begin{tikzcd}[]
         & K(E_1) = K(x_1,y_1)\\
        K(x_1) & K(E_2) = K(x_2,y_2)\\
        K(x_2) &
        \arrow["2", from=2-1, to=1-2, dash]
        \arrow["2", from=3-1, to=2-2, dash]
        \arrow["", from=2-1, to=3-1, dash]
        \arrow["", from=1-2, to=2-2, dash]
    \end{tikzcd}.$$
    In particular, $\phi^*x_2 = \xi(x_1)$ for some $\xi \in K(t)$. By tower law, $2 \text{deg}(\phi) = 2 \text{deg}(\xi) \implies \text{deg}(\phi) = \text{deg}(\xi)$. Now $K(x_2) \hookrightarrow K(x_1)$ by $x_2 \mapsto \xi(x_1) = \frac{r(x_1)}{s(x_1)}$ for $r,s \in K[t]$ coprime. Then minimal polynomial of $x_1$ over $K(x_2)$ is $F(t) =r(t) -s(t)x_2 \in K(x_2)[t]$. This is true as $F(x_1)=0$, $F$ is irreducible on $K[x_2,t]$ (since $r,s$ are coprime) and by Gauss' Lemma, $F$ is irreducible on $K(x_2)[t]$. Hence $\text{deg}(\phi) = \text{deg}(\xi) = [K(x_1):K(x_2)] = \text{deg}(F) = \max(\text{deg}(r),\text{deg}(s))$.
\end{proof}
\begin{lemma}
    $\text{deg}[2] = 4$.
\end{lemma}
\begin{proof}
    Assume $\text{char }K \neq 2,3$, so $E : y^2 = x^3+ax+b = f(x)$. If $P = (x,y)$, then $x(2P) = \left(\frac{3x^2+a}{2y}\right)^2-2x = \frac{(3x^2+a)^2-2xf(x)}{4f(x)}$. The numerator and denominator are coprime, since otherwise $\exists \theta \in \overline{K}$ with $f(\theta) = f'(\theta) = 0$, meaning $f$ has a multiple root, contradiction. We are now done by Lemma \ref{lemma5.4}, since $\text{deg}[2] = \max(3,4) = 4$.
\end{proof}
\begin{defn}
    Let $A$ be an abelian group. Then a map $q : A \to \mathbb{Z}$ is a quadratic form if
    \begin{enumerate}[(i)]
        \item $q(nx) = n^2q(x)~\forall n \in \mathbb{Z}, x \in A$.
        \item $(x,y) \mapsto q(x+y)-q(x)-q(y)$ is $\mathbb{Z}$--bilinear.
    \end{enumerate}
\end{defn}
\begin{lemma}
    $q : A \to \mathbb{Z}$ is a quadratic form if and only if it satisfies the parallelogram law $q(x+y)+q(x-y) = 2q(x)+2q(y) ~\forall x,y \in A$.
\end{lemma}
\begin{proof}
    $(\implies)$. Let $\langle x,y \rangle = q(x+y)-q(x)-q(y)$. Then $\langle x,x \rangle = q(2x) - 2q(x) = 2q(x)$ by (i) with $n=2$. By (ii), $\langle x+y, x+y \rangle + \langle x-y, x-y \rangle = 2 \langle x,x \rangle + 2 \langle y, y \rangle$, which implies $q(x+y)+q(x-y)=2q(x)+2q(y)$.
    \vspace{1mm}
     
    $(\impliedby)$. This is on Ex. Sheet 2.
\end{proof}
\marginpar{02 Jan 2024, Lecture 7}
\begin{theorem}\label{theorem5.7}
    $\text{deg}:\text{Hom}(E_1,E_2) \to \mathbb{Z}$ is a quadratic form (with $\text{deg}(0)=0$).
\end{theorem}
\begin{proof}
    Assume $\text{char }K \neq 2,3$ and write $E_2 = y^2=x^3+ax+b$. Let $P, Q \in E_2$ with $P,Q,P+Q,P-Q$ all nonzero and let $x_1,x_2,x_3,x_4$ be the $x$--coordinates of these points.
    \begin{lemma}\label{lemma5.8}
        There exist polynomials $W_0,W_1,W_2 \in \mathbb{Z}[a,b][x_1,x_2]$ of degree $\le 2$ in $x_1$ and of degree $\le 2$ in $x_2$ such that 
        \begin{align*}
            (1:x_3+x_4:x_3x_4) = (W_0:W_1:W_2)
        \end{align*}
    \end{lemma}
    \begin{proof}
        Method 1: Direct calculation (results on the formula sheet) gives the result (e.g. $W_0 = (x_1-x_2)^2$).
        \vspace{1mm}
         
        Method 2: Let $y = \lambda x + \nu$ be the line through $P$ and $Q$. Substituting, we get $x^3+ax+b -(\lambda x + \nu)^2 = (x-x_1)(x-x_2)(x-x_3) = x^3 - s_1x^2 + s_2 x - s_3$ where $s_i$ is the $i^{\text{th}}$ symmetric polynomial in $x_1,x_2,x_3$. Comparing coefficients gives $\lambda^2 = s_1,  -2\lambda \nu = s_2-a, \nu^2 = s_3+b$. Eliminating $\lambda$ and $\nu$ gives $$F(x_1,x_2,x_3) =(s_2-a)^2 - 4s_1(s_3+b) = 0,$$
        where $F$ has degree at most $2$ in each $x_i$. Hence $x_3$ is a root of the quadratic $W(t) = F(x_1,x_2,t)$. Repeating this for the line through $P$ and $-Q$ shows that $x_4$ is the other root of $W(t)$. Therefore 
        \begin{align*}
            W(t) &= W_0(t-x_3)(t-x_4) = W_0t^2 - W_1t + W_2\\
            \implies & (1:x_3+x_4:x_3x_4) = (W_0:W_1:W_2).
        \end{align*}
    \end{proof}
    We now show that if $\phi, \psi \in \text{Hom}(E_1,E_2)$, then $\text{deg}(\phi + \psi) + \text{deg}(\phi - \psi) \le 2\text{deg}(\phi) + 2\text{deg}(\psi)$. We may assume that $\phi, \psi, \phi+\psi, \phi-\psi$ are not the zero maps (otherwise we're done trivially, or use $\text{deg}[-1]=1,~\text{deg}[2]=4$). Now
    \begin{align*}
        \phi: (x,y) \mapsto (\xi_1(x), \ldots)\\
        \psi: (x,y) \mapsto (\xi_2(x), \ldots)\\
        \phi+\psi: (x,y) \mapsto (\xi_3(x), \ldots)\\
        \phi-\psi: (x,y) \mapsto (\xi_4(x), \ldots).
    \end{align*}
    Lemma \ref{lemma5.8} implies $(1:\xi_3+\xi_4: \xi_3\xi_4) = ((\xi_1-\xi_2)^2: \ldots)$. Say $\xi_i = \frac{r_i}{s_i}$ for $r_i,s_i \in K[t]$ coprime. This gives $$(s_3s_4:r_3s_4+r_4s_3:r_3r_4) \stackrel{(\star)}{=} ((r_1s_2-r_2s_1)^2 : \ldots)$$
    where every term is quadratic in $r_3,r_4,s_3$ and $s_4$. Hence (as the terms on the LHS of $(\star)$ are coprime)
    \begin{align*}
        \text{deg}(\phi+\psi) + \text{deg}(\phi-\psi) &= \max(\text{deg}(r_3),\text{deg}(s_3)) + \max(\text{deg}(r_4),\text{deg}(s_4)) \\
        &= \max(\text{deg}(s_3s_4),\text{deg}(r_3s_4+r_4s_3), \text{deg}(r_3r_4))\\
        &\le 2\max(\text{deg}(r_1),\text{deg}(s_1)) + 2\max(\text{deg}(r_2),\text{deg}(s_2))\\
        &= 2\text{deg}(\phi)+2\text{deg}(\psi).
    \end{align*}
    Now replace $\phi$ and $\psi$ by $\phi+\psi$ and $\phi-\psi$ and use $\text{deg}[2]=4$ to get \[
    4\text{deg}(\phi) + 4\text{deg}(\psi) = \text{deg}(2\phi) + \text{deg}(2\psi) \le 2\text{deg}(\phi+\psi) + 2\text{deg}(\phi-\psi).
    \]
    This gives the parallelogram law, so $\text{deg}$ is a quadratic form.
\end{proof}
\begin{cor}\label{cor5.9}
    $\text{deg}(n \phi) = n^2 \text{deg}(\phi)$. In particular, $\text{deg}[n] = n^2$.
\end{cor}
\begin{example}
    Let $E/K$ be an elliptic curve. Suppose $\text{char }K\neq 2$ and $0\neq T \in E(K)[2]$. WLOG let $E: y^2 = x(x^2+ax+b)$ for $a,b \in K, b(a^2-4b)\neq 0$ (by moving a root to zero) and WLOG $T=(0,0)$. 
    \vspace{1mm}
     
    If $P = (x,y)$ and $P' = P+T = (x',y')$, then 
    \begin{align*}
        &x' = \left(\frac{y}{x}\right)^2-a-x = \frac{x^2+ax+b}{x}-a-x = \frac{b}{x}\\
        &y' = -\left(\frac{y}{x}\right)x' = -\frac{by}{x^2}.
    \end{align*}
    We let $\xi = x + x' + a = \left(\frac{y}{x}\right)^2, \eta = y + y' = \frac{y}{x}\left(x-\frac{b}{x}\right)$.
    Then 
    \begin{align*}
        \eta^2 = \left(\frac{y}{x}\right)^2\left(\left(x+\frac{b}{x}\right)^2-4b \right) = \xi((\xi-a)^2-4b) = \xi(\xi^2-2a\xi + a^2-4b).
    \end{align*}
    Let $E' : y^2 = x(x^2+a'x+b')$ with $a'=-2a, b' = a^2-4b$. There is an isogeny $\phi : E \to E'$ given by $(x,y) \mapsto \left(\left(\frac{y}{x}\right)^2 : \frac{y(x^2-b)}{x^2} : 1 \right)$.
    \vspace{1mm}
     
    Sanity check/finding where $0_E$ maps to: $x$ is a double pole, $y$ is a triple pole, so $\left(\frac{y}{x}\right)^2$ is a double pole and $\frac{y(x^2-b)}{x^2}$ is a triple pole (and the last coordinate $1$ has degree 0). Multiplying through by a cube of a uniformizer, the degrees go from $(-2,-3,0)$ to $(1,0,3)$, so $0_E \mapsto (0:1:0)$.
    \vspace{1mm}
     
    To compute $\text{deg}(\phi)$, $\left(\frac{y}{x}\right)^2 = \frac{x^2+ax+b}{x}$ with the numerator and denominator coprime as $b \neq 0$, so by Lemma \ref{lemma5.4}, $\text{deg}(\phi)=2$. We say $\phi$ is a \textbf{2--isogeny}.
\end{example}
\section{The invariant differential}
For $C$ some algebraic curve over $K=\overline{K}$.
\begin{defn}
    The space of differentials $\Omega_C$ (sometimes called one--forms) is the $K(C)$--vector space generated by $\mathrm{d}f$ for all $f \in K(C)$ subject to the relations
    \begin{enumerate}[(i)]
        \item $\mathrm{d}(f+g)=\mathrm{d}f+\mathrm{d}g$.
        \item $\mathrm{d}(fg)=f\mathrm{d}g+g\mathrm{d}f$.
        \item $\mathrm{d}a = 0~\forall a \in K$.
    \end{enumerate}
\end{defn}
\textbf{Fact.} $\Omega_C$ is a 1--dimensional $K(C)$--vector space. 
\marginpar{05 Feb 2024, Lecture 8}
\vspace{1mm}
 
Let $0\neq \omega \in \Omega_C$, let $P \in C$ be a smooth point and let $t \in K(C)$ be a uniformizer at $P$. Then $\omega = f \mathrm{d}t$ for some $f \in K(C)^\times$. We define $\text{ord}_P(\omega) = \text{ord}_P(f)$, which is independent of the choice of $t$.
\vspace{1mm}
 
\textbf{Fact.} Suppose $f \in K(C)^\times$ with $\text{ord}_P(f) = n \neq 0$. If $\text{char }K \nmid n$, then $\text{ord}_P(\mathrm{d}f)=n-1$. 
\vspace{1mm}
 
We assume that $C$ is a smooth projective curve.
\begin{defn}
    We define $\text{div}(\omega) = \sum_{P \in C}^{} \text{ord}_P(\omega)P \in \text{Div}(C)$. Here we use the fact that $\text{ord}_P(\omega) = 0$ for all but finitely many $P \in C$.
\end{defn}
\begin{defn}
    A differential $\omega \in \Omega_C$ is regular if $\text{div}(\omega)\ge 0$. We define the genus $g(C)$ of $C$ to be \[
    g(C) = \text{dim}_K \{\omega \in \Omega_C \mid \text{div}(\omega)\ge 0\},
    \]
    where the set on the RHS is the set of regular differentials.
\end{defn}
As a consequence of Riemann--Roch, we have that if $0 \neq \omega \in \Omega_C$, then $\text{deg}(\text{div}(\omega)) = 2g(C)-2$.
\begin{lemma}
    Assume $\text{char }K \neq 2$ and let $E : y^2 = (x-e_1)(x-e_2)(x-e_3)$ for $e_1,e_2,e_3$ distinct. Then $\omega = \frac{\mathrm{d}x}{y}$ is a differential on $E$ with no zeroes or poles, which implies $g(E)=1$. In particular, the $K$--vector space of regular differentials on $E$ is 1--dimensional (see previous fact), spanned by $\omega$.
\end{lemma}
\begin{proof}
    Let $T_i = (e_i,0)$. Then $E[2]=\{0,T_1,T_2,T_3\}$ and $\text{div}(y) \stackrel{(\dagger)}{=} (T_1)+(T_2)+(T_3)-3(0)$. For $0 \neq P \in E$, $\text{div}(x-x_P) = (P)+(-P)-2(0)$. 
    \begin{itemize}
        \item If $P \in E\setminus E[2]$, then $\text{ord}(x-x_P)=1 \implies \text{ord}_P(\mathrm{d}x) = 0$.
        \item If $P = T_i$, then $\text{ord}_P(x-x_P) = 2 \implies \text{ord}_P(\mathrm{d}x)=1$. 
        \item If $P=0$, then $\text{ord}_P(x)= - 2 \implies \text{ord}_P(\mathrm{d}x) = -3$.
    \end{itemize}
    Hence $\text{div}(\mathrm{d}x) = (T_1)+(T_2)+(T_3) - 3(0)$, which with $(\dagger)$ gives $\text{div}\left(\frac{\mathrm{d}x}{y}\right) = 0$.
\end{proof}
\begin{defn}
    For $\phi : C_1 \to C_2$ a nonconstant morphism, we define 
    \begin{align*}
        \phi^* : \Omega_{C_2} &\to \Omega_{C_1}\\
        f \mathrm{d}g &\mapsto \phi^* f d(\phi^* g).
    \end{align*}
\end{defn}
\begin{lemma}
    Let $P \in E$, $\tau_P : E \to E$ by $X \mapsto X+P$ and $\omega = \frac{\mathrm{d}x}{y}$ as above. Then $\tau_P^*\omega = \omega$. We say $\omega$ is the \textbf{invariant differential}.
\end{lemma}
\begin{proof}
    $\tau_P^*\omega$ is a regular differential on $E$, so $\tau_P^* \omega = \lambda_P \omega$ for some $\lambda_P \in K^\times$. The map $E \to \mathbb{P}^1$ by $P \mapsto \lambda_P$ is a morphism of smooth projectives curves, but it is not surjective (as it misses $0$ and $\infty$). Hence it is constant by Theorem \ref{theorem2.3}, i.e. $\exists \lambda \in K^\times$ such that $\tau_P^* \omega = \lambda \omega ~\forall P \in E$. Taking $P=0$ shows $\lambda=1$.
\end{proof}
\textbf{Remark.} If $K=\mathbb{C}$ and $\mathbb{C}/\Lambda \cong E(\mathbb{C})$ by $ z\mapsto (\wp(z),\wp'(z)) := (x,y)$, then $\frac{\mathrm{d}x}{y} = \frac{\wp'(z)\mathrm{d}z}{\wp'(z)}=\mathrm{d}z$, which is invariant under $z \mapsto z + \text{const}$.
\begin{lemma}\label{lemma6.3}
    Let $\phi,\psi \in \text{Hom}(E_1,E_2)$. Let $\omega$ be the invariant differential on $E_2$. Then $(\phi+\psi)^* \omega = \phi^* \omega + \psi^* \omega$.
\end{lemma}
\begin{proof}
    Write $E$ for $E_2$. We have the maps
    \begin{align*}
        E \times E &\to E \\
        \mu: (P,Q) &\mapsto P +Q\\
        \text{pr}_1 : (P,Q)&\mapsto P\\
        \text{pr}_2 : (P,Q)&\mapsto Q. 
    \end{align*}
    \textbf{Fact.} $\Omega_{E \times E}$ is a 2--dimensional $K(E \times E)$--vector space with basis $\text{pr}_1^*\omega$ and $\text{pr}_2^*\omega$. Consequently, $\mu^* \omega \stackrel{(\dagger)}{=} f \text{pr}_1^* \omega + g \text{pr}_2^* \omega$ for some $f,g \in K(E \times E)$.
    \vspace{1mm}
     
    For fixed $Q \in E$, let $i_Q : E \to E \times E$ by $P \mapsto (P,Q)$. Applying $i^*_Q$ to $(\dagger)$ gives 
    \begin{align*}
        (\underbrace{\mu \circ i_Q}_{\tau_Q})^* \omega &= (i_Q^* f)(\underbrace{\text{pr}_1 \circ i_Q}_{\text{identity map}})^* \omega + (i_Q^* g)(\underbrace{\text{pr}_2 \circ i_Q}_{\text{constant map}})^* \omega\\
        \implies & \tau_Q^* \omega = (i_Q^*f)\omega + 0.
    \end{align*}
    As $\tau_Q^*\omega = \omega$ by the previous lemma, we conclude $i_Q^*f = 1 ~\forall q \in E$, so $f(P,Q) = 1 ~\forall P,Q \in E$. Similarly $g(P,Q) = 1 ~\forall P,Q \in E$, so $(\dagger)$ gives $\mu^* \omega = \text{pr}_1^* \omega + \text{pr}_2^* \omega$. Now pull back using
    \begin{align*}
        E_1 &\to E \times E\\
        P &\mapsto (\phi(P),\psi(P))
    \end{align*}
    to get $(\phi+\psi)^*\omega = \phi^* \omega + \psi^* \omega$.
\end{proof}
\begin{lemma}\label{lemma6.4}
    Let $\phi : C_1 \to C_2$ be a nonconstant morphism. Then $\phi$ is separable if and only if $\phi^* : \Omega_{C_2} \to \Omega_{C_1}$ is nonzero.
\end{lemma}
\begin{proof}
    Omitted.
\end{proof}
\begin{example}
    Let $\mathbb{G}_m = \mathbb{A}^1 \setminus \{0\}$ be the multiplicative group. For $n\ge 2$ an integer, consider $\phi : \mathbb{G}_m \to \mathbb{G}_m$ by $x \mapsto x^n$. Then $\phi^*(\mathrm{d}x)=\mathrm{d}(x^n)=nx^{n-1}\mathrm{d}x$. So if $\text{char }K \nmid n$, then $\phi$ is separable, so $\left|\phi^{-1}(Q)\right|=\text{deg} \phi$ for all but at most finitely many $Q \in \mathbb{G}_m$.
    \vspace{1mm}
     
    But $\phi$ is a group homomorphism, so $\left|\phi^{-1}(Q)\right|=\left|\text{ker}(Q)\right| ~\forall Q \in \mathbb{G}_m$. Hence $\left|\text{ker}Q \right| = \text{deg}\phi = n$. This shows that $K = \overline{K}$ contains exactly $n$ distinct $n^{\text{th}}$ roots of unity.
\end{example}
\marginpar{07 Feb 2024, Lecture 9}
\begin{theorem}\label{theorem6.5}
    \footnote{Remember that $\overline{K} = K$ here.}If $\text{char }K \nmid n$, then $E[n] = (\mathbb{Z}/n\mathbb{Z})^2$.
\end{theorem}
\begin{proof}
    Lemma \ref{lemma6.3} and induction imply $[n]^* \omega = n \omega$ where $\text{char }K \nmid n$, so $[n]$ is separable by Lemma \ref{lemma6.4}. Hence $\left|[n]^{-1}(Q)\right| = \text{deg}[n]$ for all but finitely many points $Q \in E$. But $[n]$ is a group homomorphism, so $\left|[n]^{-1} Q\right| = \left|E[n]\right| ~\forall Q \in E$. We conclude that $\left|E[n]\right|=\text{deg}[n] = n^2$ by Corollary \ref{cor5.9}.
    \vspace{1mm}
     
    By classification of finite abelian groups, $E[n] \cong \mathbb{Z}/d_1\mathbb{Z} \times \ldots \times \mathbb{Z}/d_t\mathbb{Z}$ with $d_1 \mid d_2 \mid \ldots \mid d_t$, but $d_t \mid n$, and if $p$ is a prime with $p \mid d_1$, then $E[p] \cong (\mathbb{Z}/p\mathbb{Z})^t$, so $\left|E[p]\right| = p^2$, so $t=2$. Hence $d_1 \mid d_2 \mid n$ with $d_1d_2=n^2$, so $d_1=d_2=n$ and so $E[n] \cong (\mathbb{Z}/n\mathbb{Z})^2$.
\end{proof}
\textbf{Remark.} If $\text{char }K = p$, then $[p]$ is inseparable. It can be shown that either $E[p^r] \cong \mathbb{Z}/p^r \mathbb{Z} ~\forall r\ge 1$ or $E[p^r] = 0~\forall r\ge 1$ (the ''ordinary'' case and the ''supersingular'' case).

\textbf{Remark about the remark.} Do not use this remark to trivialize a question on Ex. Sheet 2.

\section{Elliptic curves over finite fields}
\begin{lemma}\label{lemma7.1}
    Let $A$ be an abelian group. Let $q : A \to \mathbb{Z}$ be a positive definite quadratic form. Then \[
    \underbrace{\left|q(x+y)-q(x)-q(y)\right|}_{\langle x,y \rangle}\le 2\sqrt{q(x)q(y)}.
    \] 
\end{lemma}
\begin{proof}
    We may assume $x \neq 0$, otherwise the result is clear. Hence $q(x) \neq 0$. Let $m,n \in \mathbb{Z}$, then 
    \begin{align*}
        0&\le q(mx+ny) = \frac{1}{2}\langle mx+ny,mx+ny \rangle \\
        &= m^2q(x) + mn \langle x,y \rangle +n^2q(y) \\
        &= q(x) \left( m + \frac{\langle x,y \rangle}{2q(x)}n\right)^2 + \left(q(y)- \frac{\langle x,y \rangle^2}{4q(x)}\right)n^2.
    \end{align*}
    Get rid of the first term by taking $m = - \langle x, y \rangle$ and $n = 2q(x)$ to deduce $\langle x,y \rangle^2 \le 4q(x)q(y)$, so the result follows.
\end{proof}
\begin{theorem}[Hasse]
    Let $E/\mathbb{F}_q$ be an elliptic curve. Then \[
    \left|\# E(\mathbb{F}_q) - (q+1)\right| \le 2\sqrt{q}.
    \]    
\end{theorem}
\begin{proof}
    Recall $\text{Gal}(\mathbb{F}_{q^r}/\mathbb{F}_q)$ is cyclic of order $r$, generated by the Frobenius map $x \mapsto x^q$. Let $E$ have Weierstrass equation with coefficients $a_1,\ldots,a_6 \in \mathbb{F}_q$ (and note that $a_i^q = a_i ~\forall i$). 
    \vspace{1mm}
     
    Define the Frobenius endomorphism $\phi: E \to E$ by $(x,y) \mapsto (x^q,y^q)$, which is an isogeny of degree $q$. Then $E(\mathbb{F}_q) = \{P \in E \mid \phi(P) = P\} = \text{ker}(1-\phi)$. We have
    \begin{align*}
        \phi^* \omega = \phi^* \left(\frac{\mathrm{d}x}{y}\right) = \frac{d(x^q)}{y^q} = \frac{qx^{q-1}\mathrm{d}x}{y^q} = 0
    \end{align*}
    as $q=p^n$, so $p \mid q$. By Lemma \ref{lemma6.3},
    \begin{align*}
        (1-\phi)^* \omega = \omega - \phi^* \omega = \omega \neq 0,
    \end{align*}
    so $1-\phi$ is separable. By Theorem \ref{theorem2.3} and the fact that $1-\phi$ is a group homomorphism, we argue in the proof of Theorem \ref{theorem6.5} that $$\underbrace{\left|\text{ker}(1-\phi)\right|}_{\left|E(\mathbb{F}_q)\right|} = \text{deg}(1-\phi).$$
    The map $\text{deg}: \text{Hom}(E,E) \to \mathbb{Z}$ is a positive definite quadratic form by Theorem \ref{theorem5.7}. Hence by Lemma \ref{lemma7.1}, 
    \begin{align*}
        &\left|\text{deg}(1-\phi) - 1 - \text{deg}\phi\right|\le 2\sqrt{\text{deg}\phi}\\
        \implies&\left|\# E(\mathbb{F}_q)-q-1\right| \le 2\sqrt{q}.
        \qedhere
    \end{align*}
\end{proof}
\begin{defn}
    For $\phi, \psi \in \text{End}(E) = \text{Hom}(E,E)$, we put $\langle \phi, \psi \rangle = \text{deg}(\phi+\psi)-\text{deg}(\phi)-\text{deg}(\psi)$ and $\text{tr}(\phi) = \langle \phi,1 \rangle$.
\end{defn}
\begin{cor}\label{cor7.3}
    Let $E/\mathbb{F}_q$ be an elliptic curve and let $\phi \in \text{End}(E)$ be the $q^{\text{th}}$ power Frobenius map. Then $\# E(\mathbb{F}_q) = q+1 - \text{tr}(\phi)$ and $\left|\text{tr}(\phi)\right|\le 2\sqrt{q}$.
\end{cor}
\textbf{Zeta functions.} For $K$ a number field, 
\begin{align*}
    \zeta_K(s) = \sum_{\mathfrak{a} \subset \mathcal{O}_K}^{} \frac{1}{(N(\mathfrak{a}))^s} = \prod_{\mathfrak{p} \subset \mathcal{O}_K, \mathfrak{p} \text{ prime}}^{} \left(1-\frac{1}{(N(\mathfrak{p}))^s}\right)^{-1}.
\end{align*} 
For $K$ a function field, i.e. $K = \mathbb{F}_q(C)$ where $C$ is a smooth projective curve,
\begin{align*}
    \zeta_K(s) = \prod_{x \in \left|C\right|} \left(1-\frac{1}{(Nx)^{s}}\right)^{-1}, 
\end{align*}
where $\left|C\right| = \{\text{closed points of }C\} = \{\text{orbits for the action of Gal}(\overline{\mathbb{F}_q}/\mathbb{F}_q) \text{ on }C(\overline{\mathbb{F}_q})\}$ and $Nx = q^{\text{deg }x}$, where $\text{deg }x$ is the size of the corresponding orbit (these definitions are borrowed from scheme theory). We have $\zeta_K(s) = F(q^{-s})$ for some $F \in \mathbb{Q}[[T]]$. We have
\begin{align*}
    F(T) &= \prod_{x \in \left|C\right|}^{} \left(1-T^{\text{deg }x}\right)^{-1} \\
    \implies \log F(T) &= \sum_{x \in \left|C\right|}^{} \sum_{m=1}^{\infty} \frac{1}{m}T^{m \text{deg }x} \\
    \implies T\frac{\mathrm{d}}{\mathrm{d}T} \log F(T) &= \sum_{x \in \left|C\right|}^{} \sum_{m=1}^{\infty} \text{deg }x T^{m \text{deg }x} \\
    &\stackrel{n = m\text{deg }x}{=} \sum_{n=1}^{\infty} \left(\sum_{x \in \left|C\right|,\text{deg }x \mid n}^{} \text{deg }x \right)T^n \\
    &= \sum_{n=1}^{\infty} \# C(\mathbb{F}_{q^n})T^n \\
    \implies F(T) &= \exp \left(\sum_{n=1}^{\infty} \frac{\# C(\mathbb{F}_{q^n})}{n}T^n \right).
\end{align*}
\begin{defn}
    The zeta function of a smooth projective curve $C/\mathbb{F}_q$ is 
    \[
    Z_C(T) = \exp \left(\sum_{n=1}^{\infty} \frac{\# C(\mathbb{F}_{q^n})}{n}T^n \right).
    \]
\end{defn}
\marginpar{09 Feb 2024, Lecture 10}
\begin{theorem}\label{theorem7.4}
    Let $E/\mathbb{F}_q$ be an elliptic curve with $\#E(\mathbb{F}_q) = q+1 - a$. Then \[
    Z_E(T) = \frac{1-aT+qT^2}{(1-T)(1-qT)}.
    \]
\end{theorem}
\begin{proof}
    Let $\phi : E \to E$ be the $q$--power Frobenius map. By Corollary \ref{cor7.3}, $\# E(\mathbb{F}_q) = q+1 - \text{tr}(q)$, so $\text{tr}(\phi) = a$ and $\text{deg}(\phi) = q$. By a result from Ex. Sheet 2, $\phi^2 - a \phi + q = 0$. Hence $\phi^{n+2} - a \phi^{n+1}  + q \phi^n = 0$. As the trace is linear, $\text{tr}(\phi^{n+2})-a \text{tr}(\phi^{n+1}) + q \text{tr}(\phi^n) = 0$. The second order difference equation with initial conditions $\text{tr}(1)=\langle 1,1 \rangle = 2^2-1^2-1^2 = 2$ and $\text{tr}(\phi) = a$ has solution \[
    \text{tr}(\phi^n) = \alpha^n + \beta^n
    \]
    for $\alpha, \beta \in \mathbb{C}$ are roots of $X^2 - aX + q = 0$.\footnote{We don't need to worry about the case where the roots are equal, since we don't want a general solution, just a solution satisfying our initial conditions.} Apply Corollary \ref{cor7.3} again to get that $\#E(\mathbb{F}_{q^n}) = q^n +1 - \text{tr}(\phi^n) = 1 + q^n - \alpha^n - \beta^n$. Hence
    \begin{align*}
        Z_E(T) &= \exp \sum_{n=1}^{\infty} \left(\frac{T^n}{n}+\frac{(qT)^n}{n}-\frac{(\alpha T)^n}{n} - \frac{(\beta T)^n}{n} \right) \\
        &= \exp \left( -\log(1-T) - \log(1-qT) + \log(1-\alpha T)+ \log(1 - \beta T) \right) \\
        &= \frac{(1-\alpha T)(1 - \beta T)}{(1-T)(1-qT)} \\
        &= \frac{1-a T +qT^2}{(1-T)(1-qT)}.
        \qedhere
    \end{align*}
\end{proof}
\textbf{Remark.} Hasse's theorem tells us that $\left|a\right|\le 2\sqrt{q}$, so the discriminant $a^2-q$ is nonpositive, so the roots are complex conjugates, i.e. $\alpha = \overline{\beta}$, and $\left|\alpha\right|=\left|\beta\right|\stackrel{(\dagger)}{=} \sqrt{q}$.
\vspace{1mm}
 
Let $K = \mathbb{F}_q(E)$, then $\zeta_K(s) = 0 \implies  Z_E(q^{-s}) = 0 \implies q^{-s} \in \{\frac{1}{\alpha},\frac{1}{\beta}\} \implies q^s \in \{\alpha,\beta\} \implies q^{\text{Re}(s)} = \left|\alpha\right| = \left|\beta\right| \implies \text{Re}(s) = \frac{1}{2}$. This proves the Riemann hypothesis for elliptic curves over finite fields.

\section{Formal groups}
\begin{defn}
    Let $R$ be a ring and $I \subset R$ an ideal. The \textbf{$I$--adic topology} on $R$ has basis $\{r+I^n \mid r \in R, n\ge 1\}$.
\end{defn}
\begin{defn}
    A sequence $(x_n)$ in $R$ is \textbf{Cauchy} if $\forall k, ~ \exists N$ such that $x_m - x_n \in I^k ~\forall m,n \ge N$.
\end{defn}
\begin{defn}
    $R$ is \textbf{complete} if 
    \begin{enumerate}[(i)]
        \item $\bigcap_{n\ge 0} I^n = \{0\}$ (this is a Hausdorff--type condition).
        \item Every Cauchy sequence converges.
    \end{enumerate}
\end{defn}
\textbf{Useful remark.} If $x \in I$, then $\frac{1}{1-x}=1+x+x^2+\ldots$. This exists as the sequence of partial sums form a Cauchy sequence, and then we check that the result it converges to is an inverse for $\frac{1}{1-x}$. Hence $1-x \in R^{\times}$.
\begin{example}
    Basically the only two examples we care about in this course are:
    \begin{itemize}
        \item $R = \mathbb{Z}_p$, the $p$--adic integers, and $I = p \mathbb{Z}_p$.
        \item $R = \mathbb{Z}[[t]]$ and $I=(t)$.
    \end{itemize} 
\end{example} 
\begin{lemma}[Hensel's lemma]\label{lemma8.1}
    Let $R$ be complete with respect to an ideal $I$. Let $F \in R[X]$, $s\ge 1$ with $s \in \mathbb{Z}$. Suppose $a \in R$ satisfies 
    \begin{align*}
        &F(a) \equiv 0 \pmod{I^s}\\
        &F'(a) \in R^\times
    \end{align*}
    Then there exists a unique $b \in R$ such that $F(b)=0$ and $b \equiv a \pmod{I^s}$.
\end{lemma}
\begin{proof}
    Let $u \in R^\times$ be such that $F'(a) = u \pmod{I}$ (e.g. we could take $u = F'(a)$). Replacing $F(X)$ by $\frac{F(X+a)}{u}$ we may assume $a = 0$ and ${F'(0) \equiv 1 \pmod{I}}$. We put $x_0 = 0$ and $x_{n+1} \stackrel{(\dagger)}{=}  x_n - F(x_n)$. Each induction shows that $x_n \equiv 0 \pmod{I^s} ~\forall n ~ (\ddagger)$. Now use the useful identity $$F(X)-F(Y)=(X-Y)(F'(0) + XG(X,Y) + YH(X,Y))$$
    for some $G,H \in R[X,Y]$. Call this identity $(\star)$.
    \vspace{1mm}
     
    We claim that $x_{n+1} \equiv x_n \pmod{I^{n+s}} ~\forall n\ge 0$. To prove this, use induction. The case $n=0$ is clear. Suppose $x_n \equiv x_{n-1} \pmod{I^{n+s-1}}$. By $(\star)$, 
    \begin{align*}
        F(x_n) - F(x_{n-1}) = (x_n-x_{n-1})(1+c)
    \end{align*}
    for some $c \in I$. Modulo $I^{n+s}$ we now use $(\dagger)$ to get
    \begin{align*}
        &F(x_n) - F(x_{n-1}) \equiv x_n-x_{n-1} \pmod{I^{n+s}} \\
        \implies & x_n - F(x_n) = x_{n-1} - F(x_{n-1}) \pmod{I^{n+s}}\\
        \implies & x_{n+1} \equiv x_n \pmod{I^{n+s}}.
    \end{align*}
    Hence $(x_n)_{n\ge 0}$ is Cauchy, and $R$ is complete, so $x_n \to b$ as $n \to \infty$ for some $b \in R$. Taking the limit in $(\dagger)$ gives $b = b - F(b)$ (as the polynomial is continuous in our topology), so $F(b) = 0$. Taking the limit in $(\ddagger)$ gives $b \equiv 0 \equiv a\pmod{I^s}$.

    \vspace{1mm}
     
    For uniqueness, if $b_1,b_2$ work, then plug them into $(\star)$ and use the useful remark that $1-x$ is a unit to get that $b_1=b_2$. 
\end{proof}
Write $E: Y^2Z + a_1 XYZ + a_3 YZ^2 = X^3 + a_2 X^2Z + a_4 XZ^2 + a_6 Z^3$ and look at its affine piece $Y \neq 0$ with $t = -\frac{X}{Y}, w = -\frac{Z}{Y}$ (the minus signs are here to match Silverman's book). We get $$w = t^3 + a_1 tw + a_2 t^2w + a_3w^2 + a_4 tw^2+ a_6w^3 = f(t,w).$$
We apply Hensel's lemma (Lemma \ref{lemma8.1}) with $R = \mathbb{Z}[a_1,\ldots,a_6][[t]]$, $I = (t)$ and $F(X) = X - f(t,X) \in R[X]$. We take $s=3$, $a=0$ and check that $F(a) = F(0) = -f(t,0) = -t^3 \equiv 0 \pmod{I^3}$ and $F'(0) = 1 - a_1t - a_2t^2 \in R^x$ by our useful remark, so the assumptions hold. Hence there exists a unique $\omega(t) \in R = \mathbb{Z}[a_1,\ldots,a_6][[t]]$ such that $\omega(t) = f(t,w(t))$ and $w(t) \equiv 0 \pmod{t^3}$.

\vspace{1mm}
 
\textbf{Remarks.} 
\begin{enumerate}[(i)]
    \item Taking $u=1$ in the proof of Hensel's lemma gives $w(t) = \lim_{n \to \infty} w_n(t)$ where $w_0(t)=0$, $w_{n+1}(t) = f(t,w_n(t))$.
    \item In fact, $w(t) = t^3(1+A_1t + A_2t^2 + \ldots)$ where $A_1 = a_1$, $A_2 = a_1^2+a_2$, $A_3 = a_1^3 + 2a_1a_2 + 2a_3$, etc. (i.e. we can compute the series explicitly).
\end{enumerate} 
\marginpar{12 Feb 2024, Lecture 11}
\begin{lemma}\label{lemma8.2}
    Let $R$ be an integral domain, complete with respect to an ideal $I$. Let $a_0,\ldots,a_6 \in R$ and let $K = \text{Frac}(R)$. Then \[
    \widehat{E}(I) := \{(t,w) \in E(K) \mid t,w \in I\}
    \]
    is a subgroup of $E(K)$.
\end{lemma}
\textbf{Remark.} By uniqueness in Hensel's lemma, $\widehat{E}(I) = \{(t,w(t)) \in E(K) \mid t \in I\}$.
\begin{proof}
    Taking $(t,w) = (0,0)$ shows $0_E \in \widehat{E}(I)$. So it suffices to show that if $P_1,P_2 \in \widehat{E}(I)$, then $P_3 := -P_1 - P_2 \in \widehat{E}(I)$. Since we're working over an affine piece with the identity at 0, we know three points sum to zero if and only if they lie on the same line. Say $P_i = (t_i,w_i)$ with the line $P_1P_2$ given by $w = \lambda t + \nu$. We have $P_1,P_2 \in \widehat{E}(I) \implies t_1,t_2 \in I$ and $w_1 = w(t_1), w_2=w(t_2)$. Write $w(t) = \sum_{n=2}^{\infty} A_{n-2}t^{n+1}$ with $A_0=1$. We have 
    \begin{align*}
        &\lambda = \begin{cases}
            \frac{w(t_2)-w(t_1)}{t_2-t_1} & \text{ if }t_1 \neq t_2\\
            w'(t_1) & \text{ if }t_1=t_2
        \end{cases} = \sum_{n=2}^{\infty} A_{n-2}(t_1^n + t_1^{n-1}t_2 + \ldots + t_2^n) \in I, \\
        &\nu = w_1-\lambda t_1 \in I.
    \end{align*}
    Substituting $w = \lambda t + \nu$ into $w = f(t,w)$ gives
    \begin{align*}
        \lambda t + \nu = t^3 + a_1 t(\lambda t + \nu) + a_2 t^2 (\lambda t + \nu) + a_3 (\lambda t + \nu)^2 + a_4 t (\lambda t + \nu)^2 + a_6 (\lambda t + \nu)^3.
    \end{align*}
    Let
    \begin{align*}
        &A = (\text{coeff. of }t^3) = 1 + a_2 \lambda + a_4 \lambda^2 + a_6 \lambda^3,\\
        &B = (\text{coeff. of }t^2) = a_1 \lambda + a_2 \nu + a_3 \lambda^2 + 2 a_4 \lambda \nu + 3 a_6 \lambda^2 \nu.
    \end{align*}
    We have $A \in R^\times$, $B \in I$. Hence $t_3 = \frac{-B}{A} - t_1 - t_2 \in I$ and $w_3 = \lambda t_3 + \nu \in I$.
\end{proof}
Taking $R = \mathbb{Z}[a_1,\ldots,a_6][[t]]$ and $I = (t)$ and using Lemma \ref{lemma8.2} implies $\exists \iota \in \mathbb{Z}[a_1,\ldots,a_6][[t]]$ with $\iota(0)=0$ such that $[-1](t,w(t)) = (\iota(t), w(\iota(t)))$.
\vspace{1mm}
 
Taking $R = \mathbb{Z}[a_1,\ldots,a_6][[t_1,t_2]]$ and $I = (t_1,t_2)$ and using Lemma \ref{lemma8.2} implies $\exists F \in \mathbb{Z}[a_1,\ldots,a_6][[t_1,t_2]]$ with $F(0,0)=0$ and 
\begin{align*}
    (t_1,w(t_1)) + (t_2,w(t_2)) = (F(t_1,t_2), w(F(t_1,t_2))).
\end{align*}
In fact, $F(X,Y) = X + Y - a_1XY - a_2(X^2Y + XY^2) + \ldots$.
\vspace{1mm}
 
By properties of the group law, we deduce
\begin{enumerate}[(i)]
    \item $F(X,Y)=F(Y,X)$,
    \item $F(X,0)=X$ and $F(0,Y)=Y$,
    \item $F(X,F(Y,Z)) = F(F(X,Y),Z)$,
    \item $F(X,\iota(X))=0$.
\end{enumerate}
\begin{defn}\label{defn8.4}
    Let $R$ be a ring. A \textbf{formal group} over $R$ is a power series $F(X,Y) \in R[[X,Y]]$ satisfying the first three axioms above.
\end{defn}
An exercise on Ex. Sheet 2 asks us to show that the first three conditions imply the fourth, i.e. there is a unique $\iota(X) = -X + \ldots \in R[[X]]$ such that $F(X,\iota(X))=0$.
\begin{example}
    \begin{enumerate}[(i)]
        \item The additive formal group $F(X,Y)=X+Y$, called $\widehat{\mathbb{G}_a}$.
        \item The multiplicative formal group $F(X,Y) = X + Y +XY = (1+X)(1+Y)-1$, called $\widehat{\mathbb{G}_m}$.
        \item The formal group of an elliptic curve, $F(X,Y) = [\text{see above}]$, called $\widehat{E}$.
    \end{enumerate}
\end{example}
\begin{defn}
    Let $\mathcal{F}$ and $\mathcal{G}$ be formal groups over $R$ given by power serise $F$ and $G$.
    \begin{enumerate}[(i)]
        \item A \textbf{morphism} $\mathcal{F} \to \mathcal{G}$ is a power series $f \in R[[T]]$ such that $f(0)=0$ satisfying $f(F(X,Y))=G(f(X),f(Y))$.
        \item We say $\mathcal{F}$ is \textbf{isomorphic} to $\mathcal{G}$, i.e. $\mathcal{F}\cong \mathcal{G}$ if there exist morphisms $\mathcal{F} \stackrel{f}{\to} \mathcal{G}$ and $\mathcal{G} \stackrel{g}{\to} \mathcal{F}$ such that $f(g(T))=g(f(T))=T$.
    \end{enumerate}
\end{defn}
\begin{theorem}\label{theorem8.3}
    If $\text{char }R = 0$, then any formal group $\mathcal{f}$ over $R$ is isomorphic to $\widehat{\mathbb{G}_a}$ over $R \otimes \mathbb{Q}$. (In other words, our conditions are $\text{char }R = 0$ and ''the integers are invertible''). More precisely:
    \begin{enumerate}[(i)]
        \item There is a unique power series $\log(T) = T + \frac{a_2}{2}T^2 + \frac{a_3}{3}T^3 + \ldots$ with $a_i \in R$ such that 
        \[
        \log(F(X,Y)) = \log(X) + \log(Y).~~ (\star)
        \]
        \item There is a unique power series $\exp(T) = T + \frac{b_2}{2!}T^2 + \frac{b_3}{3!}T^3 + \ldots$ with $b_i \in R$ such that \[
        \exp(\log(T)) = \log(\exp(T)) = T.
        \]
    \end{enumerate}
\end{theorem}
\begin{proof}
    \begin{enumerate}[(i)]
        \item Notation: Write $F_1(X,Y) = \frac{\partial F}{\partial X}(X,Y)$. Uniqueness: Let $p(T) = \frac{\mathrm{d}}{\mathrm{d}T}\log T = 1 + a_2 T + 1_3 T^2 + \ldots$. Differentiating $(\star)$ with respect to $X$ gives $p(F(X,Y))F_1(X,Y) = p(X) + 0$. Putting $X=0$ gives $P(Y)F_1(0,Y)=1$, so $p(Y) = \frac{1}{F_1(0,Y)}$, proving uniqueness.
        \vspace{1mm}
         
        Existence: Let $p(T) = F_1(0,T)^{-1} = 1 + a_2T + a_3T^2 + \ldots$ for some $a_i \in R$. Define $\log T = T + \frac{a_2}{2}T^2 + \frac{a_3}{3}T^3 + \ldots$, so $p(T) = \frac{\mathrm{d}}{\mathrm{d}T}\log T$. Then 
        \begin{align*}
            &F(F(X,Y),Z) = F(X,F(Y,Z)) \\
            \stackrel{\frac{\mathrm{d}}{\mathrm{d}X}}{\implies}& F_1(F(X,Y),Z)F_1(X,Y) = F_1(X,F(Y,Z))\\
            \stackrel{X=0}{\implies}& F_1(Y,Z)p(Y)^{-1} = p(F(Y,Z))^{-1}\\
            \implies& F_1(Y,Z)p(F(Y,Z)) = p(Y) \\
            \stackrel{\text{intg. wrt }Y}{\implies}& \log(F(Y,Z)) = \log(Y) + h(Z) 
        \end{align*}
        for some power series $H$. But the symmetry in $Y$ and $Z$ implies that $h(Z)=\log Z$, so we're done.
        \marginpar{14 Feb 2024, Lecture 12}
        
        \item For this, use
        \begin{lemma}\label{lemma8.4}
            Let $f(T) = aT + \ldots \in R[[T]]$ with $a \in R^\times$. Then there exists a unique $g(T) = a^{-1}T + \ldots \in R[[T]]$ with $f(g(T))=g(f(T))=T$.
        \end{lemma}
        \begin{proof}
            We construct polynomials $g_{n}(T) \in R[T]$ such that $f(g_n(T)) \equiv T \pmod{T^{n+1}}$ and $g_{n+1}(T) \equiv g_n(T) \pmod{T^{n+1}}$. Then $g(T) = \lim_{n \to \infty}g_n(T)$ satisfies $f(g(T))=T$. To start the induction, set $g_1(T) = a^{-1}T$.
            \vspace{1mm}
             
            Now suppose $n\ge 2$, so $g_{n-1}(T)$ exists, so $f(g_{n-1}(T)) \equiv T + bT^n \pmod{T^{n+1}}$ for some $b \in R$. We put $g_n(T) = g_{n-1}(T) + \lambda T^n$ for $\lambda \in R$ to be chosen later. Then $f(g_n(T)) = f(g_{n-1}(T)+\lambda T^n) = f(g_{n-1}(T)) + \lambda a T^n \equiv T + (b+\lambda a)T^n\pmod{T^{n+1}}$, so we take $\lambda = -ba^{-1}$ (then $\lambda \in R$ as $b \in R, a \in R^\times$), completing the induction step.
            \vspace{1mm}
             
            We get $g(T) = a^{-1}T + \ldots \in R[[T]]$ such that $f(g(T))=T ~ (\dagger)$. Applying the same construction to $g$ gives $h(T) = a  + \ldots \in R[[T]]$ such that $g(h(T))=T ~ (\ddagger)$. Now note that $f(T) \stackrel{(\ddagger)}{=} f(g(h(T))) \stackrel{(\dagger)}{=} h(T)$, so $f=h$.
        \end{proof}
        The result now follows from this lemma and Ex. Sheet 2 Q5 (which allows us to control the denominators, so they'd be $n!$).
    \end{enumerate}
\end{proof} 
\textbf{Notation.} Let $\mathcal{F}$ be a formal group (e.g. $\widehat{\mathbb{G}_a}, \widehat{\mathbb{G}_m}, \widehat{E}$) given by a power series $F \in R [[X,Y]]$. Suppose $R$ is complete with respect to an ideal $I$. For $x, y \in I$. define $x \oplus_{\mathcal{F}} y = F(x,y) \in I$. Then $\mathcal{F}(I) = (I, \oplus_{\mathcal{F}})$ is an abelian group. 
\begin{example}
    \begin{itemize}
        \item $\widehat{\mathbb{G}_a}(I) = (I,+)$,
        \item $\widehat{\mathbb{G}_m}(I) = (1+I, \times)$,
        \item $\widehat{E}(I) = \text{subgroup of }E(K)$ in Lemma \ref{lemma8.2}.
    \end{itemize}
\end{example}
\begin{cor}\label{cor8.5}
    Let $\mathcal{F}$ be a formal group over $R$ and $ n \in \mathbb{Z}$. Suppose $n \in R^\times$. Then 
    \begin{enumerate}[(i)]
        \item $[n] : \mathcal{F} \to \mathcal{F}$ is an isomorphism of formal groups.
        \item If $R$ is complete with respect to an ideal $I$, then $\mathcal{F}(I) \stackrel{\times n}{\to} \mathcal{F}(I)$ is an isomorphism of groups. In particular, $\mathcal{F}(I)$ has no $n$-torsion.
    \end{enumerate}
\end{cor}
\begin{proof}
    We define $[I](T)=T$ and $[n](T) = F([n-1]T,T) ~\forall n\ge 2$. (For $n<0$, use $[-1](T) = \iota(T)$). Since $F(X,Y) = X+Y+XY(\ldots)$, we have $[2](T) = f(T,T) = 2T + \ldots$. By induction we get $[n](T) = nT + \ldots \in R[[T]]$. Lemma \ref{lemma8.4} shows that if $n \in R^\times$, then $[n]$ is an isomorphism. This proves (i). Part (ii) now follows.
\end{proof}
\section{Elliptic curves over local fields}
Let $K$ be a field, compelte with respect to a discrete valuation $v : K \to \mathbb{Z}$. (Here complete means complete with respect to the matric given by the absolute value arising from $v$.)
\begin{itemize}
    \item The \textbf{valuation ring} is $\mathcal{O}_K = \{x \in K^\times \mid v(x)\ge 0\} \cup \{0\}$.
    \item The \textbf{unit group} is $\mathcal{O}_K^\times = \{x \in K^\times \mid  v(x) = 0\}$.
    \item The \textbf{maximal ideal} is $\pi \mathcal{O}_K$, where $v(\pi)=1$.
    \item The \textbf{residue field} is $k = \mathcal{O}_K/\pi \mathcal{O}_K$.
\end{itemize}
We assume that $\text{char }K = 0$, but $\text{char}(k) = p >0$ (i.e. we are in the mixed characteristic case). The key example to keep in mind is $K = \mathbb{Q}_p, \mathcal{O}_K = \mathbb{Z}_p, k = \mathbb{F}_p$. Now let $E/K$ be an elliptic curve.
\begin{defn}
    A Weierstrass equation for $E$ with coefficients $a_1,\ldots,a_6 \in K$ is \textbf{integral} if $a_1,\ldots,a_6 \in \mathcal{O}_K$ and \textbf{minimal} if $v(\Delta)$ is minimal among all integral Weierstrass equations for $E$.
\end{defn}
\textbf{Remarks.}
\begin{enumerate}[(i)]
    \item Rescaling $x = u^2x', y= u^3y'$ gives $a_i = u^i a_i'$, so we can clear denominators, so integral Weierstrass equations exist.
    \item $a_1,\ldots,a_6 \in \mathcal{O}_K \implies \Delta \in \mathcal{O}_K \implies v(\Delta)\ge 0 \implies $ minimal Weierstrass equations exist.
    \item If $\text{char}(k) \neq 2,3$, then there exists a minimal Weierstrass equation of the form $y^2 = x^3+ax+b$.
\end{enumerate} 
\begin{lemma}\label{lemma9.1}
    Let $E/K$ have integral Weierstrass equation $y^2+a_1xy+a_3y = x^3+a_2x^2+a_4x+a_6$. Let $0 \neq P = (x,y) \in E(K)$. Then either $x,y \in \mathcal{O}_K$ or $\begin{cases}
        v(x) = -2s\\
        v(y) = -3s
    \end{cases}$
    for some $s\ge 1$. (Compare this with Ex. Sheet 1 Q5.)
\end{lemma}
\begin{proof}
    \begin{itemize}
        \item Case $v(x)\ge 0$: Suppose $v(y)<0$. Then $v(\text{LHS})=v(y^2)<0$ while $v(\text{RHS})\ge 0$, a contradiction. Hence $v(y)\ge 0$, so $x,y \in \mathcal{O}_K$.
        \item Case $v(x)<0$: We have $v(\text{LHS})\ge \min(2v(y),v(x)+v(y),v(y))$ and $v(\text{RHS})=3v(x)$. Go through 3 cases based on which element is minimal to get $v(y)<v(x)$ in every case. Now $v(\text{LHS})=2v(y)$, $v(\text{RHS})=3v(x)$, so we're done.
    \end{itemize}
\end{proof}
If $K$ is complete, then $\mathcal{O}_K$ is complete with respect to $\pi^r\mathcal{O}_K$ for any $r\ge 1$. We fix a minimal Weierstrass equation for $E/K$. This gives rise to a formal group $\widehat{E}$ over $\mathcal{O}_K$. Take $R = \mathcal{O}_K$, $I = \pi^r \mathcal{O}_K$ for $r\ge 1$ in Lemma \ref{lemma8.2} to get 
\begin{align*}
    \widehat{E}(\pi^r \mathcal{O}_K) &= \left\{(x,y) \in E(K) \mid -\frac{x}{y},-\frac{1}{y} \in \pi^r \mathcal{O}_K \right\} \cup \{0\} \\
    &= \left\{(x,y) \in E(K) \mid v \left(\frac{x}{y}\right)\ge r, v \left(\frac{1}{y}\right)\ge r\right\} \cup \{0\} \\
    &= \left\{(x,y) \in E(K) \mid v(x)=-2s, v(y)=-3s \text{ for some }s\ge r \right\} \cup\{0\} \\
    &= \left\{(x,y) \in E(K) \mid v(x) \le -2r, v(y) \le -3r \right\} \cup \{0\}.
\end{align*}
\marginpar{16 Feb 2024, Lecture 13}
By Lemma \ref{lemma8.2} this is a subgroup of $E(K)$, call it $E_r(K)$. It is also clear that $\ldots \subset E_3(K) \subset E_2(K) \subset E_1(K) \subset E(K)$. More generally, for $\mathcal{F}$ a formal group over $\mathcal{O}_K$ we have $\ldots \subset \mathcal{F}(\pi^3 \mathcal{O}_K) \subset \mathcal{F}(\pi^2 \mathcal{O}_K) \subset \mathcal{F}(\mathcal{O}_K)$. We claim that 
\begin{itemize}
    \item $\mathcal{F}(\pi^r \mathcal{O}_K) \cong (\mathcal{O}_K, +)$ for $r$ sufficiently large,
    \item $\mathcal{F}(\pi^r \mathcal{O}_K)/\mathcal{F}(\pi^{r+1} \mathcal{O}_K) \cong (k,+) ~\forall r\ge 1$.
\end{itemize}
\textbf{Reminder.} Remember that we always have $\text{char }K = 0, \text{char}(k) = p > 0$.
\begin{theorem}\label{theorem9.2}
    Let $\mathcal{F}$ be a formal group over $\mathcal{O}_K$. Let $e = v(p)$. If $r > \frac{e}{p-1}$, then \[
    \log : \mathcal{F}(\pi^r \mathcal{O}_K) \stackrel{\sim}{\longrightarrow} \widehat{\mathbb{G}_a}(\pi^r \mathcal{O}_K)
    \]
    is an isomorphism of groups with inverse \[
    \exp : \widehat{\mathbb{G}_a}(\pi^r \mathcal{O}_K) \stackrel{\sim}{\longrightarrow} \mathcal{F}(\pi^r \mathcal{O}_K). 
    \]
\end{theorem} 
\textbf{Remark.} We have $\widehat{\mathbb{G}_a}(\pi^r \mathcal{O}_K) = (\pi^r \mathcal{O}_K, +) \cong (\mathcal{O}_K, +)$.
\begin{proof}
    For $x \in \pi^r \mathcal{O}_K$, we must show that the power series $\log(x)$ and $\exp(x)$ converge to elements in $\pi^r \mathcal{O}_K$. Recall $\exp(T) = T + \frac{b_2}{2!}T^2 + \frac{b_3}{3!}T^3 + \ldots$ with $b_i \in \mathcal{O}_K$.
    \vspace{1mm}
     
    \textbf{Claim.} $v_p(n!)\le \frac{n-1}{p-1}$.
    \begin{proof}[Proof of claim.]
        Write 
        \begin{align*}
            v_p(n!) = \sum_{r=1}^{\infty} \left\lfloor \frac{n}{p^r} \right\rfloor < \sum_{r=1}^{\infty} \frac{n}{p^r} = \frac{n \cdot \frac{1}{p}}{1-\frac{1}{p}} = \frac{n}{p-1}.
        \end{align*}
        Clearing denominators, $(p-1)v_p(n!)<n \implies v_p(n!) \le \frac{n-1}{p-1}$.
    \end{proof}
    Now $v\left(\frac{b_nx^n}{n!}\right) \ge nr - e \left(\frac{n-1}{p-1}\right) = (n-1)\underbrace{\left( r - \frac{e}{p-1}\right)}_{>0} + r$. This is always $\ge r$ and tends to infinity as $n \to \infty$. Hence $\exp(x)$ converges to an element of $\pi^r \mathcal{O}_K$. The same argument works for $\log$.
\end{proof}
\begin{lemma}\label{lemma9.3}
    We have $\mathcal{F}(\pi^r \mathcal{O}_K)/\mathcal{F}(\pi^{r+1}\mathcal{O}_K) \cong (k,+) ~\forall r\ge 1$.
\end{lemma}
\begin{proof}
    Our definition of a formal group gives $F(X,Y) = X + Y + XY(\ldots)$. So if $x, y \in \mathcal{O}_K$, then $F(\pi^r x, \pi^r y) \equiv \pi^r(x+y) \pmod{\pi^{r+1}}$. Therefore $\mathcal{F}(\pi^r \mathcal{O}_K) \to (k,+)$ by $\pi^r x \mapsto x $ mod $\pi$ is a surjective group homomorphism with kernel $\mathcal{F}(\pi^{r+1}\mathcal{O}_K)$.
\end{proof}
\begin{cor}
    If $\left|k\right|<\infty$, then $\mathcal{F}(\pi \mathcal{O}_K)$ has a subgroup of finite index isomorphic to $(\mathcal{O}_K,+)$.
\end{cor}
\textbf{Notation.} Denote reduction mod $\pi$, $\mathcal{O}_K \to \mathcal{O}_K/\pi \mathcal{O}_K = k$ by $x \mapsto \widetilde{x}$.
\begin{prop}\label{prop9.4}
    Let $E/K$ be an elliptic curve. Then the reductions mod $\pi$ of any two minimal Weierstrass equations for $E$ define isomorphic curves over $k$.
\end{prop}  
\begin{proof}
    Say the Weierstrass equations are related by $[u,r,s,t]$ with $u \in K^\times, r,s,t \in K$. Then $\Delta_1 = u^{12}\Delta_2$, but both equations are minimal, so $v(u) = 0 \implies u \in \mathcal{O}_K^\times$. The transformation formulae (on the formula sheet) for the $a_i$ and $b_i$ combined with the fact that $\mathcal{O}_K$ is algebraically closed imply $r,s,t \in \mathcal{O}_K$. The Weierstrass equations of the reduction mod $\pi$ are now related by $[\widetilde{u},\widetilde{r,}\widetilde{s},\widetilde{t}]$ with $\widetilde{u} \in k^\times$, $\widetilde{r},\widetilde{s},\widetilde{t} \in k$.
\end{proof}
\begin{defn}
    The reduction $\widetilde{E}/k$ of $E/K$ is defined by the reduction mod $\pi$ of a minimal Weierstrass equation for $E$. We say $E$ has \textbf{good reduction} if $\widetilde{E}$ is nonsingular (and so $\widetilde{E}$ is an elliptic curve), otherwise $E$ has \textbf{bad reduction}.
\end{defn}
For an integral Weierstrass equation,
\begin{itemize}
    \item $v(\Delta)=0 \implies$ good reduction.
    \item $0<v(\Delta)<12 \implies $ bad reduction.
    \item $v(\Delta)\ge 12 \implies $ beware that the equation might not be minimal, more information is needed. 
\end{itemize}
There is a well-defined map $\mathbb{P}^2(K) \to \mathbb{P}^2(k)$ by $(x:y:z) \mapsto (\widetilde{x}:\widetilde{y}:\widetilde{z})$. (Here we must choose a representative for $(x:y:z)$ such that $\min(v(x),v(y),v(z))=0$.) We restrict to get a map $E(K) \to \widetilde{E}(k)$ by $P \mapsto \widetilde{P}$. 
\vspace{1mm}
 
IF $P=(x,y) \in E(K)$, then by Lemma \ref{lemma9.1}, either $x, y \in \mathcal{O}_K$, so $\widetilde{P}=(\widetilde{x},\widetilde{y}) \in \widetilde{E}(k)$, or $v(x)=-2s,v(y)=-3s$ for some $s\ge 1$, so $P=(x:y:1) = (\pi^{3s}x : \pi^{3s}y : \pi^{3s})$, so $\widetilde{P} = (0:1:0)$. Therefore \[
\widehat{E}(\pi \mathcal{O}_K) = E_1(K) = \{P \in E(K) \mid  \widetilde{P} = 0\},
\]
the \textbf{kernel of reduction}. Let \[
\widetilde{E}_{\text{ns}} = \begin{cases}
    \widetilde{E} & \text{ if }E \text{ has good reduction,}\\
    \widetilde{E}\setminus \{\text{singular point}\} & \text{ if }E \text{ has bad reduction.}
\end{cases}
\]
We have a remarkable fact: the chord and tangent process still defines a group law on $\widetilde{E}_{\text{ns}}$. However, in the case of bad reductions, either $\widetilde{E}_{\text{ns}} \cong \mathbb{G}_a$ (over $k$) or $\widetilde{E}_{\text{ns}} \cong \mathbb{G}_m$ (over $k$ or possibly over a quadratic extension of $k$). These are the additive reduction and the multiplicative reduction.
\marginpar{19 Feb 2024, Lecture 14}
\vspace{1mm}
 
For simplicity, assume $\text{char }k \neq 2$. Then for $\widetilde{E} : y^2=f(x)$, $\text{deg}f=3$, we have that $\widetilde{E}$ is singular if and only if $f$ has a repeated root.
\begin{itemize}
    \item If this is a double root, we get $\mathbb{G}_m$ (e.g. for $y^2=x^2(x+1)$, a curve with a node).
    \item If this is a triple root, we get $\mathbb{G}_a$ (e.g. for $y^2=x^3$, a curve with a cusp).
\end{itemize}
The proof of the former is on Ex. Sheet 3. For the latter, consider the map $\mathbb{G}_a \to \widetilde{E}_{\text{ns}}$ by $t \mapsto (t^{-2},t^{-3})$, so $\frac{x}{y}\mapsfrom (x,y)$ and the point at infinity $\leftrightarrow 0$. Suppose we have a line through $P_1$, $P_2$ meeting the curve again at $P_3$ (with none of these points at the origin), so this line is $ax+by=1$. Write $P_i = (x_i,y_i)$ for $i =1,2,3$, and $t_i = \frac{x_i}{y_i}$. Then 
\begin{align*}
    &x_i^3=y_i^2=y_i^2(ax_i+by_i)\\
    \implies & t_i^3at_i-b=0\\
    \implies & t_1,t_2,t_3 \text{ are roots of }X^3-aX-b=0.
\end{align*}
Looking at the coefficient of $X^2$ gives $t_1+t_2+t_3=0$, so $\widetilde{E}_{\text{ns}} \cong \mathbb{G}_a$.
\begin{defn}
    We define
    \begin{align*}
        E_0(K) = \{P \in E(K) \mid \widetilde{P} \in \widetilde{E}_{\text{ns}}(k)\}.
    \end{align*}
\end{defn}
\begin{prop}\label{prop9.5}
    $E_0(K)$ is a subgroup of $E(K)$ and reduction mod $\pi$ is a surjective group homomorphism $E_0(K) \to \widetilde{E}_{\text{ns}}(K)$.
\end{prop}
\begin{proof}
    The group homomorphism part: A line $\ell$ in $\mathbb{P}^2$ defined over $K$ has equation $\ell : aX+bY+cZ=0$ for $a,b,c \in K$, where we may assume that $\min(v(a),v(b),v(c))=0$ by scaling.
    \vspace{1mm}
     
    Reduction mod $\pi$ gives a line $\widetilde{\ell}: \widetilde{a}X+\widetilde{b}Y+\widetilde{c}Z=0$. If $P_1,P_2,P_3 \in E(K)$ with $P_1+P_2+P_3=0$, then these points lie on a line $\ell$, so $\widetilde{P}_1, \widetilde{P}_2, \widetilde{P}_3$ lie on the line $\widetilde{\ell}$. If $\widetilde{P}_1,\widetilde{P}_2 \in \widetilde{E}_{\text{ns}}(k)$, then $\widetilde{P}_3 \in \widetilde{E}_{\text{ns}}(k)$. Hence if $P_1,P_2 \in E_0(K)$, then $P_3 \in E_0(K)$ and $\widetilde{P}_1 + \widetilde{P}_2 + \widetilde{P}_3=0$. It is left as an exercise to check that this still works if $\#\{\widetilde{P}_1,\widetilde{P}_2,\widetilde{P}_3\}<3$.
    \vspace{1mm}
     
    For surjectivity, let $f(x,y) = y^2+a_1xy+a_3y-(x^3+\ldots)$. Let $\widetilde{P} \in \widetilde{E}_{\text{ns}}(k) \setminus \{0\}$, say $\widetilde{P}=(\widetilde{x}_0,\widetilde{y}_0)$ for some $x_0,y_0 \in \mathcal{O}_K$. As $\widetilde{P}$ is nonsingular, we either have $\frac{\partial f}{\partial x}(x_0,y_0) \not\equiv 0$ mod $\pi$ or $\frac{\partial f}{\partial y}(x_0,y_0) \not\equiv 0$ mod $\pi$.
    \vspace{1mm}
     
    In the first case, we put $g(t)=f(t,y_0) \in \mathcal{O}_K[t]$ to get $\begin{cases}
        g(x_0)\equiv \pmod{\pi},\\
        g'(x_0) \in \mathcal{O}_K^\times,
    \end{cases}$ so by Hensel's lemma $\exists b \in \mathcal{O}_K$ such that $\begin{cases}
        g(b)=0,\\
        b \equiv x_0 \pmod{\pi}. 
    \end{cases}$ Then $(b,y_0) \in E(K)$ has reduction $\widetilde{P}$. The second case is analogous.
\end{proof}
Recall that for $r\ge 1$, we put \[
E_r(K)= \{(x,y) \in E(K) \mid v(x)\le -2r,v(y)\le -3r\} \cup \{0\}
\]
and we have $\ldots \subset E_r(K) \subset \ldots \subset E_1(K) \subset E_0(K)\subset E_K$. Recall that $\widehat{E}(\pi^r \mathcal{O}_K) = E_r(K)$ by definition. We know that we have $E_r(K) \cong (\mathcal{O}_K,+)$ if $r>\frac{e}{p-1}$ and $E_r(K)/E_{r+1}(K) \cong (k,+) ~\forall r\ge 1$. We can extend this to include $E_0(K)/E_1(K) \cong \widetilde{E}_{\text{ns}}(K)$. What about $E_0(K)/E(K)$?
\begin{lemma}\label{lemma9.6}
    If $\left|k\right|<\infty$, then $E_0(K)\subset E_K$ has finite index.
\end{lemma}
\begin{proof}
    $\left|k\right|<\infty \implies \frac{\mathcal{O}_K}{\pi^r \mathcal{O}_K}$ is finite $~\forall r\ge 1$. Hence $\mathcal{O}_K = \lim_{\stackrel{\leftarrow}{r} } \mathcal{O}_K/\pi^r \mathcal{O}_K$ is a profinite group, hence compact. Then $\mathbb{P}^n(K)$ is a union of sets of the form \[
    \{(a_0:a_1:a_2:\ldots:a_{i-1}:1:a_{i+1}:\ldots a_n) \mid a_j \in \mathcal{O}_K\}
    \]
    and hence is compact (with respect to the $\pi$-adic topology on $K$). $E(K) \subset \mathbb{P}^2(K)$ is a closed subset and hence compact, so $E(K)$ is a compact topological group. If $\widetilde{E}$ has a singular point $(\widetilde{x}_0,\widetilde{y}_0)$, then \[
    E(K) \setminus E_0(K) = \{(x,y) \in E(K) \mid v(x-x_0)\ge 1,v(y-y_0)\ge 1\}
    \]
    is a closed subset of $E(K)$, so $E_0(K)$ is an open subgroup of $E(K)$. But the cosets of $E_0(K)$ are open, so $[E(K):E_0(K)]<\infty$ by compactness of $E(K)$.
\end{proof}
\begin{defn}
    $c_K(E) = [E_K:E_0(K)]$ is called the \textbf{Tamagawa number}.
\end{defn}
\textbf{Remarks.}
\begin{enumerate}[(i)]
    \item Good reduction implies $c_K(E)=1$, but the converse is false.
    \item It can be shown that either $c_K(E) = v(\Delta)$ or $c_K(E)\le 4$ (here it is essential that we work with a minimal Weierstrass equation). 
\end{enumerate} 
We deduce the following:
\begin{theorem}\label{theorem9.7}
    If $[K:\mathbb{Q}_p]<\infty$, then $E(K)$ contains a subgroup of finite index isomorphic to $(\mathcal{O}_K,+)$.
\end{theorem}
Some setup: Let $[K:\mathbb{Q}_p]<\infty$, $L/K$ a finite extension with residue fields $k$ and $k'$ and $f = [k':k]$. This gives us the map 
\begin{tikzcd}
    K^\times \rar{v_K} & \mathbb{Z} \dar{\times e} \\
    L^\times \arrow[draw=none]{u}[sloped,auto=false, rotate=180]{\subset}  \rar{v_L} & \mathbb{Z}
\end{tikzcd}.
\vspace{1mm}

\textbf{Facts.} 
\begin{enumerate}[(i)]
    \item $[L:K] = ef$.
    \item If $L/K$ is Galois, then the natural map $\text{Gal}(L/K) \to \text{Gal}(k'/k)$ is surjective with kernel of order $e$.
\end{enumerate}
\begin{defn}
    $L/K$ is \textbf{unramified} if $e=1$.
\end{defn}
\marginpar{21 Feb 2024, Lecture 15}
\textbf{Facts.}
\begin{enumerate}[(i)]
    \item For each $m\ge 1$, $k$ has a unique extension of degree $m$ (say $k_m$).
    \item For each $m\ge 1$, $K$ has a unique unramified extension of degree $m$ (say $K_m$).
\end{enumerate} 
These extensions are Galois with cyclic Galois group.
\begin{defn}
    We have the maximal unramified extension of $K$, $$K^{\text{nr}} = \bigcup_{m\ge 1} K_m \subset \overline{K}.$$
\end{defn}
\begin{theorem}\label{theorem9.8}
    Let $[K:\mathbb{Q}]<\infty$. Suppose $E/K$ has good reduction and $p \nmid n$. If $P \in E(K)$, then $K([n]^{-1}P)/K$ is unramified.
\end{theorem}
\textbf{Notation.} We have
\[
[n]^{-1}(P)=\{Q \in E(\overline{K}) \mid nQ=P\}
\] 
and we let 
\[
K(\{Q_1,\ldots,Q_r\})=K(x_1,y_1,\ldots,x_r,y_r),
\]
where $Q_i = (x_i,y_i)$.
\begin{proof}
    For each $m\ge 1$, there is a short exact sequence $$0 \to E_1(K_m) \to E(K_m) \to \widetilde{E}_{k_m} \to 0.$$
    Taking the union $\bigcup_{m\ge 1} $ gives us a commutative diagram with exact rows:
    $$\begin{tikzcd}[]
        0 & E_1(K^{\text{nr}}) & E(K^{\text{nr}}) & \widetilde{E}(\overline{k}) & 0 \\
        0 & E_1(K^{\text{nr}}) & E(K^{\text{nr}}) & \widetilde{E}(\overline{k}) & 0
        \arrow["", from=1-1, to=1-2]
        \arrow["", from=1-2, to=1-3]
        \arrow["", from=1-3, to=1-4]
        \arrow["", from=1-4, to=1-5]
        \arrow["", from=2-1, to=2-2]
        \arrow["", from=2-2, to=2-3]
        \arrow["", from=2-3, to=2-4]
        \arrow["", from=2-4, to=2-5]
        \arrow["\times n", from=1-2, to=2-2]
        \arrow["\times n", from=1-3, to=2-3]
        \arrow["\times n", from=1-4, to=2-4]
    \end{tikzcd},$$
    The first multiplication map is an isomorphism by Corollary \ref{cor8.5} applied to each $K_m$ (using $p \nmid n$). 
    \vspace{1mm}
    
    The third is surjective by Theorem \ref{theorem2.3}, and has kernel $\cong (\mathbb{Z}/n\mathbb{Z})^2$ by Theorem \ref{theorem6.5} (again using $p \nmid n$). Using the snake lemma on this diagram gives $E(K^{\text{nr}})[n] \cong(\mathbb{Z}/n\mathbb{Z})^2$ and $E(K^{\text{nr}})/nE(K^{\text{nr}}) = 0$. So if $P \in E(K)$, then $\exists Q \in E(K^{\text{nr}})$ with $nQ=P$ and $[n]^{-1}P = \{Q+T \mid T \in E[n]\} \subset E(K^{\text{nr}})$. Hence $K([n]^{-1}P) \subset K^{\text{nr}}$ and so $K([n]^{-1}P)/K$ is unramified.
\end{proof}
\section{Elliptic curves over number fields}
\subsection{The torsion subgroup} 
\textbf{Notation.} Let $E/K$ be an elliptic curve for $[K:\mathbb{Q}]<\infty$. We write $\mathfrak{p}$ for a prime of $K$ (i.e. of $\mathcal{O}_K$), $K_{\mathfrak{p}}$ for the $\mathfrak{p}$-adic completion of $K$, and $k_{\mathfrak{p}}=\mathcal{O}_K/\mathfrak{p}$.

\begin{defn}
    $\mathfrak{p}$ is a prime of good reduction for $E/K$ if $E/K_{\mathfrak{p}}$ has good reduction.
\end{defn}
\begin{lemma}\label{lemma10.1}
    $E/K$ has only finitely many primes of bad reduction.
\end{lemma}
\begin{proof}
    Take a Weierstrass equation for $E$ with $a_1,\ldots,a_6 \in \mathcal{O}_K$. Since $E$ is nonsingular, $0 \neq \Delta \in \mathcal{O}_K$. Write $(\Delta)=\mathfrak{p}_1^{\alpha_1}\ldots \mathfrak{p}_r^{\alpha_r}$ for the factorization into prime ideals and let $S = \{\mathfrak{p}_1,\ldots,\mathfrak{p}_r\}$. If $\mathfrak{p} \not\in S$, then $v_{\mathfrak{p}}(\Delta)=0$, so $E/K_{\mathfrak{p}}$ has good reduction. Hence $\{\text{bad primes of }E\}\subset S$ is finite.
\end{proof}
\textbf{Remark.} If $K$ has class number 1 (e.g. if $K=\mathbb{Q}$), then we can always find a Weierstrass equation for $E$ with $a_1,\ldots,a_6 \in \mathcal{O}_K$ which is minimal at all primes $\mathfrak{p}$.
\vspace{1mm}
 
\textbf{Basic group theory.} If $A$ is a finitely generated abelian group, then $$A \cong (\text{finite group}) \times \mathbb{Z}^r$$ for the finite group the \textbf{torsion subgroup} and $r$ the \textbf{rank}. 

\begin{lemma}\label{lemma10.2}
    $E(K)_{\text{tors}}$ is finite.
\end{lemma}
\begin{proof}
    Take any prime $\mathfrak{p}$. We saw that $E(K_{\mathfrak{p}})$ has a subgroup $A$ of finite index with $A \cong (\mathcal{O}_{\mathfrak{p}},+)$. In particular, $A$ is torsion-free. Hence we get \[
    E(K)_{\text{tors}} \subset E(K_{\mathfrak{p}})_{\text{tors}} \hookrightarrow E(K_{\mathfrak{p}})/A,
    \]
    and this last group is finite.
\end{proof}
\begin{lemma}\label{lemma10.3}
    Let $\mathfrak{p}$ be a prime of good reduction. Then reduction mod $\mathfrak{p}$ gives an injective group homomorphism $E(K)[n] \hookrightarrow \widetilde{E}(k_{\mathfrak{p}})$.
\end{lemma}
\begin{proof}
    Proposition \ref{prop9.5} implies that $E(K_{\mathfrak{p}}) \to \widetilde{E}(k_{\mathfrak{p}})$ is a group homomorphism with kernel $E_1(K_{\mathfrak{p}})$. Corollary \ref{cor8.5} and the fact that $\mathfrak{p}\nmid n$ imply now that $E_1(K_{\mathfrak{p}})$ has no $n$-torsion.
\end{proof}
\begin{example}
    Let $E/\mathbb{Q}$ be given by $y^2+y=x^3-x^2$. Then $\Delta=-11$, so $E$ has good reduction at all $p \neq 11$. We can count
    $$\begin{tabular}{ c|cccccc } 
        $p$ & 2 & 3 & 5 & 7 & 11 & 13 \\ 
        \hline
        $\# \widetilde{E}(\mathbb{F}_p)$ & 5 & 5 & 5 & 10 & -- & 10 \\ 
    \end{tabular}.$$
    By Lemma \ref{lemma10.3}, $\# E(\mathbb{Q})_{\text{tors}} \mid 5\cdot 2^a$ for some $a\ge 0$ and $\# E(\mathbb{Q})_{\text{tors}} \mid 5 \cdot 3^b$ for some $b\ge 0$. This implies that $\#E(\mathbb{Q}) \mid 5$. If we let $T=(0,0) \in E(\mathbb{Q})$, then calculation shows $5T=0$, so $E(\mathbb{Q})_{\text{tors}} \cong \mathbb{Z}/5\mathbb{Z}$.
\end{example}
\begin{example}
    Let $E/\mathbb{Q}$ be given by $y^2+y=x^3+x^2$. Then $\Delta=-43$, and we get
    $$\begin{tabular}{ c|cccccc } 
        $p$ & 2 & 3 & 5 & 7 & 11 & 13 \\ 
        \hline
        $\# \widetilde{E}(\mathbb{F}_p)$ & 5 & 6 & 10 & 8 & 9 & 19 \\ 
    \end{tabular}.$$
    By Lemma \ref{lemma10.3}, $\#E(\mathbb{Q})_{\text{tors}} \mid 5\cdot 2^a$ for some $a\ge 0$ and $\#E(\mathbb{Q})_{\text{tors}} \mid 9\cdot 11^b$ for some $b\ge 0$, so $E(\mathbb{Q})_{\text{tors}} = \{0\}$. Hence the point $P = (0,0)$ is a point of infinite order. In particular, $E(\mathbb{Q})$ is infinite.
\end{example}
\marginpar{23 Feb 2024, Lecture 16}
\begin{example}
    Let $E_D/\mathbb{Q}$ be given by $E_D: y^2=x^3-D^2x$ for $D \in \mathbb{Z}$ squarefree. Then $\Delta=2^6 D^6$ and we spot \[
    E_D(\mathbb{Q})_{\text{tors}} \supset \{0, (0,0), (\pm D,0)\} \cong (\mathbb{Z}/2\mathbb{Z})^2.
    \]
    Let $f(x)=x^3-D^2x$. If $p \nmid 2D$, then $$\# \widetilde{E}_D(\mathbb{F}_p) = 1 + \sum_{x \in \mathbb{F}_p}^{} \left(\left(\frac{f(x)}{p}\right)+1\right).$$ If $p \equiv 3$ mod 4, then $\# \widetilde{E}_D(\mathbb{F}_p)=p+1$, since 
    \begin{align*}
        \left( \frac{f(-x)}{p} \right) = \left( \frac{-f(x)}{p} \right) = \left( \frac{-1}{p} \right) \left( \frac{f(x)}{p} \right) = - \left( \frac{f(x)}{p} \right).
    \end{align*}
    Let $m = \#E_D(\mathbb{Q})_{\text{tors}}$. We have $4 \mid m \mid (p+1)$ for all sufficiently large primes $p$ with $p \equiv 3$ mod 4 ($p \nmid 2Dm$ suffices).
    \vspace{1mm}
     
    If $8 \mid m$ or $l \mid m$ for some odd prime $l$, then this contradicts Dirichlet's Theorem on primes in arithmetic progressions. Hence $m=4$ and so $E_D(\mathbb{Q})_{\text{tors}} = (\mathbb{Z}/2\mathbb{Z})^2$. Thus 
    \[
    \text{rank }E_D(\mathbb{Q})\ge 1 \iff \exists x,y \in \mathbb{Q} \text{ with }y\neq 0 \text{ and }y^2=x^3-D^2x.
    \]
    By Lecture 1, this is equivalent to $D$ being a congruent number.
\end{example}
\begin{lemma}\label{lemma10.4}
    Let $E/\mathbb{Q}$ be given by a Weierstrass equation with $a_1,\ldots,a_6 \in \mathbb{Z}$. Suppose $0 \neq T = (x,y) \in E(\mathbb{Q})_{\text{tors}}$. Then
    \begin{enumerate}[(i)]
        \item $4x,8y \in \mathbb{Z}$.
        \item If $2 \mid a_1$ or $2T \neq 0$, then $x,y \in \mathbb{Z}$.
    \end{enumerate}
\end{lemma}
\begin{proof}
    The Weierstrass equation defines a formal group $\widehat{E}$ over $\mathbb{Z}$. For $r \ge 1$, we have 
    \[
    \widehat{E}(p^r \mathbb{Z}_p) = \{(x,y) \in E(\mathbb{Q}_p) \mid v_p(x)\le -2r, v_p(y)\le -3r\} \cup \{0\}.
    \]
    By Theorem \ref{theorem9.2}, $\widehat{E}(p^r\mathbb{Z}_p) \cong (\mathbb{Z}_p,+)$ if $r>\frac{1}{p-1}$. Hence $\widehat{E}(4\mathbb{Z}_2)$ and $\widehat{E}(p\mathbb{Z}_p)$ for $p$ odd are torsion-free. This means that $v_2(x)\ge -2, v_2(y)\ge -3$ and $v_p(x),v_p(y)\ge 0$ for all odd primes $p$, which proves (i).
    \vspace{1mm}
     
    For the second part, suppose $T \in \widehat{E}(4\mathbb{Z}_2)$, i.e. $v_2(x)=-2,v_2(y)=-3$. Since $\widehat{E}(2\mathbb{Z}_2)/\widehat{E}(4\mathbb{Z}_2)\cong (\mathbb{F}_2,+)$ and $\widehat{E}(4\mathbb{Z}_2)$ is torsion-free, we get $2T=0$. Also, $(x,y) = T = -T = (x,-y-a_1x-a_3) \implies 2y+a_1x+a_3=0$, so $8y+a_1(4x)+4a_3=0$. Since $8y, 4x, 4a_3$ are even, we require $a_1$ to be odd. So if $2T \neq 0$ or $a_1$ is even, then $T \not\in \widehat{E}(2\mathbb{Z}_2)$, so $x, y \in \mathbb{Z}$.
\end{proof}
\begin{example}
    For $E:y^2+xy=x^3+4x+1$, $\left(-\frac{1}{4},\frac{1}{8}\right) \in E(\mathbb{Q})[2]$.
\end{example}
\begin{theorem}[Lutz-Nagell]
    Let $E/\mathbb{Q}$ be given by $y^2=x^3+ax+b=f(x)$ for $a,b \in \mathbb{Z}$. Suppose $0 \neq T = (x,y) \in E(\mathbb{Q})_{\text{tors}}$. Then $x,y \in \mathbb{Z}$ and either $y=0$ or $y^2 \mid (4a^3+27b^2)$.
\end{theorem}
\begin{proof}
    Lemma \ref{lemma10.4} implies that $x,y \in \mathbb{Z}$. If $2T=0$, then $y=0$. Otherwise, $0 \neq 2T = (x_2,y_2) \in E(\mathbb{Q})_{\text{tors}}$, so by Lemma \ref{lemma10.4}, $x_2,y_2 \in \mathbb{Z}$. But $x_2 = \left(\frac{f'(x)}{2y}\right)^2-2x \implies y \mid f'(x)$. As $E$ is nonsingular, $f(X)$ and $f'(X)$ are coprime, so $f(X)$ and $f'(X)^2$ are coprime, so $\exists g,h \in \mathbb{Q}[X]$ with $g(X)f(X)+h(X)f'(X)^2=1$. In fact, we can check that
    \begin{align*}
        (3X^2+4a)f'(X)^2-27(X^3+aX-b)f(X)=4a^3+27b^2.
    \end{align*}
    Since $y \mid f'(x)$ and $y^2 = f(x)$, we get $y^2 \mid (4a^3+27b^2)$.
\end{proof}
\textbf{Remark.} Mazur showed that if $E/\mathbb{Q}$ is an elliptic curve, then
\begin{align*}
    E(\mathbb{Q})_{\text{tors}} \cong \begin{cases}
        \mathbb{Z}/n\mathbb{Z} & 1\le n\le 12, n \neq 11,\\
        \mathbb{Z}/2\mathbb{Z} \times \mathbb{Z}/2n\mathbb{Z} & 1\le n\le 4.
    \end{cases}
\end{align*}
Moreover, all 15 possibilies occur.
\section{Kummer theory}
Let $K$ be a field with $\text{char }K \nmid n$. Assume $\mu_n \subset K$ for $\mu_n$ the set of $n^{\text{th}}$ (primitive?) roots of unity.
\begin{lemma}\label{lemma11.1}
    Let $\Delta \subset K^\times/(K^\times)^n$ be a finite subgroup and let $L = K(\sqrt[n]{\Delta})$. Then $L/K$ is Galois and $\text{Gal}(L/K) \cong \text{Hom}(\Delta,\mu_n)$.
\end{lemma}
\begin{proof}
    $L/K$ is Galois since $\mu_n \subset K \implies L/K$ normal and $\text{char }K \nmid n \implies L/K$ separable. Define the \textbf{Kummer pairing} 
    \begin{align*}
        \langle ~,~ \rangle : \text{Gal}(L/K) \times \Delta &\to \mu_n \\
        (\sigma,x) &\mapsto \frac{\sigma(\sqrt[n]{x})}{\sqrt[n]{x}}.
    \end{align*}
    This is well-defined: If $\alpha,\beta \in L$ with $\alpha^n=\beta^n=x$, then $\left(\frac{\alpha}{\beta}\right)^n=1$, so $\frac{\alpha}{\beta} \in \mu_n \subset K$, so $\sigma \left(\frac{\alpha}{\beta}\right) = \frac{\alpha}{\beta}$ and so $\frac{\sigma(\alpha)}{\alpha} = \frac{\sigma(\beta)}{\beta}$.
    \vspace{1mm}
     
    This is bilinear: we have
    \begin{align*}
        \langle \sigma \tau, x \rangle = \frac{\sigma(\tau \sqrt[n]{x})}{(\tau \sqrt[n]{x})} \frac{\tau \sqrt[n]{x}}{\sqrt[n]{x}} = \langle \sigma,x \rangle \langle \tau, x \rangle,\\
        \langle \sigma,xy \rangle = \frac{\sigma \sqrt[n]{xy}}{\sqrt[n]{xy}} = \frac{\sigma \sqrt[n]{x}}{\sqrt[n]{x}}\frac{\sigma \sqrt[n]{y}}{\sqrt[n]{y}} = \langle \sigma,x \rangle \langle \sigma,y \rangle.
    \end{align*} 
    This is nondegenerate: Let $\sigma \in \text{Gal}(L/K)$. If $\langle \sigma,x \rangle = 1 ~\forall x \in \Delta$, then $\sigma(\sqrt[n]{x})=\sqrt[n]{x} ~\forall x \in \Delta$, so $\sigma$ fixes $L$ pointwise, i.e. $\sigma=1$. Now let $x(K^\times)^n \in \Delta$. If $\langle \sigma, x \rangle = 1~\forall \sigma \in \text{Gal}(L/K)$, then $\sigma(\sqrt[n]{x})=\sqrt[n]{x}~\forall \sigma \in \text{Gal}(L/K)$, so $\sqrt[n]{x} \in K$, so $x \in (K^\times)^n$ and so $x(K^{\times})^n \in \Delta$ is trivial.
    \vspace{1mm}
     
    We get injective group homomorphisms
    \begin{enumerate}[(i)]
        \item $\text{Gal}(L/K) \hookrightarrow \text{Hom}(\Delta,\mu_n)$,
        \item $\Delta \hookrightarrow \text{Hom}(\text{Gal}(L/K),\mu_n)$.
    \end{enumerate}
    From (i), $\text{Gal}(L/K)$ is abelian and of exponent dividing $n$. Recall the following \textbf{Fact}: If $G$ is a finite abelian group of exponent dividing $n$, then $\text{Hom}(G,\mu_n)\cong G$ (non-canonically). Hence $\left|\text{Gal}(L/K)\right|\stackrel{(i)}{\le } \left|\Delta\right| \stackrel{(ii)}{\le } \left|\text{Gal}(L/K)\right|$, so (i) and (ii) are isomorphisms.
\end{proof}
\begin{example}
    $\text{Gal}(\mathbb{Q}(\sqrt{2},\sqrt{3},\sqrt{5})/\mathbb{Q}) \cong (\mathbb{Z}/2\mathbb{Z})^3$.
\end{example}
\marginpar{26 Feb 2024, Lecture 17}
Reminder: we are assuming $\text{char }K \nmid n$ and $\mu_n \subset K$.
\begin{theorem}\label{theorem11.2}
    There is a bijection 
    \begin{align*}
        \{\text{finite subgroups of }K^\times/(K^\times)^n\} &\leftrightarrow \{\text{finite abelian extensions }L/K \text{ of exponent dividing }n\}\\
        \Delta &\mapsto K(\sqrt[n]{\Delta})\\
        ((L^\times)^n \cap K^\times)/(K^\times)^n &\mapsfrom L
    \end{align*}
\end{theorem}
\begin{proof}
    (i). Let $\Delta \subset K^\times/(K^\times)^n$ be a finite subgroup. Let $L = K(\sqrt[n]{\Delta})$ and $\Delta' = ((L^\times)^n \cap K^\times)/(K^\times)^n$. We must show $\Delta= \Delta'$. Clearly $\Delta \subset \Delta'$. Also 
    \begin{align*}
        &L = K(\sqrt[n]{\Delta}) \subset K(\sqrt[n]{\Delta'}) \subset L\\
        \implies & K(\sqrt[n]{\Delta}) = K(\sqrt[n]{\Delta'}).
    \end{align*}
    Thus $\left|\Delta\right|=\left|\Delta'\right|$ by Lemma \ref{lemma11.1}. Since $\Delta \subset \Delta'$, we get $\Delta= \Delta'$.
    \vspace{1mm}
     
    (ii). Let $L/K$ be a finite abelian extension of exponent dividing $n$. Let $\Delta=((L^\times)^n\cap K^\times)/(K^\times)^n$, then $K(\sqrt[n]{\Delta})\subset L$ and we aim to prove that these are equal. Let $G = \text{Gal}(L/K)$. The Kummer pairing gives an injection $\Delta \hookrightarrow \text{Hom}(G,\mu_n)$, which we claim is surjective. Given the claim, we would have $[K(\sqrt[n]{\Delta}):K] = \left|\Delta\right|=\left|G\right| = [L:K]$ by Lemma \ref{lemma11.1}. Since $K(\sqrt[n]{\Delta})\subset L$, $L=K(\sqrt[n]{\Delta})$ follows.
    \vspace{1mm}
     
    It remains to prove the surjectivity claim. For this, let $\chi : G \to \mu_n$ be a group homomorphism. Distinct automorphisms are linearly independent, so $\exists a \in L$ such that $y := \sum_{\tau \in G}^{} \chi(\tau)^{-1} \tau(a) \neq 0$. Let $\sigma \in G$. Then 
    \begin{align*}
        &\sigma(y) = \sum_{\tau \in G}^{} \chi(\tau)^{-1}\sigma \tau(a) \\
        \stackrel{\tau \mapsto \sigma^{-1}\tau}{=}& \sum_{\tau \in G}^{} \chi(\sigma^{-1} \tau)^{-1}\tau(a)\\
        =&\chi(\sigma)\sum_{\tau \in G}^{} \chi(\tau)^{-1}\tau(a)\\
        =&\chi(\sigma)y.
    \end{align*}
    Hence $\sigma(y^n)=y^n ~\forall \sigma \in G$, so $y^n \in K$. Let $x = y^n$, then $x \in (L^\times)^{n} \cap K^\times$, so $x(K^{\times})^n \in \Delta$. Also by the calculation above, $\chi : \sigma \mapsto \frac{\sigma(y)}{y} = \frac{\sigma(\sqrt[n]{x})}{\sqrt[n]{x}}$, so the map $\Delta \hookrightarrow \text{Hom}(G,\mu_n)$ sends $x \mapsto \chi$, which proves the claim.
\end{proof}
\begin{prop}\label{prop11.3}
    Let $K$ be a number field and $\mu_n \subset K$. Let $S$ be a finite set of primes of $K$. Then there are only finitely many extensions $L/K$ such that
    \begin{enumerate}[(i)]
        \item $L/K$ is finite and abelian of exponent dividing $n$.
        \item $L/K$ is unramified at all primes $\mathfrak{p} \not\in S$.
    \end{enumerate}
\end{prop}
\begin{proof}
    Theorem \ref{theorem11.2} implies that this extension is of the form $L = K(\sqrt[n]{\Delta})$ for some finite subgroup $\Delta \subset K^\times/(K^\times)^n$. Let $\mathfrak{p}$ be a prime of $K$. We have $\mathfrak{p}\mathcal{O}_L = \mathcal{P}_1^{e_1}\ldots \mathcal{P}_r^{e_r}$ for $\mathcal{P}_i$ distinct primes of $L$. If $x \in K^\times$ represents an element of $\Delta$, then $n v_{\mathcal{P}_i}(\sqrt[n]{x}) = v_{\mathcal{P}_i}(x) = e_i v_{\mathfrak{p}}(x)$. \vspace{1mm}
     
    If $\mathfrak{p} \not\in S$, then all $e_i=1$, so $v_{\mathfrak{p}}(x) \equiv 0 \pmod{n}$. Hence $\Delta \subset K(S,n)$, where
    \begin{align*}
        K(S,n) = \{x \in K^\times/(K^\times)^n \mid v_{\mathfrak{p}}(x)\equiv 0\pmod{n}~\forall \mathfrak{p} \not\in S\}.
    \end{align*}
    We now complete the proof using the following lemma.
\end{proof}
\begin{lemma}\label{lemma11.4}
    $K(S,n)$ is finite.
\end{lemma}
\begin{proof}
    The map $K(S,n) \to (\mathbb{Z}/n\mathbb{Z})^{\left|S\right|}$ by $x \mapsto (v_{\mathfrak{p}}(x) \text{ mod }n)_{\mathfrak{p} \in S}$ is a group homomorphism with kernel $K(\varnothing,n)$. Since $\left|S\right|<\infty$, it suffices to prove the lemma with $S=\varnothing$. 
    \vspace{1mm}
     
    Now, if $x \in K^\times$ represents an element of $K(\varnothing,n)$, then $(x) = \mathfrak{a}^n$ for some fractional ideal $\mathfrak{a}$. There is an exact short sequence
    \begin{align*}
        0 \to \mathcal{O}_K^\times/(\mathcal{O}_K^\times)^n \to K(\varnothing,n) &\to \text{Cl}_K[n] \to 0\\
        x &\mapsto [\mathfrak{\alpha}].
    \end{align*}
    Since $\text{Cl}_K[n]$ is finite and $\mathcal{O}_K^\times$ is a finitely generated abelian group (by Dirichlet's unit theorem), we conclude that $K(\varnothing,n)$ is finite.
\end{proof}
\section{Elliptic curves over number fields continued}
\subsection{The weak Mordell-Weil theorem}
\begin{lemma}\label{lemma12.1}
    Let $E/K$ be an elliptic curve and $L/K$ a finite Galois extension. Then the natural map $E(K)/nE(K) \to E(L)/nE(L)$ has finite kernel.
\end{lemma}
\begin{proof}
    For each element in the kernel, we pick a coset representative $P \in E(K)$ and then $Q \in E(L)$ such that $nQ=P$. For any $\sigma \in \text{Gal}(L/K)$, $n(\sigma(Q)-Q) = \sigma P - P = 0$, so $\sigma(Q)-Q \in E[n]$. Since $\text{Gal}(L/K)$ and $E[n]$ are finite, there are only finitely many possibilities for the map $\text{Gal}(L/K) \to E[n]$ given by $\sigma \mapsto \sigma(Q)-Q$. But if $P_1,P_2 \in E(K)$ with $P_i = nQ_i$ for $Q_i \in E(L)$ and $\sigma(Q_1)-Q_1=\sigma(Q_2)-Q_2 ~\forall \sigma \in \text{Gal}(L/K)$, then $\sigma(Q_1-Q_2)=Q_1-Q_2 ~\forall \sigma \in\text{Gal}(L/K)$, so $Q_1-Q_2 \in E(K)$ and so $P_1-P_2 \in nE(K)$. We conclude that
    \begin{align*}
        \text{ker}(E(K)/nE(K) \to E(L)/nE(L)) \hookrightarrow \text{Maps}(\text{Gal}(L/K),E[n])
    \end{align*}
    and the set on the right is finite, which finishes the proof.
\end{proof}
\begin{theorem}[Weak Mordell-Weil Theorem]\label{theorem12.2}
    \marginpar{28 Feb 2024, Lecture 18}
    Let $K$ be a number field, $E/K$ an elliptic curve and $n\ge 2$ an integer. Then $E(K)/nE(K)$ is finite.
\end{theorem}
\begin{proof}
    By Lemma \ref{lemma12.1}, we may replace $K$ by a finite Galois extension of $K$. Hence WLOG assume $\mu_n \subset K$ and $E[n] \subset E(K)$. Let $$S = \{\mathfrak{p} \mid  n\} \cup \{\text{primes of bad reduction for }E/K\}.$$ For each $P \in E(K)$, the extension $K([n]^{-1}P)/K$ is unramified outside $S$ by Theorem \ref{theorem9.8}. Since $\text{Gal}(\overline{K}/K)$ acts on $[n]^{-1}P$, it follows that $K([n]^{-1}P)/K$ is a Galois extension.
    \vspace{1mm}
     
    Let $Q \in [n]^{-1}P$. Since $E[n] \subset E(K)$, we have $K(Q) = K([n]^{-1}P)$. Consider the map $\text{Gal}(K(Q)/K) \hookrightarrow E[n] \cong (\mathbb{Z}/n\mathbb{Z})^2$ by $\sigma \mapsto \sigma Q-Q$ (for $-$ being $\ominus$ here). This is a group homomorphism, as $\sigma \tau Q - Q = \sigma(\tau Q - Q)+ \sigma Q - Q$. It is also injective, as $\sigma Q = Q \implies \sigma$ fixes $K(Q)$ pointwise, i.e. $\sigma=1$. Hence $K(Q)/K$ is an abelian extension of exponent dividing $n$ unramified outside $S$. So by Proposition \ref{prop11.3}, as we vary $P \in E(K)$, there are only finitely many possibilities for $K(Q)$. Let $L$ be the composite of all such extensions of $K$. Then $L/K$ is finite and Galois and $E(K)/nE(K) \to E(L)/nE(L)$ is the zero map, so by Lemma \ref{lemma12.1}, $\left|E(K)/nE(K)\right|<\infty$.
\end{proof}
\textbf{Remark.} If $K=\mathbb{R}$,$K = \mathbb{C}$ or $[K:\mathbb{Q}_p]<\infty$, then $\left|E(K)/nE(K)\right|<\infty$, yet $E(K)$ is uncountable, so not finitely generated.
\vspace{1mm}
 
\textbf{Fact.} If $K$ is a number field, then there exists a quadratic form (known as the \textbf{canonical height}) $\widehat{h}: E(K) \to \mathbb{R}_{\ge 0}$ with the property that for any $B\ge 0$, the set $\{P \in E(K) \mid \widehat{h}(P)\le B\}$ is finite.

\begin{theorem}[Mordell-Weil Theorem]\label{theorem12.3}
    Let $K$ be a number field and $E/K$ an elliptic curve. Then $E(K)$ is a finitely generated abelian group.
\end{theorem}
\begin{proof}
    Fix an integer $n\ge 2$. By Weak Mordell-Weil, $\left|E(K)/nE(K)\right|<\infty$. Pick coset representatives $P_1,\ldots,P_m$ and let $$\Sigma = \{P \in E(K) \mid \widehat{h}(P)\le \max_{1\le i\le m}\widehat{h}(P_i)\}.$$ We claim that $\Sigma$ generates $E(K)$. Indeed, if not, then there exists an element $P \in E(K)\setminus \{\text{subgroup generated by }\Sigma\}$ of minimal height (using our fact above). Then $P = P_i + nQ$ for some $1\le i\le m$ and $Q \in E(K)$. Note that $Q \in E(K)\setminus \{\text{subgroup gen. by }\Sigma\}$ and the minimal choice of $P$ implies 
    \begin{align*}
        4\hat{h}(P)&\le 4\hat{h}(Q)\le n^2 \hat{h}(Q) =\hat{h}(nQ) = \hat{h}(P-P_i)\\
        &\le \hat{h}(P-P_i)+\hat{h}(P+P_i) = 2\hat{h}(P) + 2\hat{h}(P_i),
    \end{align*}
    (using the parallelogram law in the last step), so $\hat{h}(P)\le \hat{h}(P_i)$ and so $P \in \Sigma$, a contradiction to the choice of $P$. But by our fact, $\Sigma$ is finite, so we're done.
\end{proof}
\section{Heights}
For simplicity, take $K=\mathbb{Q}$. Write $P = (a_0:a_1:\ldots:a_n)$ for $P \in \mathbb{P}^n(\mathbb{Q})$, where we scale to have $a_0,a_1,\ldots,a_n \in \mathbb{Z}$ and $\gcd(a_0,a_1,\ldots,a_n) = 1$.
\begin{defn}\label{defn13.1}
    We define the \textbf{height} of $P$ as \[
    H(P) = \max_{0\le i\le n} \left|a_i\right|.
    \]
\end{defn}
\begin{lemma}\label{lemma13.1}
    Let $f_1,f_2 \in \mathbb{Q}[X_1,X_2]$ be coprime homogeneous polynomials of degree $d$. Let $F : \mathbb{P}^1 \to \mathbb{P}^1$ be given by $(x_1:x_2) \mapsto (f_1(x_1,x_2),f_2(x_1,x_2))$. Then $\exists c_1,c_2>0$ such that \[
    c_1 H(P)^d \le H(F(P)) \le c_2H(P)^d ~\forall P \in \mathbb{P}^1(\mathbb{Q}).
    \]
\end{lemma}
\begin{proof}
    WLOG assume $f_1,f_2 \in \mathbb{Z}[X_1,X_2]$. For the upper bound, write ${P = (a:b)}$ for $a,b \in \mathbb{Z}$ coprime, so $$H(F(P))\le \max(\left|f_1(a,b)\right|,\left|f_2(a,b)\right|) \le c_2 \max(\left|a\right|^d,\left|b\right|^d),$$
    where $c_2 = \max_{i=1,2}(\text{sum of abs. values of coeffs. of }f_i)$, so $H(F(P))\le c_2 H(P)^d$.

    \vspace{1mm}
     
    For the lower bound, we claim $\exists (g_{ij})_{1\le i,j\le 2} \in \mathbb{Z}[X_1,X_2]$, homogeneous of degree $d-1$ and $\kappa \in \mathbb{Z}_{>0}$ such that \[
    \sum_{j=1}^{2} g_{ij}f_j = \kappa X_i^{2d-1}
    \]
    for $i=1,2$. Indeed, running Euclid's algorithm on $f_1(X,1)$ and $f_2(X,1)$ gives $r,s \in \mathbb{Q}[X]$ of degree $<d$ such that $r(X)f_1(X,1)+s(X)f_2(X,1)=1$. Homogenizing and clearing denominators gives the desired result for $i=2$. The case for $i=1$ is analogous. Write $P = (a_1:a_2)$ for $a_1,a_2 \in \mathbb{Z}$ coprime. The expression above implies $\sum_{j=1}^{2} g_{ij}(a_1,a_2)f_j(a_1,a_2) = \kappa a_i^{2d-1}$ for $i=1,2$. Hence $\gcd(f_1(a_1,a_2),f_2(a_1,a_2))$ divides $\gcd(\kappa a_1^{2d-1}, \kappa a_2^{2d-1}) = \kappa$, but also
    \[
    \left|\kappa a_i^{2d-1}\right|\le \underbrace{\max_{j=1,2} \left|f_j(a_1,a_2)\right|}_{\le \kappa H(F(P))} \underbrace{\sum_{j=1}^{2}\left|g_{ij}(a_1,a_2)\right|}_{\le \gamma_i H(P)^{d-1}},
    \]
    where $\gamma_i = \sum_{j=1}^{2} (\text{sum of abs. values of coefficients of }g_{ij})$. This implies that 
    \begin{align*}
        &\kappa \left|a_i\right|^{2d-1} \le \kappa H(F(P))\gamma_i H(P)^{d-1}\\
        \implies & H(P)^{2d-1} \le \max(\gamma_1,\gamma_2)H(F(P))H(P)^{d-1}\\
        \implies & \frac{1}{\max(\gamma_1,\gamma_2)}H(P)^d \le H(F(P)).
    \end{align*}
    Taking $c_2 = \frac{1}{\max(\gamma_1,\gamma_2)}$ finishes the proof.
\end{proof}
\marginpar{01 Mar 2024, Lecture 19}
\textbf{Notation.} For $x \in \mathbb{Q}$, write $H(x)=H((x:1)) = \max(\left|r\right|,\left|s\right|)$ for $x=\frac{r}{s}$ for $r,s \in \mathbb{Z}$ coprime. 
\begin{defn}
    Let $E/\mathbb{Q}$ be an elliptic curve given by $y^2=x^3+ax+b$. The \textbf{height} is defined as 
    \begin{align*}
        H:E(\mathbb{Q}) &\to \mathbb{R}_{\ge 1}\\
        P &\mapsto \begin{cases}
            H(x) &\text{if} P = (x,y).\\
            1 &\text{if} P = 0_E.
        \end{cases}
    \end{align*}
    We also define the \textbf{logarithmic height} $h: E(\mathbb{Q}) \to \mathbb{R}_{\ge 0}$ by $P \mapsto \log H(P)$.
\end{defn}
\begin{lemma}\label{lemma13.2}
    Let $E,E'$ be elliptic curves defined over $\mathbb{Q}$ and let $\phi: E \to E'$ be an isogeny defined over $\mathbb{Q}$. Then $\exists c>0$ such that
    \begin{align*}
        \left|h(\phi(P))-(\text{deg }\phi)h(P)\right| < c ~\forall P \in E(\mathbb{Q}).
    \end{align*}
    Importantly, note that $c$ depends on $E$ and $E'$, but not on $P$.
\end{lemma}
\begin{proof}
    Recall from Lemma \ref{lemma5.4} that we have a morphism $\xi$ making our diagram commute with $\text{deg }\phi = \text{deg }\xi := d$. By Lemma \ref{lemma13.1}, $\exists c_1,c_2>0$ such that $c_1H(P)^d \le H(\phi(P)) \le c_2 H(P)^d ~\forall P \in E(\mathbb{Q})$. Taking logarithms gives 
    \begin{align*}
        \left|h(\phi(P))-d h(P)\right| \le \max(\log c_2, - \log c_1) := c
    \end{align*}
    as desired.
\end{proof}
\begin{example}
    Take $\phi = [2] : E \to E$. Then $\exists c>0$ such that $\left|h(2P)-4h(P)\right|\le c ~\forall P \in E(\mathbb{Q})$.
\end{example}
\begin{defn}
    The \textbf{canonical height} is defined as 
    \begin{align*}
        \widehat{h}(P) = \lim_{n \to \infty} \frac{1}{4^n} h(2^n P).
    \end{align*}
\end{defn}
We need to check that this converges. Let $m\ge n$, then
\begin{align*}
    & \left|\frac{1}{4^m}h(2^mP)-\frac{1}{4^n}h(2^n P)\right| \le \sum_{r=n}^{m-1} \left|\frac{1}{4^{r+1}}h(2^{r+1}P)-\frac{1}{4^r}h(2^rP)\right|\\
    =& \sum_{r=n}^{m-1} \frac{1}{4^{r+1}}\left|h(2(2^rP))-4 h(2^rP)\right| < c \sum_{r=n}^{\infty} \frac{1}{4^{r+1}} = c \cdot \frac{1}{3\cdot 4^n} \to 0
\end{align*}
as $n \to \infty$. Hence the sequence is Cauchy, so converges, so $\widehat{h}(P)$ exists.
\begin{lemma}\label{lemma13.3}
    $\left|h(P)-\widehat{h}(P)\right|$ is bounded for $P \in E(\mathbb{Q})$.
\end{lemma}
\begin{proof}
    Put $n=0$ in the above calculation to get $\left|\frac{1}{4^m}h(2^m P)-h(P)\right|\le \frac{c}{3}$. Take the limit as $m \to \infty$ to conclude.
\end{proof}
\begin{lemma}\label{lemma13.4}
    For any $B>0$,
    \begin{align*}
        \# \{P \in E(\mathbb{Q}) \mid \widehat{h}(P) \le B\} < \infty.
    \end{align*}
\end{lemma}
\begin{proof}
    $\widehat{h}(P)$ is bounded $\implies $ $h(P)$ is bounded by Lemma \ref{lemma13.3}. But there are only finitely many possibilities for $x$, and each of them gives $\le 2$ choices of $y$, so we're done.
\end{proof}
\begin{lemma}\label{lemma13.5}
    Let $\phi :E \to E'$ be an isogeny over $\mathbb{Q}$. Then
    \begin{align*}
        \widehat{h}(\phi(P)) = (\text{deg }\phi)\widehat{h}(P) ~\forall P \in E(\mathbb{Q}).
    \end{align*}
\end{lemma}
\begin{proof}
    By Lemma \ref{lemma13.2}, $\exists c>0$ such that $\left|h(\phi(P))-(\text{deg }\phi)h(P)\right|< c ~\forall P \in E(\mathbb{Q})$. Replace $P$ by $2^nP$, divide by $4^n$ and take the limit as $n \to \infty$ to conclude.
\end{proof}
\textbf{Remarks.} 
\begin{enumerate}[(i)]
    \item The case $\text{deg }\phi = 1$ shows that $\widehat{h}$ (unlike $h$) is independent of the choice of Weierstrass equation for $E$.
    \item Taking $\phi = [n]: E \to E$ shows $\widehat{h}(nP)= n^2 \widehat{h}(P) ~\forall P \in E(\mathbb{Q})$.
\end{enumerate} 
\begin{lemma}\label{lemma13.6}
    Let $E/\mathbb{Q}$ be an elliptic curve. Then $\exists c>0$ such that
    \[
    H(P+Q)H(P-Q)\le cH(P)^2H(Q)^2
    \]
    for all $P,Q \in E(\mathbb{Q})$ with $P,Q,P\pm Q \neq 0_E$.
\end{lemma}
\begin{proof}
    Let $E$ have Weierstrass equation $y^2=x^3+ax+b$ with $a,b \in \mathbb{Z}$. Let $P,Q,P+Q,P-Q$ have $x$-coordinates $x_1,x_2,x_3,x_4$. By Lemma \ref{lemma5.8}, there exist $W_0,W_1,W_2 \in \mathbb{Z}[x_1,x_2]$ of degree $\le 2$ in both $x_1$ and $x_2$ such that $(1:x_3+x_4:x_3x_4) = (W_0:W_1:W_2)$ (and $W_0 = (x_1-x_2)^2$). Write $x_i = \frac{r_i}{s_i}$ for $r_i,s_i \in \mathbb{Z}$ coprime. Then we get 
    \[
    (s_3s_4:r_3s_4+r_4s_3:r_3r_4) = ((r_1s_2-r_2s_1)^2:\ldots).
    \]
    Then 
    \begin{align*}
        H(P+Q)H(P-Q) &= \max(\left|r_3\right|,\left|s_3\right|)\max(\left|r_4\right|,\left|s_4\right|) \\
        &\le 2\max(\left|s_3s_4\right|,\left|r_3s_4+r_4s_3\right|,\left|r_3r_4\right|)\\
        &\le 2 \max(\left|(r_1s_2-r_2s_1)^2\right|,\ldots) \\
        &\le c H(P)^2H(Q)^2.
    \end{align*}
    where $c$ depends on $E$, but not on $P,Q$.\footnote{I watched this lecture online and Fisher spent a few minutes explaining how the above inequalities follow, so it might be nontrivial to deduce this (shouldn't be too hard though).}
\end{proof}
\begin{theorem}\label{theorem13.7}
    $\widehat{h}:E(\mathbb{Q}) \to \mathbb{R}_{\ge 0}$ is a quadratic form.
\end{theorem}
\begin{proof}
    By Lemma \ref{lemma13.6} and the fact that $\left|h(2P)-4h(P)\right|$ is bounded, $\exists c \in \mathbb{R}$ such that 
    \[
    h(P+Q) + h(P-Q) \le 2h(P) + 2h(Q) + c ~\forall P,Q \in E(\mathbb{Q}).
    \]
    Replacing $P,Q$ by $2^nP,2^nQ$, dividing by $4^n$ and taking the limit as $n \to \infty$ gives \[
    \widehat{h}(P+Q) + \widehat{h}(P-Q) \le 2\widehat{h}(P)+2\widehat{h}(Q).
    \]
    Replacing $P,Q$ by $P+Q,P-Q$ and using $\widehat{h}(2P) = 4 \widehat{h}(P)$ gives the reverse inequality. Hence $\widehat{h}$ satisfies the parallelogram law, so it is a quadratic form.
\end{proof}
\textbf{Remark.} For $K$ a number field and $P = (a_0:a_1:\ldots:a_n) \in \mathbb{P}^n(K)$, we define \[
H(P) = \prod_{v}^{} \max_{0\le i\le n} \left|a_i\right|_v
\]
where the product is over all places $v$ of $K$, and the absolute values are normalized such that $\prod_{v}^{} \left|\lambda\right|_v =1 ~\forall \lambda \in K^\times$. All results proved in this section then generalize from $\mathbb{Q}$ to $K$. Note further that the places are the finite places given by $\left|x\right|_{\mathfrak{p}} = c^{v_{\mathfrak{p}}(x)}$ for some $c<1$ and the infinite places $\left|x\right|_{\sigma} = \left|\sigma(x)\right|^d$ for some $d>0$ (and now we choose appropriate $c,d$ to satisfy the product formula).

\section{Dual isogenies and the Weil pairing}
\marginpar{04 Mar 2024, Lecture 20}
Let $K$ be a perfect field and $E/K$ an elliptic curve.
\begin{prop}\label{prop14.1}
    Let $\Phi \subset E(\overline{K})$ be a finite $\text{Gal}(\overline{K}/K)$-stable subgroup. Then there exists an elliptic curve $E'/K$ and a separable isogeny $\phi : E \to E'$ defined over $K$ with kernel $\Phi$ such that every isogeny $\psi : E \to E''$ with $\Phi \subset \text{ker}(\psi)$ factors uniquely via $\phi$ as 
    $$\begin{tikzcd}[]
        E & & E''\\
        & E' & 
        \arrow["\psi", from=1-1, to=1-3]
        \arrow["\phi", from=1-1, to=2-2]
        \arrow["\exists \text{ unique}", from=2-2, to=1-3]
    \end{tikzcd}.$$
\end{prop}
\begin{proof}
    Omitted. (See e.g. Silverman, Chapter 3, Proposition 4.12).
\end{proof}
\begin{prop}\label{prop14.2}
    Let $\phi :E \to E'$ be an isogeny of degree $n$. Then there exists a unique isogeny $\widehat{\phi}:E' \to E$ (called the \textbf{dual isogeny}) such that $\widehat{\phi}\phi = [n]$.
\end{prop}
\begin{proof}
    If $\phi$ is separable, then $\left|\text{ker}(\phi)\right|=n$, so $\text{ker}(\phi) \subset E[n]$. Apply Proposition \ref{prop14.1} with $\psi = [n]$ to get the result.
    \vspace{1mm}
     
    The case where $\phi$ is inseparable is omitted.
    \vspace{1mm}
     
    For uniqueness, if $\psi_1 \phi = \psi_2 \phi$, then $(\psi_1-psi_2)\phi=0$, so $\text{deg}(\psi_1-\psi_2)\text{deg}(\phi) = 0$, but $\text{deg}(\phi) \neq 0$, so $\psi_1=\psi_2$.
\end{proof}
\textbf{Remarks.}
\begin{enumerate}[(i)]
    \item Write $E_1 \sim E_2 \iff E_1,E_2$ are isogenous. Then $\sim$ is an equivalence relation.
    \item $\text{deg }[n] = n^2 \implies  \begin{cases}
        \text{deg }\phi = \text{deg }\widehat{\phi},\\
        \widehat{[n]} = [n].
    \end{cases}$ 
    \item $\phi \widehat{\phi}\phi = \phi[n]_E = [n]_{E'}\phi$, so $\phi\widehat{\phi}=[n]_{E'}$. In particular, $\widehat{\widehat{\phi}}=\phi$.
    \item If $E \stackrel{\psi}{\to} E' \stackrel{\phi}{\to} E''$, then $\widehat{\phi \psi} = \widehat{\psi}\widehat{\phi}$.
    \item If $\phi \in \text{End}(E)$, then $\phi^2-[\text{tr}(\phi)]\phi + [\text{deg }\phi]=0$, so $\phi([\text{tr}(\phi)]-\phi) = [\text{deg }\phi]$, so $\widehat{\phi} = [\text{tr}(\phi)]-\phi$, so $[\text{tr}(\phi)] = \phi + \widehat{\phi}$.
\end{enumerate} 
\begin{lemma}\label{lemma14.3}
    If $\phi,\psi \in \text{Hom}(E,E')$, then $\widehat{\phi+\psi}\widehat{\phi}+\widehat{\psi}$.
\end{lemma}
\begin{proof}
    \begin{enumerate}[(i)]
        \item If $E = E'$, then this follows from $\text{tr}(\phi+\psi)=\text{tr}(\phi)+\text{tr}(\psi)$.
        \item In general, let $\alpha : E' \to E$ be any isogeny (e.g. $\alpha = \widehat{\phi}$) and we get
        \begin{align*}
            &\widehat{\alpha \phi + \alpha \psi} = \widehat{\alpha \phi} + \widehat{\alpha \psi}\\
            \implies &\widehat{\phi+\psi}\widehat{\alpha} = (\widehat{\phi}+\widehat{\psi})\widehat{\alpha}\\
            \implies & \widehat{\phi + \psi} = \widehat{\phi}+\widehat{\psi}
        \end{align*}
        where the first line follows by (i).
    \end{enumerate}
\end{proof}
\textbf{Remark.} In Silverman's book, he proves Lemma \ref{lemma14.3} first and uses this to show that $\text{deg }: \text{Hom}(E,E') \to \mathbb{Z}$ is a quadratic form.
\begin{defn}
    We define the following map: $\text{sum}: \text{Div}(E) \to E$ by $\sum_{}^{} n_P P \mapsto \sum_{}^{} n_P P$ where on the left we have a formal sum and on the right we sum using the group law.
\end{defn}
Recall that $E \stackrel{\sim}{\to} \text{Pic}^0(E)$ by $P\mapsto [(P)-(0_E)]$. Hence $\text{sum}(D) \mapsto [D] ~\forall D \in \text{Div}^0(E)$. Thus we conclude:
\begin{lemma}\label{lemma14.4}
    Let $D \in \text{Div}(E)$. Then 
    \begin{align*}
        D \sim 0 \iff \begin{cases}
            \text{deg }D = 0,\\
            \text{sum }D = 0_E.
        \end{cases}
    \end{align*}
\end{lemma}
Now let $\phi : E \to E'$ be a isogeny of degree $n$ with dual isogeny $\widehat{\phi} :E' \to E$. Assume that $\text{char }K \nmid n$ (so $\phi, \widehat{\phi}$ are separable). Write $E[\phi]$ for $\text{ker}(\phi)$. We define the \textbf{Weil pairing}
\[
e_{\phi} : E[\phi] \times E'[\widehat{\phi}] \to \mu_n
\]
as follows: Let $T \in E'[\widehat{\phi}]$. Then $nT = 0$, so there exists $f \in \overline{K}(E')^\times$ such that $\text{div}(f)=n(T)-n(0)$. Pick $T_0 \in E(\overline{K})$ with $\phi(T_0) = T$. Then $\phi^*(T)-\phi^*(0) = \sum_{P \in E[\phi]}^{} (P+T_0) - \sum_{P \in E[\phi]}^{} (P)$ has sum $nT_0 = \widehat{\phi}\phi T_0 = \widehat{\phi}(T) = 0$. So there exists $g \in \overline{K}(E)^\times$ such that $\text{div}(g)=\phi^*(T)-\phi^*(0)$. Now $\text{div}(\phi^* f) = \phi*(\text{div }f) = n(\phi^*(T)-\phi^*(0))=\text{div}(g^n)$. Hence $\phi^* f = cg^n$ for some $c \in \overline{K}^\times$. Rescaling $f$ allows us to wlog assume $c=1$, i.e. $\phi^* f = g^n$.
\vspace{1mm}
 
If $S \in E[\phi]$, then $\tau_S^*(\text{div }g)=\text{div }g$, so $\text{div}(\tau_S^* g) =\text{div }g$ and so $\tau_S^* g = \zeta g$ for some $\times \in \overline{K}^\times$, i.e. $\zeta = \frac{g(X+S)}{g(X)}$ is independent of the choice of $X \in E(\overline{K})$. Now $\zeta^n = \frac{g(X+S)^n}{g(X)^n} = \frac{f(\phi(X+S))}{f(\phi(X))} = 1$ since $S \in E[\phi]$, so $\zeta \in \mu_n$. We hence define $e_{\phi}(S,T) = \frac{g(X+S)}{g(X)}$.

\begin{prop}\label{prop14.5}
    $e_\phi$ is bilinear and nondegenerate.
\end{prop}
\begin{proof}
    \begin{enumerate}[(i)]
        \item Linearity in the first argument:
        \begin{align*}
            e_\phi(S_1+S_2,T) = \frac{g(X+S_1+S_2)}{g(X+S_2)}\frac{g(X+S_2)}{g(X)} =e_\phi(S_1,T)e_\phi(S_2,T).
        \end{align*}
        \item Linearity in the second argument: Let $T_1,T_2 \in E'[\widehat{\phi}]$ with $\text{div}(f_i)n(T_i)-n(0)$ and $\phi^* f_i = g_i^n$ for $i=1,2$. There exists $h \in \overline{K}(E')^\times$ such that $\text{div}(h)=(T_1)+(T_2)-(T_1+T_2)-(0)$. We put $f = \frac{f_1f_2}{h^n}$ and $g = \frac{g_1g_2}{\phi^*h}$.
        \vspace{1mm}
         
        We can check $\text{div}(f) = n(T_1+T_2)-n(0)$ and $\phi^* f = \frac{\phi^* f_1 + \phi^* f_2}{(\phi^* h)^n} = \left(\frac{g_1g_2}{\phi^* h}\right)^n = g^n$, so 
        \begin{align*}
            e_{\phi}(S,T_1+T_2) &= \frac{g(X+S)}{g(X)} = \frac{g_1(X+S)}{g_1(X)}\frac{g_2(X+S)}{g_2(X)}\frac{h(\phi(X_1))}{h(\phi(X+S))} \\&= e_\phi(S,T_1)e_{\phi}(S,T_2)
        \end{align*}
        where the last term cancels since $S \in E[\phi]$.
        \item $e_\phi$ is nondegenerate. Fix $T \in E'[\widehat{\phi}]$ and suppose $e_\phi(S,T)=1 ~\forall S \in E[\phi]$, so $\tau_S^*g = g ~\forall S \in E[\phi]$. We get that $\overline{K}(E)/\phi^*\overline{K}(E')$ is a Galois extension with Galois group $E[\phi]$ (here $S \in E[\phi]$ acts via $\tau_S^*$). Hence $\tau_S^*g = g ~\forall S \in E[\phi]$, so $g = \phi^* h$ for some $h \in \overline{K}(E')$, so $\phi^* f = g^n = (\phi^* h)^n = \phi^*(h^n)$, so $f = h^n$, whence $\text{div}(h)=(T)-(0)$. This implies $T=0$ (using, I think, the sum function). We've shown $E'[\widehat{\phi}] \hookrightarrow \text{Hom}(E[\phi],\mu_n)$, which is actually an isomorphism as $\# E[\phi] = \# E'[\widehat{\phi}] = n$, so $e_\phi$ is nondegenerate.
    \end{enumerate}
\end{proof}
\textbf{Remarks.}
\begin{enumerate}[(i)]
    \marginpar{06 Mar 2024, Lecture 21}
    \item If $E,E',\phi$ are defined over $K$, then $e_\phi$ is Galois invariant, i.e. $e_\phi(\sigma(S),\sigma(T))=\sigma(e_\phi(S,T)) ~\forall \sigma \in \text{Gal}(\overline{K}/K), S \in E[\phi], T \in E'[\widehat{\phi}]$.
    \item Taking $\phi = [n]:E \to E$ (so $\widehat{\phi} = [n]$) gives $e_n = E[n] \times E[n] \to \mu_{n^2}$, but in fact the image is contained in $\mu_n \subset \mu_{n^2}$, as $E[n]\times E[n]$ has exponent $n$.
\end{enumerate} 
\begin{cor}\label{cor14.6}
    If $E[n] \subset E(K)$, then $\mu_n \subset K$.
\end{cor}
\begin{proof}
    Let $T \in E[n]$ have order $n$. Since $e_n$ is nondegenerate, $\exists S \in E[n]$ such that $e_n(S,T)$ is a primitive $n^{\text{th}}$ root of unity, say $\zeta_n$. Then $\sigma(\zeta_n)=\sigma(e_n(S,T))=e_n(\sigma(S),\sigma(T))=e_n(S,T)=\zeta_n ~\forall \sigma \in \text{Gal}(\overline{K}/K)$, so $\zeta_n \in K$.
\end{proof}
\begin{example}
    There does not exist $E/\mathbb{Q}$ with $E(\mathbb{Q})_{\text{tors}} = (\mathbb{Z}/3\mathbb{Z})^2$, since $\zeta_3 \not\in \mathbb{Q}$.
\end{example}
\textbf{Remark.} In fact, $e_n$ is alternating, i.e. $e_n(T,T)=1 ~\forall T \in E[n]$. This implies $e_n(S,T)=e_n(T,S)^{-1}$.

\section{Galois cohomology}
Let $G$ be a group and let $A$ be an abelian group that's a $G$-module, i.e. it's an abelian group equipped with a homomorphism $G \to \text{Aut}(A)$.
\begin{defn}
    We define the $0^{\text{th}}$ cohomology group as
    \begin{align*}
        H^0(G,A) = A^G = \{a \in A\mid \sigma(a)=a ~\forall \sigma \in G\}.
    \end{align*}
    We define the collection of $1^{\text{st}}$ cochains as \[
    C^1(G,A) = \{\text{maps }G \to A\}.
    \]
    This contains the collection of $1^{\text{st}}$ cocycles
    \[
    Z^1(G,A) = \{(a_\sigma)_{\sigma \in G} \mid a_{\sigma \tau} = \sigma(a_\tau)+a_\sigma ~\forall \sigma, \tau \in G\},
    \]
    which contains the collection of $1^{\text{st}}$ coboundaries
    \[
    B^1(G,A) = \{(\sigma(b)-b)_{\sigma \in G} \mid b \in A\}.
    \]
    We set the $1^{\text{st}}$ group cohomology to be
    \[
    H^1(G,A) = \frac{Z^1(G,A)}{B^1(G,A)} = \frac{\text{cocycles}}{\text{coboundaries}}.
    \]
\end{defn}
\textbf{Remark.} If $G$ acts trivially on $A$, then $H^1(G,A)=\text{Hom}(G,A)$.
\begin{theorem}\label{theorem15.1}
    A short exact sequence of $G$-modules \[
    0 \to A \stackrel{\phi}{\to } B \stackrel{\psi}{\to} C \to 0
    \]
    gives rise to a long exact sequence 
    \[
    0 \to A^G \stackrel{\phi}{\to} B^G \stackrel{\psi}{\to} C^G \stackrel{\delta}{\to}  H^1(G,A) \stackrel{\phi^*}{\to} H^1(G,B) \stackrel{\psi^*}{\to} H^1(G,C).
    \]
\end{theorem}
\begin{proof}
    Omitted.
\end{proof}
However, we give the definition of $\delta$. Let $c \in C^G$. There exists some $b \in B$ with $\psi(b)=c$. Then $\psi(\sigma(b)-b)=\sigma \psi(b)-\psi(b) = \sigma c - c = 0 ~\forall \sigma \in G$, so $\sigma b - b = \phi(a_\sigma)$ for some $a_\sigma \in A$. We can check that $(a_\sigma)_{\sigma \in G} \in Z^1(G,A)$, so we define $\delta(c) = \text{the class of }(a_\sigma)_{\sigma \in G}$ in $H^1(G,A)$.
\begin{theorem}\label{theorem15.2}
    Let $A$ be a $G$-module and $H \le G$ a normal subgroup. Then there is an inflation-restriction exact sequence 
    \[
    0 \to H^1(G/H,A^H) \stackrel{\text{inf}}{\to} H^1(G,A) \stackrel{\text{res}}{\to} H^1(H,A).
    \]
\end{theorem}
\begin{proof}
    Omitted.
\end{proof}
Let $K$ be a perfect field. $\text{Gal}(\overline{K}/K)$ is a topological group with its topology generated by the basis of open subgroups of the form $\text{Gal}(\overline{K}/L)$ for $L$ a finite extension of $K$. If $G = \text{Gal}(\overline{K}/K)$, we modify the definition of $H^1(G,A)$ by insisting that
\begin{enumerate}[(1)]
    \item The stabilizer of each $a \in A$ is an open subgroup of $G$.
    \item All cochains $G \to A$ are continuous, where $A$ is given the discrete topology.
\end{enumerate}
Then $$H^1(\text{Gal}(\overline{K}/K),A) = \lim_{\substack{\longrightarrow\\L/K \text{ finite}\\L/K \text{ Galois}}} = H^1(\text{Gal}(L/K),A^{\text{Gal}(\overline{K}/L)}),$$ the direct limit with respect to inflation maps.
\begin{theorem}[Hilbert's Theorem 90]\label{theorem15.3}
    Let $L/K$ be a finite Galois extension. Then $H^1(\text{Gal}(L/K),L^\times)=0$.
\end{theorem}
\begin{proof}
    Let $G = \text{Gal}(L/K)$ and let $(a_\sigma)_{\sigma \in G} \in Z^1(G,L^\times)$. Distinct automorphisms are linearly independent, so $\exists y$ such that \[
    x := \sum_{\tau \in G}^{} a_{\tau}^{-1} \tau(y) \neq 0.
    \]
    Then, using the fact that $a_{\sigma \tau}=\sigma(a_\tau)a_\sigma$, so $\sigma(a_\tau)^{-1} = a_\sigma a_{\sigma \tau}^{-1}$, we get
    \begin{align*}
        \sigma(x) &= \sum_{\tau \in G}^{} \sigma(a_{\tau}^{-1})\sigma \tau(y) \\
        &= a_\sigma \sum_{\tau \in G}^{} a_{\sigma \tau}^{-1} \sigma \tau(y) \\
        &= a_\sigma x \\
        &\implies  a_\sigma = \frac{\sigma(x)}{x} \\
        &\implies (a_\sigma)_{\sigma \in G} \in B^1(G,L^\times).
    \end{align*}
    Hence $H^1(G,L^\times) = 0$.
\end{proof}
\begin{cor}
    With the setup as before, $H^1(\text{Gal}(\overline{K}/K),\overline{K}^\times) = 0$.
\end{cor}
\textbf{Application.} Assume $\text{char }K \nmid n$. Then there is a short exact sequence of $\text{Gal}(\overline{K}/K)$-modules 
\begin{align*}
    0 \to \mu_n \to \overline{K}^\times &\to \overline{K}^\times \to 0 \\
    x &\mapsto x^n
\end{align*} 
giving a long exact sequence
\begin{align*}
    K^\times &\to K^\times \to H^1(\text{Gal}(\overline{K}/K),\mu_n) \to H^1(\text{Gal}(\overline{K}/K),\overline{K}^\times) = 0\\
    x &\mapsto x^n
\end{align*}
by Hilbert's Theorem 90. Consequently, $H^1(\text{Gal}(\overline{K}/K),\mu_n) \cong K^\times/(K^\times)^n$. If $\mu_n \subset K$, then \[
\text{Hom}_{\text{cts}}(\text{Gal}(\overline{K}/K),\mu_n) \cong K^\times/(K^\times)^n.
\]
Here the finite subgroups of the LHS are of the form $\text{Hom}(\text{Gal}(L/K),\mu_n)$ for $L/K$ a finite abelian extension of exponent dividing $n$. This gives another proof of Theorem \ref{theorem11.2}.

\marginpar{08 Mar 2024, Lecture 22}
\textbf{Notation.} We write $H^1(K,-)$ to mean $H^1(\text{Gal}(\overline{K}/K),-)$. 
\vspace{1mm}
 
Let $\phi : E \to E'$ be an isogeny of elliptic curves of $K$. We have a short exact sequence of $\text{Gal}(\overline{K}/K)$-modules \[
0 \to E[\phi] \to E \stackrel{\phi}{\to} E' \to 0,
\]
and a long exact sequence
\begin{align*}
    E(K) \stackrel{\phi}{\to} E'(K) \stackrel{\delta}{\to} H^1(K,E[\phi]) \to H^1(K,E) \stackrel{\phi_*}{\to} H^1(K,E'). 
\end{align*}
We get a short exact sequence $$0 \to E'[K]/\phi E[K] \to H^1(K,E[\phi]) \to H^1(K,E)[\phi_*] \to 0.$$
Now take $K$ a number field. For each place $v$ of $K$, we fix an embedding $\overline{K} \subset \overline{K}_v$. Then $\text{Gal}(\overline{K}_v/K_v) \subset \text{Gal}(\overline{K}/K)$. Taking the product over all places gives another short exact sequence compatible with the one above as \[
\begin{tikzcd}
    0 \arrow{r}{} & E'[K]/\phi E[K] \arrow{r}{}\arrow{d} & H^1(K,E[\phi]) \arrow{r}\arrow {d}{\text{res}_v}\arrow{dr}& H^1(K,E)[\phi_*] \arrow{r} \arrow{d}{\text{res}_v} & 0\\
    0 \arrow{r} & \prod_{v}^{} E'[K_v]/\phi E[K_v] \arrow{r} & \prod_{v}^{} H^1(K_v,E[\phi]) \arrow{r}& \prod_{v}^{}  H^1(K_v,E)[\phi_*] \arrow{r} & 0.
\end{tikzcd}
\]
\begin{defn}\label{defna}
    The \textbf{$\phi$-Selmer group} is 
    \begin{align*}
        S^{(\phi)}(E/K) &= \text{ker}(\searrow) \\
        &= \text{ker}\left(H^1(K,E[\phi]) \to \prod_{v}^{} H^1(K_v,E)\right)\\
        &=\{\alpha \in H^1(K,E[\phi]) \mid \text{res}_v(\alpha) \in \text{Im}(\delta_v) \forall v\}.
    \end{align*}
    The \textbf{Tate-Shafarevich group} is \[
        \Sha(E/K) = \text{ker}\left(H^1(K,E) \to \prod_{v}^{} H^1(K_v,E)\right).
    \]
\end{defn}
We get a short exact sequence 
\[
0 \to E'[K]/\phi E[K] \to S^{(\phi)}(E/K) \to \Sha(E/K)[\phi_*] \to 0.
\]
Taking $\phi = [n]$ gives \[
0 \to E[K]/nE[K] \to S^{(n)}(E/K) \to \Sha(E/K)[n] \to 0.
\]
Rearranging the proof of weak Mordell-Weil (Theorem \ref{theorem12.2}) gives
\begin{theorem}\label{theorem15.5}
    $S^{(n)}(E/K)$ is finite.
\end{theorem}
\begin{proof}
    For $L/K$ a finite Galois extension, there is an exact sequence
    \[
    \begin{tikzcd}
        0 \arrow{r} & H^1(\text{Gal}(L/K), E(L)[n]) \arrow{r}{\text{inf}} & H^1(K,E[n]) \arrow{r}{\text{res}} & H^1(L,E[n]) \\
        & & S^{(n)}(E/K) \arrow[draw=none]{u}[sloped,auto=false, rotate=0]{\subset}& S^{(n)}(E/L) \arrow[draw=none]{u}[sloped,auto=false, rotate=0]{\subset}.
    \end{tikzcd}
    \]
    Consequently, by extending our field we may assume $E[n] \subset E(K)$ and hence $\mu_n \subset K$. Hence $E[n] \cong \mu_n \times \mu_n$ as a $\text{Gal}(\overline{K}/K)$-module, so $$H^1(K,E[n]) \cong H^1(K,\mu_n) \times H^1(K,\mu_n) \cong K^\times /(K^\times)^n \times K^\times/(K^\times)^n.$$
    Let $S = \{\text{primes of bad reduction for }E\} \cup \{v \mid n \infty\}$, which is a finite set of places.
    \begin{defn}\label{defn15.6}
        The subgroup of $H^1(K,A)$ unramified outside $S$ is \[
        H^1(K,A;S) = \text{ker}\left(H^1(K,A) \to \prod_{v \not\in S}^{} H^1(K_v^{\text{nr}},A) \right).
        \]
    \end{defn}
    There is a commutative with exact rows
    \[
    \begin{tikzcd}
        E(K_v) \arrow{r}{\times n}\arrow[draw=none]{d}[sloped,auto=false, rotate=0]{\subset} & E(K_v) \arrow{r}{\delta_v}\arrow[draw=none]{d}[sloped,auto=false, rotate=0]{\subset} & H^1(K_v,E[n])\arrow{d}{\text{res}}\\
        E(K_v^{\text{nr}}) \arrow{r}{\times n} & E(K_v^{\text{nr}}) \arrow{r} & H^1(K_v^{\text{nr}}, E[n]),
    \end{tikzcd}
    \]
    where the first arrow on the second row is surjective $\forall v \not\in S$ (Theorem \ref{theorem9.8}). Hence $$S^{(n)}(E/K) \subset H^1(K,E[n];S) \cong H^1(K,\mu_n;S) \times H^1(K,\mu_n;S).$$
    But $$H^1(K,\mu_n;S) \cong \text{ker}\left(K^\times/(K^\times)^n \to \prod_{v \not\in S}^{} (K_v^{\text{nr}})^\times/((K_v^{\text{nr}})^\times)^n\right) \subset K(S,n)$$
    which is finite by Lemma \ref{lemma14.4}. (For the last inclusion we have an optional exercise: we can check that the $\subset$ is actually a $=$ using $\{v \mid n\}\subset S$.)
\end{proof}
\textbf{Remark.}$S^{(n)}(E/K)$ is finite and effectively computable. It is conjectured that $\left|\Sha(E/K)\right|<\infty$. This would imply that $\text{rank}E(K)$ is effectively computable.

\section{Descent by cyclic isogeny}
Let $E,E'$ be elliptic curves over a number field $K$ and let $\phi: E \to E'$ be an isogeny of degree $n$ defined over $K$. Suppose $E'[\widehat{\phi}] \cong \mathbb{Z}/n\mathbb{Z}$, generated by $T \in E'(K)$. Then $E[\phi] \cong \mu_n$ as a $\text{Gal}(\overline{K}/K)$-module via $S \mapsto e_\phi(S,T)$. This gives a short exact sequence of $\text{Gal}(\overline{K}/K)$-modules
\[
0 \longrightarrow \mu_n \longrightarrow E \stackrel{\phi}{\longrightarrow} E' \longrightarrow 0
\]
and a long exact sequence
\[
\begin{tikzcd}
    E(K) \arrow{r}{} & E'(K) \arrow{r}{\delta}\arrow{rd}{\alpha} & H^1(K,\mu_n) \arrow{r}\arrow[draw=none]{d}[sloped,auto=false, rotate=0]{\cong}{~,H90} & H^1(K,E) \arrow{r} & \ldots \\
    & & K^\times/(K^\times)^n. &
\end{tikzcd}
\]
\begin{theorem}\label{theorem16.1}
    Let $f \in K(E')$ and $g \in K(E)$ with $\text{div}(f) = n(T) - n(0)$ and $\phi^* f = g^n$. Then $$\alpha(P) = f(P) \pmod{(K^\times)^n} ~\forall P \in E'(K)\setminus \{0,T\}.$$
\end{theorem}
\begin{proof}
    Let $Q \in \phi^{-1}P$. Then $\delta(P) \in H^1(K,\mu_n)$ is represented by the cocycle $\sigma \mapsto \sigma Q - Q \in E[\phi] \cong \mu_n$ and 
    \begin{align*}
        e_\phi(\sigma Q - Q, T) &= \frac{g(\sigma Q - Q + X)}{g(X)} \\
        &= \frac{g(\sigma Q)}{g(Q)} \\
        &= \frac{\sigma(g(Q))}{g(Q)} \\
        &= \frac{\sigma \sqrt[n]{f(P)}}{\sqrt[n]{f(P)}},
    \end{align*}
    where the first equality holds for any $X \in E$ avoiding the zeros and poles of $g$, so we take $X=Q$ in the second equality, and for the last equality use $\phi^* f = g^n$, so $g(Q)^n = f(P)$. But $H^1(K,\mu_n) \cong K^\times/(K^\times)^n$ via $\left(\sigma \mapsto \frac{\sigma \sqrt[n]{x}}{\sqrt[n]{x}}\right) \mapsfrom x$, so $\alpha(P) \equiv f(P) \pmod{(K^\times)^n}$.
\end{proof}
\subsection{Descent by 2-isogeny}
\marginpar{11 Mar 2024, Lecture 23}
Consider $E: y^2=x(x^2+ax+b)$ with $b(a^2-4b) \neq 0$ and $E':y^2=x(x^2+a'x+b')$ with $a'=-2a, b'=a^2-4b$. Recall we have a 2-isogeny $\phi : E \to E'$ via $(x,y) \mapsto \left(\left(\frac{y}{x}\right)^2, \frac{y(x^2-b)}{x^2}\right)$ with dual $\widehat{\phi}: E' \to E$ via $(x,y) \mapsto \left(\frac{1}{4} \left(\frac{y}{x}\right)^2, \frac{y(x^2-b')}{8x^2}\right)$.

\vspace{1mm}
 
We have $E[\phi] = \{0,T\}$ for $T = (0,0) \in E(K)$ and $E'[\widehat{\phi}] = \{0,T'\}$ for $T' = (0,0) \in E'(K)$.

\begin{prop}\label{prop16.2}
    There is a group homomorphism 
    \begin{align*}
        E'(K) &\to K^\times/(K^\times)^2 \\
        (x,y) &\mapsto \begin{cases}
            x \text{ mod }(K^\times)^2 &\text{ if }x \neq 0\\
            b' \text{ mod }(K^\times)^2 &\text{ if }x = 0.
        \end{cases}
    \end{align*}    
    with kernel $\phi(E(K))$.
\end{prop}
\begin{proof}
    Either apply Theorem \ref{theorem16.1} with $f = x \in K(E')$ and $g = \frac{y}{x} \in K(E)$, or we can prove this via direct calculation (see Ex. Sheet 4).
\end{proof}
This gives injective group homomorphisms
\begin{align*}
    \alpha_E = E(K)/\widehat{\phi}E'(K) \hookrightarrow K^\times/(K^\times)^2\\
    \alpha_{E'} = E'(K)/\phi E(K) \hookrightarrow K^\times/(K^\times)^2.
\end{align*}
\begin{lemma}\label{lemma16.3}
    We have \[
    2^{\text{rank}E(K)} = \frac{\left|\text{Im }\alpha_E\right|\left|\text{Im }\alpha_{E'}\right|}{4}.
    \]
\end{lemma}
\begin{proof}
    If $A \stackrel{f}{\to} B \stackrel{g}{\to} C$ are homomorphisms of abelian groups, then there is an exact sequence 
    \begin{align*}
        0 \to \text{ker}(f) \to \text{ker}(gf) \stackrel{f}{\to} \text{ker}(g) \to \text{coker}(f) \stackrel{g}{\to} \text{coker}(gf) \to \text{coker}(g) \to 0.
    \end{align*}
    Since $\widehat{\phi}\phi = [2]_E$, we get an exact sequence
    \begin{align*}
        0 \to \underbrace{E(K)[\phi]}_{\cong \mathbb{Z}/2\mathbb{Z}} \to E(K)[2] \stackrel{\phi}{\to} \underbrace{E'(K)[\widehat{\phi}]}_{\cong \mathbb{Z}/2\mathbb{Z}} \to \underbrace{E'(K)/\phi E(K)}_{\text{Im }\alpha_{E'}} \stackrel{\widehat{\phi}}{\to} E(K)/2E(K) \to \underbrace{E(K)/\widehat{\phi}E'(K)}_{\text{Im }\alpha_E} \to 0.
    \end{align*}
    Consequently, $\frac{\left|E(K)/2E(K)\right|}{\left|E(K)[2]\right|} \stackrel{(\dagger)}{=}  \frac{\left|\text{Im }\alpha_E\right|\left|\text{Im }\alpha_{E'}\right|}{4}$. By Mordell-Weil, write $E(K) \cong \Delta \times \mathbb{Z}^r$ for $\Delta$ a finite group and $r = \text{rank}E(K)$. We have \begin{align*}
        E(K)/2E(K) &\cong \Delta/2\Delta \times (\mathbb{Z}/2\mathbb{Z})^r \\
        E(K)[2] &\cong \Delta[2]
    \end{align*}
    and $\Delta/2\Delta, \Delta[2]$ have the same order as $\Delta$ is finite. Hence $\frac{\left|E(K)/2E(K)\right|}{\left|E(K)[2]\right|} = 2^r$, which with $(\dagger)$ implies the claim.
\end{proof}
\begin{lemma}\label{lemma16.4}
    If $K$ is a number field and $a, b \in \mathcal{O}_K$, then $\text{Im}(\alpha_E)\subset K(S,2)$, where $S$ is the set of primes dividing $b$.
\end{lemma}
\begin{proof}
    We want to show that for $x,y \in K$, if $y^2=x(x^2+ax+b)$ and $v_{\mathfrak{p}}(b) = 0$, then $v_{\mathfrak{p}}(x)$ is even.
    \vspace{1mm}
     
    If $v_{\mathfrak{p}}(x)<0$, then Lemma \ref{lemma9.1} implies $\begin{cases}
        v_{\mathfrak{p}}(x) = -2r\\
        v_{\mathfrak{p}}(y) = -3r
    \end{cases}$ so we're done.
    \vspace{1mm}
     
    If $v_{\mathfrak{p}}(x)>0$, then $v_{\mathfrak{p}}(x^2+ax+b)>0$, so $v_{\mathfrak{p}}(x) = v_{\mathfrak{p}}(y^2)=2v_{\mathfrak{p}}(y)$ and we're done (and of course $v_{\mathfrak{p}}(x)=0$ is clear).
\end{proof}
\begin{lemma}\label{lemma16.5}
    If $b_1b_2=b$, then $b_1(K^\times)^2 \in \text{Im}(\alpha_E) \iff w^2 = b_1u^4+au^2v^2+b_2v^4$ is soluble for $u,v,w \in K$ not all zero.
\end{lemma}
\begin{proof}
    If $b_1 \in (K^\times)^2$ or $b_2 \in (K^\times)^2$, then both conditions are satisfied. Hence now assume that $b_1,b_2 \not\in (K^\times)^2$, i.e. they are not squares in $K$.
    \vspace{1mm}
     
    Now $b_1(K^\times)^2 \in \text{Im}(\alpha_E) \iff \exists (x,y) \in E(K)$ such that $x = b_1t^2$ for some $t \in K^\times$, so $ y^2 = b_1t^2((b_1t^2)^2+ab_1t^2+b)$ and so $\left(\frac{y}{b_1t}\right)^2 = b_1t^4 + at^2 + \underbrace{\frac{b}{b_1}}_{b_2}$, so our equation has a solution $(u,v,w) = \left(t,1,\frac{y}{b_1t}\right)$.
    \vspace{1mm}
     
    Conversely, if $(u,v,w)$ is a solution to our equation, then $uv \neq 0$ and hence $\left(b_1\left(\frac{u}{v}\right)^2, b_1 \frac{uw}{v^3}\right) \in E(K)$.
\end{proof}
Not take $K =\mathbb{Q}$.
\begin{example}
    $E: y^2=x^3-x$, so $a=0,b=-1$. Then $\text{Im}(\alpha_E) = \langle -1 \rangle \subset \mathbb{Q}^\times/(\mathbb{Q}^\times)^2$. We have $E' : y^2=x^3+4x$ and $\text{Im}(\alpha_{E'}) \subset \langle -1,2 \rangle \subset \mathbb{Q}^\times/(\mathbb{Q}^\times)^2$.
    \vspace{1mm}
     
    Applying the previous lemma (Lemma \ref{lemma16.5}) to $b_1=1, b_1 = 2, b_1=-2$ gives the equations $w^2 = -u^4 - 4 v^4$, $w^2 = 2u^4 + 2v^4, w^2  = -2u^4 - 2v^4$. The first and third have no nontrivial solutions over $\mathbb{R}$, while for the second we can take $(u,v,w)=(1,1,2)$. Hence $\text{Im}(\alpha_{E'}) = \langle 2 \rangle \subset \mathbb{Q}^\times/(\mathbb{Q}^\times)^2$ and so $2^{\text{rank}E(\mathbb{Q})}=\frac{2\cdot 2}{4}=1$, so $\text{rank}E(\mathbb{Q})=0$ and hence 1 is not a congruent number.
\end{example}
\begin{example}
    Take $E: y^2=x^3+px$ for $p$ a prime that is 5 modulo 8. Taking $b_1=-1$ gives $w^2 = -u^4-pv^4$, which has no nontrivial solutions over $\mathbb{R}$, so $\text{Im}(\alpha_{E}) = \langle p \rangle \subset \mathbb{Q}^\times/(\mathbb{Q}^\times)^2$.
    \vspace{1mm}
     
    Note that $\alpha_{E'}(T') = (-4p)(\mathbb{Q}^\times)^2 = (-p)(\mathbb{Q}^\times)^2$, so we check $b_1=2,b_1=-2,b_1=p$ which give $w^2 = 2u^4-2pv^4, w^2 = -2u^4 + 2pv^4, w^2=pu^4-4v^4$. 
    \vspace{1mm}
     
    Suppose the first of these is soluble and WLOG take $u,v,w \in \mathbb{Z}$ with $\gcd(u,v)=1$. If $p \mid u$, then $p \mid w$, so $p \mid v$, contradiction, so $w^2 \equiv 2u^4 \not\equiv 0 \pmod{p}$, so $\left(\frac{2}{p}\right) = +1$, a contradiction as $p \equiv 5 \pmod{8}$.
    \vspace{1mm}
     
    Likewise the second equation has no solutions since $\left(\frac{-2}{p}\right) = -1$. Hence $\text{Im}(\alpha_{E'}) \subset \langle -1,p \rangle \subset \mathbb{Q}^\times/(\mathbb{Q}^\times)^2$ and so 
    \[
    \text{rank}E(\mathbb{Q}) = \begin{cases}
        0 &\text{if } w^2=pu^4-4v^4 \text{ has no nontrivial solutions over }\mathbb{Q},\\
        1 &\text{if } w^2=pu^4-4v^4 \text{ has a nontrivial solution over }\mathbb{Q}.
    \end{cases}
    \]
\end{example}


\end{document}