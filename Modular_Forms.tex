\documentclass{article}
%build with recipe latexmk
\usepackage[utf8]{inputenc}
\usepackage[T1]{fontenc}
\usepackage{textcomp}
\usepackage{fancyhdr}
\pagestyle{fancy}

\usepackage{tcolorbox}
\tcbuselibrary{theorems}
\usepackage{babel}
\usepackage{enumerate}
\usepackage{amsmath, amssymb, amsthm}
%\usepackage{a4wide}
\usepackage{float}
\usepackage{tikz-cd}
\usepackage{tikz}
\usepackage{graphicx}
\usepackage{caption}
\usepackage{wrapfig}
\usepackage{setspace}
\setstretch{1.1}
\usepackage{color}
\usepackage{hyperref}
\hypersetup{
    colorlinks=true, %set true if you want colored links
    linktoc=all,     %set to all if you want both sections and subsections linked
    linkcolor=black,  %choose some color if you want links to stand out
}
\usepackage{stackrel}

\theoremstyle{definition}
\newtheorem{theorem}{Theorem}[section]
\newtheorem{lemma}[theorem]{Lemma}
\newtheorem{cor}[theorem]{Corollary}
\newtheorem{prop}[theorem]{Proposition}
\newtheorem{example}{Example}[section]
\newtheorem{defn}{Definition}[section]

\title{Part III - Modular Forms
    \\ \large
    Lectured by Jack Thorne
}
 
\author{Artur Avameri}
\date{}
 
\setcounter{section}{0}
 
\begin{document}
\maketitle
\tableofcontents
\newpage

\section{Introduction}

\marginpar{06 Oct 2022, Lecture 1}

\begin{defn}
    We define the following groups:
    \begin{align*}
        &\mathfrak{h} = \{\tau \in \mathbb{C} \mid \text{Im}(\tau)>0\}\\
        &GL_2(\mathbb{R})^{+} = \{g \in GL_2(\mathbb{R}) \mid \det(g)>0\}\\
        &\Gamma(1) = SL_2(\mathbb{Z}) = \{g \in M_2(\mathbb{Z}) \mid  \det(g)=1\} .
    \end{align*}
    Note that $\Gamma(1)$ is a subgroup of $GL_2(\mathbb{R})^+$.
\end{defn}
\begin{lemma}
    $GL_2(\mathbb{R})^+$ acts transitively on $\mathfrak{h}$ by Möbius transformations.
\end{lemma}
\begin{proof}
    Let $g = \begin{pmatrix} a & b\\c&d \end{pmatrix} \in GL_2(\mathbb{R})^+, \tau \in \mathfrak{h}$. Then \[
    \text{Im}(g \tau) = \frac{1}{2i}\left(\frac{a \tau + b}{c \tau + d} - \frac{a \overline{\tau}+ b}{c \overline{\tau } + d} \right) = \frac{1}{2i} \frac{(ad-bc)(\tau - \overline{\tau})}{|c \tau + d|^2} = \frac{\det(g) \text{Im}(\tau)}{|c \tau + d |^2} > 0,
    \]
    so $g \tau \in \mathfrak{h}$. This action is transitive since \[
    x + iy  \in \mathfrak{h} \implies \begin{pmatrix} y & x \\ 0 & 1 \end{pmatrix} i  = x + iy,
    \]
    so everything in $\mathfrak{h}$ is conjugate to $i$.   
\end{proof}
\begin{defn}
    If $g = \begin{pmatrix} a & b\\ c & d\end{pmatrix}\in GL_2(\mathbb{R})^+$ and $\tau \in \mathfrak{h}$, then define \[
    j(g, \tau) =  c \tau + d.
    \]
    This is called a \textbf{modular cocycle}.
    If $k \in \mathbb{Z}$ and $f : \mathfrak{h} \to \mathbb{C}$, then \[
    f |_k[g]: \mathfrak{h} \to \mathbb{C}
    \]
    is defined by \[
    f |_k[g](\tau) = \det(g)^{k-1} f(g \tau) j(g, \tau)^{-k}.
    \]
    This is the \textbf{weight $k$ action of $g$ on $f$}.
\end{defn}
\begin{lemma}
    This is a right action of $GL_2(\mathbb{R})^+$: if $g,h \in GL_2(\mathbb{R})^+$, then $$f|_k[gh] = (f|_k[g])|_k[h].$$
\end{lemma}
\begin{proof}
    We compute
    \begin{align*}
        &(f|_k[g])|_k[h](\tau) = \det(h)^{k-1}f|_k[g](h \tau) j(h, \tau)^{-k} = \\
        &\det(h)^{k-1}\det(g)^{k-1}f(g h \tau) j(g, h \tau)^{-k} j(h, \tau)^{-k}  \stackrel{?}{=}\\
        & \det(gh)^{k-1} f(gh \tau) j(gh, \tau)^{-k} = f|_k[gh](\tau). 
    \end{align*}
    Hence we need to check that $j(gh, \tau) = j(gh, \tau)j(h, \tau)$. Note that if $g = \begin{pmatrix} a & b \\ c & d \end{pmatrix}$, then \[
    g \begin{pmatrix} \tau \\ 1 \end{pmatrix} = \begin{pmatrix} a \tau + b\\ c \tau + d \end{pmatrix} = j(g,\tau)\begin{pmatrix} g \tau \\ 1 \end{pmatrix}.
    \]
    We now get\[
    j(gh, \tau) \begin{pmatrix} g h \tau \\ 1 \end{pmatrix} = gh \begin{pmatrix} \tau \\ 1 \end{pmatrix} = g \left( j(h, \tau) \begin{pmatrix} h \tau \\ 1 \end{pmatrix}\right) = j(h, \tau) j(g, h \tau) \begin{pmatrix} g h \tau \\ 1 \end{pmatrix},
    \]
    which finishes the computation and proof.
\end{proof}
\textbf{Formulae.} For $g \in GL_2(\mathbb{R})^+, \tau \in \mathfrak{h}$, we have \begin{align*}
    \text{Im}(g \tau) = \det(g) \frac{\text{Im}(\tau)}{|j(g,\tau)|^2} \text{ and } j(g,\tau) \begin{pmatrix} g \tau \\ 1 \end{pmatrix} = g \begin{pmatrix} \tau \\ 1 \end{pmatrix}.
\end{align*} 
\begin{defn}
    Let $k \in \mathbb{Z}$ and $\gamma \le \Gamma(1)$ of finite index\footnote{In other words, $\gamma$ is a (finite index) subgroup of $\Gamma(1)$.}. A \textbf{weakly modular function of weight $k$ and level $\Gamma$} is a meromorphic function $f : \mathfrak{h} \to \mathbb{C}$ which is invariant under the weight $k$ action of $\Gamma$, i.e. such that $$\forall \tau \in \mathfrak{h}, \forall \gamma \in \Gamma, f|_k(\gamma) = f.$$
\end{defn}
We will define modular forms next time: they are weakly modular functions which are holomorphic both in $\mathfrak{h}$ and at $\infty$.
\vspace{1mm}
 
It is a fact that modular forms of fixed weight and level live in finite-dimensional $\mathbb{C}$-vector spaces called $M_k(\Gamma)$. These form the main objects of study in this course.
\vspace{1mm}
 
\textbf{Motivation.} Why study modular forms?
\begin{enumerate}[(1)]
    \item They are related to the theory of elliptic functions.
    Let $E/\mathbb{C}$ be an elliptic curve and $\omega$ a holomorphic non--zero 1--form. Then there exists a unique lattice\footnote{i.e. a discrete cocompact subgroup, or an abelian subgroup which is freely generated by two elements that are linearly independent over $\mathbb{R}$.} $\Lambda \in \mathbb{C}$ and isomorphism $\phi : \mathbb{C}/\Lambda \to E$ such that $\phi^*(\omega) = dz$. Then $E$ is isomorphic to the elliptic curve $y^2 = 4x^3 - 60G_4(\Lambda)x - 140G_6(\Lambda)$ where if $k \in \mathbb{Z}$, then $G_k(\Lambda) = \sum_{\lambda \in \Lambda - \{0\}}^{} \lambda^{-k}$. This converges absolutely for $k>2$.

    If $\tau \in \mathfrak{h}$, then $\Lambda \tau = \mathbb{Z} \tau \oplus \mathbb{Z} \subset \mathbb{C}$ is a lattice and $G_k(\tau) = G_k(\Lambda_\tau)$. This is a modular form of weight $k$ and level $\Gamma(1)$, called an Eisenstein series.
    \vspace{1mm}
     
    $\mathfrak{h}/SL_2(\mathbb{Z})$ can be identified with the set of (isomorphism classes of) elliptic curves over $\mathbb{C}$.
    \item Modular forms $f$ have Fourier expansions $\sum_{n \in \mathbb{Z}}^{} a_n g^n$, $a_n \in \mathbb{C}$ and they often serve as a generating functions for arithmetically interesting sequences $a_n$.
    \vspace{1mm}
     
    For example, take $\theta(\tau) = \sum_{n \in \mathbb{Z}}^{} e^{\pi i n^2 \tau}$. If $k \in 2\mathbb{N}$, then $\theta^{k}$ is a modular form with $q$--expansion $\theta^{k} = \sum_{n \in \mathbb{Z}}^{} r_k(n) e^{\pi i n \tau}$, where $r_k(n)$ is the number of ways of writing $n$ as a sum of $k$ squares, i.e. $r_k(n) = |\{x \in \mathbb{Z}^k \mid \sum_{i=1}^{k} x_i^2 = n\}|$.
    By expressing $\theta^k$ in terms of other modular forms, we can prove formulae such as $r_4(n) = 8 \sum_{d \mid n, 4 \nmid d}^{} d$.
    \item The Riemann zeta function $\zeta(s)$ is an important object of study. Its pleasant features include:
    \begin{itemize}
        \item The Euler product $\zeta(s) = \prod_{p}^{} (1-p^{-s})^{-1}$.
        \item It has a meromorphic continuation to $\mathbb{C}$ and has a functional equation relating $\zeta(s)$ and $\zeta(1-s)$.
    \end{itemize}
    A Dirichlet series $\sum_{n\ge 1}^{} a_n n^{-s}$ which has similar properties (Euler product, meromorphic extension, some nice function equation) is called an $L$--function. Modular forms can be used to construct interesting examples of $L$--functions. In practice, we take $M_k(\Gamma)$ and decompose it under Hecke operators to get Hecke eigenforms, the nicest possible modular forms, which have the above properties.
    \item The Langlands program predicts a relation between modular forms and objects in arithmetic geometry. A special case of this is the modularity conjecture, which says that there is a bijective correspondence between elliptic curves $E/\mathbb{C}$ up to isogeny and the set of Hecke eigenforms of weight 2. This implies Fermat's last theorem. Note that this is formulated in the language of Hecke operators and $L$--functions.
\end{enumerate}
\textbf{Homework.} There is a handout on Moodle called ''Reminder on Complex Analysis''. Have a look at it before the next lecture.

\newpage

\section{Modular Forms on $\Gamma(1)$}

\marginpar{09 Oct 2022, Lecture 2}

\textbf{Reminder.} A \textbf{meromorphic} function in an open subset $U \subset \mathbb{C}$ is a closed subset $A \subset U$ and a holomorphic function $f : U \setminus A \to \mathbb{C}$ such that $\forall a \in A$, $\exists \delta > 0$ such that $D^*(a,\delta) \subset U\setminus A$ and $\exists n \ge 0$ such that $(z-a)^n f(z)$ extends to a holomorphic function in $D(a,\delta)$.
\vspace{1mm}
 
$f$ then has a Laurent expansion $\sum_{m \in \mathbb{Z}}^{} a_m (z-a)^{m}$ valid on $D^*(a,\delta)$.

\begin{lemma}
    Let $f$ be a weakly modular function of weight $k$ and level $\Gamma(1)$. Then there exists a meromorphic function $\tilde{f}$ in $D^*(0,1)$ such that $f(\tau) = \tilde{f}(e^{2\pi i \tau})$.
\end{lemma}
\begin{proof}
    $f$ is meromorphic in $\mathfrak{h}$ by assumption. Take $\gamma = \begin{pmatrix} 1 & 1 \\ 0 & 1 \end{pmatrix} \in \Gamma(1)$. Then $f|_h[\gamma](\tau) = f(\gamma \tau) = f(\tau)$, as $f$ is invariant under the weight $k$ action of $\gamma$. But also $f(\gamma \tau) = f(\tau+1)$, so $f$ is periodic. 
    \vspace{1mm}
     
    Now map a strip of $\mathfrak{h}$ of width 1 to $D^*(0,1)$ by $\tau \mapsto e^{2 \pi i \tau}$. Existence of $\tilde{f}$: Let $a \in D^*(0,1)$ and $\delta>0$ be such that $D(\alpha,\delta) \subset D^*(0,1)$. Define $\tilde{f}$ on $D(a,\delta)$ by $\tilde{f}(q) = f(\frac{1}{2\pi i} \log q)$, for any branch of $\log$ defined in $D(a,\delta)$. This is meromorphic and independent of the choice of the branch of log, as $f$ is periodic with period 1. This defines $\tilde{f}$ in $D^*(0,1)$.
    \vspace{1mm}
     
    $\tilde{f}$ is unique since $\tau \mapsto e^{2\pi i \tau}$ is surjective.
\end{proof}

If $\tilde{f}$ extends to a meromorphic function\footnote{This might not be the case if the set of poles has a limit inside the disk.} in $D(0,1)$, then $\exists \delta > 0$ such that $\tilde{f}$ has a Laurent expansion $\tilde{f}(q) = \sum_{n \in \mathbb{Z}}^{} a_n q^n$ valid in $D^*(0,\delta)$. 
\vspace{1mm}
 
In the region $\{\tau \in \mathfrak{h} \mid \text{Im}(\tau) > \frac{1}{2\pi} \log \delta\}$, we have $$f(\tau) = \sum_{n \in \mathbb{Z}}^{} a_nq^n,$$ where $q = e^{2\pi i \tau}$. This is called the \textbf{$q$--expansion} of the weakly modular function $f$.

\begin{defn}
    Let $f$ be a weakly modular function of weight $k$ and level $\Gamma(1)$. We say that $f$ is \textbf{meromorphic at $\infty$} if $\tilde{f}$ extends to a meromorphic function in $D(0,1)$. \vspace{1mm}
     
    We say $f$ is \textbf{holomorphic at $\infty$} if $\tilde{f}$ is meromorphic at $\infty$ and has a removable singularity at $q = 0$. In this case, we define $f(\infty) = \tilde{f}(0) = \lim_{\text{Im}(\tau) \to \infty}f(\tau)$
    \vspace{1mm}
     
    We say $f$ \textbf{vanishes at $\infty$} if $f$ is holomorphic at $\infty$ and $f(\infty)=0$.
\end{defn}
\begin{defn}
    A \textbf{modular function} (of weight $k$ and level $\Gamma(1)$) is a weakly modular function (of weight $k$ and level $\Gamma(1)$) which is meromorphic at $\infty$.
    \vspace{1mm}
     
    A \textbf{modular form} is a weakly modular function which is holomorphic in $\mathfrak{h}$ and holomorphic at $\infty$.
    \vspace{1mm}
     
    A \textbf{cuspidal modular form} is a modular form that vanishes at $\infty$.
\end{defn}

\textbf{Remark.} We let $M_k(\Gamma(1))$ denote the set of modular forms of weight $k$ and level $\Gamma(1)$. We write $S_k(\Gamma(1))$ for the set of cuspidal modular forms of weight $k$, level $\Gamma(1)$. Note $S_k(\Gamma(1)) \subset M_k(\Gamma(1))$. These are $\mathbb{C}$--vector spaces. If $k$ is odd, then these both only contain the zero function, since taking $\gamma = \begin{pmatrix} -1 & 0 \\ 0 & -1 \end{pmatrix} \in \Gamma(1)$ gives $f|_k[\gamma](\tau) = f(\tau)(-1)^k = f(\tau)$.
\vspace{1mm}
 
We now consider even weights only. If $k \in \mathbb{Z}$ is even, let \[
G_k(\tau) = \sum_{\lambda \in \Lambda_\tau \setminus 0}^{} \lambda^{-k} = \sum_{(m,n) \in \mathbb{Z}^2 \setminus  0}^{} (m \tau + n)^{-k},
\]
where $\Lambda_\tau = \mathbb{Z} \tau \oplus \mathbb{Z}$ for any $\tau \in \mathfrak{h}$. 
\vspace{1mm}
 
If $\gamma \in \Gamma(1)$, then formally we have $$G_k|_k[\gamma](\tau) = G_k(\gamma \tau)j(\gamma, \tau)^{-k} = \sum_{\lambda \in \Lambda_{\gamma \tau} \setminus 0}^{} \lambda^{-k} j(\gamma, \tau)^{-k},$$
but $\Lambda_{\gamma \tau} = \mathbb{Z} \frac{a \tau + b}{c \tau +d} \oplus \mathbb{Z} = (c \tau + d)^{-1}\left(\mathbb{Z}(a \tau + b) \oplus \mathbb{Z}(c \tau + d)\right) = (c \tau + d)^{-1} \Lambda_\tau$. Hence 
\begin{align*}
    G_k|_k[g](\tau) &= \sum_{\lambda \in (c \tau + d)^{-1} \Lambda_\tau \setminus 0}^{} \lambda^{-k} (c \tau +d)^{-k}\\ &= \sum_{\lambda \in (c \tau + d)^{-1}\Lambda_\tau \setminus 0} ((c \tau + d)^{-1}\lambda)^{-k}(c \tau + d)^{-k} = G_k(\tau).
\end{align*}

This is justified only when the series defining $G_k(\tau)$ converges absolutely. Hence:
\begin{prop}
    Let $k > 2$ be an even integer. Then $G_k(\tau)$ converges absolutely and defines a modular form of weight $k$ and level $\Gamma(1)$ with $G_k(\infty) = 2\zeta(k)$. $G_k$ is the \textbf{weight $k$ Eisenstein series}.
\end{prop}
We will later see that $M_2(\Gamma(1)) = 0$.
\begin{proof}
    We want to show absolute and locally uniform convergence in $\mathfrak{h}$. This will show that $G_k$ is holomorphic by complex analysis. Let $A\ge 2$ and define $\Omega_A = \{\tau \in \mathfrak{h} \mid \text{Im}(\tau) \ge \frac{1}{A}, \text{Re}(\tau) \in [-A,A]\}$. We show uniform convergence in $\Omega_A$. If $\tau \in \Omega_A, x \in \mathbb{R}$, then $|\tau+x| \ge \begin{cases}
        \frac{1}{A} & |x|\le 2A\\
        \frac{|x|}{2} & |x|\ge 2A.
    \end{cases}$ 
    Hence $$|\tau + x| \ge \sup\left(\frac{1}{A},\frac{|x|}{2A^2}\right) \ge \sup\left(\frac{1}{2A^2}, \frac{|x|}{2A^2}\right) = \frac{1}{2A^2}\sup(1,|x|).$$
    If $(m,n) \in \mathbb{Z}^2, m \neq 0$, then $$|m \tau + n| = |m| |\tau + \frac{n}{m}| \ge \frac{1}{2A^2} \sup\left(1, |\frac{n}{m}|\right) \cdot |m| = \frac{1}{2A^2}\sup\left(|m|, |n|\right).$$ This is also valid when $m=0$ by inspection. If $\tau \in \Omega_A$, then 
    \begin{align*}
        &\sum_{(m,n) \in \mathbb{Z}^2 \setminus  0}^{} |m \tau + n|^{-k} \\\le&~ \left(\frac{1}{2n^2}\right)^{-k}\sum_{(m,n) \in \mathbb{Z}^2 \setminus  0}^{} \sup\left(|m|, |n|\right)^{-k} \\=&~ (2A^2)^k \sum_{d \in \mathbb{N}}^{} d^{-k} \cdot \left|\{(m,n) \in \mathbb{Z}^2 \mid \sup\left(|m|, |n|\right) = d\}\right| \\=&~ (2A^2)^k \sum_{d \in \mathbb{N}}^{} d^{-k}8d = 8(2A^2)^k \sum_{d \in \mathbb{N}}^{} d^{1-k} \\<&~ \infty
    \end{align*}
    whenever $k-1>1$, i.e. $k>2$. This shows absolute convergence, and uniform convergence in $\Omega_A$ by the Weierstrass M-test\footnote{If we have a sequence of functions $f_n : \Omega \to \mathbb{C}$, $M_n>0$, $|f_n(x)|<M_n$ and $\sum_{}^{} M_n < \infty$, then $\sum_{}^{} f_n$ converges uniformly} Hence $G_k$ is holomorphic in $\mathfrak{h}$ and invariant under the weight $k$ action of $\Gamma(1)$. It remains to show that $G_k$ is holomorphic at $\infty$ with $G_k(\infty) = 2\zeta(k)$. For this, it suffices to check that \[
    \lim_{\text{Im}(\tau) \to \infty} G_k(\tau) = 2\zeta(k).
    \]
    This follows from uniform convergence in $\Omega_A$: we get 
    \begin{align*}
        \lim_{\text{Im}(\tau) \to \infty} G_k(\tau) = \sum_{(m,n) \in \mathbb{Z}^2 \setminus  0}^{} \lim_{\text{Im}(\tau) \to \infty} (m \tau + n)^{-k} = \sum_{n \in \mathbb{Z} \setminus  0}^{} n^{-k} = 2 \sum_{n\ge 1}^{} n^{-k} = 2\zeta(k).
    \end{align*}
\end{proof}


\end{document}