\documentclass{article}
%build with recipe latexmk
\usepackage[utf8]{inputenc}
\usepackage[T1]{fontenc}
\usepackage{textcomp}
\usepackage{fancyhdr}
\pagestyle{fancy}

\usepackage{tcolorbox}
\tcbuselibrary{theorems}
\usepackage{babel}
\usepackage{enumerate}
\usepackage{amsmath, amssymb, amsthm}
%\usepackage{a4wide}
\usepackage{float}
\usepackage{tikz-cd}
\usepackage{tikz}
\usepackage{graphicx}
\usepackage{caption}
\usepackage{wrapfig}
\usepackage{setspace}
\setstretch{1.1}
\usepackage{color}
\usepackage{hyperref}
\hypersetup{
    colorlinks=true, %set true if you want colored links
    linktoc=all,     %set to all if you want both sections and subsections linked
    linkcolor=black,  %choose some color if you want links to stand out
}

\theoremstyle{definition}
\newtheorem{theorem}{Theorem}[section]
\newtheorem{lemma}[theorem]{Lemma}
\newtheorem{cor}[theorem]{Corollary}
\newtheorem{prop}[theorem]{Proposition}
\newtheorem{example}{Example}[section]
\newtheorem{defn}{Definition}[section]

\title{Part III - Commutative Algebra
    \\ \large
    Lectured by Oren Becker
}
 
\author{Artur Avameri}
\date{}
 
\setcounter{section}{-1}
 
\begin{document}
\maketitle
\tableofcontents
\newpage

\section{Introduction}

\marginpar{05 Oct 2022, Lecture 1}

In this course, a ring $R$ will be a commutative ring with a 1. However, we start with a noncommutative expection: $$\text{End}(M) = \{f: M \to M \mid f \text{a group homomorphism}\},$$ the endomorphisms of an abelian group $(M, +)$ with the multiplication being given by composition.

\begin{defn}[Module]
    An $R$-module $M$ is an abelian group $M$ with a ring homomorphism $\rho : R \to \text{End}(M)$, given by $r \cdot  m =\rho(r)(m)$.
\end{defn}
We have that
\begin{enumerate}[(i)]
    \item $r(m_1 + m_2) = \rho(r)(m_1+m_2) = \rho(r)(m_1) + \rho(r)(m_2) = rm_1 + rm_2$ since $\rho(r)$ is a group homomorphism $M \to M$.
    \item $(r_1 +r_2)m = \rho(r_1+r_2)m = (\rho(r_1)+\rho(r_2))m =r_1m+r_2m$ since $\rho$ is a ring homomorphism.
\end{enumerate}

\begin{example}
    \begin{enumerate}[(i)]
        \item For $k$ a field, a $k$-module is a $k$-vector space.
        \item Every abelian group $M$ is a $\mathbb{Z}$-module in a unique way through $\mathbb{Z} \to \text{End}(M)$ given by $1_{\mathbb{Z}} \mapsto \text{id}$.
        \item Every ring $R$ is an $R$-module via $R \to \text{End}(R)$, $r_0 \mapsto (r \mapsto r_0r)$.
        \item $R^{\oplus \mathbb{N}}$, the direct sum, and $\mathbb{R}^\mathbb{N}$, the direct product, are $R$-modules.
    \end{enumerate}
\end{example}

\section{Chain Conditions}
\begin{defn}
    An $R$-module $M$ is called \textbf{Noetherian} if 
    \begin{enumerate}[(i)]
        \item every ascending chain of submodules $M_0 \subseteq M_1 \subseteq M_2 \subseteq \dots$ stabilises, i.e. there exists $n$ such that $M_n = M_{n+1} = \dots$.
        \item every nonempty set $\Sigma$ of submodules of $M$ has a maximal element, i.e. there exists $M_0 \in \Sigma$ such that for all $M' \in \Sigma$, $M_0 \subseteq M'$ implies $M' = M_0$.
    \end{enumerate}
\end{defn}
\begin{defn}
    $M$ is called \textbf{Artinian} if the same holds, but with descending chains and minimal elements.
\end{defn}
\begin{lemma}
    An $R$-module $M$ is Noetherian if and only if every submodule of $M$ is finitely generated.
\end{lemma}
In particular, every Noetherian module is finitely generated.
\begin{example}
    If $R = \mathbb{Z}[T_1,T_2,T_3,\ldots]$ and $M=R$ as an $R$-module, then is $M$ finitely generated? Yes, by $1_R$, as $1_R \in M \implies r \cdot 1_R \in M = R$.
    \vspace{1mm}
     
    On the other hand, take $M' = \langle T_1,T_2,T_3, \ldots \rangle$, which is not finitely generated, since any finite subset only involves finitely many variables.
\end{example}
\begin{defn}
    A ring $R$ is Noetherian (respectively Artinian) if $R$, as an $R$-module, is Noetherian (respectively Artinian).
\end{defn}
\begin{example}
    $\mathbb{Z}$ is a Noetherian, but not an Artinian module. It is Noetherian since every submodule is finitely generated (by one element, since it is a PID), but $\mathbb{Z} \supseteq 2\mathbb{Z} \supseteq 4\mathbb{Z} \supseteq \dots$ is a descending chain of ideals that does not stabilise. The same example works for a Noetherian non-Artinian ring.
    \vspace{1mm}
     
    For an Artinian non-Noetherian module, take $\mathbb{Z}[\frac{1}{2}]/\mathbb{Z}$, where $\mathbb{Z}[\frac{1}{2}] = \{\frac{a}{2^a} \mid a \in \mathbb{Z} \}$.
\end{example}
We will later show that a ring $R$ is Artinian $\iff$ $R$ is Noetherian and $R$ has Krull dimension zero.

\begin{defn}
    A sequence $$\ldots \to M_{i-1} \to M_i \to M_{i+1} \to \ldots$$
    of $R$-modules and $R$-module homomorphisms is \textbf{exact} if $\text{Im}(f_i)= \text{ker}(f_{i+1}) ~\forall i$.
\end{defn}
\begin{defn}
    A \textbf{short exact sequence} (SES) is an exact sequence of the form $$0 \to M' \to M \to M'' \to 0.$$
    Hence $M'' \equiv M/i(M')$.
    todo later: import stackrel, the second arrow is injective and denoted by $i$ and the third is surjective.
\end{defn}
\begin{lemma}
    Let $0 \to N \to M \to L \to 0$ be a short exact sequence of $R$-modules. Then $M$ is Noetherian (Artinian) $\iff$ $N$ and $L$ are Noetherian (Artinian).
    \vspace{1mm}
     
    In other words, a module is Noetherian if and only if both the submodule and the quotient are Noetherian.
\end{lemma}
\begin{cor}
    If $M_1,\ldots,M_n$ are Noetherian (Artinian) modules, then so is $M_1 \oplus \ldots \oplus M_n$.
\end{cor}
\textbf{Reminder.} A module homomorphism from $M_1 \oplus \ldots \oplus M_n \to L$ (for all the $M_i$ $R$-modules) is just a collection of homomorphisms $\phi_i: M_i \to L$. 
\begin{prop}
    For a Noetherian (Artinian) ring $R$, every finitely generated $R$-module is Noetherian (Artinian).
\end{prop}
\begin{proof}
    $M$ is finitely generated $\iff$ there exists $n\ge 1$ and a surjective map $\phi : R^n \to M$.
\end{proof}
So $R^n$ is Noetherian and being Noetherian passes to quotients.

\begin{defn}[Algebra]
    An \textbf{$R$-algebra} is a ring $A$ together with a fixed ring homomorphism $\rho: R \to A$. We write $ra$ for $\rho(r)a$. Hence $\rho(r) = \rho(r) \cdot 1_A = r \cdot 1_A$, so we don't need to explicitly define $\rho$.
\end{defn}
\begin{example}
    A polynomial algebra $k \to k[T_1,\ldots,T_n]$. As a module, this is infinite dimensional. As a $k$-algebra, it is generated by $T_1,\ldots,T_n$.
\end{example}
\begin{defn}
    An $R$-algebra is \textbf{finitely generated} if and only if there exists $n \ge 0$ and a surjective $R$-algebra homomorphism $\phi: R[T_1,\ldots,T_n] \to A$. 
\end{defn}
\begin{defn}
    $\phi: A \to B$ is an $R$-algebra \textbf{homomorphism} if and only if $\phi$ is a ring homomorphism and $\phi(r \cdot 1_A) = r \cdot 1_B$ for all $r \in R$.
\end{defn}

\end{document}
 