\documentclass{article}
%build with recipe latexmk
\usepackage[utf8]{inputenc}
\usepackage[T1]{fontenc}
\usepackage{textcomp}
\usepackage{fancyhdr}
\pagestyle{fancy}

\usepackage{tcolorbox}
\tcbuselibrary{theorems}
\usepackage{babel}
\usepackage{enumerate}
\usepackage{amsmath, amssymb, amsthm}
%\usepackage{a4wide}
\usepackage{float}
\usepackage{tikz-cd}
\usepackage{tikz}
\usepackage{graphicx}
\usepackage{caption}
\usepackage{wrapfig}
\usepackage{setspace}
\setstretch{1.1}
\usepackage{color}
\usepackage{hyperref}
\hypersetup{
    colorlinks=true, %set true if you want colored links
    linktoc=all,     %set to all if you want both sections and subsections linked
    linkcolor=black,  %choose some color if you want links to stand out
}

\theoremstyle{definition}
\newtheorem{theorem}{Theorem}[section]
\newtheorem{lemma}[theorem]{Lemma}
\newtheorem{cor}[theorem]{Corollary}
\newtheorem{prop}[theorem]{Proposition}
\newtheorem{example}{Example}[section]
\newtheorem{defn}{Definition}[section]

\title{Part III - Algebraic Geometry
    \\ \large
    Lectured by Dhruv Ranganathan 
}
 
\author{Artur Avameri}
\date{}
 
\setcounter{section}{-1}
 
\begin{document}
\maketitle
\tableofcontents
\newpage
 
\section{Introduction}

The course consists of four parts.
\begin{enumerate}[(1)]
    \item Basics of sheaves on topological spaces.
    \item Definition of schemes and morphisms.
    \item Properties of schemes (e.g. the algebraic geometry notion of compactness and other properties).
    \item A rapid introduction to the cohomology of schemes.
\end{enumerate}

The main reference for the course is Hartshorne's \textit{Algebraic Geometry}.

\section{Beyond algebraic varieties}
\subsection{Summary of classical algebraic geometry}
We let $k = \overline{k}$ be a algebraically closed field and consider $\mathbb{A}_k^n = \mathbb{A}^n = k^n$ as a set.

\begin{defn}
    An \textbf{affine variety} is a subset $V \subset \mathbb{A}^n$ of the form $\mathbb{V}(S)$ with $S \subset k[x_1,\ldots,x_n]$, where $\mathbb{V}$ is the common vanishing locus.
\end{defn}
Note that $\mathbb{V}(S) = \mathbb{V}(I(S))$ (the ideal generated by $S$). By Hilbert Basis Theorem (since $k[x_1,\ldots,x_n]$ is noetherian), $\mathbb{V}(I(S)) = \mathbb{V}(S')$ for some finite set $S \subset k[x_1,\ldots,x_n]$.
\vspace{1mm}
 
In fact, $\mathbb{V}(I) = \mathbb{V}(\sqrt{I})$, where \[
\sqrt{I} = \{ f \in k[x_1,\ldots,x_n] \mid f^m \in I \text{ for some } m\ge 0\}
\] 
is the \textbf{radical} of $I$.
For example, in $k[x]$, if $I = (x^2)$, then $\sqrt{I} = (x)$.

\begin{defn}
    Given varieties $V \subset \mathbb{A}^n$ and $W \subset  \mathbb{A}^m$, a \textbf{morphism} is a (set-theoretic) map $\phi : V \to W \subset \mathbb{A}_k^m$ such that if $\phi = (f_1,\ldots,f_m)$, then each $f_i$ is the restriction of a polynomial in $\{x_1,\ldots,x_n\}$.
    \vspace{1mm}
     
    An \textbf{isomorphism} is a morphism with a two--sided inverse.  
\end{defn}

Our basic correspondence is 
\begin{align*}
    \{\text{Affine varieties over }k\}&/\text{up to isomorphism}\\  &\leftrightarrow \\ \{\text{finitely generated }k\text{--algebras }A &\text{ without nilpotent elements}\}
\end{align*}
A finitely generated $k$--algebra is just a quotient of a polynomial ring in finitely many variables. A nilpotent element is such that some power of it is zero. For example, in $k[x]/(x^2)$, the element $x$ is nilpotent.
\vspace{1mm}
 
How does this correspondence work? Given a variety $V$ (representing an isomorphism class), we write $V = \mathbb{V}(I)$ for $I \subset k[x_1,\ldots,x_n]$ a radical ideal\footnote{A radical ideal is an ideal equal to its radical.}, and map $V \mapsto k[x_1,\ldots,x_n]/I$.
\vspace{1mm}
 
For the reverse, if $A$ is a finitely generated nilpotent free algebra, then $A \cong k[y_1,\ldots,y_m]/J$ where we can choose $J$ to be radical (exercise: why?).
\vspace{1mm}
 
We have to check that this is independent of our choice on both sides (exercise: think through this, it should be clear).

\begin{defn}
    The algebra associated to $V$ is classically denoted $k[V]$ and called the \textbf{coordinate ring of} $V$.
\end{defn}

We have the compatibility of morphisms with our basic correspondence: there is a bijection between \[
\text{Morphisms}(V, W) \leftrightarrow \text{Ring homomorphisms}_k(k[W], k[V])
\]
(here $\text{RingHom}_k$ means that our homomorphisms preserve $k$).

We can now make our set into a topological space:
\begin{defn}
    Let $V = \mathbb{V}(I) \subset \mathbb{A}^n$ be a variety with coordinate ring $k[V]$. The \textbf{Zariski topology} on $V$ is defined such that the closed sets are $\mathbb{V}(S)$, where $S \subset k[V]$.
\end{defn}
If $V \cong W$, then the Zariski topological spaces are homeomorphic as varieties (exercise).

\begin{theorem}[Nullstellensatz]
    Fix $V$ a variety and let $k[V]$ be its coordinate ring. Given $p \in V$, we can produce a homomorphism $\text{ev}_p : k[V] \to k$ by sending $f \mapsto f(p)$. Note that $\text{ev}_p$ is surjective (since we have constant functions), hence $\text{ker}(\text{ev}_p) = m_p$ is a maximal ideal, giving us a map \[
        \{\text{points of }V\} \rightarrow \{\text{maximal ideals in }k[V]\}.
    \]
    Nullstellensatz says that this is actually a bijection. For the converse map, given $m \subset k[V]$, we get a quotient $k[V] \to k[V]/m = k$ (Nullstellensatz says this extension is finite, hence must be $k$). So using/choosing a representation for $V$ in $k[x_1,\ldots,x_n]$ gives a surjective homomorphism onto $k$ and specifies a bunch of points.
\end{theorem}
\subsection{Limitations of classical algebraic geometry}
\textbf{Question.} What is an abstract variety, i.e. ''some ''space'' $X$ such that locally as a cover $\{U_i\}$, each $U_i$ is an affine variety, compatible with overlaps''. 

\begin{example}[non--algebraically closed fields]
    Take $I = (x^2+y^2+1) \subset \mathbb{R}[x,y]$. Then $\mathbb{V}(I) = \varnothing \subset \mathbb{R}^2$, but $I$ is prime, so radical, so nullstellensatz fails.
\end{example}

\textbf{Question.} On what topological space is $\mathbb{R}[x,y]/(x^2+y^2+1)$ ''naturally'' the set of functions? (or $\mathbb{Z}$, or $\mathbb{Z}[x]$).

\begin{example}[Why restrict to radical ideals?]
    Take $C = \mathbb{V}(y-x^2) \subset \mathbb{A}_k^2$ and $D = \mathbb{V}(x,y)$, so $C \cap D = \mathbb{V}(y, y-x^2) = \mathbb{V}(x,y) = \{(0,0)\}$. This is a single point, but if $D_\delta = \mathbb{V}(y+\delta)$ for some $\delta \in k$, then $C \cap D_\delta = \{\pm \sqrt{\delta}\}$, which is 2 points for all $\delta \neq 0$. In other words, intersections of varieties don't want to be varieties.
\end{example}

\end{document}