\documentclass{article}
%build with recipe latexmk
\usepackage[utf8]{inputenc}
\usepackage[T1]{fontenc}
\usepackage{textcomp}
\usepackage{fancyhdr}
\pagestyle{fancy}

\usepackage{tcolorbox}
\tcbuselibrary{theorems}
\usepackage{babel}
\usepackage{enumerate}
\usepackage{stmaryrd}
\usepackage{amsmath, amssymb, amsthm}
%\usepackage{a4wide}
\usepackage{float}
\usepackage{tikz-cd}
\usepackage{tikz}
\usepackage{graphicx}
\usepackage{caption}
\usepackage{wrapfig}
\usepackage{setspace}
\setstretch{1.1}
\usepackage{color}
\usepackage{hyperref}
\hypersetup{
    colorlinks=true, %set true if you want colored links
    linktoc=all,     %set to all if you want both sections and subsections linked
    linkcolor=black,  %choose some color if you want links to stand out
}

\theoremstyle{definition}
\newtheorem{theorem}{Theorem}[section]
\newtheorem{lemma}[theorem]{Lemma}
\newtheorem{cor}[theorem]{Corollary}
\newtheorem{prop}[theorem]{Proposition}
\newtheorem{example}{Example}[section]
\newtheorem{defn}{Definition}[section]

\title{Part III - Algebraic Geometry
    \\ \large
    Lectured by Dhruv Ranganathan 
}
 
\author{Artur Avameri}
\date{}
 
\setcounter{section}{-1}
 
\begin{document}
\maketitle
\tableofcontents
\newpage
 
\section{Introduction}

\marginpar{6 Oct 2022, Lecture 1}


The course consists of four parts.
\begin{enumerate}[(1)]
    \item Basics of sheaves on topological spaces.
    \item Definition of schemes and morphisms.
    \item Properties of schemes (e.g. the algebraic geometry notion of compactness and other properties).
    \item A rapid introduction to the cohomology of schemes.
\end{enumerate}

The main reference for the course is Hartshorne's \textit{Algebraic Geometry}.

\section{Beyond algebraic varieties}
\marginpar{08 Oct 2022, Lecture 2}

\subsection{Summary of classical algebraic geometry}
We let $k = \overline{k}$ be a algebraically closed field and consider $\mathbb{A}_k^n = \mathbb{A}^n = k^n$ as a set.

\begin{defn}
    An \textbf{affine variety} is a subset $V \subset \mathbb{A}^n$ of the form $\mathbb{V}(S)$ with $S \subset k[x_1,\ldots,x_n]$, where $\mathbb{V}$ is the common vanishing locus.
\end{defn}
Note that $\mathbb{V}(S) = \mathbb{V}(I(S))$ (the ideal generated by $S$). By Hilbert Basis Theorem (since $k[x_1,\ldots,x_n]$ is noetherian), $\mathbb{V}(I(S)) = \mathbb{V}(S')$ for some finite set $S \subset k[x_1,\ldots,x_n]$.
\vspace{1mm}
 
In fact, $\mathbb{V}(I) = \mathbb{V}(\sqrt{I})$, where \[
\sqrt{I} = \{ f \in k[x_1,\ldots,x_n] \mid f^m \in I \text{ for some } m\ge 0\}
\] 
is the \textbf{radical} of $I$.
For example, in $k[x]$, if $I = (x^2)$, then $\sqrt{I} = (x)$.

\begin{defn}
    Given varieties $V \subset \mathbb{A}^n$ and $W \subset  \mathbb{A}^m$, a \textbf{morphism} is a (set-theoretic) map $\phi : V \to W \subset \mathbb{A}_k^m$ such that if $\phi = (f_1,\ldots,f_m)$, then each $f_i$ is the restriction of a polynomial in $\{x_1,\ldots,x_n\}$.
    \vspace{1mm}
     
    An \textbf{isomorphism} is a morphism with a two--sided inverse.  
\end{defn}

Our basic correspondence is 
\begin{align*}
    \{\text{Affine varieties over }k\}&/\text{up to isomorphism}\\  &\leftrightarrow \\ \{\text{finitely generated }k\text{--algebras }A &\text{ without nilpotent elements}\}
\end{align*}
A finitely generated $k$--algebra is just a quotient of a polynomial ring in finitely many variables. A nilpotent element is such that some power of it is zero. For example, in $k[x]/(x^2)$, the element $x$ is nilpotent.
\vspace{1mm}
 
How does this correspondence work? Given a variety $V$ (representing an isomorphism class), we write $V = \mathbb{V}(I)$ for $I \subset k[x_1,\ldots,x_n]$ a radical ideal\footnote{A radical ideal is an ideal equal to its radical.}, and map $V \mapsto k[x_1,\ldots,x_n]/I$.
\vspace{1mm}
 
For the reverse, if $A$ is a finitely generated nilpotent free algebra, then $A \cong k[y_1,\ldots,y_m]/J$ where we can choose $J$ to be radical (exercise: why?).
\vspace{1mm}
 
We have to check that this is independent of our choice on both sides (exercise: think through this, it should be clear).

\begin{defn}
    The algebra associated to $V$ is classically denoted $k[V]$ and called the \textbf{coordinate ring of} $V$.
\end{defn}

We have the compatibility of morphisms with our basic correspondence: there is a bijection between \[
\text{Morphisms}(V, W) \leftrightarrow \text{Ring homomorphisms}_k(k[W], k[V])
\]
(here $\text{RingHom}_k$ means that our homomorphisms preserve $k$).

We can now make our set into a topological space:
\begin{defn}
    Let $V = \mathbb{V}(I) \subset \mathbb{A}^n$ be a variety with coordinate ring $k[V]$. The \textbf{Zariski topology} on $V$ is defined such that the closed sets are $\mathbb{V}(S)$, where $S \subset k[V]$.
\end{defn}
If $V \cong W$, then the Zariski topological spaces are homeomorphic as varieties (exercise).

\begin{theorem}[Nullstellensatz]
    Fix $V$ a variety and let $k[V]$ be its coordinate ring. Given $p \in V$, we can produce a homomorphism $\text{ev}_p : k[V] \to k$ by sending $f \mapsto f(p)$. Note that $\text{ev}_p$ is surjective (since we have constant functions), hence $\text{ker}(\text{ev}_p) = m_p$ is a maximal ideal, giving us a map \[
        \{\text{points of }V\} \rightarrow \{\text{maximal ideals in }k[V]\}.
    \]
    Nullstellensatz says that this is actually a bijection. For the converse map, given $m \subset k[V]$, we get a quotient $k[V] \to k[V]/m = k$ (Nullstellensatz says this extension is finite, hence must be $k$). So using/choosing a representation for $V$ in $k[x_1,\ldots,x_n]$ gives a surjective homomorphism onto $k$ and specifies a bunch of points.
\end{theorem}
\subsection{Limitations of classical algebraic geometry}
\textbf{Question.} What is an abstract variety, i.e. ''some ''space'' $X$ such that  locally as a cover $\{U_i\}$, each $U_i$ is an affine variety, compatible with overlaps''. 

\begin{example}[non--algebraically closed fields]
    Take $I = (x^2+y^2+1) \subset \mathbb{R}[x,y]$. Then $\mathbb{V}(I) = \varnothing \subset \mathbb{R}^2$, but $I$ is prime, so radical, so nullstellensatz fails.
\end{example}

\textbf{Question.} On what topological space is $\mathbb{R}[x,y]/(x^2+y^2+1)$ ''naturally'' the set of functions? (or $\mathbb{Z}$, or $\mathbb{Z}[x]$).

\begin{example}[Why restrict to radical ideals?]
    Take $C = \mathbb{V}(y-x^2) \subset \mathbb{A}_k^2$ and $D = \mathbb{V}(x,y)$, so $C \cap D = \mathbb{V}(y, y-x^2) = \mathbb{V}(x,y) = \{(0,0)\}$. This is a single point, but if $D_\delta = \mathbb{V}(y+\delta)$ for some $\delta \in k$, then $C \cap D_\delta = \{\pm \sqrt{\delta}\}$, which is 2 points for all $\delta \neq 0$. In other words, intersections of varieties don't want to be varieties.
\end{example}

\subsection{The spectrum of a ring}

\marginpar{11 Oct 2022, Lecture 3}

Let $A$ be a commutative ring with identity. We will define a topological space on which $A$ is the ring of functions.

\begin{defn}
    The \textbf{Zariski spectrum} of $A$ is $$\text{Spec }A = \{\mathfrak{p} \subset A \mid \mathfrak{p} \text{ is a prime ideal}\}.$$
\end{defn}
A ring homomorphism $\phi : A \to B$ induces a map $\phi^{-1} : \text{Spec }B \to \text{Spec }A$ by $q \mapsto \phi^{-1}(q)$. In general, the preimage of a prime ideal is a prime ideal.
\vspace{1mm}
 
\textbf{Warning.} This would fail if we only considered maximal ideals, since the preimage of a maximal ideal need not be maximal. 
\vspace{1mm}
 
Given $f \in A$ and $\mathfrak{p} \in \text{Spec}(A)$, we have an induced $\overline{f} \in A/\mathfrak{p}$ obtained via a quotient. Informally, we can evaluate any $f \in A$ at points $\mathfrak{p} \in \text{Spec}(A)$ with the caveat that the codomain of this evaluation depends on $\mathfrak{p}$.

\begin{example}
    Take $A = \mathbb{Z}$. Then $\text{Spec }A  = \text{Spec }(\mathbb{Z}) = \{p \mid p \text{ is prime}\} \cup \{(0)\}$. Let's pick an element in $\mathbb{Z}$, say $132 \in \mathbb{Z}$. Given a prime $p$, we can look at ${132\pmod{p}} \in \mathbb{Z}/p$. The takeaway here is that 
    \begin{align*}
        \text{Spec }\mathbb{Z} &\rightarrow \text{Space}\\
        132 \in \mathbb{Z} &\rightarrow \text{a function}\\
        132 \text{ (mod }p) &\rightarrow \text{value of that function at }p.
    \end{align*}
    Note that based on the value of $p$, our codomain changes from point to point.
\end{example}
\begin{example}
    Take $A = \mathbb{R}[x]$, then $\text{Spec }\mathbb{R}[x] = \mathbb{C}/ \text{complex conjugation} \cup \{(0)\}$. 
\end{example}
\textbf{Exercise.} Draw $\text{Spec }\mathbb{Z}[x]$ and $\text{Spec }k[x]$ for $k$ any field (i.e. describe all prime ideals and their containment). This is on example sheet 1.
\begin{example}
    If $A = \mathbb{C}[x]$, then $\text{Spec }A = \mathbb{C} \cup\{(0)\}$, where given $a \in \mathbb{C}$, we send it to the maximal ideal $\langle z-a\rangle$.
\end{example}
\subsection{A topology on Spec $A$}
Fix $f \in A$. Then $\mathbb{V}(f) = \{\mathfrak{p} \in \text{Spec }A \mid f \equiv 0 \pmod{\mathfrak{p}}\} \subset \text{Spec }A$. (Note that $f \equiv 0 \pmod{\mathfrak{p}}$ is the same as $f  \in \mathfrak{p}$). \vspace{1mm}
 
Similarly for $J \subset A$ an ideal, $\mathbb{V}(J) = \{\mathfrak{p} \in \text{Spec }A \mid f \in \mathfrak{p} ~\forall f \in J\}$.

\begin{prop}
    The sets $\mathbb{V}(J) \subset \text{Spec }A$ ranging over all ideals $J$ form the closed sets of a topology on $\text{Spec }A$. This topology is called the \textbf{Zariski topology}.
\end{prop}
\begin{proof}
    Easy fact: $\varnothing$ and $\text{Spec }A$ are closed, since we have functions $1$ (vanishing nowhere) and $0$ (vanishing everywhere). Since $\mathbb{V}(\sum_{\alpha}^{} I_\alpha) = \bigcap_\alpha \mathbb{V}(I_\alpha)$ (this is because $I_1 + I_2$ is the smallest ideal containing $I_1 \cup I_2$), arbitrary intersections are closed.
    \vspace{1mm}
     
    Finally, we claim $\mathbb{V}(I_1) \cup \mathbb{V}(I_2) = \mathbb{V}(I_1 \cap I_2)$. The containment $\subset $ is clear: if a prime ideal contains $I_1$ or $I_2$, it contains $I_1 \cap I_2$. Conversely, $I_1I_2 \subset I_1 \cap I_2$, so if $I_1 I_2 \subset I_1 \cap I_2 \subset \mathfrak{p}$, then by primality $I_1 \subset \mathfrak{p}$ or $I_2 \subset \mathfrak{p}$. 
\end{proof}

\begin{example}
    Let $k = \mathbb{C}$ and consider $\text{Spec }\mathbb{C}[x,y]$. We make a few observations:
    \begin{itemize}
        \item The point $(0) \in \text{Spec }\mathbb{C}[x,y]$ is dense in the Zariski topology, i.e. $\overline{\{(0)\}} = \text{Spec }\mathbb{C}[x,y]$ because every prime ideal contains $(0)$ (because we are in an integral domain).
        \item Consider the prime ideal $(y^2-x^3)$ (which is prime since the quotient is an integral domain). Consider a maximal ideal $\mathfrak{m}_{a,b} = (x-a, y-b)$. We can ask: when is $\mathfrak{m}_{a,b} \in \overline{\{(y^2-x^3)\}}$? The answer: if and only if $b^2 = a^3$, e.g. $(1,1)$ (see example sheet 1). The lesson here is that points are not closed in the Zariski topology.
    \end{itemize}
\end{example}
\subsection{Functions on opens}
\begin{defn}
    Let $f \in A$. Define the \textbf{distinguished open} corresponding to $f$ to be \[
    \mathcal{U}_{f} = (\text{Spec}(A))/\mathbb{V}(f).
    \]
\end{defn}
\begin{example}
    \begin{itemize}
        \item Let $A = \mathbb{C}[x]$, so $\text{Spec }A = \mathbb{C} \cup \{(0)\}$ (with the Zariski topology). Take $f = x$ and consider $\mathcal{U}_x$. Recall the bijection $\text{Spec }\mathbb{C} \leftrightarrow \mathbb{C} \cup \{(0)\}$ by $(x-a) \mapsfrom a \in \mathbb{C}$ and $(0) \mapsfrom (0)$. Then $\mathbb{V}(x) = \{\mathfrak{p} \in \text{Spec }A \mid x \in \mathfrak{p}\} = \{(x)\}$, so $\mathcal{U}_f = \text{Spec }A \setminus \{(x)\}$.
        \item More generally, suppose we fix $a_1,\ldots,a_r \in \mathbb{C}$, then $\text{Spec }A\setminus \{(x-a_i)\}_{i=1}^r = \mathcal{U}$ and $\mathcal{U} = \mathcal{U}_f$, where $f = \prod_{i=1}^{r} (x-a_i)$.
    \end{itemize}
\end{example}
\begin{lemma}
    The distinguished opens $\mathcal{U}_f$ taken over all $f \in A$ form a basis for the Zariski topology on $\text{Spec }A$.
\end{lemma}
\begin{proof}
    Left as an exercise on example sheet 1.
\end{proof}

A bit of commutative algebra: 
\begin{defn}
    Given $f \in A$, the \textbf{localization of $A$ at $f$} is $A_f = A[x]/(xf-1)$, which we can informally think of as $A_f = A[\frac{1}{f}]$.
\end{defn}
\begin{lemma}
    The distinguished open $\mathcal{U}_f \subset \text{Spec }A$ is naturally homeomorphic to $\text{Spec }A_f$ via the ring homomorphism $A \stackrel{j}{\to} A_f$, which produces the inverse $j^{-1}: \text{Spec }A_f \to \text{Spec }A$.
\end{lemma}
\marginpar{13 Oct 2022, Lecture 4}
\begin{proof}
    Primes in the ring $A_f$ are in bijection with primes of $A$ that miss $f$ via $j^{-1}$. We exhibit this bijection: 
    \begin{itemize}
        \item Given $q \subset A_f$ prime, take $j^{-1}(q) \subset A$, which is prime. 
        \item Given $p \subset A$ a prime ideal, take $p_f = j(p) A_f$. We claim $p_f$ is a prime exactly when $f \not\in p$.
        \begin{itemize}
            \item If $f \in p$, then $p_f$ contains $f$, which is a unit, so $p_f = (1)$ is not prime.
            \item If $f \not\in p$, then $\left( A_f/p_f \right) \cong \left( A/p\right)_{\overline{f}}$, where $\overline{f}$ is $f + p$, a coset (exercise: check this formally). Hence $\left(A/p \right)_{\overline{f}} \subset FF \left(A/p \right)$ (FF stands for fraction field), so it is an integral domain, so $p_f$ is prime.
        \end{itemize}
    \end{itemize}
    Finally we need to check that these maps are inverses. This is left as an exercise.
\end{proof}

Facts about distinguished opens:
\begin{itemize}
    \item $U_f \cap U_g = U_{fg}$ (easy fact).
    \item $U_{f^n} = U_f$ for all $n \ge 1$ (easy fact).
    \item The rings $A_f$ and $A_{f^n}$ for $n\ge 1$ are isomorphic. Why? Since $A_f = A[x]/(xf-1)$ and $A_{f^n} = A[y]/(yf^n-1)$, the isomorphism is given by $A_f \to A_{f^n}$ by $x \mapsto f^{n-1}y$ and $A_{f^n} \to A_f$ by $y \mapsto x^n$ (check these are inverses).
    \item Containment. $U_{f} \subset U_{g} \iff f^n$ is a multiple of $g$ for some $n\ge 1$. To orient ourselves: if $f = g f'$, then $U_f \subset U_g$.
    \begin{proof}
        The $(\implies )$ direction is clear by the orientation above. Conversely, suppose $U_f \subset U_g$, so $\mathbb{V}(f) \supset \mathbb{V}(g)$. The set $\mathbb{V}(f)$ is the set of all primes containing $(f)$. We claim that $\sqrt{(f)} \subset \sqrt{(g)}$. But what is the radical of $I$? It is the intersection of all primes containing the ideal $I$.
    \end{proof}
\end{itemize}

Foreshadowing: fix $A$. We've made an assignment from distinguished opens in $\text{Spec }A$ to rings by mapping $U_f \mapsto A_f$. The association is ''functorial'', i.e. if $U_{f_1} \subset U_{f_2}$, then we can assume that $f_1^n = f_2 f_3$, so $U_{f_1} = U_{f_1^n} = U_{f_2f_3} \subset U_{f_2}$, so there is a homomorphism $A_{f_2} \to A_{f_1}$. This is the restriction map.

\vspace{1mm}
 
Question: can we extend this association to all open sets? See notes for the answer (yes).

\section{Sheaves}
\subsection{Presheaves}
Let $X$ be a topological space. 
\begin{defn}
    A \textbf{presheaf $\mathcal{F}$ on $X$ of abelian groups} is an association from the set of open sets in $X$ to abelian groups given by $U \mapsto \mathcal{F}(U)$ and for $U \subset V$ opens, a homomorphism $\text{res}_u^v : \mathcal{F}(V) \to \mathcal{F}(U)$ (a \textbf{restriction map}) such that $\text{res}_u^u = \text{id}$ and $\text{res}^v_u \circ \text{res}^w_v = \text{res}^w_u$ when $U \subset V \subset W$ are opens.
\end{defn}
\begin{example}
    For any space $X$, take $\mathcal{F}(U) = \{f : U \to \mathbb{R} \mid f \text{ continuous}\}$ with the usual restriction.
\end{example}
Similarly we can get sheaves of rings, sets, modules over a fixed ring, etc.

\begin{defn}
    A \textbf{morphism} $\phi : \mathcal{F} \to \mathcal{G}$ of presheaves on $X$ is, for each $U \subset X$ open, a homomorphism $\phi(u) : \mathcal{F}(u) \to \mathcal{G}(u)$ compatible with restriction maps, i.e. if $V \subset U$, then the following diagram commutes. 
    \[\begin{tikzcd}[row sep=large,column sep=large]
        \mathcal{F}(u) & \mathcal{G}(u) \\
        \mathcal{F}(v) & \mathcal{G}(v)
        \arrow["\phi(u)", from=1-1, to=1-2]
        \arrow["\text{res}^u_v",from=1-1, to=2-1]
        \arrow["\text{res}^u_v", from=1-2, to=2-2]
        \arrow["\phi(v)"', from=2-1, to=2-2]
    \end{tikzcd}\]
\end{defn}
\begin{defn}
    A morphism $\phi : \mathcal{F} \to \mathcal{G}$ of preshaves is injective (surjective) if $\phi(U) : \mathcal{F}(U) \to \mathcal{G}(U)$ is injective (surjective) for all $U \subset X$.
\end{defn}

\subsection{Sheaves}

\marginpar{16 Oct 2022, Lecture 5}

\begin{defn}
    A \textbf{sheaf} is a presheaf $\mathcal{F}$ such that 
    \begin{enumerate}[(1)]
        \item If $U \subset X$ is open and $\{U_i\}$ is an open cover of $U$, then for $s \in \mathcal{F}(U)$, if $s|_{U_i} = \text{res}^U_{U_i}(s) = 0$ for all $i$, then $s = 0$.
        \item If $U$ and $\{U_i\}$ are as in (1), then given $s_i \in \mathcal{F}(U_i)$ with $s_i |_{U_i \cap U_j} = s_j |_{U_i \cap U_j}$ for all $i,j$, then there exists $s \in \mathcal{F}(U)$ with $s |_{U_i} = s_i$.
    \end{enumerate}
\end{defn}
\textbf{Remark.} These axioms imply $\mathcal{F}(\varnothing) = 0$ (exercise).
\vspace{1mm}
 
A \textbf{morphism} of sheaves is a morphism of the underlying presheaves.

\begin{example}
    If $X$ is a topological space, $\mathcal{F}(U) = \{f : U \to \mathbb{R} \mid f \text{ continuous}\}$, then $f$ is a sheaf.
\end{example}
\textbf{Non--example.} Let $X = \mathbb{C}$ with the Euclidean topology and take $\mathcal{F}(U) = \{f: U \to \mathbb{C} \mid f \text{ holomorphic and bounded}\}$. Then $\mathcal{F}$ is not a sheaf, since bounded functions may glue to unbounded functions. For example, take $U = \mathbb{C}$ and $U_i = D(0,i)$. Then $f(z) = z$ is bounded on each $U_i$, but not on $U$. In general, the characterization of elements of a sheaf should be purely local, and being bounded is not a local condition.
\vspace{1mm}
 
\textbf{Non--example.} Fix a group $G$ and a set $\mathcal{F}(U)=G$ (the \textbf{constant presheaf}). If $U_1, U_2$ are disjoint, then $\mathcal{F}(U_1 \cup U_2) = G \times G$.

\begin{example}
    Give $G$ the discrete topology (every subset is open and closed) and define $$\mathcal{F}(U) = \{f : U \to G \text{ continuous}\} = \{f : U \to G \mid f \text{ is locally constant}\}.$$
    This is the \textbf{constant sheaf}.
\end{example}
\begin{example}
    If $V$ is an irreducible variety, then 
    \begin{align*}
        \mathcal{O}_V(v) = \{f \in k[V] \mid f \text{ is regular at }p ~\forall p \in U\}.
    \end{align*}
    Here regular at $p$ means that $f = \frac{g}{h}$ in a neighborhood of $p$ with $g,h$ polynomials and $h(p) \neq 0$. $\mathcal{O}_V$ is the \textbf{structure sheaf} of $V$.
    \vspace{1mm}
     
    This is a sheaf, since we have a local condition.
\end{example}

\subsection{Basic constructions}

\textbf{Terminology.} A \textbf{section} of $\mathcal{F}$ over $U$ is an element $s \in \mathcal{F}(U)$.

\vspace{1mm}
 
\textbf{Construction of stalks.} Fix $p \in X$ and $\mathcal{F}$ a presheaf on $X$. Then $\mathcal{F}_p$, the \textbf{stalk} of $\mathcal{F}$ at $p$, is defined to be $$\mathcal{F}_p = \{(U,s) \mid s \in \mathcal{F}(U), p \in U\}/\sim$$
with $(U,s) \sim (V, s')$ if $\exists W \subset U \cap V$ with $p \in W$ such that $s|_W = s'|_W$. 
\vspace{1mm}
 
The elements of $\mathcal{F}_p$ are called \textbf{germs}.

\begin{example}
    Take $\mathbb{A}^1$, the affine line, then $\mathcal{O}_{\mathbb{A}^1,0} = \left\{\frac{f(t)}{g(t)} \mid g(0)\neq 0\right\} = k[t]_{(t)} \subset k(t) $.
\end{example}
\begin{prop}
    If $f : \mathcal{F} \to \mathcal{G}$ is a morphism of sheaves on $X$ such that for all $p \in X$, the induced map $f_p :\mathcal{F}_p \to \mathcal{G}_p$ is an isomorphism, then $f$ is an isomorphism.
    \vspace{1mm}
     
    Here $f_p((U,s)) = (U, f_U(s))$, which is well--defined.
\end{prop}
\begin{proof}
    We will show $f_U : \mathcal{F}(U) \to \mathcal{G}(U)$ is an isomorphism for each $U$, and we can then define $f^{-1}$ by $(f^{-1})_U = (f_U)^{-1}$.
    \vspace{1mm}
     
    $f_U$ is injective: suppose $s \in \mathcal{F}(U)$ with $f_U(s) = 0$. Since $f_p$ is injective, $(U,s) = 0$ in $\mathcal{F}_p$ for every $p \in U$. Thus for every $p \in U$, there exists an open neighborhood $U_p$ of $p$ such that $s|_{U_p} = 0$. But $\{U_p \mid p \in U\}$ is a cover of $U$, so $s = 0$ in $\mathcal{F}(U)$ by the first condition of being a sheaf.
    \vspace{1mm}
     
    $f_U$ is surjective: take $t \in \mathcal{G}(U)$. For each $p \in U$, we have $(U_p, s_p) \in \mathcal{F}_p$ with $f_p(U_p, s_p) = (U,t) \in \mathcal{G}_p$. By shrinking $U_p$ if necessary, we can assume $f_{U_p}(s_p) = t|_{U_p}$. For points $p, p' \in U$, $$f(_{U_p \cap U_{p'}}) \left(s_p|_{U_p \cap U_{p'}} \setminus s_{p'}|_{U_p \cap U_{p'}}\right) = t|_{U_p \cap U_{p'}} - t |_{U_p \cap U_{p'}} = 0.$$
    Thus $s_p |_{U_p \cap U_{p'}} - s_{p'}|_{U_p \cap U_{p'}} = 0$ by the injectivity of $f_{U_p \cap U_{p'}}$. Thus by the second sheaf axiom, $\exists s \in \mathcal{F}(U)$ with $s|_{U_p} = s_p$. Now $f_U(s)|_{U_p} = f_{U_p}(s|_{U_p}) = f_{U_p}(s_p) = t|_{U_p}$. Thus $f_U(s) = t$ by the first sheaf axiom.
\end{proof}
We emphasize that this proof is asymmetric in the sense that we need to first prove injectivity to be able to prove surjectivity.

\marginpar{18 Oct 2022, Lecture 6}
\vspace{1mm}
 
\textbf{Exercises.}
\begin{enumerate}[(i)]
    \item There is a map $\mathcal{F}(U) \to \prod_{p \in U}^{} \mathcal{F}_p$ mapping $s \mapsto ((U,s))_{p \in U}$. The claim is that this map is injective (by sheaf axiom 1).
    \item Given two maps $\phi, \psi : \mathcal{F} \to \mathcal{G}$ with $\phi_p = \psi_p ~\forall p \in X$, we have $\phi = \psi$.
\end{enumerate} 

\begin{defn}[Sheafification]
    If $\mathcal{F}$ is a presheaf on $X$, then a morphism $\text{sh}: \mathcal{F} \to \mathcal{F}^{\text{sh}}$ to the sheaf $\mathcal{F}^{\text{sh}}$ is a \textbf{sheafification} if for any morphism of presheaves $\phi : \mathcal{F} \to \mathcal{G}$ for $\mathcal{G}$ a sheaf there is a unique commutative diagram of the following form:
    \[\begin{tikzcd}[row sep=large,column sep=large]
        \mathcal{F} & \mathcal{F}^{\text{sh}} \\
         & \mathcal{G}
        \arrow["\text{sh}", from=1-1, to=1-2]
        \arrow["\exists !", from=1-2, to=2-2]
        \arrow["\phi", from=1-1, to=2-2]
    \end{tikzcd}\]
\end{defn}
\textbf{Remark.} Since this is a definition by universal property, $\mathcal{F}^{\text{sh}}$ and the map $\mathcal{F} \to \mathcal{F}^{\text{sh}}$ are unique (up to unique isomorphism).
\vspace{1mm}
 
A morphism of presheaves $\mathcal{F} \to \mathcal{G}$ induces a morphism of sheaves $\mathcal{F}^{\text{sh}} \to \mathcal{G}^{\text{sh}}$.

\begin{prop}
    Sheafification exists.
\end{prop}
\begin{proof}
    Given a presheaf $\mathcal{F}$ on $X$, define 
    \begin{align*}
        \mathcal{F}^{\text{sh}}(U) = \{f : U \to \bigsqcup_{p \in U}^{} \mathcal{F}_p \mid& f(p) \in \mathcal{F}_p \text{ and for all } p \in U, \text{ there exists an open neighborhood }\\&V_p \subset U \text{ and }s \in \mathcal{F}(V_p) \text{ such that } (V_p, q) = f(q) \in \mathcal{F}_q ~\forall q \in V_p\}.
    \end{align*}
    This is clearly a sheaf. Verifying the universal property is left as an exercise.
\end{proof}
\begin{cor}
    The stalks of $\mathcal{F}$ and $\mathcal{F}^{\text{sh}}$ coincide.
\end{cor}
\begin{proof}
    Easy exercise from the definitions.
\end{proof}
\textbf{Exercise.} Find a nonzero presheaf $\mathcal{F}$ with $\mathcal{F}^{\text{sh}} = 0$. (Comment by Dhruv: this is rather stupid). 

\subsection{Kernels, cokernels, etc.}

Let $\phi : \mathcal{F} \to \mathcal{G}$ be a morphism of presheaves. Then we can define presheaves $\text{ker } \phi$, $\text{coker } \phi$, $\text{im } \phi$ by \begin{align*}
    &(\text{ker }\phi)(u) = \text{ker }\phi_u : \mathcal{F}(U) \to \mathcal{G}(U)\\
    &(\text{coker }\phi)(u) = \text{coker }\phi_u\\
    &(\text{im }\phi)(u) = \text{im }\phi_u.
\end{align*}
These are all presheaves.

\textbf{Exercise.} The presheaf kernel for a morphism of sheaves $\phi: \mathcal{F} \to \mathcal{G}$ is also a sheaf.
\vspace{1mm}
 
This is not true for $\text{coker }\phi$ in general!
\begin{example}
    Take $X = \mathbb{C}$ with the Euclidean topology, and let $\mathcal{O}_X$ be the sheaf of holomorphic functions on $X$ (with addition as its group operation). Let $\mathcal{O}_X^*$ be the sheaf of nowhere vanishing holomorphic functions (with multiplication as its group operation). 
    \vspace{1mm}
     
    We have a morphism of sheaves $\text{exp}: \mathcal{O}_X \to \mathcal{O}_X^*$ by $f \in \mathcal{O}_X(U) \mapsto \text{exp}(f) \in\mathcal{O}_X^*(U)$. Thus $\text{ker}(\text{exp}) = 2 \pi i \mathbb{Z}$ with $\mathbb{Z}$ the constant sheaf, but $\text{coker}(\text{exp})$ is not a sheaf: if we let $U_1 = \mathbb{C}\setminus [0,\infty)$, $U_2 = \mathbb{C}\setminus (-\infty,0]$ and $U = U_1 \cup U_2 = \mathbb{C}\setminus \{\infty\}$ and we let $f(z)=z \in \mathcal{O}_X^*(U)$, then it is not in the image of $\text{exp}:\mathcal{O}_X(U) \to \mathcal{O}_X^*(U)$ since $\log z$ is not single--valued on $U$. Thus $f$ defines a nonzero section of $(\text{coker exp})(U)$. But $f|_{U_i}$ is in the image of $\text{exp}_{U_i}$, since we just choose some branch of $\log z$. Thus $f|_{U_i} = 1$ in $\text{coker exp}$, so sheaf axiom 1 fails.
\end{example}
\begin{defn}
    For a morphism $\phi : \mathcal{F} \to \mathcal{G}$ of sheaves, we define the \textbf{sheaf cokernel} and the \textbf{sheaf image} to be the sheafification of the presheaf cokernel and the presheaf image.
\end{defn}
\textbf{Remark.} Crucial fact: there is an exact sequence of sheaves \[
0 \to 2\pi i \mathbb{Z} \to \mathcal{O}_X \stackrel{\text{exp}}{\to} \mathcal{O}_X^* \to 1.
\]
In other words, $2\pi i \mathbb{Z} = \text{ker}(\text{exp})$ and $\text{coker}(\text{exp}) = 1$ (the first of these we showed, the second of this we will show once we've developed the necessary theory). 
\vspace{1mm}
 
\textbf{Remark.} $\text{ker }\phi, \text{coker }\phi$ satisfy the category theoretic definitions of kernels and cokernels, i.e. they are universal in the appropriate sense. For example, for the kernel, if $\text{ker }\phi : \mathcal{F} \to \mathcal{G}$, then for any other sheaf $\mathcal{L}$ with a map  $\psi $ to $\mathcal{F}$ such that $\phi \circ \psi = 0$, this map factors uniquely through the kernel. This is easy to check and left as an exercise.
\[\begin{tikzcd}[row sep=large,column sep=large]
    & \mathcal{L} & \\
    \text{ker }\phi & \mathcal{F} & \mathcal{G}
    \arrow["\psi", from=1-2, to=2-2]
    \arrow["\phi", from=2-2, to=2-3]
    \arrow["\phi \circ \psi = 0", from=1-2, to=2-3]
    \arrow[,from = 2-1, to = 2-2]
    \arrow["\exists !", from=1-2, to=2-1, dashed]
    \arrow["0", from = 2-1, to = 2-3, bend right = 23]
\end{tikzcd}\]
For the cokernel, reverse all the arrows and check that $\text{coker }\phi$ satisfies the universal property (exercise).

\textbf{Adjacent notions.}
\begin{enumerate}[(i)]
    \item \textbf{Subsheaves}. $\mathcal{F} \subset \mathcal{G}$ is there exist inclusions $\mathcal{F}(U) \subset \mathcal{G}(U)$ compatible with restrictions. For example, $\text{ker}(\phi : \mathcal{F} \to \mathcal{G}) \subset \mathcal{F}$.
    \item \textbf{Quotient sheaves}. To be added at a later date.
\end{enumerate} 

\subsection{Moving between spaces}

\marginpar{20 Oct 2022, Lecture 7}


\begin{defn}
    Given $f: X \to Y$ continuous with sheaves $\mathcal{F}$ on $X$ and $\mathcal{G}$ on $Y$, the \textbf{presheaf pushforward} $f_{\star}\mathcal{F}$ is defined by \[
    \mathcal{U} \mapsto \mathcal{F}(f^{-1}(\mathcal{U}))
    \]
    for an open set $\mathcal{U} \subset \mathcal{Y}$. 
\end{defn}
\begin{prop}
    The presheaf pushforward of a sheaf is a sheaf.
\end{prop}
\begin{proof}
    Trivial.
\end{proof}
\begin{defn}
    Given $f: X \to Y$ continuous with sheaves $\mathcal{F}$ on $X$ and $\mathcal{G}$ on $Y$, the \textbf{inverse image presheaf} $(f^{-1}\mathcal{G})^{\text{pre}}$ is defined by (for $V$ open in $X$) \[
    (f^{-1}\mathcal{G})^{\text{pre}}(V) = \{(s_U,U) \mid U \text{ is an open set containing }V, s_U \in \mathcal{G}(U)\}/\sim
    \]
    where $\sim$ is an equivalence relation that identifies pairs that agree on a smaller open set containing $V$.
    \vspace{1mm}
     
    The \textbf{inverse image sheaf} is given by $f^{-1}\mathcal{G} = ((f^{-1}\mathcal{G})^{\text{pre}})^{\text{sh}}$.
\end{defn}
\begin{example}
    Take $Y$ a topological space and set $X = Y \sqcup Y$. Take $\mathcal{G} = \mathbb{Z}$ to be the constant sheaf and $\mathcal{F} = (f^{-1} \mathcal{G})^{\text{pre}}$. Fix $U \subset Y$ open and $V = f^{-1}(U)$. Then $\mathcal{F}(V) = \mathcal{G}(U) = \mathbb{Z}$, constant (assuming $U$ is connected). But $V = U \sqcup U$, so $\mathcal{F}^{\text{sh}}(V) = \mathcal{G}(U) \times \mathcal{G}(U) = \mathbb{Z}^2$. This happens because this isn't a local condition.
\end{example}
\begin{example}
    Let $\mathcal{F}$ be a sheaf of $X$ and $\pi : X \to \text{point}$. Then $\pi_{\star}\mathcal{F}$ is a sheaf on a point, i.e. an abelian group, specifically $\mathcal{F}(\pi^{-1}(\text{point})) = \mathcal{F}(X)$.
    \vspace{1mm}
     
    \textbf{Notation.} We write $\mathcal{F}(X) = \Gamma(X,\mathcal{F})$, called the global section, and $\mathcal{F}(X)=H^0(X,\mathcal{F})$, the $0^{\text{th}}$ cohomology of coefficients in $\mathcal{F}$.
    \vspace{1mm}
     
    For $p \in X$, $i : \{p\} \to X$, $\mathcal{G}$ a sheaf on the point, i.e. an abelian group $A$, we can consider $i_{\star}\mathcal{G}$. This is the sheaf on $X$ such that $i_{\star}(\mathcal{G})(U) = \begin{cases}
        0 & p \not\in U.\\
        A & p \in U.
    \end{cases}$ This is called the skyscraper at $p$ with value $A$.
\end{example}

\section{Schemes}
The summary: $\text{Spec}(A)$ has a sheaf $\mathcal{O}_{\text{Spec}(A)}$ such that the value on a distinguished open $\mathcal{U}_f = A_f$, and then globalize this to get a scheme. We now spell this out in detail.

\begin{defn}
    Let $A$ be a ring and $S \subset A$ a set that is closed under multiplication. The two examples we should keep in mind are $S = \{1,f,f^2,f^3,\ldots\}$ or $S = A\setminus \mathfrak{p}$ for $\mathfrak{p}$ a prime ideal. The \textbf{localization} of $A$ at $S$ is \[
    S^{-1}A = \{(a,s) \mid a \in A, s \in S\}/\sim
    \]
    where $(a,s) \sim (a',s') \iff \exists s'' \in S$ such that $s''(as'-a's) = 0 \in A$. We read $\frac{a}{s}$ for the equivalence class of $(a,s)$.
\end{defn}

\textbf{Warning.} The map $A \to S^{-1}A$ need not be injective, e.g. if $S$ contains a zero divisor.
\vspace{1mm}
 
What's going to happen now? We will define a sheaf $\mathcal{O}_{\text{Spec}(A)}$ on the topological space $\text{Spec}(A)$ with two features: 
\begin{itemize}
    \item The stalk at a prime $\mathfrak{p}$ will be $(A\setminus \mathfrak{p})^{-1}A$.
    \item If $\mathcal{U}_f$ is a distinguished open, then $\mathcal{O}_{\text{Spec}(A)}(\mathcal{U}_f) = A_f$.
\end{itemize}
\vspace{1mm}
 
\textbf{A sheaf on a base.} Fix a topological space $X$ and $\mathcal{B}$ a basis for the topology. A \textbf{sheaf on the base $\mathcal{B}$}, $\mathfrak{F}$, consists of assignments $B_i \mapsto \mathfrak{F}(B_i)$ on abelian groups/rings/some objects with restriction maps $\mathfrak{F}(B_i) \to \mathfrak{F}(B_j)$ whenever $B_j \subset B_i$. These satisfy the usual commutativity and the identities when $B_i \subset B_j \subset B_k$ or $B_i = B_j$, as well as the following two conditions:
\begin{enumerate}[(1)]
    \item If $B = \cup B_i$ with $B, B_i \in \mathcal{B}$ and $f, g \in \mathfrak{F}(B)$ such that $f|_{B_i} = g|_{B_i} ~\forall i$, then $f = g$. 
    \item If $B = \cup B_i$ as above with $f_i \in \mathfrak{F}(B_i)$ such that they agree where assigned (i.e. $\forall i,j$, if $B' = B_i \cap B_j$, then $f_i|_{B'} = f_j|_{B'}$), then $\exists f \in \mathfrak{F}(B)$ with $f|_{B_i} = f_i$.
\end{enumerate}
\begin{prop}
    Let $F$ be a sheaf on a base $\mathcal{B}$ of a topological space $X$. Then this uniquely (up to unique isomorphism) determines a sheaf $\mathfrak{F}$ by $\mathfrak{F}(B_i) = F(B_i)$ agreeing with restriction maps.
\end{prop}
\begin{proof}
    Define the stalks of $\mathfrak{F}$ first, i.e. $\mathfrak{F}_\mathfrak{p} = \{(S_B,B) \mid B \text{ is a basic open containing }\mathfrak{p}\}/\sim$. Now use the sheafification trick to define $$\mathfrak{F}(U) = \{(f_p \in \mathfrak{F}_p)_{p \in U} \mid ~\forall p \in U, \exists \text{ basic open }B \text{ containing }p \text{ and } s \in F(B) \text{ with }s_q = f_q \text{ in }\mathcal{F}_q ~\forall q \in B\}.$$
    Thirdly, the natural maps $F(B) \to \mathfrak{F}(B)$ are isomorphic by sheaf axioms. The final fact that this is unique (up to unique isomorphism) is left as an exercise.
\end{proof}

\marginpar{23 Oct 2022, Lecture 8}

Setup so far: $\text{Spec }A$ is a topological space with base $\{U_f\}$ for $U_f =\mathbb{V}(f)^C$ over $f \in A$. Recall also that $U_f = U_g$ if and only if $\sqrt{(f)} = \sqrt{(g)}$. Also, if $U_f = U_g$, then the localizations $A_f \cong A_g$ are isomorphic. Therefore, the assignment $U_f \mapsto A_f$ is well--defined.

\begin{prop}
    The assignment $U_f \to A_f$ defines a sheaf (of rings) on the base of the topology of $\text{Spec }A$ given by distinguished opens.
    \vspace{1mm}
     
    As a consequence, $\text{Spec }A$ inherits a sheaf of rings, denoted $\mathcal{O}_{\text{Spec }A}$ and called \textbf{the structure sheaf}.
\end{prop}
\textbf{Prelude.} Suppose $\{U_{f_i}\}_{i \in I}$ covers $\text{Spec }A$. Then there exists a finite subcover. In other words, $\text{Spec }A$ is quasi--compact. Why? Since the $U_{f_i}$ cover, there exists no prime ideal $\mathfrak{p} \subset A$ containing all $(f_i) \iff \sum_{i \in I}^{} (f_i) = (1)$. Hence $ 1 = \sum_{i}^{} a_i f_i$, where all but finitely many $a_i=0$. So if $J \subset I$ are the indices with nonzero coefficient, then $\{U_{f_i}\}_{i \in J}$ cover. 
\begin{proof}
    We need to check that axioms 1 and 2 of a sheaf hold. We will check these for a the basic open $B = \text{Spec }A$ itself (the general case is similar, restrict to a basic open and repeat the proof).
    \vspace{1mm}
     
    Axiom 1: Suppose $\text{Spec }A = \bigcup_{i=1}^n U_{f_i}$ (finite by the prelude). Given $s \in A$ such that $s|_{U_{f_i}} = 0 ~\forall i$, by the definition of localization $f_i^m s = 0$ for some $m$ large enough. But $(1) = (f_i^m)_{i=1}^n$ for any $m>0$ because $\{U_{f_i}\}$ cover $\text{Spec }A$, so hence so do $\{U_{f_i^m}\}$. Hence $1 = \left(\sum_{}^{} r_i f_i^m \right)$ and multiplying by $s$ on both sides gives us $s = 0$.
    \vspace{1mm}
     
    Axiom 2: Say $\text{Spec }A = \bigcup_{i \in I} U_{f_i}$ and choose elements in each $A_{f_i}$ that agree in $A_{f_i f_j}$, i.e. if $s_i \in A_{f_i}$, then the images of $s_i$ and $s_j$ in $A_{f_i f_j}$ coincide. We need to build an element in $A$ with these restrictions. 
    \vspace{1mm}
     
    First suppose $I$ is finite. On $U_{f_i}$, we've chosen an element $\frac{a_i}{f_i^{l_i}} \in A_{f_i}$. Write $g_i = f_i^{l_i}$, noting $U_{f_i} = U_{g_i}$. On overlaps, restrict to $A_{g_ig_j}$. The condition for the second axiom is $(g_i g_j)^{m_j}(a_i g_j - a_j g_i) = 0$. Rewriting this using algebra and $U_f = U_{f^k} ~\forall k\ge 1$, we may assume $m = m_{ij}$ by taking the largest. Write $b_i = a_i g_i^m$ and $h_i = g_i^{m+1}$, so on each $U_{h_i}$ we've chosen an element $\frac{b_i}{h_i}$. \footnote{So far, this is just rewriting everything symbolically with no actual content.} But $U_{h_i}$ cover $\text{Spec }A$, so $1 = \sum_{i}^{} r_i h_i$ for $r_i \in A$. Now we construct $r = \sum_{}^{} r_i b_i$ with $r_i$ as above. This restricts correctly to $\frac{b_i}{h_i}$ on $U_{h_i}$ (i.e. in the localization $A_{h_i}$).
    \vspace{1mm}
     
    When $I$ is infinite, pick a finite subcover $(f_1,\ldots,f_n) = A$ such that $U_{f_i}$ form a cover and use the above to build $r$. But given $(f_1,\ldots,f_n, f_{\alpha}) = A$, the same construction gives a ''new'' $r'$. But $r' = r$ by the first axiom. 
\end{proof}
\begin{defn}
    The structure sheaf on $\text{Spec }A$ is the sheaf associated to the sheaf on the base sending $U_f \mapsto A_f$, denoted $\mathcal{O}_{\text{Spec }A}$.
\end{defn}
\textbf{Observation.} The stalk $\mathcal{O}_{\text{Spec }A, \mathfrak{p}} = A_{\mathfrak{p}}$.
\vspace{1mm}
 
\textbf{Terminology.} A \textbf{ringed space} $(X, \mathcal{O}_X)$ is a topological space $X$ with a sheaf of rings $\mathcal{O}_X$. An isomorphism of ringed spaces $(X, \mathcal{O}_X) \to (Y, \mathcal{O}_Y)$ is the combination of a homomorphism $\pi : X \to Y$ and an isomorphism of sheaves on $Y$, $\mathcal{O}_Y \stackrel{\sim}{\to} \pi_{\star}\mathcal{O}_X$.

\begin{defn}
    An \textbf{affine scheme} is a ringed space $(X, \mathcal{O}_X)$ that is isomorphic (as a ringed space) to $(\text{Spec }A, \mathcal{O}_{\text{Spec }A})$.
\end{defn}
\begin{defn}
    A \textbf{scheme} is a ringed space $(X, \mathcal{O}_X)$ that is locally isomorphic to an affine scheme.
\end{defn}
Intuitively, every point $p \in X$ has a neighborhood $U_p$ such that the ringed space $(U_p, \mathcal{O}_{U_p})$ is isomorphic to some affine scheme (possibly depending on $p$).
Note that if $U \subset X$ is open, then $U$ is naturally a ringed space with $\mathcal{O}_U(V) = \mathcal{O}_X(V)$.

\begin{example}
    $\text{Spec }A$ for various rings $A$.
\end{example}
\begin{example}
    Take $X = \text{Spec }\mathbb{C}[x,y]$ and $U = \{(x,y)\}^C$. Then the scheme $U$ is not an affine scheme.
\end{example}
\marginpar{25 Oct 2022, Lecture 9}

\begin{example}
    Open subschemes. Let $X$ be a scheme and $U \subset X$ be open.\footnote{From now on, whenever we say ''let $X$ be a scheme'', we silently take that to mean $(X, \mathcal{O}_X)$.} Write $i : U \to X$ for the inclusion map. Take $\mathcal{O}_U = \mathcal{O}_X |_U = i^{-1} \mathcal{O}_X$ to be the structure sheaf of $U$.
\end{example}
\begin{prop}
    The ringed space $(U, \mathcal{O}_U)$ is a scheme.
\end{prop}
Simple case: take $X = \text{Spec }A$, $U = U_f$ for $f \in A$. Then $(U, \mathcal{O}_U) \cong (\text{Spec }A_f, \mathcal{O}_{\text{Spec }A_f})$.
\begin{proof}
    Let $p \in U \subset X$ be a point. Since $X$ is a scheme, we can find some $(V_p, \mathcal{O}_X |_{V_p})$ inside $X$ with $p \in V$ such that $V_p$ is isomorphic to an affine scheme. Take $V_p \cap U \subset U$ with the structure sheaf via restriction. However, this may not be affine. But $V_p$ is affine, say $V \cong \text{Spec }B$, and the distinguished opens in $\text{Spec }B$ form a basis for the topology. Hence we've reduced to the ''simple case'' and we're done.
\end{proof}
\begin{example}
    Define $\mathbb{A}_k^n = \text{Spec }k[x_1,\ldots,x_n]$. Take $U = \mathbb{A}^{n^2} - \{\det(x_{ij}) = 0\}$, i.e. ''$U = GL_2(k)$''.
\end{example}
\begin{example}
    A non--affine scheme. Take $X = \mathbb{A}_k^2 = \text{Spec }k[x,y]$ and $U = \mathbb{A}_k^2 \setminus \{(x,y)\}$.\footnote{Illegally, we are allowed to think of this as $\mathbb{R}^2 \setminus \{(0,0)\}$.} Then the claim is that $U$ is not affine. We will calculate $\mathcal{O}_U(U)$. Write $U_x = \mathbb{V}(x)^{C} \subset \mathbb{A}^2$ and $U_y = \mathbb{V}(y)^{C} \subset \mathbb{A}^2$. Note that $U = U_x \cup U_y$ and $U_x \cap U_y = \mathbb{A}^2 \setminus \mathbb{V}(xy)$. \vspace{1mm}
     
    We have $\mathcal{O}_U(U_x) = k[x,x^{-1}, y]$, $\mathcal{O}_U(U_y) = k[x,y,y^{-1}]$, and $\mathcal{O}_U(U_x \cap U_y) = k[x,x^{-1},y,y^{-1}]$. Also, the restriction maps $\mathcal{O}_U(U_x) \to \mathcal{O}_U(U_{xy})$ are the obvious ones.
    \vspace{1mm}
     
    By sheaf axioms, $\mathcal{O}_U(U) = k[x,x^{-1},y] \cap k[x,y,y^{-1}]$ (inside $k[x,x^{-1},y,y^{-1}]$). Hence $\mathcal{O}_U(U) = k[x,y]$. This is a contradiction. Why? One way: there exists (in $(U, \mathcal{O}_U)$) a maximal ideal in the global sections ring with empty vanishing locus, namely $(x,y) \subset k[x,y]$. On the other hand, there is no maximal ideal in $\text{Spec }k[x,y]$ with empty vanishing locus. 
    \vspace{1mm}
     
    This is a bit of a hack and there is a better conceptual approach that we will discover soon.
    \vspace{1mm}
     
    A little more on $U = \mathbb{A}_k^2 \setminus  \{(x,y)\}$ not being affine (this was talked about at the beginning of the following lecture). Let $X$ be a scheme and $f \in \Gamma(X,\mathcal{O}_X) = \mathcal{O}_X(X)$. Fix $p \in X$. Then there's a well--defined stalk $\mathcal{O}_{X,p}$ of $\mathcal{O}_X$ at $p$.
    The stalk is of the form $A_\mathfrak{p}$, where $A$ is a ring and $\mathfrak{p}\subset A$ is a prime ideal. In particular, $A_\mathfrak{p}$ has a unique maximal ideal, namely $\mathfrak{p} A_\mathfrak{p}$. Say $f$ vanishes at $\mathfrak{p}$ if its image in $A_\mathfrak{p}/\mathfrak{p}A_\mathfrak{p}$ is 0 (i.e $f \in \mathfrak{p}A_{\mathfrak{p}}$). (Here we're using an isomorphism $p \ni V_p$ open to $\text{Spec }A$). For $f \in \Gamma(X,\mathcal{O}_X)$, $\mathbb{V}(f)$, the vanishing locus of $f \subset X$ is well--defined.
\end{example}
\subsection{Interlude: gluing sheaves}
Let $X$ be a topological space with cover $\{U_{\alpha}\}$, sheaves $\{\mathcal{F}_{\alpha}\}$ on $\{U_\alpha\}$ and isomorphisms (of sheaves) $\phi_{\alpha,\beta} : \mathcal{F}_{\alpha}|_{U_\alpha \cap U_{\beta}} \to \mathcal{F}_\beta |_{U_\alpha \cap U_\beta}$ such that $\phi_{\alpha,\alpha} = \text{id}$, $\phi_{\alpha \beta} = \phi_{\beta \alpha}^{-1}$ and $\phi_{\beta \gamma} \circ \phi_{\alpha \beta} = \phi_{\alpha \gamma}$ on $U_{\alpha} \cap U_\beta \cap U_\gamma$ (the \textbf{cocycle condition}).
\vspace{1mm}
 
\textbf{Construction.} We will build a sheaf $\mathcal{F}$ on $X$. Given $V \subset X$ open, define 
\begin{align*}
    \mathcal{F}(V) = \{(S_\alpha)_{\alpha}, S_\alpha \in \mathcal{F}_{\alpha}(U_\alpha \cap V)\mid \phi_{\alpha,\beta}(S_\alpha |_{V \cap U_\alpha \cap U_\beta}) = S_\beta|_{V \cap U_\alpha \cap U_\beta}\}.
\end{align*} 
$\mathcal{F}$ is a presheaf, since given $(S_\alpha) \in \mathcal{F}(V)$ and $W \subset V$ open, we can take $(S_a)|_W = \left(\text{res}^{V \cap U_\alpha}_{W \cap U_\alpha}(S_\alpha)\right)_\alpha$. This lies in $\mathcal{F}(W)$ by sheaf axioms.
\begin{prop}
    $\mathcal{F}$ is a sheaf and $\mathcal{F}|_{U_\alpha} = \mathcal{F}_\alpha$ on $U_\alpha$.
\end{prop}
\begin{proof}
    It is a presheaf, and both sheaf axioms are clear (exercise: check this). But we need to check/build an isomorphism $\mathcal{F}|_{U_\gamma} \to \mathcal{F}_\gamma$. Given $V \subset U_\gamma$ and $S \in \mathcal{F}_{\gamma}(V)$, define its image in $\mathcal{F}_{U_\gamma}$ to be $(\phi_{\gamma, \alpha}(S|_{V \cap U_\alpha}))_\alpha$. We need to check that this lies in $\mathcal{F}|_{U_\gamma}(V)= \mathcal{F}(V)$, but this follows from the cocycle condition: $\phi_{\alpha,\beta} \circ \phi_{\gamma,\alpha} (S |_{V \cap U_\alpha \cap U_\beta}) = \phi_{\gamma, \beta}(S|_{V \cap U_\alpha \cap U_\beta})$.
\end{proof}
\subsection{More schemes}
Take schemes $(X, \mathcal{O}_X)$ and $(Y, \mathcal{O}_Y)$ with opens $U \subset X, V \subset Y$ and an isomorphism $(U, \mathcal{O}_X|_U) \stackrel{\sim}{\to} (V, \mathcal{O}_Y|_V)$. We can glue both the topological spaces and the schemes: $X \sqcup Y / (U \sim V)$ with the sheaf glued as in the previous construction.
\vspace{1mm}
 
\marginpar{27 Oct 2022, Lecture 10}

How to glue: Take $(X \sqcup Y)/(U \sim V)$. By definition of the quotient topology, the image of $X,Y$ in $S$ form an open cover and their intersection is the image of $U$ (or $V$). Now glue the structure sheaves on these opens as in the previous lecture (to get $(S, \mathcal{O}_S)$). Note that there is no cocycle condition, since we only have the intersection of two and not three opens.

\begin{example}
    The bug--eyed line, i.e. the line with two origins. Let $k$ be a field and $U \subset X = \text{Spec }k[t]$, $V \subset Y = \text{Spec }k[u]$, $U = \text{Spec }k[t,t^{-1}]$, $V = \text{Spec }k[u,u^{-1}]$. We have the isomorphism $U \to V$ by $t \mapsto u$. (Really, this is an isomorphism of rings $k[u,u^{-1}] \to k[t,t^{-1}]$ with $u \mapsto t$ and now take Spec).
    \vspace{1mm}
     
    On the level of topological spaces, $X = \mathbb{A}_k^1$ and $Y = \mathbb{A}_k^1$ with $U = \mathbb{A}^1\setminus \{(t)\}$ (i.e. ''$U$ minus a point'', similarly for $V$). Hence $X \sqcup Y/\sim$ gives the line with two origins.
    \vspace{1mm}
     
    What are the types of opens in this scheme? 
    \begin{itemize}
        \item $W$ could be contained inside $X$ or $Y$ (inside $S$). There are nice, easy open sets.
        \item $W = S \setminus \{p_1,\ldots,p_r\}$ where $p_i \in U$ or $p_i \in V$. The simplest of these is when $W = S$. 
    \end{itemize}
    What is $\mathcal{O}_S(S)$? Use sheaf axioms to find that $\mathcal{O}_S(S) \cong k[t]$. Conclusion: $S$ is not affine.
\end{example}
\begin{example}
    $\mathbb{P}_k^1$. Same setup: $X = \text{Spec }k[t], Y = \text{Spec }k[s]$, $U = \text{Spec }k[t,t^{-1}]$, $V = \text{Spec }k[s,s^{-1}]$. We glue via the isomorphism $s \mapsto t^{-1}$. Then $\mathbb{P}_k^1$ is the result of the gluing (we can consider this as a definition for now).
\end{example}
\begin{prop}
    $\mathcal{O}_{\mathbb{P}^1}(\mathbb{P}^1) \cong k$. 
\end{prop}
\begin{proof}
    Important exercise: the only elements of $k[t,t^{-1}]$ that are both polynomials in $t$ and $t^{-1}$ are the constants. (Do this!). In particular, $\mathbb{P}^1$ is not affine.
\end{proof}
\begin{example}
    Similarly we can build $S =$ ''$\mathbb{A}_k^2$ with doubled origin'' -- this has the interseting property that there exist affine open subschemes $U_1, U_2 \subset S$ such that $U_1 \cap U_2$ is not affine. We flag this example for later.
\end{example}

\textbf{Gluing schemes}. (Example sheet 1). Given schemes $X_i$ for $i \in I$, open subschemes $X_{ij} \subset X_i$ with $X_{ii} = X_i$, isomorphisms $f_{ij}: X_{ij} \stackrel{\sim}{\to} X_{ji}$ with $f_{ii} = \text{id}_{X_i}$, and the cocycle condition $f_{ik}|_{X_{ij} \cap X_{ik}} = f_{jk}|_{X_{ji} \cap X_{jk}} \circ f_{ij}|_{X_{ij} \cap X_{ik}}$, there is a unique scheme $X$ with an open cover given by $X_i$, glued along $X_{ij} \cong X_{ji}$.

\begin{example}[Key example]
    Take $A$ any ring, $X_i = \text{Spec }A\left[\frac{x_0}{x_i},\ldots,\frac{x_n}{x_i}\right]$, $X_{ij} = \mathbb{V}\left(\frac{x_j}{x_i}\right)^C \subset X_i$, and isomorphisms $X_{ij} \to X_{ji}$ by $\frac{x_k}{x_i}\mapsto \frac{x_k}{x_j}\left(\frac{x_i}{x_j}\right)^{-1}$. The resulting glued scheme is called the \textbf{projective $n$--space}, denoted $\mathbb{P}_A^n$.
\end{example}
\textbf{Exercise/calculation.} $\Gamma(\mathbb{P}_A^n, \mathcal{O}_{\mathbb{P}_A^n}) = A$.

\subsection{Proj construction}
\begin{defn}
    A \textbf{$\mathbb{Z}$--grading} on a ring $A$ is a decomposition $A = \bigoplus_{i \in \mathbb{Z}}A_i$ as abelian groups such that $A_i A_j \subset A_{i+j}$.
\end{defn}
\begin{example}
    Take $A = k[x_0,\ldots,x_n]$ and write $A_d = \{\text{degree }d \text{ homogeneous polynomials}\} \cup \{0\}$.
    \vspace{1mm}
     
    Also: Let $I \subset k[x_0,\ldots,x_n]$ be a homogeneous ideal (i.e. generated by homogeneous elements of possibly different degree). Then $k[x_0,\ldots,x_n]/I$ is also naturally graded. (Exercise: think about how).
\end{example}
\textbf{Remark.} $A_0$ is always a subring of $A$.
\vspace{1mm}
 
\textbf{Assumption.} From now on, we will assume that the degree 1 elements generate $A$ as an algebra over $A_0$.

\marginpar{30 Oct 2022, Lecture 11}

\vspace{1mm}
 
\textbf{Another assumption.} We assume $A_i = 0$ for all $i < 0$.
\vspace{1mm}
 
\textbf{Terminology.} $A_+ = \bigoplus_{i\ge 1} A_i \subset A$ is the subgroup of positive degree elements. It forms an ideal, called the \textbf{irrelevant ideal}. 
\vspace{1mm}
 
A \textbf{homogeneous element} $f \in A$ is an element contained in some $A_d$. 
\vspace{1mm}
 
An ideal $I \subset A$ is called \textbf{homogeneous} if it is generated by homogeneous elements (possibly of different degrees).

\begin{defn}
    The set $\text{Proj }A$ is the set of all homogeneous primes in $A$ that do not contain the irrelevant ideal $A_+$.
    \vspace{1mm}
     
    If $I \subset A$ is homogeneous, then $\mathbb{V}(I) = \{\mathfrak{p} \in \text{Proj }A \mid \mathfrak{p} \text{ contains }I\}$. Given this, the Zariski topology on $\text{Proj }A$ has closed sets $\mathbb{V}(I)$ for $I \subset A$ homogeneous.
\end{defn}

Let $f \in A_i$ and $U_f = \text{Proj }A\setminus \mathbb{V}(f)$. Then observe that $\{U_f\}_{f \in A_1}$ covers $\text{Proj }A$. Also, the ring $A[\frac{1}{f}] = A_f$ is naturally $\mathbb{Z}$--graded by $\text{deg}(f^{-1})= - \text{deg}(f)$.

\begin{example}
    Let $ A = k[x_0,x_1]$ and $f = x_0$, then $A[\frac{1}{f}] = k[x_0,x_1,x_0^{-1}]$. The degree 0 elements of this include $\lambda (\lambda \in k), \frac{x_1}{x_0}, \frac{x_1^2+x_1x_0}{x_0^2}$, etc. Similarly, degree 1 elements include $\frac{x_1^2}{x_0}$, etc.  
\end{example}
\begin{prop}
    There is a natural bijection between
    \begin{align*}
        \{\text{homogeneous primes in }A \text{ missing }f\} \leftrightarrow \{\text{primes in }(A_f)_{\text{degree 0}}\}.
    \end{align*}
    (Equivalently, the LHS is the homogeneous primes in $A_f$).
\end{prop}
\begin{proof}[Proof and construction:]
    Primes in $A$ missing $f$ are naturally in bijection with homogeneous primes in $A_f$. Suppose $\mathfrak{q} \subset (A[\frac{1}{f}])_0$ is a prime. Then let $\Psi(\mathfrak{q})$ be generated by \[
    \bigcup_{d\ge 0} \left\{a \in A_d \mid \frac{a}{f^d} \in \mathfrak{q}\right\} \subset A.
    \]
    Exercise/easy check: this is prime.
    \vspace{1mm}
     
    Given $\mathfrak{p} \subset A$ homogeneous missing $f$, take $\phi(\mathfrak{p}) = \left(\mathfrak{p} A[\frac{1}{f}] \cap(A[\frac{1}{f}]_0)\right)$.
    \vspace{1mm}
     
    We need to check two compositions. $\phi \circ \Psi =\text{id}$ is easy and left as an exercise. However, $\Psi \circ \phi$ is trickier. We will show $\mathfrak{p} = \Psi(\phi(p))$ by exhibiting both containments. Suppose $\mathfrak{p} \in U_f \subset \text{Proj }A$. Then if $a \in \mathfrak{p} \cap A_d$, then $\frac{a}{f^d} \subset \phi(p)$, so $a \in \Psi(\phi(\mathfrak{p}))$. Conversely, if $a \in \Psi(\phi(\mathfrak{p}))$, then $\frac{a}{f^d} \in \phi(\mathfrak{p})$ for some $d$. Hence there exists $b \in \mathfrak{p}$ such that $\frac{b}{f^e} = \frac{a}{f^d}$ in the localization after inverting $f$. For some $k\ge 0$, we have $f^k(f^d b - f^e a) = 0$, but $f^{e+k} \not\subset \mathfrak{p}$. Hence by primality, $a \in \mathfrak{p}$, giving the reverse containment.
\end{proof}
\textbf{Remark.} The bijection we constructed is compatible with ideal containment, so it gives a homeomorphism from $U_f$ to $\text{Spec}(A_f)_0$.
\vspace{1mm}
 
$\text{Proj }A$ is covered by open sets, each isomorphic to $\text{Spec }(A_f)_0$ for some $f$. If $f, g \in A_1$, then $U_f \cap U_g$ is naturally homeomorphic to $\text{Spec}(A[\frac{1}{f}])_0[\frac{f}{g}] = \text{Spec}(A[f^{-1},g^{-1}])_0$. Call this property $(\star)$. 
\vspace{1mm}
 
Take the open cover $\{U_f\}$ with structure sheaf $\mathcal{O}_{\text{Spec}(A_f)_0}$ on each $U_f$ with isomorphisms on $U_f \cap U_g$ by $(\star)$. The cocycle condition follows immediately from the formal properties of localization (exercise: check this).
\vspace{1mm}
 
\textbf{Terminology.} If $A = k[x_0,\ldots,x_n]$ with the standard grading, then $\text{Proj }A$ is denoted as $\mathbb{P}_k^n$. This is the same as the projective $n$--space defined earlier, but we need further work to show this.

\section{Morphisms}

We have already seen a few ''examples''. For example, we have a morphism given by inclusion for $U \subset X$. Similarly, if $A \to B$ is a ring homomorphism, then $\text{Spec }B \to \text{Spec }A$ should be another morphism.

\subsection{Morphisms of schemes and locally ringed spaces}

Given a scheme $(X, \mathcal{O}_X)$, the stalks $\mathcal{O}_{X,p}$ are \textbf{local rings} (i.e. they have a unique maximal ideal). Given a function $f \in \mathcal{O}_X(U)$ with $p \in U$, we can ask whether $f$ vanishes at $p$, i.e. is the image of $f$ in $\mathcal{O}_{X,p}$ contained in the maximal ideal?

\begin{defn}
    A morphism of \textbf{ringed spaces} (i.e. a topological space and a sheaf of rings) is $(f,f^\#)$ such that:
    \begin{itemize}
        \item $f: (X, \mathcal{O}_X) \to (Y, \mathcal{O}_Y)$
        \item $f : X \to Y$ is continuous.
        \item $f^\# : \mathcal{O}_Y \to \mathcal{O}_X$, a morphism of sheaves of rings on $Y$.
    \end{itemize}
\end{defn}

\marginpar{01 Nov 2022, Lecture 12}

\textbf{Warning.} It is possible to find $(f,f^\#)$ from $(X, \mathcal{O}_X)$ to $(Y, \mathcal{O}_Y)$ of schemes such that there exists $U \subset Y$ open with $q \in U$ and $h \in \mathcal{O}_Y(U)$ such that $h$ vanishes at $q$, but $f^\#(h) \in \mathcal{O}_X(f^{-1}(U))$ does not vanish at $p \in X$ such that $f(p) = q$.
\vspace{1mm}
 
\textbf{Observation.} Given $f : X \to Y$ a ringed space morphism and $p \in X$, there is an induced map $f^\# : \mathcal{O}_{Y,f(p)} \to \mathcal{O}_{X,p}$. To spell this out, given $s \in \mathcal{O}_{Y,f(p)}$, we can represent it by $(S_U,U)$ for $U$ open, $f(p) \in U$ and $S_U \in \mathcal{O}_Y(U)$. Therefore $f^\#(S_U) \in \mathcal{O}_X(f^{-1}(U))$, so the pair $(f^\#(S_U), f^{-1}(U))$ defines an element in $\mathcal{O}_{X,p}$.

\begin{defn}
    $(X,\mathcal{O}_X)$, a ringed space, is called \textbf{locally ringed} if $\forall p \in X$, $\mathcal{O}_{X,p}$ is a local ring (i.e. has a unique maximal ideal). A morphism of locally ringed spaces $(f,f^\#) : (X, \mathcal{O}_X) \to (Y, \mathcal{O}_Y)$ is a morphism as ringed spaces such that if $\mathfrak{m}_p$ denotes the maximal ideal in $\mathcal{O}_{X,p}$, then $f^\#(\mathfrak{m}_{f(p)}) \subset \mathfrak{m}_{p}$ (in the stalks). 
\end{defn}
\begin{defn}
    A \textbf{morphism of schemes} $X \to Y$ is a morphism as locally ringed spaces.
\end{defn}
\begin{theorem}
    There is a natural bijection between 
    \begin{align*}
        \{\text{Scheme theoretic morphisms }\text{Spec }B \to \text{Spec }A\} \leftrightarrow \{\text{Ring homomorphisms }A \to B\}.
    \end{align*}
\end{theorem}
\textbf{Prologue.} Recall that section of a sheaf $\mathcal{F}$ on $U$, i.e. $s \in \mathcal{F}(U)$, is a coherent collection of elements $s(p) \in \mathcal{F}_p$ (the stalk) for all $p \in U$.
\begin{proof}
    We will first show that every ring map $A \to B$ induces a scheme map (we will construct it), and then conversely we will show that every scheme map $\text{Spec }B \to \text{Spec }A$ arises via our construction.
    \vspace{1mm}
     
    Given $\phi : A \to B$, we can take $\phi^{-1} : \text{Spec }B \to \text{Spec }A$ as the topological part (continuity is formal). Now we build $\phi^\# : \mathcal{O}_{\text{Spec }A} \to \phi^{-1}_{\star}\mathcal{O}_{\text{Spec }B}$. First, at the stalk level, take $A_{\phi^{-1}(p)} \to B_p$ sending $\frac{a}{s} \mapsto \frac{\phi(a)}{\phi(s)}$ induced by $\phi$. This makes sense because if $s \not\in \phi^{-1}(p)$, then $\phi(s) \not\in p$ (we're treating $p$ as a prime ideal here). Observe that this is automatically a local homomorphism of local rings.
    \vspace{1mm}
     
    Secondly, on the open set level, given $U \subset \text{Spec }A$, we need to define $\phi^\# : \mathcal{O}_{\text{Spec }A}(U) \to \mathcal{O}_{\text{Spec }B}((\phi^{-1})^{-1}(U))$ (this just means take the preimage of $U$ inside $\text{Spec } B$). An element $s \in \mathcal{O}_{\text{Spec }A}(U)$ is a collection of assignments $[p \mapsto s(p)]_{p \in U}$ with $p \in U$, $s(p) \in A_p$. We define $\phi^\#$ by sending $[p \mapsto s(p)]_{p \in U} \mapsto [q \mapsto \phi_q(s(\phi^{-1}(q)))]_{q \in (\phi^{-1})^{-1}(U)}$, where $\phi_q$ is the map on stalks at $q$. We can check that this glues (see official notes if we're still in disbelief).
    \vspace{1mm}
     
    Conversely, suppose we're given $(f,f^\#) \colon \text{Spec }B \to \text{Spec }A$. Using $\mathcal{O}_{\text{Spec }A}(\text{Spec }A) \to \mathcal{O}_{\text{Spec }B}(\text{Spec }B)$, we get $g: A \to B$ a ring homomorphism. We need to check that $g^{-1}$ gives the right topological map and that the construction from the first part gives the right map on sheaves.
    \vspace{1mm}
     
    For the first part, the maps on stalks are compatible with restriction. For instance, $\Gamma(\text{Spec }A, \mathcal{O}_{\text{Spec }A}) \to \Gamma(\text{Spec }B, \mathcal{O}_{\text{Spec }B})$ is compatible with restricting to the stalks: $\mathcal{O}_{\text{Spec }A,f(p)} \to \mathcal{O}_{\text{Spec }B,p}$ for all $p \in \text{Spec }B$. Hence the following map commutes.
    \[\begin{tikzcd}[]
        \Gamma(\text{Spec }A, \mathcal{O}_{\text{Spec }A}) & \Gamma(\text{Spec }B, \mathcal{O}_{\text{Spec }B})\\
        \mathcal{O}_{\text{Spec }A,f(p)} & \mathcal{O}_{\text{Spec }B,p}
        \arrow["", from=1-1, to=1-2]
        \arrow["",from=1-1, to=2-1]
        \arrow["", from=1-2, to=2-2]
        \arrow["", from=2-1, to=2-2]
    \end{tikzcd}\]
    Equivalently, the following map commutes for all $p \in \text{Spec }B$.
    \[\begin{tikzcd}[]
        A & B \\
        A_{f(p)} & B_p
        \arrow["g", from=1-1, to=1-2]
        \arrow["",from=1-1, to=2-1]
        \arrow["", from=1-2, to=2-2]
        \arrow["f^\#", from=2-1, to=2-2]
    \end{tikzcd}\]
    Since the morphism is local: $(f^\#)^{-1} p B_p = f(p) A_{f(p)}$. By commutativity, $g^{-1} = f$ topologically. The structure sheaf maps agree at the stalk level by construction, so we're done.
\end{proof}

\textbf{Housekeeping.} 

\begin{defn}
    Let $X,Y$ be schemes. A morphism $f : X \to Y$ is an \textbf{open immersion} if $f$ induces an isomorphism of $X$ onto an open subscheme of $Y$ (i.e. $(U, \mathcal{O_Y}|_U)$ for $U \subset Y$ open).
    \vspace{1mm}
     
    $g : X \to Y$ is a \textbf{closed immersion} if the topological map is a homeomorphism onto a closed subset of $Y$ and the map $g^\# : \mathcal{O}_Y \to g_{\star}\mathcal{O}_X$ is surjective.
\end{defn}
\begin{example}
    Take $k[t] \to k[t]/t^2$ and take $\text{Spec}$. This is a closed immersion.
\end{example}

\marginpar{03 Nov 2022, Lecture 13}

\begin{defn}
    Let $Y$ be a scheme. A \textbf{closed subscheme} of $Y$ is an equivalence class of closed immersions $\{X \to Y\}$, where $X \stackrel{f}{\to}  Y$ and $X' \stackrel{f'}{\to} Y$ are equivalent if there is a commuting triangle:
    \[\begin{tikzcd}
        X & &X' \\
         & Y &
        \arrow["\text{isom}", from=1-1, to=1-3]
        \arrow["f", from=1-1, to=2-2]
        \arrow["f'", from=1-3, to=2-2]
    \end{tikzcd}\]
\end{defn}
A typical example of a closed immersion: if $A$ is a ring and $I \subset A$ is an ideal, then the natural map $\text{Spec }A/I \to \text{Spec }A$ is a closed immersion.

\subsection{Fiber products}
Fiber products will simultaneously generalize:
\begin{itemize}
    \item the ''product'' of schemes.
    \item if $X_1,X_2 \subset Y$ are closed subschemes,then $X_1 \cap X_2 \subset Y$ is a fibre product.
    \item Given a morphism $X \stackrel{f}{\to} Y$ and a subscheme $Z \subset Y$, the preimage $f^{-1}(Z)$ is a subscheme of $X$, described by the fiber product. 
\end{itemize}  
\begin{defn}
    Consider a diagram $\begin{tikzcd}[]
         & X\\
        Y & S
        \arrow["", from=1-2, to=2-2]
        %\arrow["",from=1-1, to=2-1]
        \arrow["", from=2-1, to=2-2]
        %\arrow["", from=2-1, to=2-2]
    \end{tikzcd}$. The \textbf{fiber product} is a scheme $X \times_S Y$ fitting into a diagram $
        \begin{tikzcd}[]
           X \times_S Y & X\\
           Y & S
           \arrow["", from=1-2, to=2-2]
           \arrow["p_Y",from=1-1, to=2-1]
           \arrow["", from=2-1, to=2-2]
           \arrow["p_X", from=1-1, to=1-2]
       \end{tikzcd}
    $
    which commutes such that for any other scheme $Z$ with a commuting diagram of this form ($\begin{tikzcd}[]
        Z & X\\
        Y & S
        \arrow["", from=1-2, to=2-2]
        \arrow["",from=1-1, to=2-1]
        \arrow["", from=2-1, to=2-2]
        \arrow["", from=1-1, to=1-2]
    \end{tikzcd}$) there is a unique morphism $Z \to X \times_S Y$ such that 
    \[
        \begin{tikzcd}[]
            Z & &\\
            &X \times_S Y & X\\
            & Y & S
            \arrow["", from=2-3, to=3-3]
            \arrow["p_Y",from=2-2, to=3-2]
            \arrow["", from=3-2, to=3-3]
            \arrow["p_X", from=2-2, to=2-3]
            \arrow["", from=1-1, to=2-2]
            \arrow["q_X",from=1-1, to=2-3, bend left=20]
            \arrow["q_Y",from=1-1, to=3-2, bend right=30]
        \end{tikzcd}
    \]
    commutes.
\end{defn}
\textbf{Observation.} If $X \times_S Y$ exists, then it is unique up to unique isomorphism.
\vspace{1mm}
 
\textbf{Remarks.} 
\begin{itemize}
    \item Similarly we can define these product in Sets, so if $X,Y,S$ are sets with maps $X \stackrel{r_X}{\to} S$ and $Y \stackrel{r_Y}{\to} S$, then $$X \times_S Y = \{(x,y) \in X \times Y \mid r_X(x) = r_Y(y)\}.$$
    This is left as an important exercise: check that this is the fibre product.
    \item Fibre products also exist in topological spaces with the same definition as in Sets (given the subspace (of product) topology).
    \item Say $X \stackrel{r_X}{\to} S$ is a map of sets and say $Y = \{\star\}$ with $r_Y(\star) = s \in S$. Then $X \times_S Y = r_X^{-1}(s)$.
\end{itemize}
\begin{theorem}
    Fibre products of schemes exist.
\end{theorem}
This is Hartshorne chapter 2, Theorem 3.3, which we should read to understand all the details.
\begin{proof}
    Affine case: Let $X,Y,S$ be affine schemes with associated rings $A,B,R$. Then the fibre product $X \times_S Y$ exists and is isomorphic to $\text{Spec }(A \otimes_R B)$. We check that the universal property is satisfied, i.e. given any scheme $Z$ with morphisms $\begin{tikzcd}[]
        Z & X\\
        Y & S
        \arrow["", from=1-2, to=2-2]
        \arrow["",from=1-1, to=2-1]
        \arrow["", from=2-1, to=2-2]
        \arrow["", from=1-1, to=1-2]
    \end{tikzcd}$, there is a unique morphism $Z \to \text{Spec }(A \otimes_R B)$. Fact (from example sheet 2): a scheme theoretic map $Z \to \text{Spec }A \otimes_R B \iff A \otimes_R B \to \Gamma(Z,\mathcal{O}_Z)$.
    \vspace{1mm}
     
    Now we start with a general $X \times_S Y$ and ''cover by affines''. 
    \begin{itemize}
        \item If $X \times_S Y$ exists and $U \subset X$ is an open subscheme, then $U \times_S Y$ also exists, since we take the inverse image of $U$ under $X \times_S Y \stackrel{p_X}{\to} X$ with the open subscheme structure.
        \item If $X$ is covered by opens $\{X_i\}$, then if $X_i \times_S Y$ exists for all $i$, then $X \times_S Y$ exists, since we can just glue (there's no cocycle condition).
    \end{itemize}
    Now for any $X$ and $S,Y$ affine, $X \times_S Y$ exists by above (i.e. we can cover $X$ by affines). But $X$ and $Y$ are interchangable, so $X \times_S Y$ exists for all $S$ affine and $X,Y$ arbitrary. Now we cover $S$ by affines $\{S_i\}$. Let $X_i$ and $Y_i$ be the preimages of $S_i$ in $X$ and $Y$ respectively. Now $X_i \times_{S_i} Y_i$ exists. By the universal property, $X_i \times_{S_i} Y_i = X_i \times_S Y_i$, so glue in to form $X \times_S Y$.
\end{proof}
\begin{example}
    \begin{enumerate}[(i)]
        \item $\mathbb{P}_{\mathbb{C}}^n = \mathbb{P}_{\mathbb{Z}}^n \times_{\text{Spec }\mathbb{Z}} \text{Spec }\mathbb{C}$, where $\text{Spec }\mathbb{C} \to \text{Spec }\mathbb{Z}$ is induced by $\mathbb{Z} \hookrightarrow \mathbb{C}$ and $\mathbb{P}_{\mathbb{Z}}^n \to \text{Spec }\mathbb{Z}$ is induced locally by $\mathbb{Z} \hookrightarrow \mathbb{Z}\left[\frac{x_0}{x_i},\ldots,\frac{x_n}{x_i}\right]$. Compare this with the fact that we already know, $\mathbb{Z}[\overline{X}] \otimes_{\mathbb{Z}} \mathbb{C} = \mathbb{C}[\overline{X}]$.
        \item Let $C = \text{Spec }\mathbb{C}[x,y]/(y-x^2)$ and $L = \text{Spec }\mathbb{C}[x,y]/(y)$. We have natural morphisms $\mathbb{C} \to \mathbb{A}_{\mathbb{C}}^2, L \to \mathbb{A}_{\mathbb{C}}^2$. By algebra, $C \times_{\mathbb{A}^2} L = \text{Spec}(\mathbb{C}[x]/(x^2))$.
    \end{enumerate}
\end{example}
\marginpar{06 Nov 2022, Lecture 14}

Example sheet 2 is now up, on which many basic definitions are defined for structure sheafs -- reduced, integral, irreducible, noetherian, etc. Have a look at the sheet, since we will use these definitions from now on.
\vspace{1mm}
 
\textbf{Language.} In scheme theory, we often fix a ''base scheme'' $S$ and consider all other schemes $X$ with a fixed morphism $X \to S$, which we often refer to as $\text{Sch}/S$, ''schemes over $S$''. A typical case to keep in mind is $S = \text{Spec }k$ or $S = \text{Spec }\mathbb{Z}$. The product in $\text{Sch}/S$ is the fibre product $X \times_S Y$.

\subsection{Separated morphisms}
\textbf{Motivation.} A topological space $X$ is Hausdorff if and only if the diagonal $\Delta_X = \{(x,x) \mid x \in X\} \subset X \times X$ is closed. 

\begin{defn}
    Let $X \to S$ be a morphism of schemes. Then the \textbf{diagonal} is the morphism $\Delta_{X/S} : X \to X \times_S X$ induced by the univeral property via:
    \[
        \begin{tikzcd}[]
            X & &\\
            &X \times_S X & X\\
            & X & S
            \arrow["", from=2-3, to=3-3]
            \arrow["",from=2-2, to=3-2]
            \arrow["", from=3-2, to=3-3]
            \arrow["", from=2-2, to=2-3]
            \arrow["\exists !", from=1-1, to=2-2, dotted]
            \arrow["\text{id}_X",from=1-1, to=2-3, bend left=20]
            \arrow["\text{id}_X",from=1-1, to=3-2, bend right=30]
        \end{tikzcd}
    \]
    (We use $\Delta$ instead of $\Delta_{X/S}$ when $X$ and $S$ are clear).
\end{defn}
\textbf{Orientation.} If $U, V \subset X$ are opens and $S =\text{Spec }k$ for $k$ a field, then $\Delta^{-1}(U \times_S V) = U \cap V$.
\begin{defn}
    A morphism $X \to S$ is \textbf{separated} if $\Delta_{X/S} : X \to X \times_S X$ is a closed immersion.
\end{defn}
\begin{example}\label{example4.3}
    Say $X = \text{Spec }\mathbb{C}[t]$ and $S = \text{Spec }\mathbb{C}$, $X \to S$ is induced by $\mathbb{C} \to \mathbb{C}[t]$. Then $X \times_S X = \text{Spec}(\mathbb{C}[t] \otimes_\mathbb{C} \mathbb{C}[t])$, the diagonal $\Delta$ applied to $\text{Spec}(\mathbb{C}[t]\otimes_{\mathbb{C}} \mathbb{C}[t] \to \mathbb{C}[t])$ via the multiplication map (i.e. $\Delta$ means take spec of the map of rings given by multiplication).
    \vspace{1mm}
     
    $\Delta$ is closed because $\mathbb{C}[t]\otimes_\mathbb{C} \mathbb{C}[t] \to \mathbb{C}[t]$ is surjective.
\end{example}
\textbf{Clarificational category stuff.} There are two ways to describe fiber products, which look different, but are basically the same thing. These two are given by: $\begin{tikzcd}[]
    A \times_C B & A\\
    B & C
    \arrow["", from=1-2, to=2-2]
    \arrow["",from=1-1, to=2-1]
    \arrow["", from=2-1, to=2-2]
    \arrow["", from=1-1, to=1-2]
\end{tikzcd}$ and $\begin{tikzcd}[]
    A \times_C B & A \times B\\
    C & C \times C
    \arrow["", from=1-2, to=2-2]
    \arrow["",from=1-1, to=2-1]
    \arrow["\Delta", from=2-1, to=2-2]
    \arrow["", from=1-1, to=1-2]
\end{tikzcd}$.
\begin{prop}\label{prop4.3}
    Let $X \to S$ be a scheme morphism. Then there is a factorization of $\Delta_{X/S}$ given by $\begin{tikzcd}[]
        X & U & X \times_s X\\
        \arrow["", from=1-1, to=1-2]
        \arrow["",from=1-2, to=1-3]
        \arrow["\Delta_{X/S}", from=1-1, to=1-3, bend right=30]
    \end{tikzcd}$ such that the first arrow is a closed immersion and the second arrow is an open immersion, i.e. a ''locally closed immersion''.
\end{prop}
\begin{proof}
    Let $g : X \to S$. Say $S$ is covered by open affine schemes $\{V_i\}$ and suppose $X$ is covered by $\{U_{ij}\}$, where for fixed $i$, $\{U_{ij}\}$ cover $g^{-1}(V_i)$. We have morphisms $U_{ij} \to V_i$ induced by the fibre product $\begin{tikzcd}[]
        U_{ij} & g^{-1}(V_i) & V_i\\
        & X & S
        \arrow["", from=1-1, to=1-2]
        \arrow["", from=1-2, to=1-3]
        \arrow["", from=1-2, to=2-2]
        \arrow["", from=2-2, to=2-3]
        \arrow["", from=1-3, to=2-3]
    \end{tikzcd}$. 
    Now observe $U_{ij} \times_{V_i} U_{ij}$ is affine open in $X \times_S X$ and their union contains the image of $\Delta_{X/S}$. Also, $\Delta^{-1}(U_{ij} \times_{V_i} U_{ij}) = U_{ij} \subset X$. Take $U$ in the statement to be the union of $(U_{ij} \times_{V_i} U_{ij})$ over all $i,j$. The second map is clearly an open immersion. To check that a map $T \to T'$ is a closed immersion, we can do it locally on the codomain. For $U_{ij}$ affine, the diagonal gives a map $U_{ij} \to U_{ij}\times_{V_i} U_{ij}$, which is clearly a closed immersion (if this is not clear, see Example \ref{example4.3}).
\end{proof}
\begin{prop}
    If $X \to S$ is a morphism of affine schemes, then $\Delta_{X/S}$ is a closed immersion.
\end{prop}
\begin{proof}
    If $X = \text{Spec }A$, $S = \text{Spec }B$, $X \to S = \text{Spec }(B \to A)$, then $A \otimes_B A \to A$ is surjective.
\end{proof}
\begin{example}
    Recall the bug--eyed line $X = \mathbb{A}_k^1 \sqcup \mathbb{A}_k^1 / \sim$ with $\mathbb{A}_k^1 \setminus \{0\} = U \subset \mathbb{A}_k^1$ and $V$ similarly and the equivalence $V \stackrel{\sim}{\to} U$ given by $k[u,u^{-1}] \to k[t,t^{-1}]$ via $u \mapsto t$. We claim this is not separated over $\text{Spec }k$. What is $X \times_S X$? We can compute it via a gluing construction of the fibre product. Hence the output is a plane with doubled axes and four origins. But the diagonal only contains two out of the four origins (exercise), so it is not a closed subset.
\end{example}
\begin{example}
    As we will be able to check/work out in a few lectures' time, open and closed immersions are always separated.
\end{example}
An easy consequence of Proposition \ref{prop4.3}: Let $X \to S$ be a morphism of schemes. If $\text{Im}(\Delta_{X/S})$ is closed as a topological subspace, then $X \to S$ is separated.
\begin{prop}
    Let $A$ be any ring. Then the morphism $\mathbb{P}_A^n \to \text{Spec }A$ is separated.
\end{prop}
\begin{prop}
    Let $k$ be a field and $X \to \text{Spec }k$ a scheme morphism. Let $U, V \subset X$ be affine opens. Then if $X \to \text{Spec }k$ is separated, then $U \cap V$ is also affine.
\end{prop}
\marginpar{08 Nov 2022, Lecture 15}

\begin{example}
    Another example: $\mathbb{A}_k^n \to \text{Spec }k$ is separated, so $\mathbb{A}_k^n \times_{\text{Spec }k}\mathbb{A}_k^n = \text{Spec }k[x]\otimes_k k[y]$ and the map $\Delta$ is induced by multiplication $k[\overline{x}]\otimes k[\overline{y}] \to k[\overline{x}]$ by $f(x)\otimes g(y) \mapsto f(x)g(x)$.  
\end{example}    
\textbf{Properties.} (All of these are on example sheet 3).
\begin{itemize}
    \item Open and closed immersions are always separated. For example, for closed immersions, the key observation is $A/J \otimes_A A/I \cong A/(I+J)$, so $\Delta$ is the identity and so closed (see example sheet 3).
    \item Compositions of separated morphisms are separated.
    \item Base extensions: suppose $X \to S$ is separated and $S' \to S$ is arbitrary. Then the natural map $X \times_S S' \to S'$ coming from $\begin{tikzcd}[]
        X \times_S S' & X\\
        S' & S
        \arrow["", from=1-2, to=2-2]
        \arrow["",from=1-1, to=2-1]
        \arrow["", from=2-1, to=2-2]
        \arrow["", from=1-1, to=1-2]
    \end{tikzcd}$ through the fibre product is also separated.
\end{itemize} 
\begin{prop}
    Let $R$ be any ring. Then $\mathbb{P}_R^n \to \text{Spec }R$ is always separated.
\end{prop}
\begin{proof}
    We want to show that in our following fibre product $\Delta$ is closed.
    $$\begin{tikzcd}[]
        \mathbb{P}_R^n & \mathbb{P}_R^n \times_{R = \text{Spec }R} \mathbb{P}_R^n & \mathbb{P}_R^n\\
        & \mathbb{P}_R^n & \text{Spec }R
        \arrow["", from=1-1, to=1-2]
        \arrow["", from=1-2, to=1-3]
        \arrow["", from=1-2, to=2-2]
        \arrow["", from=2-2, to=2-3]
        \arrow["", from=1-3, to=2-3]
    \end{tikzcd}$$ 
    It suffices to check this on an open cover of $\mathbb{P}^n \times_R \mathbb{P}^n$ (note that here and before we abuse notation by writing $R$ instead of $\text{Spec }R$ as a subscript). Let $A = R[x_0,\ldots,x_n]$ with the usual grading. Let $U_i = \text{Spec} \left(A[\frac{1}{x_i}]\right)_0$. From our proj discussion, $\{U_i\}_{i=0}^n$ cover $\mathbb{P}_R^n$.
    \vspace{1mm}
     
    Now $U_i \times_R U_j = \text{Spec }R\left[\frac{x_0}{x_i},\ldots,\frac{x_n}{x_i},\frac{y_0}{y_j},\ldots, \frac{y_n}{y_j}\right]$. Now observe (exercise) that the restriction of $\Delta$ to $\Delta^{-1}(U_i \times_R U_j)$ is $U_i \cap U_j \to U_i \times_R U_j$, given on rings by $R\left[\frac{x_0}{x_i},\ldots,\frac{x_n}{x_i}\right]\left[\frac{x_i}{x_j}\right] \leftarrow R\left[\frac{x_0}{x_i},\ldots,\frac{x_n}{x_i},\frac{y_0}{y_j},\ldots, \frac{y_n}{y_j}\right]$ with the map ''change $y$ to $x$''. This map is clearly surjective and $U_i \times_R U_j$ cover $\mathbb{P}^n \times_R \mathbb{P}^n$, so $\Delta$ is closed.
\end{proof}

Let $k = \overline{k}$ be an algebraically closed field and let $X \to \text{Spec }k$ be a scheme over $\text{Spec }k$. Recall (from ES2) that $X$ is of finite type over $\text{Spec }k$ if there is a cover of $X$ by affines $\{U_\alpha\}_{\alpha}$ such that $\mathcal{O}_{X}(U_{\alpha})$ is a finitely generated $k$--algebra (i.e. a quotient of a polynomial ring by an ideal in finitely many variables). Also from ES2, $X$ is reduced if and only if for all opens $U \subset X$, $\mathcal{O}_{X}(U)$ has no nilpotent elements.

\begin{defn}
    $X \to \text{Spec }k$ is a \textbf{variety} if it is reduced, of finite type, and separated.
\end{defn}
\begin{example}
    For many examples, see the Part II Algebraic Geometry course or Chapter 1 of Hartshorne.
\end{example}
\subsection{Properness}
Let $f:X \to S$ be a morphism of schemes. Then $f$ is of \textbf{finite type} if there exists an affine cover of $S$ by opens $\{V_{\alpha}\}$ with each $V_{\alpha} = \text{Spec}(A_\alpha)$ and corresponding covers $\{U_{\alpha,\beta}\}_{\beta}$ of $f^{-1}(V_{\alpha})$ by open affines such that $U_{\alpha,\beta} = \text{Spec}(B_{\alpha,\beta})$ such that $B_{\alpha,\beta}$ is a finitely generated $A_{\alpha}$--module and $\{U_{\alpha,\beta}\}_{\beta}$ can be chosen to be finite.

\begin{defn}
    $f:X \to S$ is \textbf{closed} if it is a closed topological map. It is \textbf{universally closed} if for any $S' \to S$, the induced $X \times_S S' \to S'$ is also closed.
\end{defn}
\begin{defn}
    $f: X \to S$ is \textbf{proper} if it is separated, of finite type, and universally closed.
\end{defn}
\begin{example}
    Closed immersions are proper (check or wait to find out why this is the case later).
\end{example}
\textbf{Non--example.} The obvious map $\mathbb{A}_k^1 \to \text{Spec }k$ is not proper.
\begin{proof}
    It is separated and of finite type, but we show it's not universally closed. $\mathbb{A}_k^1 \to \text{Spec }k$ is closed (as it consists of one point). Consider the fibre product $$\begin{tikzcd}[]
        \mathbb{A}_k^2 & \mathbb{A}_k^1\\
        \mathbb{A}_k^1 & \text{Spec }k
        \arrow["f", from=1-2, to=2-2]
        \arrow["f'",from=1-1, to=2-1]
        \arrow["", from=2-1, to=2-2]
        \arrow["", from=1-1, to=1-2]
    \end{tikzcd}$$
    Take $\mathbb{V}(xy-1) =Z \subset \mathbb{A}_k^2 = \text{Spec }k[x,y]$. Then $f'(Z)$ is not Zariski closed.
\end{proof}
\textbf{Observation.} If $X \to S$ is proper, then any base extension $X \times_S S' \to S'$ is also proper.
\marginpar{10 Nov 2022, Lecture 16}

We have two notions ''separated'' and ''proper'' for morphisms. If $X \to \text{Spec }k$ is a morphism, then the terminology is to say ''$X$ is separated'' or ''$X$ is proper''.

\begin{example}
    \begin{itemize}
        \item $\mathbb{A}_k^1$ is separated, but not proper.
        \item The line with two origins is neither separated nor proper (as its not universally closed.)
    \end{itemize}
\end{example}
\begin{prop}
    Let $R$ be a commutative ring. Then $\mathbb{P}_R^n \to \text{Spec }R$ is proper.
\end{prop} 
\begin{proof}
    Universal closedness of $X \to S$ is stable under base extension, i.e. for $S' \to S$ universally closed, $X \times_S S' \to S'$ is also universally closed. Since we already checked that $\mathbb{P}^n_R \to \text{Spec }R$ is separated, and finite type is immediate by construction, it suffices to prove the case $R = \mathbb{Z}$ since $\mathbb{P}^n_R = \mathbb{P}^n_\mathbb{Z} \times_{\text{Spec }\mathbb{Z}} \text{Spec }R$.
    \vspace{1mm}
     
    We must show that for any $Y \to \text{Spec }\mathbb{Z}$, the base extension $\mathbb{P}^n_\mathbb{Z} \times_{\text{Spec }\mathbb{Z}} Y \to Y$ is closed. But $Y$ is covered by affine schemes of the form $\text{Spec }R$, and closedness is local on the target, it suffices to show that $\mathbb{P}^n_R \to \text{Spec }R$ is closed.
    
    Let $Z \subset \mathbb{P}^n_R$ be Zariski closed, i.e. it is the vanishing locus of homogeneous polynomials $g_1,g_2,\ldots$. Our goal: if $\pi: \mathbb{P}^n_R \to \text{Spec }R$, then we need to show $\pi(Z)$ is closed. Let $K(p) = FF(R/p)$. We have a morphism $\text{Spec }K(p) \to \text{Spec }R$. We want to know for which $p$ is $Z_p = Z \times_{\text{Spec }R} K(p)$ nonempty. But $Z_p$ is nonempty $\iff \overline{g}_1, \overline{g}_2,\ldots$ cut out the origin in $\mathbb{A}^{n+1}_{K(p)}$ in $\mathbb{P}^n_{K(p)}$. Thus $Z_p$ is nonempty $\iff \sqrt{(\overline{g}_1,\overline{g}_2,\ldots)} \not\supset (x_0,\ldots,x_n)$ (where $\mathbb{P}^n_R = \text{Proj }R[x_0,\ldots,x_n]$). Equivalently, for all positive integers $d$, $(x_0,\ldots,x_n)^d \not\subset (\overline{g}_1,\overline{g}_2,\ldots)$. 
    \vspace{1mm}
     
    Write $A = R[\overline{x}]$ with the usual grading. The non--containment is equivalent to the map $\bigoplus A_{d-\text{deg}(g_i)} \to A_d$ given by $f_i \mapsto f_i g_i$ in the $i^{\text{th}}$ factor to be non--surjective mod $p$ (or equivalently in $K(p)$) for all degrees $d$. This condition is given by the vanishing of maximal minors of the matrix associated to $\bigoplus A_{d-\text{deg}(g_i)} \to A_d$, which is infinitely many polynomials, each in the coefficients of the $g_i$.
\end{proof}

From this point onwards, we will assume that all schemes are Noetherian (i.e. they have a finite cover by Noetherian rings).

\subsection{Valuative criteria (for separatedness and properness)}
A discrete valuation ring is a local PID.
\begin{example}
    \begin{itemize}
        \item $\mathbb{C}[[t]]$ is a DVR.
        \item $\mathcal{O}_{\mathbb{A}^1,0} = \left\{\frac{f(t)}{g(t)} \mid g(0) \neq 0\right\}$ is a DVR.
        \item $\mathbb{Z}_{(p)}$, $\mathbb{Z}_p$ for $p$--adic integers are DVRs.
    \end{itemize}
\end{example}

\textbf{Terminology/observations.} Let $A$ be a valuation ring (it is a discrete valuation ring, since it is noetherian). Then $\text{Spec }A$ consists of two points $(0) \subset A$ and $\mathfrak{m} \subset A$, the unique maximal ideal.
\vspace{1mm}
 
The topology on $\text{Spec }A = \{(0), \mathfrak{m}\}$. $(0)$ is dense, and the closure of $\{(0)\} = \text{Spec }A$, and $\mathfrak{m}$ is closed.
\vspace{1mm}
 
Any generator $\pi$ for the maximal ideal $\mathfrak{m}$ is called either a \textbf{uniformizer} or \textbf{uniformizing parameter}.
\vspace{1mm}
 
Any element $a \in A$ can be written as $u \pi^k$ where $u$ is a unit and $k$ is unique. The integer $k$ is called the \textbf{valuation} of $a$. This gives a map $\text{val}: A\setminus \{0\} \to \mathbb{N}$ by $a \mapsto k$ (independent of the choice of $\pi$). 

\marginpar{13 Nov 2022, Lecture 17}

This allows us to construct a \textbf{valued field} on $K = FF(A)$, where the valuation extends to $K\setminus \{0\} \to \mathbb{Z}$ by $\frac{a}{b} \to \text{val}(a)-\text{val}(b)$.

\begin{example}
    $A = K[[t]]$, so $FF(A)=K((t))$ and the valuation of $a = \alpha_k t^k + \ldots + \alpha_0$ is $k$.
\end{example}
\begin{theorem}
    Let $f: X \to Y$ be a morphism of schemes. Then $f$ is separated if and only if for any (discrete) valuation ring $A$ with factor field $K$, given the diagram of solid arrows $\begin{tikzcd}[]
        \text{Spec }K & X\\
        \text{Spec }A & Y
        \arrow["", from=1-2, to=2-2]
        \arrow["",from=1-1, to=2-1]
        \arrow["", from=2-1, to=2-2]
        \arrow["", from=1-1, to=1-2]
        \arrow["", from=2-1, to=1-2, dotted]
    \end{tikzcd}$, there exists at most one choice of lift to fill in the dotted arrow.
    \vspace{1mm}
     
    Similarly, $f$ is universally closed if and only if there exists at least one choice of lift for the dotted arrow.
\end{theorem}
\begin{proof}
    Omitted and non--examinable.
\end{proof}
\begin{cor}
    \begin{enumerate}[(i)]
        \item $\mathbb{P}^n_R \to \text{Spec }R$ is proper.
        \item $\mathbb{A}^n_R \to \text{Spec }R$ is not proper, but it is separated.
        \item Closed immersions are proper, so in particular if $Z \to \mathbb{P}^n_R$ is closed, then $Z \to \text{Spec }R$ is proper.
        \item Compositions of proper (or separated) morphisms remain so.
        \item (Base extension) If $X \stackrel{f}{\to} Y$ is proper and $Y' \to Y$ is arbitrary, then $X \times_{Y} Y' \to Y'$ is also proper.
    \end{enumerate}
\end{cor}

Some sample verifications: (i) $\mathbb{A}^1_k \to \text{Spec }k$ is not proper (not universally closed). Write $\mathbb{A}^1_k = \text{Spec }k[x]$ and consider $A=k[[t]], K=k((t))$. We have the diagram $\begin{tikzcd}[]
    \text{Spec }k((t)) & \mathbb{A}^1_k\\
    \text{Spec }k[[t]] & \text{Spec }k
    \arrow["", from=1-2, to=2-2]
    \arrow["",from=1-1, to=2-1]
    \arrow["", from=2-1, to=2-2]
    \arrow["\phi", from=1-1, to=1-2]
\end{tikzcd}$ where we let $\phi$ be induced by $k[x] \to k((t))$ via $x \mapsto \frac{1}{t}$.

\textbf{Exercise.} Use the valuative criterion to show that if $\text{Spec }A$ is proper, then $\text{Spec }A$ is finite as a topological space.
\vspace{1mm}
 
Observe that if $\mathbb{A}^1_k$ is replaced with $\mathbb{P}^1_k$, then this is always an affine chart in $\mathbb{P}^1$ such that the map above looks like $x \mapsto t$. 

\section{Modules over $\mathcal{O}_X$}
\begin{example}
    Let $\mathbb{C} \mathbb{P}^n$ be the variety $C^{n+1}\setminus \{0\}/\sim \text{ modulo scaling}$. We have the structure sheaf: $\mathcal{O}_{\mathbb{C} \mathbb{P}^n}$, if $U \subset \mathbb{C} \mathbb{P}^n$ is Zariski open, then $\mathcal{O}_{\mathbb{C} \mathbb{P}^n}(U) = \left\{\frac{P(X)}{Q(X)} \mid P,Q \text{ homogeneous of same degree and the ratio is regular at all }p \in U\right\}$.
    \vspace{1mm}
     
    Also, for $d \in \mathbb{Z}$, consider a sheaf $\mathcal{O}_{\mathbb{C} \mathbb{P}^n}(d)$ with $$\mathcal{O}_{\mathbb{C} \mathbb{P}^n}(d)(U) = \left\{\frac{P(X)}{Q(X)} \mid P,Q \text{ homogeneous with deg}(P) = \text{deg}(Q) =d  \text{ and regularity at all }p \in U\right\}.$$
    Notice that $\mathcal{O}_{\mathbb{C} \mathbb{P}^n}(d)(U)$ is a $\mathcal{O}_{\mathbb{C} \mathbb{P}^n}(U)$--module.
\end{example}
\begin{example}
    Let $A$ be a ring and $M$ an $A$--module. Define a sheaf $\mathcal{F}_M$ on $\text{Spec }A$: if $U \subset \text{Spec }A$ is a distinguished open $U= U_{f}$, then set $\mathcal{F}_M(U)=M_f$ (on general opens use sheaf on a base construction).
\end{example}

\subsection{Definition of $\mathcal{O}_X$--modules}
Fix $(X, \mathcal{O}_X)$ a ringed space.
\begin{defn}
    A sheaf of $\mathcal{O}_X$--modules on $X$ is a sheaf $\mathcal{F}$ of groups together with a multiplication $\mathcal{F}(U) \times \mathcal{O}_X(U) \to \mathcal{F}(U)$ giving a module (compatible with restriction).
\end{defn}
Similarly we can build a sheaf of $\mathcal{O}_X$--algebras.

\begin{defn}
    A morphism between sheaves of modules $\phi : \mathcal{F} \to \mathcal{G}$ on $X$ is a homomorphism of sheaves of abelian groups compatible with multiplication.
\end{defn}

\end{document} 