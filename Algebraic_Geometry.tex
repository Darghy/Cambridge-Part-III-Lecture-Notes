\documentclass{article}
%build with recipe latexmk
\usepackage[utf8]{inputenc}
\usepackage[T1]{fontenc}
\usepackage{textcomp}
\usepackage{fancyhdr}
\pagestyle{fancy}

\usepackage{tcolorbox}
\tcbuselibrary{theorems}
\usepackage{babel}
\usepackage{enumerate}
\usepackage{stmaryrd}
\usepackage{amsmath, amssymb, amsthm}
%\usepackage{a4wide}
\usepackage{float}
\usepackage{tikz-cd}
\usepackage{tikz}
\usepackage{graphicx}
\usepackage{caption}
\usepackage{wrapfig}
\usepackage{setspace}
\setstretch{1.1}
\usepackage{color}
\usepackage{hyperref}
\hypersetup{
    colorlinks=true, %set true if you want colored links
    linktoc=all,     %set to all if you want both sections and subsections linked
    linkcolor=black,  %choose some color if you want links to stand out
}

\theoremstyle{definition}
\newtheorem{theorem}{Theorem}[section]
\newtheorem{lemma}[theorem]{Lemma}
\newtheorem{cor}[theorem]{Corollary}
\newtheorem{prop}[theorem]{Proposition}
\newtheorem{example}{Example}[section]
\newtheorem{defn}{Definition}[section]

\title{Part III - Algebraic Geometry
    \\ \large
    Lectured by Dhruv Ranganathan 
}
 
\author{Artur Avameri}
\date{}
 
\setcounter{section}{-1}
 
\begin{document}
\maketitle
\tableofcontents
\newpage
 
\section{Introduction}

\marginpar{6 Oct 2022, Lecture 1}


The course consists of four parts.
\begin{enumerate}[(1)]
    \item Basics of sheaves on topological spaces.
    \item Definition of schemes and morphisms.
    \item Properties of schemes (e.g. the algebraic geometry notion of compactness and other properties).
    \item A rapid introduction to the cohomology of schemes.
\end{enumerate}

The main reference for the course is Hartshorne's \textit{Algebraic Geometry}.

\section{Beyond algebraic varieties}
\marginpar{08 Oct 2022, Lecture 2}

\subsection{Summary of classical algebraic geometry}
We let $k = \overline{k}$ be a algebraically closed field and consider $\mathbb{A}_k^n = \mathbb{A}^n = k^n$ as a set.

\begin{defn}
    An \textbf{affine variety} is a subset $V \subset \mathbb{A}^n$ of the form $\mathbb{V}(S)$ with $S \subset k[x_1,\ldots,x_n]$, where $\mathbb{V}$ is the common vanishing locus.
\end{defn}
Note that $\mathbb{V}(S) = \mathbb{V}(I(S))$ (the ideal generated by $S$). By Hilbert Basis Theorem (since $k[x_1,\ldots,x_n]$ is noetherian), $\mathbb{V}(I(S)) = \mathbb{V}(S')$ for some finite set $S \subset k[x_1,\ldots,x_n]$.
\vspace{1mm}
 
In fact, $\mathbb{V}(I) = \mathbb{V}(\sqrt{I})$, where \[
\sqrt{I} = \{ f \in k[x_1,\ldots,x_n] \mid f^m \in I \text{ for some } m\ge 0\}
\] 
is the \textbf{radical} of $I$.
For example, in $k[x]$, if $I = (x^2)$, then $\sqrt{I} = (x)$.

\begin{defn}
    Given varieties $V \subset \mathbb{A}^n$ and $W \subset  \mathbb{A}^m$, a \textbf{morphism} is a (set-theoretic) map $\phi : V \to W \subset \mathbb{A}_k^m$ such that if $\phi = (f_1,\ldots,f_m)$, then each $f_i$ is the restriction of a polynomial in $\{x_1,\ldots,x_n\}$.
    \vspace{1mm}
     
    An \textbf{isomorphism} is a morphism with a two--sided inverse.  
\end{defn}

Our basic correspondence is 
\begin{align*}
    \{\text{Affine varieties over }k\}&/\text{up to isomorphism}\\  &\leftrightarrow \\ \{\text{finitely generated }k\text{--algebras }A &\text{ without nilpotent elements}\}
\end{align*}
A finitely generated $k$--algebra is just a quotient of a polynomial ring in finitely many variables. A nilpotent element is such that some power of it is zero. For example, in $k[x]/(x^2)$, the element $x$ is nilpotent.
\vspace{1mm}
 
How does this correspondence work? Given a variety $V$ (representing an isomorphism class), we write $V = \mathbb{V}(I)$ for $I \subset k[x_1,\ldots,x_n]$ a radical ideal\footnote{A radical ideal is an ideal equal to its radical.}, and map $V \mapsto k[x_1,\ldots,x_n]/I$.
\vspace{1mm}
 
For the reverse, if $A$ is a finitely generated nilpotent free algebra, then $A \cong k[y_1,\ldots,y_m]/J$ where we can choose $J$ to be radical (exercise: why?).
\vspace{1mm}
 
We have to check that this is independent of our choice on both sides (exercise: think through this, it should be clear).

\begin{defn}
    The algebra associated to $V$ is classically denoted $k[V]$ and called the \textbf{coordinate ring of} $V$.
\end{defn}

We have the compatibility of morphisms with our basic correspondence: there is a bijection between \[
\text{Morphisms}(V, W) \leftrightarrow \text{Ring homomorphisms}_k(k[W], k[V])
\]
(here $\text{RingHom}_k$ means that our homomorphisms preserve $k$).

We can now make our set into a topological space:
\begin{defn}
    Let $V = \mathbb{V}(I) \subset \mathbb{A}^n$ be a variety with coordinate ring $k[V]$. The \textbf{Zariski topology} on $V$ is defined such that the closed sets are $\mathbb{V}(S)$, where $S \subset k[V]$.
\end{defn}
If $V \cong W$, then the Zariski topological spaces are homeomorphic as varieties (exercise).

\begin{theorem}[Nullstellensatz]
    Fix $V$ a variety and let $k[V]$ be its coordinate ring. Given $p \in V$, we can produce a homomorphism $\text{ev}_p : k[V] \to k$ by sending $f \mapsto f(p)$. Note that $\text{ev}_p$ is surjective (since we have constant functions), hence $\text{ker}(\text{ev}_p) = m_p$ is a maximal ideal, giving us a map \[
        \{\text{points of }V\} \rightarrow \{\text{maximal ideals in }k[V]\}.
    \]
    Nullstellensatz says that this is actually a bijection. For the converse map, given $m \subset k[V]$, we get a quotient $k[V] \to k[V]/m = k$ (Nullstellensatz says this extension is finite, hence must be $k$). So using/choosing a representation for $V$ in $k[x_1,\ldots,x_n]$ gives a surjective homomorphism onto $k$ and specifies a bunch of points.
\end{theorem}
\subsection{Limitations of classical algebraic geometry}
\textbf{Question.} What is an abstract variety, i.e. ''some ''space'' $X$ such that  locally as a cover $\{U_i\}$, each $U_i$ is an affine variety, compatible with overlaps''. 

\begin{example}[non--algebraically closed fields]
    Take $I = (x^2+y^2+1) \subset \mathbb{R}[x,y]$. Then $\mathbb{V}(I) = \varnothing \subset \mathbb{R}^2$, but $I$ is prime, so radical, so nullstellensatz fails.
\end{example}

\textbf{Question.} On what topological space is $\mathbb{R}[x,y]/(x^2+y^2+1)$ ''naturally'' the set of functions? (or $\mathbb{Z}$, or $\mathbb{Z}[x]$).

\begin{example}[Why restrict to radical ideals?]
    Take $C = \mathbb{V}(y-x^2) \subset \mathbb{A}_k^2$ and $D = \mathbb{V}(x,y)$, so $C \cap D = \mathbb{V}(y, y-x^2) = \mathbb{V}(x,y) = \{(0,0)\}$. This is a single point, but if $D_\delta = \mathbb{V}(y+\delta)$ for some $\delta \in k$, then $C \cap D_\delta = \{\pm \sqrt{\delta}\}$, which is 2 points for all $\delta \neq 0$. In other words, intersections of varieties don't want to be varieties.
\end{example}

\subsection{The spectrum of a ring}

\marginpar{11 Oct 2022, Lecture 3}

Let $A$ be a commutative ring with identity. We will define a topological space on which $A$ is the ring of functions.

\begin{defn}
    The \textbf{Zariski spectrum} of $A$ is $$\text{Spec }A = \{\mathfrak{p} \subset A \mid \mathfrak{p} \text{ is a prime ideal}\}.$$
\end{defn}
A ring homomorphism $\phi : A \to B$ induces a map $\phi^{-1} : \text{Spec }B \to \text{Spec }A$ by $q \mapsto \phi^{-1}(q)$. In general, the preimage of a prime ideal is a prime ideal.
\vspace{1mm}
 
\textbf{Warning.} This would fail if we only considered maximal ideals, since the preimage of a maximal ideal need not be maximal. 
\vspace{1mm}
 
Given $f \in A$ and $\mathfrak{p} \in \text{Spec}(A)$, we have an induced $\overline{f} \in A/\mathfrak{p}$ obtained via a quotient. Informally, we can evaluate any $f \in A$ at points $\mathfrak{p} \in \text{Spec}(A)$ with the caveat that the codomain of this evaluation depends on $\mathfrak{p}$.

\begin{example}
    Take $A = \mathbb{Z}$. Then $\text{Spec }A  = \text{Spec }(\mathbb{Z}) = \{p \mid p \text{ is prime}\} \cup \{(0)\}$. Let's pick an element in $\mathbb{Z}$, say $132 \in \mathbb{Z}$. Given a prime $p$, we can look at ${132\pmod{p}} \in \mathbb{Z}/p$. The takeaway here is that 
    \begin{align*}
        \text{Spec }\mathbb{Z} &\rightarrow \text{Space}\\
        132 \in \mathbb{Z} &\rightarrow \text{a function}\\
        132 \text{ (mod }p) &\rightarrow \text{value of that function at }p.
    \end{align*}
    Note that based on the value of $p$, our codomain changes from point to point.
\end{example}
\begin{example}
    Take $A = \mathbb{R}[x]$, then $\text{Spec }\mathbb{R}[x] = \mathbb{C}/ \text{complex conjugation} \cup \{(0)\}$. 
\end{example}
\textbf{Exercise.} Draw $\text{Spec }\mathbb{Z}[x]$ and $\text{Spec }k[x]$ for $k$ any field (i.e. describe all prime ideals and their containment). This is on example sheet 1.
\begin{example}
    If $A = \mathbb{C}[x]$, then $\text{Spec }A = \mathbb{C} \cup\{(0)\}$, where given $a \in \mathbb{C}$, we send it to the maximal ideal $\langle z-a\rangle$.
\end{example}
\subsection{A topology on Spec $A$}
Fix $f \in A$. Then $\mathbb{V}(f) = \{\mathfrak{p} \in \text{Spec }A \mid f \equiv 0 \pmod{\mathfrak{p}}\} \subset \text{Spec }A$. (Note that $f \equiv 0 \pmod{\mathfrak{p}}$ is the same as $f  \in \mathfrak{p}$). \vspace{1mm}
 
Similarly for $J \subset A$ an ideal, $\mathbb{V}(J) = \{\mathfrak{p} \in \text{Spec }A \mid f \in \mathfrak{p} ~\forall f \in J\}$.

\begin{prop}
    The sets $\mathbb{V}(J) \subset \text{Spec }A$ ranging over all ideals $J$ form the closed sets of a topology on $\text{Spec }A$. This topology is called the \textbf{Zariski topology}.
\end{prop}
\begin{proof}
    Easy fact: $\varnothing$ and $\text{Spec }A$ are closed, since we have functions $1$ (vanishing nowhere) and $0$ (vanishing everywhere). Since $\mathbb{V}(\sum_{\alpha}^{} I_\alpha) = \bigcap_\alpha \mathbb{V}(I_\alpha)$ (this is because $I_1 + I_2$ is the smallest ideal containing $I_1 \cup I_2$), arbitrary intersections are closed.
    \vspace{1mm}
     
    Finally, we claim $\mathbb{V}(I_1) \cup \mathbb{V}(I_2) = \mathbb{V}(I_1 \cap I_2)$. The containment $\subset $ is clear: if a prime ideal contains $I_1$ or $I_2$, it contains $I_1 \cap I_2$. Conversely, $I_1I_2 \subset I_1 \cap I_2$, so if $I_1 I_2 \subset I_1 \cap I_2 \subset \mathfrak{p}$, then by primality $I_1 \subset \mathfrak{p}$ or $I_2 \subset \mathfrak{p}$. 
\end{proof}

\begin{example}
    Let $k = \mathbb{C}$ and consider $\text{Spec }\mathbb{C}[x,y]$. We make a few observations:
    \begin{itemize}
        \item The point $(0) \in \text{Spec }\mathbb{C}[x,y]$ is dense in the Zariski topology, i.e. $\overline{\{(0)\}} = \text{Spec }\mathbb{C}[x,y]$ because every prime ideal contains $(0)$ (because we are in an integral domain).
        \item Consider the prime ideal $(y^2-x^3)$ (which is prime since the quotient is an integral domain). Consider a maximal ideal $\mathfrak{m}_{a,b} = (x-a, y-b)$. We can ask: when is $\mathfrak{m}_{a,b} \in \overline{\{(y^2-x^3)\}}$? The answer: if and only if $b^2 = a^3$, e.g. $(1,1)$ (see example sheet 1). The lesson here is that points are not closed in the Zariski topology.
    \end{itemize}
\end{example}
\subsection{Functions on opens}
\begin{defn}
    Let $f \in A$. Define the \textbf{distinguished open} corresponding to $f$ to be \[
    \mathcal{U}_{f} = (\text{Spec}(A))/\mathbb{V}(f).
    \]
\end{defn}
\begin{example}
    \begin{itemize}
        \item Let $A = \mathbb{C}[x]$, so $\text{Spec }A = \mathbb{C} \cup \{(0)\}$ (with the Zariski topology). Take $f = x$ and consider $\mathcal{U}_x$. Recall the bijection $\text{Spec }\mathbb{C} \leftrightarrow \mathbb{C} \cup \{(0)\}$ by $(x-a) \mapsfrom a \in \mathbb{C}$ and $(0) \mapsfrom (0)$. Then $\mathbb{V}(x) = \{\mathfrak{p} \in \text{Spec }A \mid x \in \mathfrak{p}\} = \{(x)\}$, so $\mathcal{U}_f = \text{Spec }A \setminus \{(x)\}$.
        \item More generally, suppose we fix $a_1,\ldots,a_r \in \mathbb{C}$, then $\text{Spec }A\setminus \{(x-a_i)\}_{i=1}^r = \mathcal{U}$ and $\mathcal{U} = \mathcal{U}_f$, where $f = \prod_{i=1}^{r} (x-a_i)$.
    \end{itemize}
\end{example}
\begin{lemma}
    The distinguished opens $\mathcal{U}_f$ taken over all $f \in A$ form a basis for the Zariski topology on $\text{Spec }A$.
\end{lemma}
\begin{proof}
    Left as an exercise on example sheet 1.
\end{proof}

A bit of commutative algebra: 
\begin{defn}
    Given $f \in A$, the \textbf{localization of $A$ at $f$} is $A_f = A[x]/(xf-1)$, which we can informally think of as $A_f = A[\frac{1}{f}]$.
\end{defn}
\begin{lemma}
    The distinguished open $\mathcal{U}_f \subset \text{Spec }A$ is naturally homeomorphic to $\text{Spec }A_f$.
\end{lemma}

\end{document}