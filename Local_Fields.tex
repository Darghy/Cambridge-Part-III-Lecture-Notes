\documentclass{article}
%build with recipe latexmk
\usepackage[utf8]{inputenc}
\usepackage[T1]{fontenc}
\usepackage{textcomp}
\usepackage{fancyhdr}
\pagestyle{fancy}

\usepackage{tcolorbox}
\tcbuselibrary{theorems}
\usepackage{babel}
\usepackage{enumerate}
\usepackage{mathtools}
\usepackage{amsmath, amssymb, amsthm}
%\usepackage{a4wide}
\usepackage{float}
\usepackage{tikz-cd}
\usepackage{tikz}
\usepackage{stmaryrd}
\usepackage{graphicx}
\usepackage{caption}
\usepackage{wrapfig}
\usepackage{setspace}
\setstretch{1.1}
\usepackage{color}
\usepackage{hyperref}
\hypersetup{
    colorlinks=true, %set true if you want colored links
    linktoc=all,     %set to all if you want both sections and subsections linked
    linkcolor=black,  %choose some color if you want links to stand out
}

\theoremstyle{definition}
\newtheorem{theorem}{Theorem}[section]
\newtheorem{lemma}[theorem]{Lemma}
\newtheorem{cor}[theorem]{Corollary}
\newtheorem{prop}[theorem]{Proposition}
\newtheorem{example}{Example}[section]
\newtheorem{defn}{Definition}[section]

\title{Part III - Local Fields
    \\ \large
    Lectured by Rong Zhou 
}
 
\author{Artur Avameri}
\date{Michaelmas 2023}
 
\setcounter{section}{-1}
 
\begin{document}
\maketitle
\tableofcontents
\newpage
 
\section{Introduction}

This is a first class in graduate algebraic number theory. Something we'd like to do is solve diophantine equations, e.g. $f(x_1,\ldots,x_r) \in \mathbb{Z}[x_1,\ldots,x_r]$. In general, solving $f(x_1,\ldots,x_r) = 0$ is very difficult. A simpler question we might consider is solving $f(x_1,\ldots,x_r) \equiv 0 \pmod{p}$, or $\pmod{p^2}, \pmod{p^3}$, etc. Local fields package all of this information together. 

\section{Absolute values}
\begin{defn}
    Let $K$ be a field. An \textbf{absolute value} on $K$ is a function $|\cdot| : K \to \mathbb{R}_{\ge 0}$ satisfying:
    \begin{enumerate}[(1)]
        \item $|x| = 0 \iff x = 0$.
        \item $|xy| = |x||y| ~\forall x,y \in K$.
        \item $|x+y|\le |x|+|y| ~\forall x,y \in K$ (triangle inequality).
    \end{enumerate}
\end{defn}
We say that $(K, |\cdot|)$ is a \textbf{valued field}.
Examples:
\begin{itemize}
    \item Take $K=\mathbb{Q},\mathbb{R},\mathbb{C}$ with the usual absolute value $|a+ib| = \sqrt{a^2+b^2}$. We call this $|\cdot |_{\infty}$.
    \item For $K$ any field, we have the trivial absolute value $|x| = \begin{cases}
        0 & \text{ if } x = 0\\
        1 & \text{ else.}
    \end{cases}$
    We will ignore this in this course.
    \item Take $K= \mathbb{Q}$ and $p$ a prime. For $0 \neq x \in \mathbb{Q}$, write $x = p^n \frac{a}{b}$ where $(a,p)=(b,p)=1$. Then the \textbf{$p$--adic absolute value} is defined to be \[
    |x|_p = \begin{cases}
        0 & x=0\\
        p^{-n} & x = p^n\frac{a}{b}.
    \end{cases}
    \] 
    We can check the axioms:
    \begin{enumerate}[(1)]
        \item The first axiom is clear.
        \item $$|xy|_p = \left|p^{n+m}\frac{ac}{bd}\right|_p = p^{-(n+m)} = |x|_p|y|_p.$$
        \item WLOG let $m\ge n$. Then \[
        |x+y|_p = \left|p^n\left(\frac{ad+p^{m-n}bc}{bd}\right)\right|_p \le p^{-n} = \max(|x|_p, |y|_p).
        \]
    \end{enumerate}
\end{itemize}

Any absolute value $|\cdot |$ on $K$ induces a metric $d(x,y) = |x-y|$ on $K$, hence induces a topology on $K$.

\begin{defn}
    Suppose we have two absolute values $|\cdot |, |\cdot |'$ on $K$. We say these absolute values are \textbf{equivalent} if they induce the same topology. An equivalence class is called a \textbf{place}.
\end{defn}
\begin{prop}
    Let $|\cdot|, |\cdot|'$ be (nontrivial) absolute values on $K$. Then the following are equivalent:
    \begin{enumerate}[(i)]
        \item $|\cdot|$ and $|\cdot|'$ are equivalent.
        \item $|x| < 1 \iff |x|' < 1 ~\forall x \in K$.
        \item $\exists c \in \mathbb{R}_{>0}$ such that $|x|^c = |x'| ~\forall x \in K$.
    \end{enumerate}
\end{prop}
\begin{proof}
    (i) $\implies $(ii): $|x|<1 \iff x^n \to 0$ with respect to $|\cdot| \iff x^n \to 0$ with respect to $|\cdot|'$ (since the topologies are the same) $\iff |x|'<1$. 
    \vspace{1mm}
     
    (ii) $\implies $(iii): Note that $|x|^c = |x|' \iff c \log |x| = \log |x|'$. Take $a \in K^\times$ such that $|a| > 1$. This exists since $|\cdot|$ is nontrivial. We need to show that $\forall x \in K^\times,$\[
     \frac{\log |x|}{\log |a|} = \frac{\log|x|'}{\log|a|'}.
    \]
    Assume $\frac{\log |x|}{\log |a|} < \frac{\log|x|'}{\log|a|'}.$ Choose $m, n \in \mathbb{Z}$ such that $\frac{\log |x|}{\log |a|} < \frac{m}{n} < \frac{\log|x|'}{\log|a|'}.$ We then have 
    \begin{align*}
        &\begin{cases}
            &n \log |x| < m \log |a|\\
            &n \log |x|' > m \log|a|'
        \end{cases}\\
        \implies & \left|\frac{x^n}{a^m}\right|<1, \left|\frac{x^n}{a^m}\right|' > 1,
    \end{align*}
    a contradiction. The other inequality is analogous.
    \vspace{1mm}
     
    (iii) $\implies$ (i): Clear, since they have the same open balls.
\end{proof}

\textbf{Remark.} $|\cdot|_{\infty}^2$ on $\mathbb{C}$ is not an absolute value by our definition (doesn't satisfy the triangle inequality). Some authors replace the triangle inquality by the condition $|x+y|^{\beta} \le |x|^{\beta} + |y|^{\beta}$ for some fixed $\beta \in \mathbb{R}_{>0}$. The equivalence classes are the same in either case.

In this course, we will mainly be interested in the following:

\begin{defn}
    An absolute value $|\cdot|$ on $K$ is said to be \textbf{non-archimedean} if it satisfies the \textbf{ultrametric inequality} \[
    |x+y|\le \max(|x|,|y|).
    \]
    If $|\cdot|$ is not non-archimedean, we say it is \textbf{archimedean}. 
\end{defn}
\begin{example}
    \begin{itemize}
        \item $|\cdot|_{\infty}$ on $\mathbb{R}$ is archimedean.
        \item $|\cdot|_{p}$ on $\mathbb{Q}$ is non--archimedean.
    \end{itemize}
\end{example}

\begin{lemma}[All triangles are isosceles]
    Let $(K, |\cdot|)$ be non--archimedean and $x,y \in K$. If $|x|<|y|$, then $|x-y|= |y|$.
\end{lemma}
\begin{proof}
    On the one hand, $|x-y|\le \max(|x|,|y|) = |y|$ (using $|x|=|-x|$).
    \vspace{1mm}
     
    On the other, $|y| \le \max(|x|, |x-y|) = |x-y|$.
\end{proof}

Convergence is easier in non--archimedean fields:
\begin{prop}
    Let $(K,|\cdot|)$ be non--archimedean and $(x_n)_{n=1}^{\infty}$ a sequence on $K$. If $|x_n-x_{n+1}| \to 0$, then $(x_n)_{n=1}^{\infty}$ is Cauchy. In particular, if $K$ is complete, then the sequence converges.
\end{prop}
\begin{proof}
    For $\epsilon > 0 $, choose $N$ such that $|x_n-x_{n+1}| < \epsilon$ for $n \ge N$. Then for $N<n<m$, \[
    |x_n - x_m| = |(x_n - x_{n+1}) + (x_{n+1} - x_{n+2}) + \ldots + (x_{m-1}-x_m)| < \epsilon,
    \]
    so $(x_n)$ is Cauchy.
\end{proof}
\begin{example}
    For $p=5$, we can construct a sequence in $\mathbb{Q}$ satisfying:
    \begin{enumerate}[(i)]
        \item $x_n^2+1 \equiv  0 \pmod{5^n}$,
        \item $x_n \equiv x_{n+1} \pmod{5^n}$.
    \end{enumerate}
    We construct it by induction. Take $x_1=2$. Now suppose we've constructed $x_n$ and write $x_n^2+1 = a\cdot 5^n$ and set $x_{n+1} = x_n + b\cdot 5^n$. We compute \[
    x_{n+1}^2 + 1 = x_n^2 + 2b x_n 5^n + b^2 5^{2n} + 1 = a5^n + 2bx_n 5^n + \underbrace{b^2 5^{2n}}_{\equiv 0 \text{ (mod }5^{n+1})} + 1. 
    \]
    Hence we choose $b$ such that $a+2bx_n \equiv 0 \pmod{5}$ and we're done. 
    \vspace{1mm}
     
    Now (ii) tells us that $(x_n)$ is Cauchy, but we claim it doesn't converge. Suppose it does, $x_n \to l \in \mathbb{Q}$. Then $x_n^2 \to l^2 \in \mathbb{Q}$. But by (i), $x_n^2 \to -1$, so $l^2 = -1$, a contradiction.
\end{example}
This tells us that $(\mathbb{Q}, |\cdot|_{5})$ is not complete.
\begin{defn}
    The $p$--adic numbers $\mathbb{Q}_p$ are the completion of $\mathbb{Q}$ with respect to $|\cdot|_p$.
\end{defn}

\marginpar{10 Oct 2022, Lecture 2}

Let $(K,|\cdot|)$ be a non--archimedean valued field. For $x \in K$ and $r \in \mathbb{R}_{>0}$, we define $B(x,r) = \{y \in K \mid |y-x|<r\}$ and $\overline{B} = \{y \in K \mid |y-x|\le r\}$ to be the open and closed balls of radius $r$.

\begin{lemma}
    \begin{enumerate}[(i)]
        \item If $z \in B(x,r)$, then $B(z,r) = B(x,r)$, i.e. open balls don't have centers.
        \item If $z \in \overline{B}(x,r)$, then $\overline{B}(x,r) = \overline{B}(z,r)$.
        \item $B(x,r)$ is closed.
        \item $\overline{B}(x,r)$ is open.
    \end{enumerate}
\end{lemma}
\begin{proof}
    \begin{enumerate}[(i)]
        \item Let $y \in B(x,r)$. Then $|x-y|<r \implies |z-y| = |(z-x) + (x-y)| \le \max(|z-x|,|x-y|) < r$, so $B(x,r) \subset B(z,r)$. The reverse inclusion is analogous.
        \item Analogous to (i) by replacing $<$ with $\le$.
        \item Let $y \in K \setminus B(x,r)$. If $z \in B(x,r) \cap B(y,r)$, then $B(x,r) = B(z,r) = B(y,r)$ by (i), so $y \in B(x,r)$, a contradiction. Hence $B(x,r) \cap B(y,r) = \varnothing$. Since $y$ was arbitrary, $K \setminus B(x,r)$ is open, so $B(x,r)$ is closed.
        \item If $z \in \overline{B}(x,r)$, then $B(z,r) \subset \overline{B}(z,r) \stackrel{\text{(ii)}}{=} \overline{B}(x,r)$.
    \end{enumerate}
\end{proof}

\section{Valuation rings}
\begin{defn}
    Let $K$ be a field. A \textbf{valuation} on $K$ is a function $v : K^\times \to \mathbb{R}$ such that 
    \begin{enumerate}[(i)]
        \item $v(xy) = v(x)+v(y)$.
        \item $v(x+y) \ge \min(v(x),v(y))$.
    \end{enumerate}
\end{defn}
Fix $0<\alpha<1$. If $v$ is a valuation on $K$, then $|x| = \begin{cases}
    \alpha^{v(x)} & x \neq 0\\
    0 & x = 0
\end{cases}$
determines a non--archimedean absolute value on $K$. Conversely, a non--archimedean absolute value on $K$ determines a valuation $v(x) = \log_{\alpha} |x|$.

\textbf{Remark.} We ignore the trivial evaluation $v(x)=0 ~\forall x \in K$, which  corresponds to the trivial absolute value.

\begin{defn}
    We say valuations $v_1,v_2$ are equivalent if $\exists c \in \mathbb{R}_{>0}$ such that $v_1(x) = cv_2(x) ~\forall x \in K^{\times}$.
\end{defn}
\begin{example}
    \begin{itemize}
        \item If $K=\mathbb{Q}$, $v_p(x) = -\log_p |x|_p$ is the $p$--adic valuation.
        \item Let $k$ be a field. Let $K=k(t) = \text{Frac}(k[t])$ be a rational function field. We let \[
        v\left(t^n \frac{f(t)}{g(t)}\right) = n
        \]
        for $f, g \in k[t]$, $f(0)\neq 0, g(0) \neq 0$. This is called a $t$--adic valuation.
        \item Let $K = k((t)) = \text{Frac}(k[[t]]) = \{\sum_{i=n}^{\infty} a_i t^i \mid a_i \in k, n \in \mathbb{Z}\}$, the field of formal Laurent series over $k$. We define \[
        v\left(\sum_{i}^{} a_i t^i\right) = \min \{i \mid a_i \neq 0\},
        \]
        the $t$--adic valuation on $K$.
    \end{itemize}
\end{example}
\begin{defn}
    Let $(K, |\cdot|)$ be a non--archimedean valued field. The \textbf{valuation ring} of $K$ is defined to be \[
    \mathcal{O}_K = \{x \in K \mid |x|\le 1\}.
    \]
    (i.e. the closed unit ball, $\mathcal{O}_K = \overline{B}(0,1)$, or ${\mathcal{O}_K = \{x \in K^{\times} \mid v(x) \ge  0\} \cup \{0\}}$).
\end{defn}
\begin{prop}
    \begin{enumerate}[(i)]
        \item $\mathcal{O}_K$ is an open subring of $K$.
        \item The subsets $\{x \in K \mid |x|\le r\}$ and $\{x \in K \mid |x|< r\}$ for $r\le 1$ are open ideals in $\mathcal{O}_K$.
        \item $\mathcal{O}_K^\times = \{x \in K \mid |x|=1\}$.
    \end{enumerate}
\end{prop}
\begin{proof}
    \begin{enumerate}[(i)]
        \item We find:
        \begin{itemize}
            \item $|0|=0$ and $|1|=1$, so $0,1 \in \mathcal{O}_K$.
            \item If $x \in \mathcal{O}_K$, then $|-x|=|x| \implies -x \in \mathcal{O}_K$. 
            \item If $x,y \in \mathcal{O}_K$, then $|x+y|\le \max(|x|,|y|)\le 1$, so $x+y \in \mathcal{O}_K$. 
            \item If $x,y \in \mathcal{O}_K$, then $|xy| = |x||y| \le 1$, so $xy \in \mathcal{O}_K$.
        \end{itemize}
        Thus $\mathcal{O}_K$ is a subring, and since $\mathcal{O}_K = \overline{B}(0,1)$, it is open.
        \item As $r\le 1$, $\{x \in K \mid |x|\le r\} = \overline{B}(0,r) \subset \mathcal{O}_K$, so it is open. We find:
        \begin{itemize}
            \item If $x, y \in \overline{B}(0,r)$, then $|x+y|\le \max(|x|,|y|) \le r$, so $x+y \in \overline{B}_r$.
            \item If $x \in \mathcal{O}_K, y \in \overline{B}_r$, then $|xy|=|x||y|\le 1 \cdot |y|\le r$, so $xy \in \overline{B}_r$.
        \end{itemize}
        Hence this is an open ideal. The proof for $\{x \in K \mid |x|< r\}$ is analogous.
        \item Note that $|x||x^{-1}|= |x x^{-1}| = 1$. Thus $|x| = 1 \iff |x^{-1}| = 1 \iff x,x^{-1} \in \mathcal{O}_K \iff x \in \mathcal{O}_K^\times$.
    \end{enumerate}
\end{proof}

\textbf{Notation.} Let $\mathfrak{m} = \{x \in \mathcal{O}_K \mid |x|<1\}$. It turns out this is a maximal ideal in $\mathcal{O}_K$. Also let $k = \mathcal{O}_K/\mathfrak{m}$, the residue field. 

\begin{cor}
    $\mathcal{O}_K$ is a \textbf{local ring} (i.e. a ring with a unique maximal ideal) with unique maximal ideal $\mathfrak{m}$.
\end{cor}
\begin{proof}
    Let $\mathfrak{m'}$ be a maximal ideal. If $\mathfrak{m'} \neq \mathfrak{m}$, then $\exists x \in \mathfrak{m'}\setminus \mathfrak{m}$. Hence $|x|= 1$, so by (iii) above, $x$ is a unit, so $\mathfrak{m'}=\mathcal{O}_K$, a contradiction.
\end{proof}

\begin{example}
    $K=\mathbb{Q}$ with $|\cdot|_p$. Then $\mathcal{O}_K = \mathbb{Z}_{(p)} = \{\frac{a}{b} \in \mathbb{Q} \mid p \nmid b\}$. In this case, $\mathfrak{m} = p\mathbb{Z}_{(p)}$ and $k = \mathbb{F}_p$.
\end{example}

\begin{defn}
    Let $v:K^{\times} \to \mathbb{R}$ be a valuation. If $v(K^{\times}) \cong \mathbb{Z}$, then we say $v$ is a \textbf{discrete valuation}. In this case, $K$ is said to be a \textbf{discretely valued field}. 
    \vspace{1mm}
     
    An element $\pi \in \mathcal{O}_K$ is said to be a \textbf{uniformizer} if $v(\pi)>0$ and $v(\pi)$ generates $v(K^\times)$.
\end{defn}
\begin{example}
    \begin{itemize}
        \item $K=\mathbb{Q}$ with the $p$--adic valuation and $K=k(t)$ with the $t$--adic valuation are discretely valued fields.
        \item $K=k(t)(t^{\frac{1}{2}}, t^{\frac{1}{4}}, t^{\frac{1}{8}},\ldots)$ with the $t$--adic valuation is not a discretely valued field.
    \end{itemize}
\end{example}
\textbf{Remark.} If $v$ is a discrete valuation, we can scale $v$, i.e. replace it with an equivalent valuation such that $v(K^\times) = \mathbb{Z}$. Such $v$ are called \textbf{normalized valuations}. Then $\pi$ is a uniformizer $\iff v(\pi) = 1$.

\begin{lemma}
    Let $v$ be a valuation on $K$. Then the following are equivalent:
    \begin{enumerate}[(i)]
        \item $v$ is discrete;
        \item $\mathcal{O}_K$ is a PID;
        \item $\mathcal{O}_K$ is Noetherian;
        \item $\mathfrak{m}$ is principal.
    \end{enumerate}
\end{lemma}
\begin{proof}
    (i) $\implies$ (ii): $\mathcal{O}_K \subset K$, so $\mathcal{O}_K$ is an integral domain. Let $I \subset \mathcal{O}_K$ be a nonzero ideal and pick $x \in I$ such that $v(x) = \min \{v(a) \mid a \in I, a \neq 0\}$, which exists as $v$ is discrete. Then we claim that $x \mathcal{O}_K = \{a \in \mathcal{O}_K \mid v(a)\ge v(x)\}$ is equal to $I$. The inclusion $x\mathcal{O}_K\subset I$ is clear, as $I$ is an ideal. For $x\mathcal{O}_K\supset I$, let $y \in I$, then $v(x^{-1}y) = v(y) - v(x) \ge 0 \implies y = x(x^{-1}y) \in x \mathcal{O}_K$.
    \vspace{1mm}
     
    (ii) $\implies $ (iii): Clear, as being a PID means every ideal is generated by one element, i.e. by finitely many. 
    \vspace{1mm}
     
    (iii) $\implies $ (iv): Write $\mathfrak{m} = x_1 \mathcal{O}_K +\ldots+ x_n \mathcal{O}_K$ and WLOG assume $v(x_1)\le v(x_2)\le \ldots\le v(x_n)$. Then $x_2,\ldots,x_n \in x_1 \mathcal{O}_K$, since $x_1 \mathcal{O}_K = \{a \in \mathcal{O}_K \mid v(a) \ge v(x_1)\}$, so $\mathfrak{m} = x_1 \mathcal{O}_K$.
    \vspace{1mm}
     
    (iv) $\implies $ (i): Let $\mathfrak{m} = \pi \mathcal{O}_K$ for some $\pi \in \mathcal{O}_K$ and let $c = v(\pi)$. Then if $v(x)>0$, i.e. $x \in \mathfrak{m}$, then $v(x) \ge c$. Thus $v(K^\times) \cap (0,c) = \varnothing$. Since $v(K^\times)$ is a subgroup of $(\mathbb{R}, +)$, we have $v(K^{\times}) = c\mathbb{Z}$.
\end{proof}

\marginpar{12 Oct 2022, Lecture 3}

\textbf{Remark.} Let $v$ be a discrete valuation on $K$, $\pi \in \mathcal{O}_K$ a uniformizer. For $x \in K^{\times}$, let $n \in \mathbb{Z}$ such that $v(x) = n v(\pi)$. Then $u = x \pi^{-n} \in \mathcal{O}_K^{\times}$ and $x = u \pi^n$. In particular, $K = \mathcal{O}_K \left[\frac{1}{\pi}\right]$ and hence $K = \text{Frac}(\mathcal{O}_K)$.

\begin{defn}
    A ring $R$ is called a \textbf{discrete valuation ring} (DVR) if it is a PID with exactly one nonzero prime ideal (which is then necessarily maximal).
\end{defn}
\begin{lemma}
    \begin{enumerate}[(i)]
        \item Let $v$ be a discrete valuation on $K$. Then $\mathcal{O}_K$ is a DVR.
        \item Let $R$ be a DVR. Then there exists a valuation $v$ on $K = \text{Frac}(R)$ such that $R = \mathcal{O}_K$.
    \end{enumerate}
\end{lemma}
\begin{proof}
    \begin{enumerate}[(i)]
        \item $\mathcal{O}_K$ is a PID by the previous lemma, hence any nonzero prime ideal is maximal. Since $\mathcal{O}_K$ is a local ring, it is a DVR.
        \item Let $R$ be a DVR with maximal ideal $\mathfrak{m}$. Then $\mathfrak{m} = (\pi)$ for $\pi \in R$. Since PIDs are UFDs, we can write any nonzero $x \in R$ uniquely as $\pi^n u$ for some $n \ge 0$, $u$ a unit (since $\pi$ is the only prime). Then any $y \in K^{\times}$ can be written uniquely as $\pi^m u$, $m \in \mathbb{Z}$. Define $v(\pi^m u) = m$. We can check that this is a valuation with $R = \mathcal{O}_K$.
    \end{enumerate}
\end{proof}
\begin{example}
    $\mathbb{Z}_{(p)}$, $R[[t]]$ for $R$ a field are DVRs.
\end{example}

\section{$p$--adic numbers}

Recall that $\mathbb{Q}_p$ is the completion of $\mathbb{Q}$ with respect to $|\cdot|_p$. It is an exercise on example sheet 1 to show that $\mathbb{Q}_p$ is a field. Moreover, $|\cdot|_p$ extends to $\mathbb{Q}_p$ and the associated valuation is discrete (example sheet again).
\begin{defn}
    The \textbf{ring of $p$--adic integers} $\mathbb{Z}_p$ is the valuation ring  \[
    \mathbb{Z}_p = \{x \in \mathbb{Q}_p \mid |x|_p \le 1\}.
    \]
\end{defn}
\textbf{Facts.} $\mathbb{Z}_p$ is a DVR and has a principal maximal ideal $p\mathbb{Z}_p$. In $\mathbb{Z}_p$, all nonzero ideals are given by $p^n \mathbb{Z}_p$. 
\begin{prop}
    $\mathbb{Z}_p$ is the closure of $\mathbb{Z}$ inside $\mathbb{Q}_p$. In particular, $\mathbb{Z}_p$ is the completion of $\mathbb{Z}$ with respect to $|\cdot|_p$.
\end{prop}
\begin{proof}
    We need to show $\mathbb{Z}$ is dense in $\mathbb{Z}_p$. Note $\mathbb{Q}$ is dense in $\mathbb{Q}_p$. Since $\mathbb{Z}_p \subset \mathbb{Q}_p$ is open, $\mathbb{Z}_p \cap \mathbb{Q}$ is dense in $\mathbb{Z}_p$. But $$\mathbb{Z}_p \cap \mathbb{Q} = \{x \in \mathbb{Q}\mid |x|_p \le 1\} = \left\{\frac{a}{b} \in \mathbb{Q} \mid  p \nmid b\right\} = \mathbb{Z}_{(p)}.$$
    Thus it suffices to show that $\mathbb{Z}$ is dense in $\mathbb{Z}_{(p)}$. Let $\frac{a}{b} \in \mathbb{Z}_{(p)}$ with $a,b \in \mathbb{Z}$ and $p \nmid b$. For $n \in \mathbb{N}$, choose $y_n \in \mathbb{Z}$ such that $by_n \equiv a \pmod{p^n}$. Then $y_n \to \frac{a}{b}$ as $n \to \infty$. 
    \vspace{1mm}
     
    For the last part, note that $\mathbb{Z}_p$ is complete (as it is a closed subset of a complete space) and $\mathbb{Z} \subset \mathbb{Z}_p$ is dense.
\end{proof}

\textbf{Inverse limits.} Let $(A_n)_{n=1}^\infty$ be a sequence of sets/groups/rings together with homomorphisms $\phi_n : A_{n+1} \to A_n$ (called \textbf{transition maps}). Then the \textbf{inverse limit} of $(A_n)_{n=1}^\infty$ is the set/group/ring \[
\underset{n}{\varprojlim} A_n = \left\{(a_n)_{n=1}^\infty \in \prod_{n=1}^{\infty} A_n \mid \phi_n(a_{n+1}) = a_n ~\forall n\right\}. 
\]
\textbf{Fact.} If $A_n$ is a group/ring, then the inverse limit is also a group/ring. Here the group/ring operations are defined componentwise. Let $\theta_m : \underset{n}{\varprojlim} A_n \to A_m$ denote the natural projection.
\vspace{1mm}
 
The inverse limit satisfies the following universal property:
\begin{prop}
    For any set/group/ring $B$ together with homomorphisms $\psi_n : B \to A_n$ such that the following diagram commutes,
    \[\begin{tikzcd}[row sep=large,column sep=large]
        B & A_{n+1} \\
        & A_n
        \arrow["\psi_{n+1}", from=1-1, to=1-2]
        \arrow["\psi_n", from=1-1, to=2-2]
        \arrow["\phi_n", from=1-2, to=2-2]
    \end{tikzcd}\]
    there exists a unique homomorphism $\psi : B \to \underset{n}{\varprojlim} A_n$ such that $\theta_n \circ \psi = \psi_n$ for all $n$.
\end{prop} 
\begin{proof}
    Define $\psi : B \to \prod_{n=1}^{\infty} A_n$ by $b \mapsto (\psi_n(b))_{n=1}^{\infty}$. Then $\psi_n = \theta_n \circ \psi_{n+1} \implies \psi(b) \in \underset{n}{\varprojlim} A_n$. This map is clearly unique (determined by $\psi_n = \phi_n \circ \psi_{n+1}$), and is a homomorphism of sets/groups/rings.
\end{proof}
\begin{defn}
    Let $I \subset R$ be an ideal (in a ring $R$). The \textbf{$I$--adic completion of $R$} is the ring $\hat{R} = \underset{n}{\varprojlim} R/I^n$ where $R/I^{n+1} \to R/I^n$ is the natural projection.
\end{defn}
Note that there exists a natural map $i : R \to \hat{R}$ by the universal property (since there exist maps $R \to R/I^n$).
\begin{defn}
    We say $R$ is \textbf{$I$--adically complete} if $i$ is an isomorphism.
\end{defn}
\textbf{Fact.} $\text{ker}(i : R \to \hat{R}) = \bigcap_{n=1}^{\infty} I^n$ (check!). 
\vspace{1mm}
 
Let $(K, |\cdot|)$ be a non--archimedean valued field and $\pi \in \mathcal{O}_K$  such that $|\pi|<1$.
\begin{prop}
    Assume $K$ is complete with respect to $|\cdot|$. Then:
    \begin{enumerate}[(i)]
        \item $\mathcal{O}_K \stackrel{i}{\cong} \underset{n}{\varprojlim} \mathcal{O}_K/\pi^n \mathcal{O}_K$ (i.e. $\mathcal{O}_K$ is $\pi$--adically complete)\footnote{There a bit of abuse of notation here -- really, $\mathcal{O}_K$ is $(\pi)$--adically complete.}.
        \item Every $x \in \mathcal{O}_K$ can be written uniquely as $x = \sum_{i=0}^{\infty} a_i \pi^i$ with $a_i \in A$, where $A \subset \mathcal{O}_K$ is a set of coset representatives for $\mathcal{O}_K/\pi \mathcal{O}_K$. Moreover, any such power series converges (in $\mathcal{O}_K$).
    \end{enumerate}
\end{prop} 
\begin{proof}
    \begin{enumerate}[(i)]
        \item $K$ is complete and $\mathcal{O}_K \subset K$ is closed, so $\mathcal{O}_K$ is complete. If $x \in \bigcap_{n=1}^{\infty} \pi^n \mathcal{O}_K$, then $v(x) \ge n v(\pi) ~\forall n \implies x = 0$, hence the natural map $\mathcal{O}_K \to \underset{n}{\varprojlim} \mathcal{O}_K/\pi^n \mathcal{O}_K$ is injective.
        
        For surjectivity, let $(x_n)_{n=1}^{\infty} \in \underset{n}{\varprojlim}\mathcal{O}_K/ \pi^n \mathcal{O}_K$ and for each $n$, let $y_n \in \mathcal{O}_K$ be a lifting\footnote{Given a surjective map $G \to G'$, a lift of an element $x \in G'$ is a choice of $y \in G$ such that $y \mapsto x$ under this map.} of $x_n \in \mathcal{O}_K/\pi^n \mathcal{O}_K$. Then $y_n - y_{n+1}  \in \pi^n \mathcal{O}_K$, thus $(y_n)_{n=1}^{\infty}$ is a Cauchy sequence in $\mathcal{O}_K$. Let $y_n \to y \in \mathcal{O}_K$. Then $y$ maps to $(x_n)_{n=1}^\infty$ in $\underset{n}{\varprojlim} \mathcal{O}_K/\pi^n \mathcal{O}_K$.
        \item Left as exercise on example sheet 1.
    \end{enumerate}
\end{proof}
\begin{cor}\label{cor3.4}
    \begin{enumerate}[(i)]
        \item $\mathbb{Z}_p \cong \underset{n}{\varprojlim} \mathbb{Z}/p^n \mathbb{Z}$.
        \item Every element in $\mathbb{Q}_p$ can be written uniquely as $x = \sum_{i=n}^{\infty} a_i p^i$ where we have $a_i \in \{0,1,\ldots,p-1\}$.
    \end{enumerate}
\end{cor}
\marginpar{14 Oct 2022, Lecture 4}
\begin{proof}
    \begin{enumerate}[(i)]
        \item By the previous proposition we just need to show $\mathbb{Z}/p^n\mathbb{Z} \cong \mathbb{Z}_p/p^n\mathbb{Z}_p$. Let $f_n : \mathbb{Z} \to \mathbb{Z}_p/p^n \mathbb{Z}_p$ be the natural map. Then $$\text{ker}(f_n) =\{x \in \mathbb{Z} \mid |x|_p \le p^{-n}\} = p^n\mathbb{Z},$$ thus the natural map $\mathbb{Z}/p^n\mathbb{Z} \to \mathbb{Z}_p/p^n \mathbb{Z}_p$ is injective. 
        \vspace{1mm}
         
        For surjectivity, take $\overline{z} \in \mathbb{Z}_p/p^n \mathbb{Z}_p$ and $c \in \mathbb{Z}_p$ a lift. Since $\mathbb{Z}$ is dense in $\mathbb{Z}_p$, there exists $x \in \mathbb{Z}$ such that $x \in c + p^n \mathbb{Z}_p$ ($p^n \mathbb{Z}_p$ is open in $\mathbb{Z}_p$). Then $f_n(x) = \overline{z}$, so $\mathbb{Z}/p^n \mathbb{Z} \to \mathbb{Z}_p / p^n \mathbb{Z}_p$ is surjective.
        \item Follows from Corollary \ref{cor3.4} (ii) applied to $p^{-n} x \in \mathbb{Z}_p$ for some $n \in \mathbb{Z}$.
    \end{enumerate}
\end{proof}

\begin{example}
    We have $\frac{1}{1-p} = 1 + p + p^2 + p^3 + \ldots$ in $\mathbb{Q}_p$.
\end{example}

\section{Complete valued fields}

\subsection{Hensel's lemma}

\begin{theorem}[Hensel's lemma, version 1]
    Let $(K, |\cdot|)$ be a complete discretely valued field. Let $f(x) \in \mathcal{O}_K[x]$ and assume $\exists a \in \mathcal{O}_K$ such that $|f(a)| < |f'(a)|^2$ for $f'(a)$ the formal derivative. Then there exists a unique $x \in \mathcal{O}_K$ such that $f(x)=0$ and $|x-a|< |f'(a)|$.
\end{theorem}
\begin{proof}
    Let $\pi \in \mathcal{O}_K$ be a uniformizer and let $r = v(f'(a))$ for $v$ a normalized valuation, i.e. $v(\pi)=1$. We inductively construct a sequence $(x_n)$ in $\mathcal{O}_K$ such that
    \begin{enumerate}[(i)]
        \item $f(x_n) \equiv 0 \pmod{\pi^{n+2r}}$.
        \item $x_{n+1} \equiv x_n \pmod{\pi^{n+r}}$.
    \end{enumerate}
    Take $x_1 = a$, so $f(x_1) \equiv 0 \pmod{\pi^{1+2r}}$. Now suppose we've constructed $x_1, \ldots, x_n$ satisfying the conditions. Then define $x_{n+1} = x_n - \frac{f(x_n)}{f'(x_n)}$. Since $x_n \equiv x_1 \pmod{\pi^{r+1}}$, $v(f'(x_n)) = v(f'(x_1)) = r$ and hence $\frac{f(x_n)}{f'(x_n)} \equiv 0 \pmod{\pi^{n+r}}$ by (i). It follows that $x_{n+1} \equiv x_n \pmod{\pi^{n+r}}$, so (ii) holds.
    \vspace{1mm}
     
    Note that for $X, Y$ indeterminates, we can write $f(X+Y) = f_0(X) + f_1(X)Y + f_2(X)Y^2 + \ldots$, where $f_i \in \mathcal{O}_K[X]$ and $f_0(X) = f(X), f_1(X) = f'(X)$. Thus $f(x_{n+1}) = f(x_n) + f'(x_n)c + f_2(x_n)c^2 + \ldots$ for $c = - \frac{f(x_n)}{f'(x_n)}$. Since $c \equiv 0 \pmod{\pi^{n+r}}$ and $v(f_i(x_n)) \ge 0$, we have $f(x_{n+1}) \equiv f(x_n) + c f'(x_n) \pmod{\pi^{n+2r+1}}$ (since the other terms vanish), but this is $\equiv 0 \pmod{\pi^{n+2r+1}}$, so (i) holds.
    \vspace{1mm}
     
    This gives the construction of $(x_n)$. Property (ii) implies that $(x_n)$ is Cauchy, so let $x \in \mathcal{O}_K$ be the limit, $x_n \to x$. Then $f(x) = \lim_{n \to \infty} f(x_n) = 0$ by property (i). Moreover, (ii) implies $a = x_1 \equiv x_n \pmod{\pi^{r+1}} ~\forall n$, so $a \equiv x \pmod{\pi^{r+1}}$, thus $|x-a| < |f'(a)|$.
    \vspace{1mm}
     
    For uniqueness, suppose $x'$ also satisfies $f(x') = 0$ and $|x'-a|<|f'(a)|$. Set $\delta = x' - x \neq 0$. Then $|x'-a|<|f'(a)|$ and $|x-a|<|f'(a)|$, so the ultrametric inequality implies $|\delta| = |x'-x| < |f'(a)| = |f'(x)|$ (since $a \equiv x \pmod{\pi^{r+1}}$). But \[
    0 = f(x') = f(x+\delta) = \underbrace{f(x)}_{=0} + f'(x)\delta + \underbrace{\delta^2 \ldots}_{|\cdot|\le |\delta|^2}.
    \]
    Hence $|f'(x) \delta|\le |\delta|^2 \implies |f'(x)| \le |\delta|$, a contradiction.
\end{proof}
\begin{cor}
    Let $(K,|\cdot|)$ be a complete discretely valued field, let $f(x) \in \mathcal{O}_K[x]$ and let $\overline{c} \in k = \mathcal{O}_K/\mathfrak{m}$ be a simple root of $\overline{f}(x) = f(x) \pmod{\mathfrak{m}} \in k[x]$. Then there exists a unique $x \in \mathcal{O}_K$ such that $f(x) = 0$ and $x \equiv \overline{c} \pmod{\mathfrak{m}}$.
\end{cor}
\begin{proof}
    Apply Hensel's lemma to a lift $c \in \mathcal{O}_K$ of $\overline{c}$. Then $|f(c)| < 1 = |f'(c)|^2$ since $f'(c)$ is a simple root.
\end{proof}
\begin{example}
    Consider $f(x) = x^2-2$, which has a simple root mod 7. Thus $\sqrt{2} \in \mathbb{Z}_7 \subset \mathbb{Q}_7$.
\end{example}
\begin{cor}
    $\mathbb{Q}_p^\times/(\mathbb{Q}_p^\times)^2 \cong \begin{cases}
        (\mathbb{Z}/2\mathbb{Z})^2 \text{ if } p>2.\\
        (\mathbb{Z}/2\mathbb{Z})^3 \text{ if } p=2.
    \end{cases}$
\end{cor}
\begin{proof}
    First consider $p>2$. Let $b \in \mathbb{Z}_p^\times$. Applying the previous corollary to $f(x) = x^2-b$, we find that $b \in (\mathbb{Z}_p^\times)^2$ if and only if $b \in (\mathbb{F}_p^\times)^2$. Thus $\mathbb{Z}_p^\times \to \mathbb{F}_p^\times/(\mathbb{F}_p^\times)^2$ has kernel $(\mathbb{Z}_p^\times)^2$, so induces an isomorphism $\mathbb{Z}_p^\times/(\mathbb{Z}_p^\times)^2 \to \mathbb{F}_p^\times/(\mathbb{F}_p^\times)^2 \cong (\mathbb{Z}/2\mathbb{Z})$ (since $\mathbb{F}_p^{\times} = \mathbb{Z}/(p-1)\mathbb{Z}$).
    \vspace{1mm}
     
    We have an isomorphism $\mathbb{Z}_p^\times \times \mathbb{Z} \to \mathbb{Q}_p^{\times}$ given by $(u,n) \mapsto up^n$. Then $\mathbb{Q}_p^\times / (\mathbb{Q}_p^\times)^2 \cong (\mathbb{Z}/2\mathbb{Z})^2$.
    \vspace{1mm}
     
    If $p=2$, let $b \in \mathbb{Z}_2^\times$. Consider $f(x) = x^2 - b$, so $f'(x) = 2x \equiv 0 \pmod{2}$. Instead now let $b \equiv 1 \pmod{8}$. Then $|f(1)|_2 \le 2^{-3} < 2^{-2} = |f'(1)|_2^2$. Hensel's lemma now implies that $b \in (\mathbb{Z}_2^\times)^2 \iff b \equiv 1 \pmod{8}$. Thus $\mathbb{Z}_2^\times / (\mathbb{Z}_2^\times)^2 \cong (\mathbb{Z}/8\mathbb{Z})^\times = (\mathbb{Z}/2\mathbb{Z})^2$. Again using $\mathbb{Q}_2^\times \cong \mathbb{Z}_2^\times \times \mathbb{Z}$, we obtain that $\mathbb{Q}_2^\times/(\mathbb{Q}_2^\times)^2 \cong (\mathbb{Z}/2\mathbb{Z})^3$. 
\end{proof}
\textbf{Remark.} The proof of Hensel's lemma uses the iteration $x_{n+1} = x_n - \frac{f(x_n)}{f'(x_n)}$. We can think of the proof as the non--archimedean analogue of the Newton-Raphson method.

\begin{theorem}[Hensel's lemma, version 2]
    Let $(K, |\cdot|)$ be a complete discretely valued field and $f(x) \in \mathcal{O}_K[x]$. Suppose $\overline{f}(x) = f(x) \pmod{\mathfrak{m}} \in k[x]$ factorizes as $\overline{f}(x) = \overline{g}(x)\overline{h}(x) \in k[x]$ with $\overline{g}(x), \overline{h}(x)$ coprime. Then there is a factorization $f(x) = g(x)h(x)$ in $\mathcal{O}_K[x]$ with $\overline{g}(x) \equiv g(x) \pmod{\mathfrak{m}}$, $\overline{f}(x) \equiv f(x) \pmod{\mathfrak{m}}$ and $\text{deg}(\overline{g}) = \text{deg}(g)$.
\end{theorem}
\begin{proof}
    Left as an exercise on example sheet 1.
\end{proof}

\marginpar{17 Oct 2022, Lecture 5}

\begin{cor}
    Let $f(x) = a_nx^n + \ldots + a_0 \in k[x]$ with $a_0 \ldots a_n \neq 0$. If $f(x)$ is irreducible, then $|a_i|\le \max(|a_0|,|a_n|)$ for all $i$.
\end{cor}
\begin{proof}
    By scaling, assume $f(x) \in \mathcal{O}_K[x]$ with $\max(|a_i|) = 1$. Then we need to show that $\max(|a_0|,|a_n|)=1$. If not, let $r$ be minimal such that $|a_r|=1$, so $0<r<n$. Then $$\overline{f}(x) = x^r(a_r + \ldots a_n x^{n-r}) \pmod{\mathfrak{m}}.$$
    By Hensel's lemma version 2, $f(x)=g(x)h(x)$ with $\text{deg}(g)=r$, contradicting irreducibility.
\end{proof}

\section{Teichmüller lifts}
\begin{defn}
    A ring $R$ of characteristic $p>0$ is \textbf{perfect} if the Frobenius map $x \mapsto x^p$ is a bijection.
    \vspace{1mm}
     
    A field of characteristic $p$ is \textbf{perfect} if it is perfect as a ring.
\end{defn}
\textbf{Remark.} Since $\text{char }R = p$, $(x+y)^p = x^p + y^p$, so the Frobenius map is a ring homomorphism.

\begin{example}
    \begin{enumerate}[(i)]
        \item $\mathbb{F}_{p^n}$ is perfect and $\overline{\mathbb{F}_p}$ is perfect.
        \item Non--example. $\mathbb{F}_p[t]$ is not perfect since $t \not\in \text{Im}(\text{Frob})$.
        \item $\mathbb{F}_p(t^{\frac{1}{p^{\infty}}}) = \mathbb{F}_p\left(t,t^{\frac{1}{p}},t^{\frac{1}{p^2}},\ldots \right)$ is a perfect field, known as the \textbf{perfection} of $\mathbb{F}_p(t)$.
    \end{enumerate}
\end{example}
\textbf{Fact.} A field $k$ of characteristic $p>0$ is perfect if and only if any finite extension of $k$ is separable.

\begin{theorem}
    Let $(K,|\cdot|)$ be a complete discretely valued field such that the residue field $k = \mathcal{O}_K/\mathfrak{m}$ is a perfect field of characteristic $p>0$. Then there exists a unique map $[]: k \to \mathcal{O}_K$ such that 
    \begin{enumerate}[(i)]
        \item $a \equiv [a] \pmod{\mathfrak{m}} ~\forall a \in k$,
        \item $[ab] = [a][b] ~\forall a,b \in k$.
    \end{enumerate}
    Moreover, if $\text{char }\mathcal{O}_K = p$, then $[]$ is a ring homomorphism (i.e. it also preserves addition).
\end{theorem}
\begin{defn}
    The element $[a] \in \mathcal{O}_K$ is called the \textbf{Teichmüller lift} of $a$.
\end{defn}
\begin{lemma}
    Let $(K,|\cdot|)$ be a complete discretely valued field\footnote{(do we need the residue field to be perfect here? lectures said let $(K,|\cdot|)$ be as in above theorem). ANSWER: we need $\mathcal{O}_K/\mathfrak{m}$ to be of characteristic $p$, but not necessarily perfect.} and fix $\pi \in \mathcal{O}_K$ a uniformizer. Let $x,y \in \mathcal{O}_K$ be such that $x \equiv y \pmod{\pi^k}$ for $k\ge 1$. Then $x^p \equiv y^p \pmod{\pi^{k+1}}$. 
\end{lemma}
\begin{proof}
    Let $x = y + u\cdot \pi^k$ for some $u \in \mathcal{O}_K$. Then \begin{align*}
        x^p = \sum_{i=0}^{p} {p\choose i}y^{p-i}(u \pi^k)^{i} = y^p +\sum_{i=1}^{p} {p \choose i}y^{p-i}(u \pi^k)^{i}.
    \end{align*}
    Since $\text{char }\mathcal{O}_K/\pi \mathcal{O}_K = p$, we have $p \in \pi \mathcal{O}_K$. Thus ${p\choose i}y^{p-i}(u \pi^k)^i \in \pi^{k+1}\mathcal{O}_K ~\forall i\ge 1$, so $x^p \equiv y^p \pmod{\pi^{k+1}}$.
\end{proof}
\begin{proof}[Proof of Theorem 5.1]
    Let $a \in k$. For each $i>0$, we choose a lift $y_i \in \mathcal{O}_K$ of $a^{\frac{1}{p^i}}$ and define $x_i = y_i^{p^i}$. We claim that $(x_i)$ is a Cauchy sequence and its limit $x_i \to x$ is independent of the choice of $y_i$.
    \vspace{1mm}
     
    By construction, $y_i  \equiv  y_{i+1}^p \pmod{\pi}$. By our previous lemma and induction on $k$, we have that $y_i^{p^k} \equiv y_{i+1}^{p^{k+1}}\pmod{\pi^{k+1}}$ and hence $x_i \equiv x_{i+1} \pmod{\pi^{i+1}}$ (by taking $k=i$) and hence $(x_i)$ is Cauchy, so $x_i \to x \in \mathcal{O}_K$. \vspace{1mm}
     
    Suppose $(x_i')$ arises from another choice of $y_i'$ lifting $a_i^{\frac{1}{p^i}}$. Then $(x_i')$ is Cauchy and $x_i' \to x'$. Let \[
    x'' = \begin{cases}
        x_i &i \text{ even.}\\
        x_i' &i \text{ odd}.
    \end{cases}
    \]
    Then $x_i''$ arises from the lifting $y'' = \begin{cases}
        y_i &i \text{ even.}\\
    y_i' &i \text{ odd}.\end{cases}$. Then $x_i''$ is Cauchy with subsequences converging to both $x$ and $x'$, so $x = x'$, so our limit is independent of the choice of liftings $(y_i)$.
    \vspace{1mm}
    We define $[a] = x$. Then $x_i \equiv y_i^{p^i} \equiv \left(a^{\frac{1}{p^i}}\right)^{p^i} \equiv a \pmod{\pi}$, so $x \equiv a \pmod{\pi}$, giving us the first property.
    \vspace{1mm}
     
    Now let $b \in k$ and choose $u_i \in \mathcal{O}_K$ a lift of $b^{\frac{1}{p^i}}$ and let $z_i = u_i^{p^i}$. Then $[b] = \lim_{i \to \infty}z_i$. Now $u_iy_i$ is a lift of $(ab)^{\frac{1}{p^i}}$, hence $$[ab] = \lim_{i \to \infty} (u_iy_i)^{p^i} = \lim_{i \to \infty} x_i z_i = \lim_{i \to \infty} x_i \lim_{i \to \infty}z_i = [a][b],$$
    giving us the second property.
    \vspace{1mm}
    
    If $\text{char }K = p$, then $u_i+y_i$ is a lift of $a^{\frac{1}{p^i}}+b^{\frac{1}{p^i}} = \left(a+b \right)^{\frac{1}{p^i}}$. Then \[
    [a+b] = \lim_{i \to \infty} (y_i+u_i)^{p^i} = \lim_{i \to \infty}y_i^{p^i} + u_i^{p^i} = \lim_{i \to \infty}x_i + z_i = [a] + [b].
    \]
    Finally, it is easy to check that $[0] = 0$ and $[1] = 1$ (take $y_i=0$ and $y_i=1$). So $[]$ is a ring homomorphism.
    \vspace{1mm}
     
    For uniqueness, let $\phi: K \to \mathcal{O}_K$ be another map of the desired form. Then for $a \in k$, $\phi\left(a^{\frac{1}{p^i}}\right)$ is a lift of $a^{\frac{1}{p^i}}$. It follows that $$[a] = \lim_{i \to \infty} \phi\left(a^{\frac{1}{p^i}}\right)^{p^i} = \lim_{i \to \infty} \phi(a) = \phi(a).$$
\end{proof}
\begin{example}
    For $K=\mathbb{Q}_p$, what does $[]: \mathbb{F}_p \to \mathbb{Z}_p$ look like? Take $a \in \mathbb{F}_p^\times$, so $[a]^{p-1}=[a^{p-1}] = [1] = 1$. Hence $[a]$ is a $(p-1)^{\text{th}}$ root of unity.
\end{example}
More generally:
\begin{lemma}
    Let $(K,|\cdot|)$ be a complete discretely valued field. If $k = \mathcal{O}_K/\mathfrak{m} \subset \overline{\mathbb{F}_p}$ (which implies that $k$ is perfect), then $[a] \in \mathcal{O}_K$ is a root of unity $\forall a \in k^\times$.
\end{lemma}
\begin{proof}
    $a \in k^\times \implies a \in \mathbb{F}_{p^n}$ for some $n \implies [a]^{p^n-1}=[a^{p^n-1}] = [1] = 1$.
\end{proof}
\begin{theorem}
    Let $(K, |\cdot|)$ be a complete discretely valued field of characteristic $p > 0$. Assume $k = \mathcal{O}_K/\mathfrak{m}$ is perfect. Then $K \cong k((t))$.
\end{theorem}
\begin{proof}
    Since $K = \text{Frac}(\mathcal{O}_K)$, it suffices to show that $\mathcal{O}_K \cong k[[t]]$. For this, fix $\pi \in \mathcal{O}_K$ a uniformizer and let $[]: k \to \mathcal{O}_K$ be the Teichmüller map. Define $\phi : k[[t]] \to \mathcal{O}_K$ by $\phi\left(\sum_{i=0}^{\infty} a_i t^i \right) = \sum_{i=0}^{\infty} [a_i] \pi^i$. Then $\phi$ is a ring homomorphism since $[]$ is a ring homomorphism, but it is also a bijection by Proposition 3.3.
\end{proof}

\section{Extensions of complete valued fields}

\marginpar{19 Oct 2022, Lecture 6}

\begin{theorem}\label{theorem6.1}
    Let $(K,|\cdot|)$ be a complete discretely valued field and let $L/K$ be a finite extension of degree $n$. Then:
    \begin{enumerate}[(i)]
        \item $|\cdot|$ extends uniquely to an absolute value $|\cdot|_L$ on $L$ defined by $$|y|_L = |N_{L/K}(y)|^{1/n}.$$
        \item $L$ is complete with respect to $|\cdot|_L$.
    \end{enumerate}
\end{theorem}
\textbf{Recall.} If $L/K$ is a finite extension, then $N_{L/K}: L \to K$ is defined by $N_{L/K}(y) = \det_K(\text{mult}(y))$ where $\text{mult}(y): L \to L$ is the $K$--linear map given by multiplication by $y$.
\vspace{1mm}
 
\textbf{Facts:} 
\begin{itemize}
    \item The norm is multiplicative, i.e. $N_{L/K}(xy) = N_{L/K}(x)N_{L/K}(y)$.
    \item Let $X^n + a_{n-1}X^{n-1}+\ldots+a_0 \in K[X]$ be the minimal polynomial of $y \in L$. Then $N_{L/K}(y) = \pm a_0^m$ for some $m\ge 1$. In particular, $N_{L/K}(x)=0 \iff x = 0$.
\end{itemize} 
\begin{defn}
    Let $(K,|\cdot|)$ be a nonarchimedean valued field and $V$ a vector space over $K$. Then a \textbf{norm} on $V$ is a function $||\cdot || : V \to \mathbb{R}_{\ge 0}$ satisfying
    \begin{itemize}
        \item $||x|| = 0 \iff x = 0$.
        \item $||\lambda x|| = |\lambda| \cdot ||x|| ~\forall x \in V, \lambda \in K$.
        \item $||x+y|| \le \max(||x||,||y||) ~\forall x,y \in V$. 
    \end{itemize}
\end{defn}
\begin{example}
    If $V$ is finite--dimensional and $e_1,\ldots,e_n$ is a basis for $V$, then the \textbf{sup norm} $||\cdot ||_{\text{sup}}$ on $V$ is defined by $||x||_{\text{sup}} = \max_i |x_i|$, where $x = \sum_{i=1}^{n} x_ie_i$.
\end{example}
\textbf{Exercise}: $||\cdot ||_{\text{sup}}$ is a norm.
\begin{defn}
    Two norms $||\cdot ||_1, ||\cdot ||_2$ on $V$ are \textbf{equivalent} if there exist constants $C, D \in \mathbb{R}_{>0}$ such that \[
    C ||x||_1 \le ||x||_2 \le D ||x||_1 ~\forall x \in V.
    \]
\end{defn}
\textbf{Fact.} A norm defines a topology on $V$ and equivalent norms induce the same topology (since an open ball in one topology is both contained in and contains an open ball in the other topology).

\begin{prop}
    Let $(K,|\cdot|)$ be complete and nonarchimedean and let $V$ be a finite dimensional vector space over $K$. Then $V$ is complete with respect to $||\cdot ||_{\text{sup}}$.
\end{prop}
\begin{proof}
    Let $(v_i)$ be a Cauchy sequence in $V$ and let $e_1,\ldots,e_n$ be a basis for $V$. Write $V_i = \sum_{j=1}^{n} x_j^i e_j$, then $(x^i_j)_{i=1}^{\infty}$ is a Cauchy sequence in $K$. Let $x_j^i \to x_j \in K$, then we can check that $v_i \to v = \sum_{j=1}^{n} x_je_j$.
\end{proof}
\begin{theorem}
    Let $(K,|\cdot|)$ be complete and nonarchimedean and let $V$ be a finite dimensional vector space over $K$. Then any two norms on $V$ are equivalent. In particular, $V$ is complete with respect to any norm.
\end{theorem}
\begin{proof}
    Since equivalence defines an equivalence relation on the set of norms, it suffices to show that any norm $||\cdot ||$ is equivalent to the sup norm $||\cdot ||_{\text{sup}}$ with respect to some basis. Let $e_1,\ldots,e_n$ be a basis for $V$.
    \vspace{1mm}
     
    For the upper bound, set $D = \max ||e_i||$. Then if $x = \sum_{i=1}^{n} x_i e_i$, then $||x|| = \max_i ||x_i e_i|| = \max_i |x_i|||e_i|| \le D \max_i |x_i| = D ||x||_{\text{sup}}$.
    \vspace{1mm}
     
    To find $C$ such that $C ||\cdot ||_{\text{sup}} \le ||\cdot ||$, we induct on $n = \text{dim }V$. If $n=1$, then $||x|| = ||x_1 e_1|| = |x_1|||e_1|| = ||x||_{\text{sup}}||e_1||$, so take $C = ||e_1||$.
    \vspace{1mm}
     
    For $n>1$, set $V_i = \langle e_1,\ldots,e_{i-1}, e_{i+1},\ldots,e_n \rangle$. By induction, the norm on $V_i$ is equivalent to the sup norm, so $V_i$ is complete with respect to $||\cdot||$, hence closed. Then the translate $e_i + V_i$ is also closed for all $i$, hence \[
    S = \bigcup_{i=1}^n e_i + V_i
    \]
    is a closed subset not containing zero. Hence $\exists C > 0$ such that $S \cap B(0,C) = \varnothing$, where $B(0,C) = \{x \in V \mid ||x||<C\}$. We claim this $C$ works. To see this, let $0 \neq x = \sum_{i=1}^{n} x_ie_i$ and suppose $|x_j| = \max_i |x_i|$. Then $||x||_{\text{sup}} = |x_j|$ and $\frac{1}{x_j}x \in S$ (since the $j^{\text{th}}$ coefficient will be equal to 1). Thus $||\frac{1}{x_j}x||\ge C$, so $||x||\ge C |x_j| = C ||x||_{\text{sup}}$.
    \vspace{1mm}
     
    Finally, $V$ is complete since it is complete with respect to $||\cdot||_{\text{sup}}$.
\end{proof}
\begin{proof}[Proof of Theorem 6.1]
    We first show that $|\cdot|_L = |N_{L/K}(\cdot )|^{1/n}$ satisfies the three absolute value axioms.
    \begin{enumerate}[(i)]
        \item $|y|_L = 0 \iff |N_{L/K}(y)|^{1/n} = 0 \iff N_{L/K}(y) = 0 \iff y = 0$.
        \item $|y_1y_2|_L = |N_{L/K}(y_1y_2)|^{1/n} = |N_{L/K}(y_1)|^{1/n}|N_{L/K}(y_2)|^{1/n} = |y_1|_L|y_2|_L$.
        \item For this, we need some preparation:
    \end{enumerate}
    \begin{defn}
        Let $R \subset S$ be a subring. We say $s \in S$ is \textbf{integral} over $R$ if $s$ is a root of a monic polynomial with coefficients in $R$, i.e. $\exists$ monic $f \in R[X]$ such that $f(s)=0$.
        \vspace{1mm}
         
        The \textbf{integral closure} $R^{\text{int}(S)}$ of $R$ in $S$ is the set of elements of $S$ that are integral over $R$, i.e. \[
        R \subset R^{\text{int}(S)} = \{s \in S \mid s \text{ is integral over }R\}.
        \]
        We say $R$ is \textbf{integrally closed} in $S$ if $R^{\text{int}(S)} = R$.
    \end{defn}
    \begin{prop}
        $R^{\text{int}(S)}$ is a subring of $S$. Moreover, $R^{\text{int}(S)}$ is integrally closed in $S$.
    \end{prop}
    \begin{proof}
        Exercise on example sheet 2.
    \end{proof}
    \begin{lemma}\label{lemma6.5}
        Let $(K,|\cdot|)$ be a nonarchimedean valued field. Then $\mathcal{O}_K$ is integrally closed in $K$.
    \end{lemma}
    \begin{proof}
        Let $x \in K$ be integral over $\mathcal{O}_K$. WLOG assume $x \neq 0$. Let $f(X) = X^n + a_{n-1}X^{n-1}+\ldots+a_0 \in \mathcal{O}_K[X]$ such that $f(x)=0$. Then \[
        x = -a_{n-1} - \ldots - a_0 \frac{1}{x^{n-1}}.
        \] 
        If $|x|>1$, then we have that $\left|-a_{n-1}- \ldots - a_0\frac{1}{x^{n-1}}\right|\le 1$ by the ultrametric inequality, contradiction. Thus $|x| \le 1$, so $x \in \mathcal{O}_K$.
    \end{proof}
    Now we show (iii): Set $\mathcal{O}_L = \{y \in L \mid |y|_L \le 1\}$. We claim that $\mathcal{O}_L$ is the integral closure of $\mathcal{O}_K$ inside $L$. In particular, $\mathcal{O}_L$ is a subring of $L$. 
    \vspace{1mm}
     
    Assuming this, let $x, y \in L$ and WLOG assume $|x|_L \le |y|_L$. Then we have $\left|\frac{x}{y}\right|_L\le 1 \implies \frac{x}{y} \in \mathcal{O}_L$. Since $\mathcal{O}_L$ is a ring, $1 \in \mathcal{O}_L$, so $1 + \frac{x}{y} \in \mathcal{O}_L$ and hence $\left|1 +\frac{x}{y}\right|_L \le 1$, so $|x+y|_L \le |y|_L = \max (|x|_L, |y|_L)$, giving the ultrametric inequality property.

    \marginpar{21 Oct 2022, Lecture 7}
    
    To prove the claim, take $0 \neq y \in L$ and let $f(X) = X^d + a_{d-1}X^{d-1} + \ldots + a_0 \in K[X]$ be the minimal monic polynomial for $y$. We claim $y$ is integral over $\mathcal{O}_K \iff f(X) \in \mathcal{O}_K[X]$.
    \vspace{1mm}
     
    $(\impliedby)$: This direction is clear.
    \vspace{1mm}
     
    $(\implies )$: Let $g(x) \in \mathcal{O}_K[X]$ be monic such that $g(y)=0$. Then $f \mid g$ in $K[X]$ and hence every root of $f$ is a root of $g$. Hence every root of $f$ considered in $\overline{K}$ is integral over $\mathcal{O}_K$. Hence the $a_i$ are integral over $\mathcal{O}_K$ for $0\le i \le d-1$. Hence $a_i \in \mathcal{O}_K$ by a lemma from last time, proving our claim.
    \vspace{1mm}
     
    Now, by the corollary of the second version of Hensel's lemma, $|a_i|\le \max(|a_0|,1)$. By a property of the norm $N_{L/K}$, we have $N_{L/K}(y) = \pm a_0^m \in \mathcal{O}_K$. Hence $y \in \mathcal{O}_L \iff |N_{L/K}(y)|\le 1 \iff |a_0|\le 1$, so by our corollary this happens $\iff |a_i|\le 1~\forall i$, i.e. $a_i \in \mathcal{O}_K ~\forall i$, so $y$ is integral.
    \vspace{1mm}
     
    Since $N_{L/K}(x)=x^n$ for $x \in K$, $|x|_L$ extends $|\cdot|$ on $K$. If $|\cdot|_L'$ is another absolute value on $L$ extending $|\cdot|$, then $|\cdot|_L, |\cdot|_L'$ are norms on $L$, which are equivalent and hence induce the same topology on $L$, so $|\cdot|_L' = |\cdot|_L^c$ for some $c>0$. But since they both extend $|\cdot|$ on $K$, we must have $c=1$.
    \vspace{1mm}
     
    (ii): Theorem 6.3 implies the result, as $L$ is complete with respect to the sup norm.
\end{proof}
\begin{cor}
    Let $(K,|\cdot|)$ be a complete, nonarchimedean discretely valued field and $L/K$ a finite extension. Then
    \begin{enumerate}[(i)]
        \item $L$ is discretely valued with respect to $|\cdot|_L$.
        \item $\mathcal{O}_L$ is the integral closure of $\mathcal{O}_K$ in $L$.
    \end{enumerate}
\end{cor}
\begin{proof}
    \begin{enumerate}[(i)]
        \item Fix $v$, the valuation on $K$ responding to our absolute value, and let $v_L$ be the valuation on $L$ extending $v$. Let $n = [L:K]$. For $y \in L^\times$, $|y|_L = |N_{L/K}(y)|^{1/n}$, so $v_L(y)=\frac{1}{n}v(N_{L/K}(y))$, so $v_L(L^\times) \subset \frac{1}{n}v(K^{\times})$. Since $v(K^\times)$ is discrete, so is $v_L$.
        \item This was proved in the proof of the previous theorem.
    \end{enumerate}
\end{proof}
\begin{cor}
    Let $(K, |\cdot|)$ be complete, nonarchimedean, and discretely valued and let $\overline{K}/K$ be the algebraic closure of $K$. Then $|\cdot|$ extends uniquely to an absolute value $|\cdot|_{\overline{K}}$ on $\overline{K}$.
\end{cor}
\begin{proof}
    Let $x \in \overline{K}$, then $x \in L$ for some finite extension $L/K$. Define $|x|_{\overline{K}} = |x|_L$. This is well--defined (i.e. independent of $L$) by uniqueness in Theorem \ref{theorem6.1} (for any $L,L'$, consider an extension containing both).
    \vspace{1mm}
     
    The axioms for $|x|_{\overline{K}}$ to be an absolute value can be checked over finite extensions.
    \vspace{1mm}
     
    Uniqueness again follows from the finite case: if two absolute values disagree on some value, then consider a finite extension containing that value.
\end{proof}
\textbf{Remark.} $|\cdot|_{\overline{K}}$ on $\overline{K}$ is never discrete. For example, if $K = \mathbb{Q}_p$, then $\sqrt[n]{p} \in \overline{\mathbb{Q}_p}$ and $\forall n\ge 0$, $v_p(\sqrt[n]{p}) = \frac{1}{n}v_p(p) = \frac{1}{n}$, giving a non--discrete valuation. Furthermore, $\overline{\mathbb{Q}_p}$ is not complete with respect to $|\cdot |_{\overline{\mathbb{Q}}_p}$. Showing this is an exercise on example sheet 2. On the sheet we also show that if we take $\mathbb{C}_p$, the completion of $\overline{\mathbb{Q}_p}$ with respect to $|\cdot|_{\overline{\mathbb{Q}_p}}$, then $\mathbb{C}_p$ is algebraically closed.

\begin{prop}\label{prop6.8}
    Let $L/K$ is a finite extension of complete discretely valued fields with $n = [L:K]$. Assume that
    \begin{enumerate}[(i)]
        \item $\mathcal{O}_K$ is compact.
        \item The extension $k_{L}/k$ of residue fields is finite and separable.
    \end{enumerate}
    Then there exists $\alpha \in \mathcal{O}_L$ such that $\mathcal{O}_L = \mathcal{O}_K[\alpha]$.
\end{prop}
\textbf{Remark.} We will later see that (i) implies (ii).
\begin{proof}
    We'll choose $\alpha \in \mathcal{O}_L$ such that:
    \begin{enumerate}[(i)]
        \item $\exists \beta \in \mathcal{O}_K[\alpha]$ a uniformizer for $\mathcal{O}_L$.
        \item $\mathcal{O}_K[\alpha] \to k_L$ is surjective.
    \end{enumerate}
    First note that $k_L/k$ is separable, so $\exists \overline{\alpha} \in k_L$ such that $k_L = k(\overline{\alpha})$. Let $\alpha \in \mathcal{O}_L$ be a lift of $\overline{\alpha}$ and $g(X) \in \mathcal{O}_K[X]$ a monic lift of the minimal polynomial of $\overline{\alpha}$. Also fix $\pi_L \in \mathcal{O}_L$ a uniformizer. Then $\overline{g}(X) \in k[X]$ is irreducible and separable, so $\overline{\alpha}$ is a simple root of $\overline{g}$, so $g(\alpha) \equiv 0 \pmod{\pi_L}$ and $g'(\alpha) \not\equiv 0 \pmod{\pi_L}$.
    \vspace{1mm}
     
    If $g(\alpha) \equiv  0 \pmod{\pi_L^2}$, then \[
    g(\alpha + \pi_L) \equiv  g(\alpha) + \pi_L g'(\alpha) \pmod{\pi_L^2}.
    \]
    Thus $v_L(g(\alpha+\pi_L)) = v_L(\pi_Lg'(\alpha)) = v_L(\pi) = 1$ for $v_L$ the normalized valuation on $L$. Hence either $v_L(g(\alpha))= 1$ or $v_L(g(\alpha+\pi_L)) = 1$. Possibly replacing $\alpha$ by $\alpha + \pi_L$, we may assume that $g(\alpha)$ is a uniformizer, i.e. $v_L(g(\alpha)) = 1$.
    \vspace{1mm}
     
    Now set $\beta = g(\alpha) \in \mathcal{O}_K[\alpha]$, a uniformizer. Then $\mathcal{O}_K[\alpha] \subset L$ is the image of a continuous map $\mathcal{O}_K^n \to L$ given by $(x_0,\ldots,x_{n-1}) \mapsto \sum_{i=0}^{n-1} x_i \alpha^i$. Since $\mathcal{O}_K$ is compact, $\mathcal{O}_K[\alpha]$ is compact, hence closed.
    \vspace{1mm}
     
    We have a closed subring of $\mathcal{O}_L$, so to show it is $\mathcal{O}_L$, it is enough to show it is dense. Since $k_L = k(\overline{\alpha})$, $\mathcal{O}_K[\alpha]$ contains a set of coset representatives for the residue field $k_L = \mathcal{O}_L/\beta \mathcal{O}_L$. Take $y \in \mathcal{O}_L$. By Proposition 3.3, we can write $y = \sum_{i=0}^{\infty} \lambda_i \beta^i$ with $\lambda_i \in \mathcal{O}_K[\alpha]$. Then $y_m = \sum_{i=0}^{m} \lambda_i \beta^i \in \mathcal{O}_K[\alpha]$ gives a Cauchy sequence converging to $y$. Then $y \in \mathcal{O}_K[\alpha]$ since $\mathcal{O}_K[\alpha]$ is closed.
\end{proof}
\section{Local fields}
\begin{defn}
    Let $(K,|\cdot|)$ be a valued field. We say $K$ is a \textbf{local field} if it is complete and locally compact (i.e. every point contains a compact neighborhood).
\end{defn}
\begin{example}
    $\mathbb{R}$ and $\mathbb{C}$ are local fields.
\end{example}
\begin{prop}
    Let $(K,|\cdot|)$ be a nonarchimedean complete valued field. Then the following are equivalent:
    \begin{enumerate}[(i)]
        \item $K$ is locally compact (so $K$ is a nonarchimedean local field).
        \item $\mathcal{O}_K$ is compact.
        \item The associated valuation $v$ is discrete and $k = \mathcal{O}_K/\mathfrak{m}$ is finite.
    \end{enumerate}
\end{prop}
\begin{proof}
    \marginpar{24 Oct 2022, Lecture 8}
    (i) $\implies $ (ii): Let $\mathcal{U} \ni 0$ be a compact neighborhood of 0 (i.e. $0 \in \mathcal{U} \subset K$ for $U$ open, $K$ compact). Then $\exists x \in \mathcal{O}_K$ such that $x \mathcal{O}_K \subset \mathcal{U}$. Since $x \mathcal{O}_K$ is closed, it is compact, so $\mathcal{O}_K$ is compact (as it is homeomorphic to $x \mathcal{O}_K$ by the homeomorphism $x \mathcal{O}_K \stackrel{\times x^{-1}}{\to} \mathcal{O}_K$).
    \vspace{1mm}
     
    (ii) $\implies$ (i): $\mathcal{O}_K$ compact $\implies a + \mathcal{O}_K$ compact $\forall a \in K$, so $K$ is locally compact.
    \vspace{1mm}
     
    (ii) $\implies $ (iii): Let $x \in \mathfrak{m}$ and let $A_x \subset \mathcal{O}_K$ be the set of coset representatives for $\mathcal{O}_K/x \mathcal{O}_K$. Then $\mathcal{O}_K = \bigcup_{y \in A_x} (y+ x\mathcal{O}_K)$, which is a disjoint open cover. By compactness, $A_x$ is finite. Hence $\mathcal{O}_K/x \mathcal{O}_K$ is finite and so $\mathcal{O}_K/\mathfrak{m}$ is finite. Now suppose $v$ is not discrete. Then let $x=x_1,x_2,x_3,\ldots$ be elements such that $v(x_1) > v(x_2) > \ldots > 0$. Then $x \mathcal{O}_K \subsetneq x_2 \mathcal{O}_K \subsetneq x_3 \mathcal{O}_K \subsetneq \ldots \subsetneq\mathcal{O}_K$. But $\mathcal{O}_K/x \mathcal{O}_K$ is finite, so it can only have finitely many subgroups, a contradiction.
    \vspace{1mm}
     
    (iii) $\implies $ (ii): Since $\mathcal{O}_K$ is a metric space, it suffices to show that $\mathcal{O}_K$ is sequentially compact, i.e. that every sequence has a convergent subsequence. Let $(x_n)$ be a sequence in $\mathcal{O}_K$ and fix $\pi \in \mathcal{O}_K$ a uniformizer. Note that $\pi^i \mathcal{O}_K/\pi^{i+1}\mathcal{O}_K \cong k$, so $\mathcal{O}_K/\pi^i \mathcal{O}_K$ is finite $\forall i$ (as $\mathcal{O}_K \supset \pi \mathcal{O}_K \supset \ldots \supset \pi^i \mathcal{O}_K$ are all finite). Since $\mathcal{O}_K/\pi \mathcal{O}_K$ is finite, $\exists a_1 \in \mathcal{O}_K / \pi \mathcal{O}_K$ and a subsequence $(x_{1,n})_{n=1}^{\infty}$ such that $x_{1,n} \equiv a_1 \pmod{\pi}$. Since $\mathcal{O}_K/ \pi^2 \mathcal{O}_K$ is finite, $\exists a_2 \in \mathcal{O}_K/ \pi^2 \mathcal{O}_K$ and a subsequence $(x_{2,n})_{n=1}^{\infty}$ of $(x_{1,n})$ such that $x_{2,n} \equiv a_2 \pmod{\pi^2}$. Continuing in this fashion, we obtain sequences $(x_{i,n})_{n=1}^{\infty}$ for $i = 1,2,3,\ldots $ such that 
    \begin{enumerate}[(i)]
        \item $(x_{i+1,n})$ is a subsequence of $(x_{i,n})$ for all $i$.
        \item For any $i$, $\exists a_i \in \mathcal{O}_K/\pi^i \mathcal{O}_K$ such that $x_{i,n} \equiv a_i \pmod{\pi^i}$ for all $n$.
    \end{enumerate}
    Then $a_i \equiv a_{i+1} \pmod{\pi^i}$. Now choose $y_i = x_{i,i}$. This defines a subsequence of $(x_n)$ with $y_i \equiv a_i \equiv a_{i+1} \equiv y_{i+1} \pmod{ \pi^i}$. Thus $(y_i)$ is Cauchy, hence converges by completeness.
\end{proof}
\begin{example}
    \begin{enumerate}[(i)]
        \item $\mathbb{Q}_p$ is a local field, as it is discretely valued and has finite residue field $\mathbb{F}_p$.
        \item $\mathbb{F}_p((t))$ is a local field.
    \end{enumerate}
\end{example}
More on inverse limits: Again let $(A_n)_{n=1}^{\infty}$ be a sequence of sets/groups/rings and let $\phi_n : A_{n+1} \to A_n$ be homomorphisms (transition maps).
\begin{defn}
    Assume each $A_n$ is finite. Then the \textbf{profinite topology} on $A = \underset{n}{\varprojlim} A_n $ is the weakest topology on $A$ such that the projection maps $\theta_n : A \to A_n$ are continuous for all $n$, where all $A_n$ are equipped with the discrete topology.
\end{defn}
\textbf{Fact.} $A = \underset{n}{\varprojlim} A_n $ with the profinite topology is compact, totally disconnected and Hausdorff.
\begin{prop}
    Let $K$ be a nonarchimedean local field. Under the isomorphism $\mathcal{O}_K \cong \underset{n}{\varprojlim} \mathcal{O}_K/\pi^n \mathcal{O}_K$ (for $\pi \in \mathcal{O}_K$ a uniformizer), the topology on $\mathcal{O}_K$ coincides with the profinite topology.
\end{prop}
\begin{proof}[Proof sketch:]
    Check that the sets $B = \{a + \pi^n \mathcal{O}_K \mid n \in \mathbb{Z}_{\ge 1}, a \in \mathcal{O}_K\}$ are a basis of open sets in both topologies. 
    \vspace{1mm}
    
    For the topology arising from $|\cdot|$, this is clear (for any open ball, we can find a closed ball of smaller radius contained inside it). 
    \vspace{1mm}
    
    For the profinite topology, $\mathcal{O}_K \to \mathcal{O}_K/\pi^n \mathcal{O}_K$ is continuous if and only if $a + \pi^n \mathcal{O}_K$ is open $\forall a \in \mathcal{O}_K$.
\end{proof}
\begin{lemma}
    Let $K$ be a nonarchimedean local field and $L/K$ a finite extension. Then $L$ is a local field.
\end{lemma}
\begin{proof}
    Theorem \ref{theorem6.1} shows that $L$ is complete and discretely valued, so it suffices to show that $k_L = \mathcal{O}_L/\mathfrak{m}_L$ is finite. Let $\alpha_1,\ldots,\alpha_n \in L$ be a basis for $L$ as a $K$--vector space. Then $||\cdot||_{\text{sup}}$, the sup norm, is equivalent to $|\cdot|_L$, so there exists $r>0$ such that $\mathcal{O}_L \subset \{x \in L \mid ||x||_{\text{sup}} \le r\}$. Then take $a \in K$ such that $|a|\ge r$, then $\mathcal{O}_L \subset \bigoplus_{i=1}^n a \alpha_i \mathcal{O}_K \subset L$. But this is a finitely generated module over a PID, hence Noetherian, so $\mathcal{O}_L$ is finitely generated as an $\mathcal{O}_K$--module, so $k_L$ is finitely generated over $k$.
\end{proof}
\begin{defn}
    A nonarchimedean valued field $(K, |\cdot|)$ has \textbf{equal characteristic} if $\text{char}(K) = \text{char}(k)$. Otherwise, $K$ has \textbf{mixed characteristic}.
\end{defn}
\begin{example}
    $\mathbb{Q}_p$ has mixed characteristic, whereas $\mathbb{F}_p((t))$ has equal characteristic $p>0$.
\end{example}
It turns out equal characteristic local fields are very easy to classify:
\begin{theorem}
    Let $K$ be a nonarchimedean local field of equal characteristic $p>0$.\footnote{Note the residue field of an equal characteristic nonarchimedean local field is finite, so the characteristic must be positive.} Then \[
    K \cong \mathbb{F}_{p^n}((t))
    \]
    for some $n\ge 1$.
\end{theorem}
\begin{proof}
    $K$ is complete and discretely valued with $\text{char}(K)>0$. Moreover, $k$ is finite, so $k \cong \mathbb{F}_{p^n}$ for some $n$, so $k$ is perfect. Now by Theorem 5.4, $K \cong \mathbb{F}_{p^n}((t))$.
\end{proof}
\begin{lemma}
    An absolute value $|\cdot|$ on a field $K$ is nonarchimedean $\iff$ $|n|$ is bounded $\forall n \in \mathbb{Z}$.
\end{lemma}
\begin{proof}
    ($\implies $): Since $|-1|=|1|$, $|-n|=|n|$. Thus it suffices to show that $|n|$ is bounded for $n\ge 1$, but $|n| = |1| + \ldots +|1| \le |1| = 1$ by the ultrametric inequality.
    \vspace{1mm}
     
    ($\impliedby$): Suppose $|n|\le B~\forall n \in \mathbb{Z}$. Take $x,y \in K$ with $|x|\le |y|$. Then we have 
    \begin{align*}
        |x+y|^m = \left|\sum_{i=0}^{m} {m\choose i}x^i y^{m-i}\right| \le \sum_{i=0}^{m} \left|{m\choose i}x^i y^{m-i}\right| \le |y|^m B (m+1).
    \end{align*}
    Take $m^{\text{th}}$ roots to get $|x+y|\le |y|\sqrt[n]{B(n+1)} \stackrel{n \to \infty}{\to} |y| = \max(|x|,|y|)$.
\end{proof}

\marginpar{26 Oct 2022, Lecture 9}

\begin{theorem}[Ostrowski's Theorem]\label{theorem7.6}
    Any nontrivial absolute value on $\mathbb{Q}$ is equivalent to either $|\cdot|_{\infty}$ or the $p$--adic absolute value $|\cdot|_{p}$ for some prime $p$.
\end{theorem}
\begin{proof}
    Case 1: $|\cdot|$ is archimedean. Then fix $b>1$ such that $|b|>1$, where such a $b$ exists by the previous lemma. Take $a>1$ another integer and write $b^n$ in base $a$, i.e. $b^n = c_ma^m + c_{m-1}a^{m-1} + \ldots + c_0$ for $0\le c_i<a$ and $c_m \neq 0$. 
    \vspace{1mm}
     
    Let $B = \max_{0\le c<a}(|c|)$, then $|b^n| \le (m+1)B \max(|a|^m,1)$. Hence 
    \begin{align*}
        |b| =& \underbrace{\left[(n \log_{a}b+1)B\right]^{1/n}}_{\to 1 \text{ as }n \to \infty}\max(|a|^{\log_a(b)},1) \\
        \implies & |b|\le \max(|a|^{\log_a(b)},1).
    \end{align*}
    Then $|a|>1$ and $|b|\le |a|^{\log_a(b)}$ ($\dagger$). Switching the roles of $a$ and $b$ we also find $|a|\le |b|^{\log_b(a)}$ ($\ddagger$). Then $(\dagger)$ and $(\ddagger)$ imply $\frac{\log|a|}{\log a} = \frac{\log |b|}{\log b} = \lambda \in \mathbb{R}_{> 0}$. Hence $|a| = a^{\lambda} ~\forall a \in \mathbb{Z}_{\ge 1}$, so $|x|=|x|_{\infty}^{\lambda}~\forall x \in \mathbb{Q}$, so $|\cdot|$ is equivalent to $|\cdot|_{\infty}$.
    \vspace{1mm}
     
    Case 2: $|\cdot|$ is non--archimedean. As in the previous inequality, we have $|n|\le 1 ~\forall n \in \mathbb{Z}$. Since this absolute value is nontrivial, $\exists n \in \mathbb{Z}_{\ge 1}$ such that $|n|<1$. Write $n = p_1^{e_1}\ldots p_r^{e_r}$. Then $|p|<1$ for some $p \in \{p_1,\ldots,p_r\}$. Now suppose $|q|<1$ for some prime $q \neq p$. Then write $1 = rp+sq$ for some $r,s \in \mathbb{Z}$. Then $1 = |rp+sq|\le \max(|rp|, |sq|) < 1$, a contradiction. Thus $|p| = \alpha<1$ and $|q| = 1$ for all primes $q \neq p$. Hence $|\cdot|$ is equivalent to $|\cdot|_p$.
\end{proof}
\begin{theorem}
    Let $(K,|\cdot|)$ be a nonarchimedean local field of mixed characteristic. Then $K$ is a finite extension of $\mathbb{Q}_p$.
\end{theorem}
\begin{proof}
    $K$ has mixed characteristic $\implies \text{char}(K) = 0 \implies \mathbb{Q} \subset K$. Also, $K$ is nonarchimedean $\implies |\cdot| |_{\mathbb{Q}} \sim |\cdot|_p$ for some $p$. Since $K$ is complete, $\mathbb{Q}_p \subset K$. Hence it suffices to show that $\mathcal{O}_K$ is finite as a $\mathbb{Z}_p$--module.
    \vspace{1mm}
     
    Let $\pi \in \mathcal{O}_K$ be a uniformizer and $v$ a normalized valuation on $K$. Set $v(p) = e$. Then $\mathcal{O}_K/p \mathcal{O}_K \cong \mathcal{O}_K/\pi^e \mathcal{O}_K$, which is finite (since $\pi^i \mathcal{O}_K/\pi^{i+1}\mathcal{O}_K \cong k$ is finite). $\mathbb{F}_p = \mathbb{Z}_p/ p\mathbb{Z}_p \hookrightarrow \mathcal{O}_K/p\mathcal{O}_K$, so $\mathcal{O}_K/p \mathcal{O}_K$ is a finite--dimensional vector space over $\mathbb{F}_p$. Let $x_1,\ldots,x_n \in \mathcal{O}_K$ be coset representatives for the $\mathbb{F}_p$--basis of $\mathcal{O}_K/p \mathcal{O}_K$. Then \[
    \left\{\sum_{i=1}^{n} a_j x_j \mid a_j \in \{0,\ldots,p-1\}\right\}
    \] gives a set of coset representatives for $\mathcal{O}_K/p\mathcal{O}_K$.
    \vspace{1mm}
     
    Now apply Proposition 3.3 (ii) to write (for $a_{ij} \in \{0,\ldots,p-1\}$)
    \begin{align*}
        y = \sum_{i=0}^{\infty} \left(\sum_{j=1}^{n} a_{ij}x_j \right)p^i = \sum_{j=1}^{n} \underbrace{\left(\sum_{i=0}^{\infty} a_{ij}p^i \right)}_{\in \mathbb{Z}_p} x_j.
    \end{align*}
    Hence $\mathcal{O}_K$ is finite over $\mathbb{Z}_p$.
\end{proof}
On example sheet 2, we show that if $K$ is a complete archimedean field, then $K \cong \mathbb{R}$ or $K \cong \mathbb{C}$.
\vspace{1mm}
 
In summary, if $K$ is a local field, then either:
\begin{enumerate}[(i)]
    \item $K$ is archimedean, so $K \cong \mathbb{R}$ or $K \cong \mathbb{C}$.
    \item $K$ is nonarchimedean of equal characteristic, so $K \cong \mathbb{F}_{p^n}((t))$.
    \item $K$ is nonarchimedean of mixed characteristic, so $K$ is a finite extension of $\mathbb{Q}_p$.
\end{enumerate}

\section{Global fields}
\begin{defn}
    A \textbf{global field} is a field which is either 
    \begin{enumerate}[(i)]
        \item an algebraic number field.
        \item a global function field, i.e. a finite extension of $\mathbb{F}_p(t)$.
    \end{enumerate}
\end{defn}
\begin{lemma}\label{lemma8.1}
    Let $(K,|\cdot|)$ be a complete discretely valued field and $L/K$ a finite Galois extension with absolute value $|\cdot|_L$ extending $|\cdot|_K$. Then for $x \in L$ and $\sigma \in \text{Gal}(L/K)$, we have $|\sigma(x)|_L = |x|_L$.
\end{lemma}
\begin{proof}
    Since $x \mapsto |\sigma(x)|_L$ is an absolute value on $L$ (as we can check) extending $|\cdot|_K$, our result follows from uniqueness of extensions of absolute values.
\end{proof}
\begin{lemma}[Krasner's lemma]
    Let $(K,|\cdot|)$ be discretely valued and let $f(X) \in K[X]$ be a separable irreducible polynomial with roots $\alpha_1,\ldots,\alpha_n \in \overline{K}$, the separable closure of $K$. Suppose $\beta \in \overline{K}$ is such that \[
    |\beta-\alpha_1|< |\beta-\alpha_i| ~\forall 2\le i\le n.
    \]
    Then $\alpha_1 \in K(\beta)$.
\end{lemma}
\begin{proof}
    Let $L = K(\beta)$ and $L' = L(\alpha_1,\ldots,\alpha_n)$. Then $L'/L$ is a Galois extension. Let $\sigma \in \text{Gal}(L'/L)$. We have $|\beta - \sigma(\alpha_1)| = |\sigma(\beta-\alpha_1)| = |\beta-a_1|$ by the previous lemma and hence $\sigma(\alpha_1)=\alpha_1$, so $\alpha_1 \in K(\beta)$.
\end{proof}
\begin{prop}\label{prop8.3}
    Let $(K,|\cdot|)$ be a complete discretely valued field and let $f(X) = \sum_{i=0}^{n} a_iX^i \in \mathcal{O}_K[X]$ be a separable irreducible monic polynomial. Let $\alpha \in \overline{K}$ be a root of $f$. Then $\exists \epsilon>0$ such that for any other polynomial $g(x) = \sum_{i=0}^{n} b_iX^i \in \mathcal{O}_K[X]$ monic with $|a_i-b_i|<\epsilon ~\forall i$, there exists a root $\beta$ of $g(x)$ such that $K(\alpha)=K(\beta)$.
\end{prop}
Informally, ''nearby'' polynomials define the same extension.
\begin{proof}
    Let $\alpha=\alpha_1,\alpha_2,\ldots,a_n \in \overline{K}$ be the roots of $f$, which are distinct. Then $f'(\alpha_1) \neq 0$. We choose $\epsilon$ such that any $g$ satisfying the conditions has $|g(\alpha_1)|<|f'(\alpha_1)|^2$ and $|f'(\alpha_1) - g'(\alpha_1)| < |f'(\alpha_1)|$. Then  $|g(\alpha_1)| < |f'(\alpha_1)^2| = |g'(\alpha_1)^2|$ (as all triangles are isosceles). By Hensel's lemma applied to the field $K(\alpha_1)$, there exists $\beta \in K(\alpha_1)$ such that $g(\beta)=0$ and $|\beta-\alpha_1|<|g'(\alpha_1)|$. But $$|g'(\alpha_1)| = |f'(\alpha_1)| = \prod_{i=2}^{n} |\alpha_1-\alpha_i| \le |\alpha_1 - \alpha_i|$$ for $2\le i\le n$ (using $|\alpha_1-\alpha_i|\le 1$ since $\alpha_i$ is integral as $f$ is monic). Since $|\beta-\alpha_1|<|\alpha_1-\alpha_i| = |\beta-\alpha_i|$ (again by isosceles condition), Krasner's lemma tells us that $\alpha \in K(\beta)$ and so $K(\alpha)=K(\beta)$.
\end{proof}
\marginpar{29 Oct 2022, Lecture 10}
\begin{theorem}
    Let $K$ be a local field. Then $K$ is the completion of a global field.
\end{theorem}
\begin{proof}
    Case 1: $|\cdot|$ is archimedean. Then $\mathbb{R}, \mathbb{C}$ are the completions of $\mathbb{Q},\mathbb{Q}(i)$, respectively, with respect to $|\cdot|_{\infty}$.
    \vspace{1mm}
     
    Case 2: $|\cdot|$ is non--archimedean and of equal characteristic. Then $K \cong \mathbb{F}_p((t))$, and so $K$ is the completion of $\mathbb{F}_p(t)$ with respect to the $t$--adic absolute value.
    \vspace{1mm}
     
    Case 3: $|\cdot|$ is non--archimedean and of mixed characteristic. Then $K = \mathbb{Q}_p(\alpha)$ for $\alpha$ a root of a monic irreducible polynomial $f(X) \in \mathbb{Z}_p[X]$ (primitive element theorem). Since $\mathbb{Z}$ is dense in $\mathbb{Z}_p$, we choose $g(X) \in \mathbb{Z}[X]$ as in Proposition \ref{prop8.3}. Then $K = \mathbb{Q}_p(\beta)$ for $\beta$ a root of $g(X)$. Since $\mathbb{Q}(\beta)$ is dense in $\mathbb{Q}_p(\beta) =K$, $K$ is the completion of $\mathbb{Q}(\beta)$.
\end{proof}
\section{Dedekind domains}
\begin{defn}
    A Dedekind domain is a ring $R$ such that 
    \begin{enumerate}[(i)]
        \item $R$ is a Noetherian integral domain.
        \item $R$ is integrally closed in $\text{Frac}(R)$.
        \item Every nonzero prime ideal of $R$ is maximal.
    \end{enumerate}
\end{defn}
\begin{example}
    The ring of integers in a number field is a Dedekind domain (we will show this later). This is the prototypical example. Also, any PID (hence DVR) is a Dedekind domain.
\end{example}
\begin{theorem}\label{theorem9.1}
    A ring is a DVR $\iff$ $R$ is a Dedekind domain with exactly one nonzero prime ideal.
\end{theorem}
We start with two lemmas.
\begin{lemma}\label{lemma9.2}
    Let $R$ be a Noetherian ring and $I \subset R$ a nonzero ideal. Then there exist nonzero prime ideals $\mathfrak{p}_1,\ldots,\mathfrak{p}_r$ such that $\mathfrak{p}_1\ldots \mathfrak{p}_r \subset I$.
\end{lemma}
\begin{proof}
    Suppose not. Since $R$ is Noetherian, we can choose $I$ maximal with this property. Then $I$ is not prime, so $\exists x,y \in R\setminus I$ such that $xy \in I$. Let $I_1 = I + (x)$ and $I_2 = I + (y)$. Then by the maximality of $I$, $\exists \mathfrak{p}_1,\ldots,\mathfrak{p}_r$ and $\mathfrak{q}_1,\ldots,\mathfrak{q}_s$ such that $\mathfrak{p}_1\ldots \mathfrak{p}_r \subset I_1$ and $\mathfrak{q_1}\ldots \mathfrak{q}_s \subset I_2$, so $\mathfrak{p}_1 \ldots \mathfrak{p}_r \mathfrak{q}_1 \ldots \mathfrak{q}_s \subset I_1 I_2 \subset I$, a contradiction.
\end{proof}
\begin{lemma}\label{lemma9.3}
    Let $R$ be an integral domain which is integrally closed in $K =\text{Frac}(R)$. Let $0 \neq I \subset R$ be finitely generated and let $x \in K$. If $x I \subset I$, then $x \in R$.
\end{lemma}
\begin{proof}
    Let $I = (c_1,\ldots,c_n)$. We write $xc_i = \sum_{j=1}^{n} a_{ij}c_j$ for $a_{ij} \in R$. Let $A = (a_{ij})$ be the matrix given by the $a_{ij}$ and set $B = xI - A \in M_{n \times n}(K)$. Let $\text{Adj}(B)$ be the adjugate matrix for $B$. Then $B \begin{pmatrix} c_1 \\ \vdots \\ c_n \end{pmatrix} = 0$ in $K^n$, so multiplying by the adjugate gives $\det(B) I \begin{pmatrix} c_1 \\ \vdots \\ c_n \end{pmatrix} = 0 \implies \det(B) = 0$. But $\det(B)$ is just a monic polynomial in $x$ with coefficients in $R$. Thus $x$ is integral over $R$, so $x \in R$ as $R$ is integrally closed.
\end{proof}
\begin{proof}[Proof of Theorem 9.1]
    $(\implies )$: This is clear, as any PID, so any DVR, is a Dedekind domain.
    \vspace{1mm}
     
    $(\impliedby)$: We need to show that $R$ is a PID. The assumption implies that $R$ is a local ring with unique maximal ideal $\mathfrak{m}$. 
    \vspace{1mm}
     
    Step 1: $\mathfrak{m}$ is principal. Let $0 \neq x \in \mathfrak{m}$. By Lemma \ref{lemma9.2}, $(x) \supset \mathfrak{m}^n$ for some $n\ge 1$. Let $n$ be minimal such that $(x) \supset \mathfrak{m}^n$. Then we may choose $y \in \mathfrak{m}^{n-1} \setminus (x)$. Set $\pi = \frac{x}{y}$. Then we have $y \mathfrak{m} \subset \mathfrak{m}^n \subset (x) \implies p^{-1}\mathfrak{m} \subset R$. If $\pi$ is a proper ideal and not the whole ring, then $\pi^{-1} \mathfrak{m} \subset \mathfrak{m}$, so $\pi^{-1} \in R$ by Lemma \ref{lemma9.3}. Thus $y \in (x)$, a contradiction. Hence $\pi^{-1} \mathfrak{m} = R \implies \mathfrak{m} = \pi R$ is principal.
    \vspace{1mm}
     
    Step 2: $R$ is a PID. Let $I \subset R$ be a nonzero ideal. Consider the sequence of fractional ideals $I \subset \pi^{-1} I \subset \pi^{-2} I  \subset \ldots$ in $K$. Since $\pi^{-1} \not\in R$, we have $\pi^{-k}I \neq \pi^{-k+1}I ~\forall k$ by Lemma \ref{lemma9.3}. Since $R$ is Noetherian, we may choose $n$ maximal such that $\pi^{-n} I \subset R$. If $\pi^{-n}I \subset \mathfrak{m} = (\pi)$, then $\pi^{-(n+1)}I \subset R$, contradicting the maximality of $R$. Hence $\pi^{-n} I = R \implies I = \pi^n R$.
\end{proof}

\begin{defn}
    Let $R$ be an integral domain and let $S \subset R$ be a multiplicatively closed subset (i.e. $1 \in S$ and $x,y \in S \implies xy \in S$). The \textbf{localization} $S^{-1}R$ of $R$ with respect to $S$ is the ring \[
S^{-1}R = \left\{\frac{r}{s} \mid r \in R, s \in S\right\} \subset \text{Frac}(R).
\]
\end{defn}
If $\mathfrak{p}$ is a prime ideal in $R$, we write $R_{(\mathfrak{p})}$ for the localization with respect to $S = R \setminus \mathfrak{p}$.
\begin{example}
    \begin{itemize}
        \item If $\mathfrak{p} = 0$, then $R_{(\mathfrak{p})} = \text{Frac}(R)$.
        \item If $R = \mathbb{Z}$, then $\mathbb{Z}_{(p)} = \left\{\frac{a}{b} \mid a, b \in \mathbb{Z}, (b,p)=1\right\}$ (as seen before as a valuation ring).
    \end{itemize}
\end{example}
\textbf{Fact.} $R$ Noetherian $\implies $ $S^{-1}R$ Noetherian.
\vspace{1mm}
 
\textbf{Fact.} There exists a bijection between
\begin{align*}
    \{\text{prime ideals in }S^{-1}R\} \leftrightarrow& \{\text{prime ideals } \mathfrak{p} \text{ in }R\text{ with }\mathfrak{p} \cap S = \varnothing\}.\\
    \mathfrak{p}S^{-1}R \mapsfrom & ~\mathfrak{p}.
\end{align*}
\begin{cor}\label{cor9.4}
    Let $R$ be a Dedekind domain and $\mathfrak{p} \subset R$ a nonzero prime ideal. Then $R_{(\mathfrak{p})}$ is a DVR. \footnote{This is the correct way to think about Dedekind domains.}
\end{cor}
\begin{proof}
    By properties of localization, $R_{(\mathfrak{p})}$ is a Noetherian integral domain with a unique nonzero prime ideal $\mathfrak{p}R_{(\mathfrak{p})}$. It suffices to show that $R_{(\mathfrak{p})}$ is integrally closed in $\text{Frac}(R_{(\mathfrak{p})}) = \text{Frac}(R)$, since then the localization of $\mathfrak{p}$ is a Dedekind domain by Theorem \ref{theorem9.1}. \vspace{1mm}
     
    Let $x \in \text{Frac}(R)$ be integral over $R_{(\mathfrak{p})}$. Multiplying out by the denominators of a monic polynomial satisfied by $x$, we obtain \[
    sx^n + a_{n-1}x^{n-1} + \ldots + a_0  = 0
    \]
    where $a_i \in R, s \in S$. Multiply this by $s^{-1}$ to get that $xs$ is integral over $R$ and hence $xs \in R$, thus $x \in R_{(\mathfrak{p})}$.
\end{proof}

\marginpar{31 Oct 2022, Lecture 11}

\begin{defn}
    If $R$ is a Dedekind domain and $\mathfrak{p} \subset R$ is a nonzero prime ideal, we write $v_{\mathfrak{p}}$ for the normalized valuation on $\text{Frac}(R) = \text{Frac}(R_{(\mathfrak{p})})$ corresponding to the DVR $R_{(\mathfrak{p})}$.
\end{defn}
\begin{example}
    If $R = \mathbb{Z}$ and $\mathfrak{p} = (p)$, then $v_p$ is the $p$--adic valuation.
\end{example}
\begin{theorem}\label{theorem9.5}
    Let $R$ be a Dedekind domain. Then every nonzero prime ideal $R$ can be written uniquely as a product of prime ideals.
\end{theorem}
\textbf{Remark.} This is clear for PIDs (as PID $\implies $ UFD).
\begin{proof}[Sketch of proof.]
    We quote the following properties of localization:
    \begin{enumerate}[(i)]
        \item $I=J \iff IR_{(\mathfrak{p})} = JR_{(\mathfrak{p})} ~\forall \mathfrak{p}$ prime ideals (and $I, J \subset R$ ideals).
        \item If $R$ is a Dedekind domain and $\mathfrak{p}_1, \mathfrak{p}_2$ are nonzero prime ideals, then $\mathfrak{p}_1 R_{(\mathfrak{p}_2)} = \begin{cases}
            R_{(\mathfrak{p}_2)} & \mathfrak{p}_1 \neq \mathfrak{p}_2.\\
            \mathfrak{p}_2 R_{(\mathfrak{p}_2)} & \mathfrak{p}_1 = \mathfrak{p}_2.
        \end{cases}$
    \end{enumerate}
    Let $I \subset R$ be a nonzero ideal. Then by Lemma \ref{lemma9.2} there exist distinct prime ideals $\mathfrak{p}_1,\ldots,\mathfrak{p}_r$ such that $\mathfrak{p}_1^{\beta_1}\ldots \mathfrak{p}_r^{\beta_r} \subset I$, where $\beta_i > 0$. Let $0 \neq \mathfrak{p}$ be a prime ideal, $\mathfrak{p} \not\in \{\mathfrak{p_1},\ldots,\mathfrak{p}_r\}$. Then by (ii), $\mathfrak{p}_i R_{(\mathfrak{p})} = R_{(\mathfrak{p})}$ and hence $IR_{(\mathfrak{p})} = R_{(\mathfrak{p})}$.
    \vspace{1mm}
     
    By Corollary \ref{cor9.4}, $I R_{(\mathfrak{p}_i)} = (\mathfrak{p}_i R_{(\mathfrak{p}_i)})^{\alpha_i} = \mathfrak{p}_i^{\alpha_i}R_{(\mathfrak{p}_i)}$ for some $0\le \alpha_i \le \beta_i$. Thus $I = \mathfrak{p}_1^{\alpha_1} \ldots \mathfrak{p}_r^{\alpha_r}$ by (i).
    \vspace{1mm}
     
    For uniqueness, if $I = \mathfrak{p}_1^{\alpha_1}\ldots \mathfrak{p}_r^{\alpha_r} = \mathfrak{p}_1^{\gamma_1} \ldots \mathfrak{p}_r^{\gamma_r}$, then $\mathfrak{p}_i^{\alpha_i}R_{(p_i)} = \mathfrak{p}_i^{\gamma_i}R_{(\mathfrak{p}_i)} \implies \alpha_i = \gamma_i$ by unique factorization in DVRs.
\end{proof}

\section{Dedekind domains and extensions}

Let $L/K$ be a finite extension. For $x \in L$, we write $\text{Tr}_{L/K}(x)$ for the trace of the $K$--linear map $L \to L$ mapping $y \mapsto xy$. If $L/K$ is separable of degree $n$ and $\sigma_1,\ldots,\sigma_n : L \to \overline{K}$ are the set of embeddings of $L$ into an algebraic closure $\overline{K}$ of $K$, then $\text{Tr}_{L/K}(x)=\sum_{i=1}^{n} \sigma_i(x) \in K$.

\begin{lemma}
    Let $L/K$ be a finite separable extension of fields. Then the symmetric bilinear pairing $(\cdot ,\cdot ) : L \times L \to K$ by $(x,y) \mapsto \text{Tr}_{L/K}(xy)$ is non--degenerate.
\end{lemma}
\begin{proof}
    $L/K$ is separable, so $L=K(\alpha)$ for some $\alpha \in L$. Consider the matrix $A$ for $(\cdot ,\cdot )$ in the $K$--basis for $L$ given by $\{1,\alpha,\alpha^2,\ldots,\alpha^{n-1}\}$. Then $A_{ij} = \text{Tr}_{L/K}(\alpha^{i+j}) = [BB^T]_{ij}$ for $B = \begin{pmatrix} 1 & 1 & \ldots & 1 \\
    \sigma_1(\alpha) & \sigma_2(\alpha) & \ldots & \sigma_n(\alpha) \\
\vdots & & \\
\sigma_1(\alpha^{n-1}) & \sigma_2(\alpha^{n-1}) & \ldots & \sigma_n(\alpha^{n-1}) \end{pmatrix}$. Then $\det A =(\det B)^2$, but $\det B = \prod_{1\le i < j\le n}^{} (\sigma_i(\alpha)-\sigma_j(\alpha))$, the Vandermonde determinant. Hence $\det A$ is nonzero since $\sigma_i(\alpha) \neq \sigma_j(\alpha)$ for $i \neq j$ by separabalility.
\end{proof}
The converse is also true and is left as an exercise on example sheet 3: A finite extension $L/K$ is separable if and only if the trace form is nondegenerate.

\begin{theorem}\label{theorem10.2}
    Let $\mathcal{O}_K$ be a Dedekind domain and $L$ a finite separable extension of $K = \text{Frac}(\mathcal{O}_K)$. Then the integral closure $\mathcal{O}_L$ of $\mathcal{O}_K$ in $L$ is a Dedekind domain. 
\end{theorem}
\begin{proof}
    $\mathcal{O}_L$ is the subring of $L$, so $\mathcal{O}_L$ is an integral domain. Hence we need to show:
    \begin{enumerate}[(i)]
        \item $\mathcal{O}_L$ is Noetherian.
        \item $\mathcal{O}_L$ is integrally closed in $L$.
        \item Every nonzero prime ideal $\mathfrak{p}$ in $\mathcal{O}_L$ is maximal.  
    \end{enumerate}
    We prove:
    \begin{enumerate}[(i)]
        \item Let $e_1,\ldots,e_n \in L$ be a $K$--basis for $L$. Upon scaling by $K$, we may assume $e_i \in \mathcal{O}_L ~\forall i$. Let $f_i \in L$ be the dual basis with respect to the trace form $(\cdot ,\cdot )$. Let $x \in \mathcal{O}_L$ and write $x = \sum_{i=1}^{n} \lambda_i f_i$ for $\lambda_i \in K$. Then $\lambda_i = \text{Tr}_{L/K}(x e_i) \in \mathcal{O}_K$. Hence for any $z \in \mathcal{O}_L$, $\text{Tr}_{L/K}(z)$ is a sum of elements in $\overline{K}$ which are integral over $\mathcal{O}_K \implies \text{Tr}_{L/K}(z) \in K$ is integral over $\mathcal{O}_K$, so $\text{Tr}_{L/K}(z) \in \mathcal{O}_K$. Thus $\mathcal{O}_L \subset \mathcal{O}_Kf_1 + \ldots + \mathcal{O}_Kf_n$. Since $\mathcal{O}_K$ is Noetherian, $\mathcal{O}_L$ is finitely generated as an $\mathcal{O}_K$--module, hence $\mathcal{O}_L$ is Noetherian.
        \item Left as an exercise on example sheet 2.
        \item Let $P$ be a nonzero prime ideal in $\mathcal{O}_L$ and define $\mathfrak{p} = P \cap \mathcal{O}_K$, a prime ideal of $\mathcal{O}_K$. Let $0 \neq x \in P$, then $x$ satisfies the equation $x^n + a_{n-1}x^{n-1}+ \ldots + a_0$, where $a_i \in \mathcal{O}_K$ and $a_0 \neq 0$. Then $0 \neq a_0 \in \mathcal{O}_K \cap P = \mathfrak{p}$, so $\mathfrak{p}$ is nonzero and hence maximal.
        \vspace{1mm}
         
        We have an injection $\mathcal{O}_K/\mathfrak{p} \to \mathcal{O}_L/P$ and $\mathcal{O}_L/P$ is a finite--dimensional vector space over $\mathcal{O}_K/\mathfrak{p}$. Since $\mathcal{O}_L/P$ is an integral domain, it is a field (e.g. by applying rank--nullity to the multiplication map $y \mapsto zy$). Hence $P$ is maximal.
    \end{enumerate}
\end{proof}
\textbf{Remark.} This theorem holds even without the assumption that $L/K$ is separable.
\begin{cor}
    The ring of algebraic integers in a number field is a Dedekind domain.
\end{cor}
\textbf{Convention.} For $\mathcal{O}_K$ the ring of integers of a number field and $\mathfrak{p} \subset \mathcal{O}_K$ a nonzero prime ideal, we normalize $|\cdot|_{\mathfrak{p}}$ (the absolute value associated to $v_\mathfrak{p}$) by $|x|_{\mathfrak{p}} = N_{\mathfrak{p}}^{-v_\mathfrak{p}(x)}$ for $N_{\mathfrak{p}} = |\mathcal{O}_K/\mathfrak{p}|$.

\marginpar{02 Nov 2022, Lecture 12}

Let us fix $\mathcal{O}_K$ to be a Dedekind domain with fraction field $K = \text{Frac}(\mathcal{O}_K)$. Let $L/K$ be a finite separable extension and $\mathcal{O}_L$ the integral closure of $\mathcal{O}_K$ inside $L$ (which is a Dedekind domain by Theorem \ref{theorem10.2}).

\begin{lemma}\label{lemma10.4}
    Let $0 \neq x \in \mathcal{O}_K$. Then $$(x) = \prod_{p \neq 0 \text{ prime}}^{} p^{v_p(x)}.$$
\end{lemma}
\begin{proof}
    $x \mathcal{O}_{K,(p)} = (p \mathcal{O}_{K,(p)})^{v_p(x)}$ by definition of $v_p(x)$. In particular, $\{p \neq 0 \mid v_p(x) \neq 0\}$ is finite. Then the lemma follows from properties of localization stated last time: $I = J \iff I \mathcal{O}_{K,(p)} = J \mathcal{O}_{K,(p)} ~\forall $ prime ideals $p$.
\end{proof}
\textbf{Notation.} $\mathcal{P} \subset \mathcal{O}_L$ and $\mathfrak{p} \subset \mathcal{O}_K$ will always denote prime ideals. We write $\mathcal{P} \mid \mathfrak{p}$ if $\mathfrak{p}\mathcal{O}_L = \mathcal{P}_1^{e_1}\ldots \mathcal{P}_r^{e_r}$ and $\mathcal{P} \in \{\mathcal{P}_1,\ldots,\mathcal{P}_r\}$ for $e_i>0$ and $\mathcal{P}_i$ distinct prime ideals. 

\begin{theorem}\label{theorem10.5}
    Let $\mathcal{O}_K, \mathcal{O}_L, K, L$ be as above. For $\mathfrak{p}$ a nonzero prime ideal of $\mathcal{O}_K$, write $\mathfrak{p}\mathcal{O}_L = \mathcal{P}_1^{e_1}\ldots \mathcal{P}_r^{e_r}$. Then the absolute values on $L$ extending $|\cdot|_{\mathfrak{p}}$ (up to equivalence) are precisely $|\cdot|_{\mathcal{P}_1},\ldots,|\cdot|_{\mathcal{P}_r}$.
\end{theorem}
\begin{proof}
    By Lemma \ref{lemma10.4}, for any $0 \neq x \in \mathcal{O}_K$ and $1\le i\le r$, we have $v_{\mathcal{P}_i}(x) = e_i v_{\mathfrak{p}}(x)$. Hence up to equivalence, $|\cdot|_{\mathcal{P}_i}$ does extend $|\cdot|_{\mathfrak{p}}$.
    \vspace{1mm}
     
    Conversely, suppose $|\cdot|$ is an absolute value on $L$ which extends $|\cdot|_{\mathfrak{p}}$. Then $|\cdot|_{\mathfrak{p}}$ is bounded on $\mathbb{Z}$ and hence $|\cdot|$ is non--archimedean. Now let $$R = \{ x \in L \mid |x|\le 1\} \subset L$$ be the valuation for $L$ with respect to $|\cdot|$. Then $\mathcal{O}_K \subset R$ and since $R$ is integrally closed in $L$ (by Lemma \ref{lemma6.5}), we have $\mathcal{O}_L \subset R$. Set $\mathcal{P} = \{ x \in \mathcal{O}_L \mid  |x|<1\} = \mathfrak{m}_R \cap \mathcal{O}_L$. Then $\mathcal{P}$ is a prime ideal in $R$ and it is nonzero as it contains $\mathfrak{p}$. Then $\mathcal{O}_{L,(\mathcal{P})} \subset R$ because $s \in \mathcal{O}_L\setminus \mathcal{P} \implies |s| = 1$. But $\mathcal{O}_{L,(\mathcal{P})}$ is a DVR, hence a maximal subring of $L \implies \mathcal{O}_{L,(\mathcal{P})} = R$. Hence $|\cdot|$ is equivalent to $|\cdot|_{\mathcal{P}}$. Since $|\cdot|$ extends to $|\cdot|_{\mathfrak{p}}$, $\mathcal{P} \cap \mathcal{O}_K = \mathfrak{p}$, so $\mathcal{P}_1^{e_1}\ldots \mathcal{P}_r^{e_r} \subset \mathcal{P} \implies \mathcal{P} = \mathcal{P}_i$ for some $i$.
\end{proof}
Let $K$ be a number field. If $\sigma : K \twoheadrightarrow \mathbb{R}, \mathbb{C}$ is a real or complex embedding, then $x \mapsto |\sigma(x)|_{\infty}$ defines an absolute value on $K$, denoted by $|\cdot|_{\sigma}$. (This is on example sheet 2).

\begin{cor}
    Let $K$ be a number field with ring of integers $\mathcal{O}_K$. Then any absolute value on $K$ is equivalent to either 
    \begin{enumerate}[(i)]
        \item $|\cdot|_{\mathfrak{p}}$ for some nonzero prime ideal $\mathfrak{p} \subset \mathcal{O}_K$.
        \item $|\cdot|_{\sigma}$ for some embedding $\sigma : K \twoheadrightarrow \mathbb{R}, \mathbb{C}$.
    \end{enumerate}
\end{cor}
\begin{proof}
    Case 1: $|\cdot|$ is non--archimedean. Then $|\cdot|_{\mathbb{Q}}$ is equivalent to $|\cdot|_{p}$ for some prime $p$ by Ostrowski's theorem (Theorem \ref{theorem7.6}). Then by Theorem \ref{theorem10.5}, $|\cdot|$ is equivalent to $|\cdot|_{\mathfrak{p}}$ for some $\mathfrak{p} \mid p$ a prime ideal in $\mathcal{O}_K$.
    \vspace{1mm}
     
    Case 2: $|\cdot|$ is archimedean. This is an exercise on example sheet 2.
\end{proof}
\subsection{Completions}
Let $\mathcal{O}_K$ be a Dedekind domain and $L/K$ a finite separable extension. Let $\mathfrak{p} \subset \mathcal{O}_K, \mathcal{P} \subset \mathcal{O}_L$ be nonzero prime ideals with $\mathcal{P} \mid  \mathfrak{p}$. We write $K_{\mathfrak{p}}$ and $L_{\mathcal{P}}$ for the completions of $K$ and $L$ with respect to the absolute values $|\cdot|_{\mathfrak{p}}$ and $|\cdot|_{\mathcal{P}}$ respectively.

\begin{lemma}\label{lemma10.7}
    \begin{enumerate}[(i)]
        \item The natural map $\Pi_p : L \otimes_K K_{\mathfrak{p}} \to L_{\mathcal{P}}$ is surjective.
        \item $[L_{\mathcal{P}}:K_{\mathfrak{p}}]\le [L:K]$.
    \end{enumerate}
\end{lemma}
\begin{proof}
    Let $M = \text{Im}(\Pi_p) = LK_{\mathfrak{p}} \subset L_\mathcal{P}$. Write $L = K(\alpha)$, so $M = K_{\mathfrak{p}}(\alpha)$. Hence $M$ is a finite extension of $K_{\mathfrak{p}}$ and $[M:K_{\mathfrak{p}}]\le [L:K]$. Moreover, $M$ is complete (by Theorem \ref{theorem6.1}) and $L \subset M \subset L_{\mathcal{P}}$, hence $M = L_{\mathcal{P}}$, so both results follow.
\end{proof}
\begin{lemma}[CRT for commutative rings]
    Let $R$ be a ring and $I_1,\ldots,I_n \subset R$ be ideals such that $I_i + I_j = R ~\forall i \neq j$ (i.e. the ideals are pairwise coprime). Then:
    \begin{enumerate}[(i)]
        \item $\bigcap_{i=1}^n I_i = \prod_{i=1}^{n} I_i$ (call this product $I$).
        \item $R/I \cong \prod_{i=1}^{n} (R/I_i)$.
    \end{enumerate}
\end{lemma}
\begin{proof}
    Exercise on example sheet 2.
\end{proof}
\begin{theorem}
    The natural map $L \otimes_K K_{\mathfrak{p}} \to \prod_{\mathcal{P} \mid \mathfrak{p}}^{} L_{\mathcal{P}}$ is an isomorphism.
\end{theorem}
\begin{proof}
    Write $L = K(\alpha)$ and let $f(X) \in K[X]$ be the minimal polynomial of $\alpha$. Then we have $f(X) = f_1(X)\ldots f_r(X)$ in $K_{\mathfrak{p}}[X]$ for $f_i(X) \in K_{\mathfrak{p}}[X]$ distinct and irreducible (also separable). Since $L = K[X]/f(X)$, 
    \begin{align*}
        L \otimes_K K_{\mathfrak{p}} \cong K_{\mathfrak{p}}[X]/f(X) \stackrel{\text{CRT}}{=} \prod_{i=1}^{r} K_{\mathfrak{p}}[X]/f_i(X). 
    \end{align*}
    Set $L_i = K_{\mathfrak{p}}[X]/f_i(X)$, a finite extension of $K$. Then $L_i$ contains both $L$ and $K_{\mathfrak{p}}$ (using the fact that $K[X]/f(X) \to K_{\mathfrak{p}}[X]/f_i(X)$ is injective, since it is a morphism of fields). Moreover, $L$ is dense inside $L_i$ (since we can approximate coefficients of $K_{\mathfrak{p}}[X]/f_i(X)$ with an element $K[X]/f(X)$ and all norms on this finite--dimensional vector space are equivalent). The theorem now follows from the following three claims:
    \begin{enumerate}[(i)]
        \item $L_i \cong L_{\mathcal{P}}$ for some prime $\mathcal{P} \subset \mathcal{O}_L$ with $\mathcal{P} \mid \mathfrak{p}$.
        \item Each $\mathcal{P}$ appears at most once.
        \item Each $\mathcal{P}$ appears at least once.
    \end{enumerate}
    To prove these:
    \begin{enumerate}[(i)]
        \item Since $[L_i : K_{\mathfrak{p}}]<\infty$, there is a unique absolute value $|\cdot|$ on $L_i$ extending $|\cdot|_{\mathfrak{p}}$ on $K_{\mathfrak{p}}$. Then Theorem \ref{theorem10.5} implies that $|\cdot|_{L_i}$ is equivalent to $|\cdot|_{\mathcal{P}}$ for some $\mathcal{P} \mid \mathfrak{p}$. Since $L$ is dense in $L_i$ and $L_i$ is complete, we must have $L = L_{\mathcal{P}}$.
        \item Suppose $\phi : L_i \to L_j$ is an isomorphism preserving $L$ and $K_{\mathfrak{p}}$. Then $\phi: K_{\mathfrak{p}}[X]/f_i(X) \to K_{\mathfrak{p}}[X]/f_j(X)$ takes $X$ to $X$ and hence $f_i(X) = f_j(X) \implies i = j$.
        \item By Lemma \ref{lemma10.7}, the natural map $\Pi_p : L \otimes_K K_{\mathfrak{p}} \to L_\mathcal{P}$ is surjective for any $\mathcal{P} \mid  \mathfrak{p}$. Since $L_\mathcal{P}$ is a field, $\Pi_p$ factors through $L_i$ for some $i$ and hence $L_i \cong L_{\mathcal{P}}$ by surjectivity.
    \end{enumerate}
\end{proof}

\begin{example}
    If $K=\mathbb{Q}$ and $L = \mathbb{Q}(i)$, then $f(X)=X^2+1$. So either by Hensel or the computation done in the first lecture, $i \in \mathbb{Q}_{5}$. Hence $(5)$ splits in $\mathbb{Q}(i)$, so $5\mathcal{O}_L = \mathfrak{p}_1 \mathfrak{p}_2$.
\end{example}

\marginpar{04 Nov 2022, Lecture 13}

\begin{cor}\label{cor10.10}
    Take $0 \neq \mathfrak{p} \subset \mathcal{O}_K$ a prime ideal. For $x \in L$, $$N_{L/K}(x) = \prod_{\mathcal{P} \mid  \mathfrak{p}}^{} N_{L_{\mathcal{P}}/K_{\mathfrak{p}}}(x).$$
\end{cor}
\begin{proof}
    Let $B_1,\ldots,B_r$ be a basis for $L_{\mathcal{P}_1},\ldots, L_{\mathcal{P}_r}$ as $K_{\mathfrak{p}}$--vector spaces (here $\mathfrak{p}\mathcal{O}_L = \mathcal{P}_1^{e_1}\ldots \mathcal{P}_r^{e_r}$). Then $B = \bigcup_{i} B_i$ is a basis for $L \otimes_K K_{\mathfrak{p}}$ over $K_{\mathfrak{p}}$. Let $[\text{mult}(x)]_{B}$ (respectively $\text{mult}(x)_{B_i}$) denote the matrix for the multiplication by $x$ map $\text{mult}(x) : L \otimes_K K_{\mathfrak{p}} \to L \otimes_K K_{\mathfrak{p}}$ (respectively $L_{\mathcal{P}_i} \to L_{\mathcal{P}_i}$) with respect to $B$ (respectively the $B_i$). Then we get a block matrix
    \begin{align*}
        [\text{mult}(x)]_B = \begin{pmatrix} [\text{mult}(x)]_{B_1} & & \\
        & [\text{mult}(x)]_{B_2} & \\ & &\ddots & \\ & & & [\text{mult}(x)]_{B_r} \end{pmatrix}
        \\
        \implies N_{L/K}(x) = \det([\text{mult}(x)]_B) = \prod_{i=1}^{r} \det([\text{mult}(x)]_{B_i}) = \prod_{i=1}^{r} N_{L_{\mathcal{P}_i}/K_{\mathfrak{p}}}(x).
    \end{align*}
\end{proof}

\section{Decomposition groups}
As before, let us work over a finite separable Dedekind domain. Let $\mathfrak{p}$ be a nonzero prime ideal of $\mathcal{O}_K$ and write $\mathfrak{p}\mathcal{O}_L = \mathcal{P}_1^{e_1}\ldots \mathcal{P}_r^{e_r}$.
\vspace{1mm}
 
\textbf{Note.} For any $i$, $\mathfrak{p} \subset \mathcal{P}_i \cap \mathcal{O}_K \subsetneq \mathcal{O}_K$, hence $\mathfrak{p} = \mathcal{P}_i \cap \mathcal{O}_K$.

\begin{defn}
    \begin{enumerate}[(i)]
        \item We say $\mathfrak{p}$ \textbf{ramifies} in $L$ if $e_i>1$ for some $i$.
        \item The $e_i$ are called the \textbf{ramification indices} of $\mathcal{P}_i$ over $\mathfrak{p}$.
    \end{enumerate}
\end{defn}
\begin{example}
    If $\mathcal{O}_K=\mathbb{C}[t], \mathcal{O}_L=\mathbb{C}[T]$, then consider the map $\mathcal{O}_K \to \mathcal{O}_L$ by $t \mapsto T^n$. Then $t \mathcal{O}_L = T^n \mathcal{O}_L$, so the ramification index of $(T)$ over $(t)$ is $n$.
    \vspace{1mm}
     
    This corresponds geometrically to the degree $n$ covering of Riemann surfaces $\mathbb{C} \to \mathbb{C}$ by $x \mapsto x^n$. This map is ramified at $0$ with ramification index $n$.
\end{example}

\begin{defn}
    We define $f_i = [\mathcal{O}_L/\mathcal{P}_i : \mathcal{O}_K/\mathfrak{p}]$, called the \textbf{residue class degree} of $\mathcal{P}_i$ over $\mathfrak{p}$.
\end{defn}

\begin{theorem}\label{theorem11.1}
    $\sum_{i=1}^{r} e_i f_i = [L:K]$.
\end{theorem}
\begin{proof}
    Let $S = \mathcal{O}_K/\mathfrak{p}$. The following properties of localization are left as an exercise:
    \begin{enumerate}[(1)]
        \item $S^{-1}\mathcal{O}_L$ is the integral closure of $S^{-1}\mathcal{O}_K$ in $L$.
        \item $S^{-1}\mathfrak{p} S^{-1}\mathcal{O}_L \cong S^{-1} \mathcal{P}_1^{e_1}\ldots S^{-1}\mathcal{P}_r^{e_r}$.
        \item $S^{-1}\mathcal{O}_L/S^{-1}\mathcal{P}_i \cong \mathcal{O}_L/\mathcal{P}_i$ and $S^{-1}\mathcal{O}_K/S^{-1}\mathfrak{p} \cong \mathcal{O}_K/\mathfrak{p}$.
    \end{enumerate}
    In particular, (2) and (3) imply that $e_i, f_i$ don't change when we replace $\mathcal{O}_K$ and $\mathcal{O}_L$ by $S^{-1}\mathcal{O}_K$ and $S^{-1}\mathcal{O}_L$. Thus we may assume that $\mathcal{O}_K$ is a DVR (and hence a PID). By CRT, we have $$\mathcal{O}_L/\mathfrak{p}\mathcal{O}_L \cong \prod_{i=1}^{r} \mathcal{O}_L/\mathcal{P}_i^{e_i}.$$ Now it suffices to count dimensions on both sides as $k=\mathcal{O}_K/\mathfrak{p}$--vector spaces.
    \vspace{1mm}
     
    RHS: For each $i$, we have a decreasing sequence of $k$--subspaces 
    \begin{align*}
        0 \subset \mathcal{P}_i^{e_i-1}/\mathcal{P}_i^{e_i} \subset \ldots \subset  \mathcal{P}_i/\mathcal{P}_i^{e_i} \subset \mathcal{O}_L/\mathcal{P}_i^{e_i}.
    \end{align*}
    Note that $\mathcal{P}_i^j/\mathcal{P}_i^{j+1}$ is an $\mathcal{O}_L/\mathcal{P}_i$ module that is generated by $ x \in \mathcal{P}_i^j / \mathcal{P}_i^{j+1}$. (For example, we can prove this after localizing at $\mathcal{P}_i$). Then $\text{dim}_k(\mathcal{P}_i^j/\mathcal{P}_i^{j+1}) = f_i$ and we have $\text{dim}_k(\mathcal{O}_L/\mathcal{P}_i^{e_i})= e_if_i$. Hence $\text{dim}_k(\text{RHS})=\sum_{i=1}^{r} e_if_i$.
    \vspace{1mm}
     
    LHS: The structure theorem for finitely generated modules over PID's tells us that $\mathcal{O}_L$ is a free module over $\mathcal{O}_K$ of rank $[L:K]$. Thus $\mathcal{O}_L/\mathfrak{p}\mathcal{O}_L \cong (\mathcal{O}_K/\mathfrak{p})^{n}$ as $\mathcal{O}_K$--modules and hence $\text{dim}_k(\mathcal{O}_L/\mathfrak{p}\mathcal{O}_L)=n$.
\end{proof}
Geometric analogue: Let $X \to Y$ be a degree $n$ cover of compact Riemann surfaces. For $y \in Y$, $n = \sum_{x \in f^{-1}(y)} e_x$ for $e_x$ the ramification index of $x.$
\vspace{1mm}
 
Now assume $[L:K]$ is Galois. Then for any $\sigma \in \text{Gal}(L/K)$, $\sigma(\mathcal{P}_i) \cap \mathcal{O}_K =\mathfrak{p}$, hence $\sigma(\mathcal{P}_i) \in \{\mathcal{P}_1,\ldots,\mathcal{P}_r\}$, i.e. $\text{Gal}(L/K)$ acts on $\{\mathcal{P}_1,\ldots,\mathcal{P}_r\}$.
\begin{prop}\label{prop11.2}
    The action of $\text{Gal}(L/K)$ on $\{\mathcal{P}_1,\ldots,\mathcal{P}_r\}$ is transitive.
\end{prop}
\begin{proof}
    Suppose not, so $\exists i \neq j$ such that $\sigma(\mathcal{P}_i) \neq \mathcal{P}_j ~\forall \sigma \in \text{Gal}(L/K)$. By CRT, we may choose $x \in \mathcal{O}_L$ such that $x \equiv 0 \pmod{\mathcal{P}_i}$ and $x \equiv 1 \pmod{\sigma(\mathcal{P}_j)}~\forall \sigma \in \text{Gal}(L/K)$. Then 
    \begin{align*}
        N_{L/K}(x) = \prod_{\sigma \in \text{Gal}(L/K)}^{} \sigma(x) \in \mathcal{O}_K \cap\mathcal{P}_i = \mathfrak{p} \subset \mathcal{P}_j. 
    \end{align*}
    Since $\mathcal{P}_j$ is prime, there must exist some $\tau \in \text{Gal}(L/K)$ such that $\tau(x) \in \mathcal{P}_j \implies x \in \tau^{-1}(\mathcal{P}_j)$, so $x \equiv 0 \pmod{\tau^{-1}(\mathcal{P}_j)}$, a contradiction.
\end{proof}
\begin{cor}\label{cor11.3}
    Suppose $L/K$ is Galois. Then $e_1 = e_2 = \ldots = e_r = e$, $f_1 = \ldots = f_r = f$ and hence $n = efr$.
\end{cor}
\begin{proof}
    For any $\sigma \in \text{Gal}(L/K)$, we have 
    \begin{enumerate}[(i)]
        \item $\mathfrak{p} = \sigma(\mathfrak{p}) = \sigma(\mathcal{P}_1)^{e_1}\ldots \sigma(\mathcal{P}_r)^{e_r}$. Hence $e_1 = \ldots = e_r$ since the Galois group acts transitively.
        \item $\mathcal{O}_L/\mathcal{P}_i \cong \mathcal{O}_L/\sigma(\mathcal{P}_i)$ via $\sigma$, so $f_1 = \ldots = f_r$.
    \end{enumerate}
    The formula now follows from Theorem \ref{theorem11.1}.
\end{proof}
If $L/K$ is an extension of complete discretely valued fields with normalized valuations $v_L$ and $v_K$ with uniformizers $\pi_L$, $\pi_K$, then the ramification index is $e = e_{L/K} = v_L(\pi_K)$ (i.e. $\pi_K \mathcal{O}_L = \pi_L^e \mathcal{O}_L$). The residue class degree is $f = f_{L/K} = [k_L:k]$.
\begin{cor}\label{cor11.4}
    Let $L/K$ be finite and separable. Then $[L:K] = ef$. \footnote{I think this should instead read "with $L/K$ as in the setup above, $[L:K] = ef$".}
\end{cor}
\textbf{Remark.} This corollary holds even if $L/K$ is not separable.

\vspace{1mm}
Now let $\mathcal{O}_K$ be a Dedekind domain again.
\begin{defn}
    Let $L/K$ be a finite Galois extension. The \textbf{decomposition group} at a prime $\mathcal{P}$ of $\mathcal{O}_L$ is the subgroup of $\text{Gal}(L/K)$ defined by 
    \begin{align*}
        G_{\mathcal{P}} = \{\sigma \in \text{Gal}(L/K) \mid \sigma(\mathcal{P}) = \mathcal{P}\}.
    \end{align*}
\end{defn}
By Proposition \ref{prop11.2}, for any $\mathcal{P}, \mathcal{P}'$ dividing $\mathfrak{p}$, $G_{\mathcal{P}}$ and $G_{\mathcal{P}'}$ are conjugate and hence have size $ef$ by the orbit--stabilizer theorem.

\marginpar{07 Nov 2022, Lecture 14}

\begin{prop}
    Suppose $\mathcal{P} \mid \mathfrak{p} \subset \mathcal{O}_K$. Then 
    \begin{enumerate}[(i)]
        \item $L_{\mathcal{P}}/K_{\mathfrak{p}}$ is Galois.
        \item There is a natural map $\text{res}: \text{Gal}(L_{\mathcal{P}}/K_{\mathfrak{p}}) \to \text{Gal}(L/K)$ which is injective and has image $G_{\mathcal{P}}$.
    \end{enumerate} 
\end{prop}
\begin{proof}
    \begin{enumerate}[(i)]
        \item $L/K$ is Galois, so $L$ is the splitting field of a separable polynomial $f(X) \in K[X]$. Then $L_{\mathcal{P}}$ is the splitting field of $f(X)$ over $K_{\mathfrak{p}}[X]$. Hence $L_{\mathcal{P}}/K_\mathfrak{p}$ is Galois.
        \item Let $\sigma \in \text{Gal}(L_{\mathcal{P}}/K_{\mathfrak{p}})$. Then $\sigma(L)=L$ since $L/K$ is normal. Hence we have a map $\text{res}: \text{Gal}(L_{\mathcal{P}}/K_{\mathfrak{p}}) \to \text{Gal}(L/K)$ by $\sigma \mapsto \sigma|_L$. Since $L$ is dense in $L_{\mathcal{P}}$, $\text{res}$ is injective. By Lemma \ref{lemma8.1}, $|\sigma(x)|_\mathcal{P}=|x|_{\mathcal{P}}~\forall \sigma \in \text{Gal}(L_{\mathcal{P}}/K_{\mathfrak{p}}), x \in L_{\mathcal{P}}$. Hence $\sigma(\mathcal{P}) = \mathcal{P} ~\forall \sigma \in \text{Gal}(L_{\mathcal{P}}/K_{\mathfrak{p}})$ and thus $\text{res}(\sigma) \in G_{\mathcal{P}}~\forall \sigma \in \text{Gal}(L_{\mathcal{P}}/K_{\mathfrak{p}})$. 
        \vspace{1mm}
         
        To show injectivity, it suffices to show that $[L_{\mathcal{P}}: K_{\mathfrak{p}}] = ef = |G_{\mathcal{P}}|$.
        \begin{itemize}
            \item $|G_{\mathcal{P}}|=ef$ follows from Proposition \ref{prop11.2}, Corollary \ref{cor11.3} and the orbit--stabilizer theorem.
            \item $[L_{\mathcal{P}}:K_{\mathfrak{p}}]=ef$ follows from Corollary \ref{cor11.4}, noting that $e$ and $f$ don't change when we take completions.
        \end{itemize}
    \end{enumerate}
\end{proof}

\section{Ramification theory}
\subsection{The different and discriminant}
In this section, assume that $L/K$ is an extension of algebraic number fields with $[L:K] = n$ and $\mathcal{O}_K, \mathcal{O}_L$ are the rings of integers.
\vspace{1mm}
 
\textbf{Notation.} For $x_1,\ldots,x_n \in L$, set $\Delta(x_1,\ldots,x_n)= \det(\text{Tr}_{L/K}(x_ix_j)) \in K$. We can show that $\Delta(x_1,\ldots,x_n) = \det(\sigma_i(x_j))^2$ for $\sigma_i : L \to \overline{K}$ the embeddings.

Note that if $y_i = \sum_{j=1}^{n} a_{ij}x_j$ for $a_{ij} \in K$, then \[
\Delta(y_1,\ldots,y_n) = (\det A)^2\Delta(x_1,\ldots,x_n).
\]
If $x_1,\ldots,x_n \in \mathcal{O}_L$, then $\Delta(x_1,\ldots,x_n) \in \mathcal{O}_K$.
\begin{lemma}\label{lemma12.1}
    Let $k$ be a perfect field and $R$ a $k$--algebra which is finite--dimensional as a $k$--vector space. Then the trace form $(\cdot ,\cdot ) : R \times R \to R$ given by $(x,y) \mapsto \text{Tr}_{R/k}(xy) = \text{Tr}_k(\text{mult}(xy))$ is nondegenerate if and only if $R = k_1 \times \ldots \times k_n$ where $k_i/k$ are finite (hence separable) field extensions.
\end{lemma}
\begin{proof}
    This is on example sheet 3.
\end{proof}
\begin{theorem}
    Let $\mathfrak{p} \subset \mathcal{O}_K$ be a nonzero prime ideal.
    \begin{itemize}
        \item If $\mathfrak{p}$ ramifies in $L$, then $\forall x_1,\ldots,x_n \in \mathcal{O}_L$, $\Delta(x_1,\ldots,x_n) \equiv 0 \pmod{\mathfrak{p}}$.
        \item If $\mathfrak{p}$ is unramified in $L$, then $\exists x_1,\ldots,x_n \in \mathcal{O}_L$ such that $\mathfrak{p} \nmid \Delta(x_1,\ldots,x_n)$.
    \end{itemize}
\end{theorem}
\begin{proof}
    \begin{enumerate}[(i)]
        \item Let $\mathfrak{p}\mathcal{O}_L = \mathcal{P}_1^{e_1}\ldots \mathcal{P}_r^{e_r}$ with the $\mathcal{P}_i$ distinct and $e_i > 0$. CRT implies that \[
        R := \mathcal{O}_L/\mathfrak{p}\mathcal{O}_L \cong \prod_{i=1}^{r} \mathcal{O}_L/\mathcal{P}_i^{e_i}.
        \]
        If $\mathfrak{p}$ ramifies in $L$, then $\mathcal{O}_L/\mathfrak{p}\mathcal{O}_L$ has nilpotent elements. By Lemma \ref{lemma12.1}, the trace form $\text{Tr}_{R/k}$ (for $k$ the residue field at $\mathfrak{p}$) is degenerate, so $\Delta(\overline{x_1},\ldots,\overline{x_n}) = 0 ~\forall \overline{x_i} \in \mathcal{O}_L/\mathfrak{p}\mathcal{O}_L$. Hence $\Delta(x_1,\ldots,x_n) \equiv 0 \pmod{\mathfrak{p}}$ for any $x_1,\ldots,x_n \in \mathcal{O}_L$ through the commutativity of the diagram 
        $\begin{tikzcd}[]
            \mathcal{O}_L & R\\
            \mathcal{O}_K & k = \mathcal{O}_K/\mathfrak{p}
            \arrow["", from=1-1, to=1-2]
            \arrow["\text{Tr}_{R/k}", from=1-2, to=2-2]
            \arrow["\text{Tr}_{L/K}", from=1-1, to=2-1]
            \arrow["", from=2-1, to=2-2]
        \end{tikzcd}$. 
        \item If $\mathfrak{p}$ is unramified in $L$, then $\mathcal{O}_L/\mathfrak{p}\mathcal{O}_L$ is a product of finite extensions of $k$, so by Lemma \ref{lemma12.1} the trace form is nondegenerate. Hence we can pick a basis $\overline{x_1},\ldots,\overline{x_n}$ of $\mathcal{O}_L/\mathfrak{p}\mathcal{O}_L$ as a $k$--vector space, so $\Delta(\overline{x_1},\ldots,\overline{x_n}) \neq 0$. Hence $\exists x_1,\ldots,x_n \in \mathcal{O}_L$ such that $\Delta(x_1,\ldots,x_n) \neq 0 \pmod{\mathfrak{p}}$.
    \end{enumerate}
\end{proof}
\begin{defn}
    The \textbf{discriminant} is the ideal $d_{L/K} \subset \mathcal{O}_K$ generated by $\Delta(x_1,\ldots,x_n)$ for all choices $x_1,\ldots,x_n \in \mathcal{O}_L$.
\end{defn}
\begin{cor}
    A prime ideal $\mathfrak{p}$ ramifies in $L \iff \mathfrak{p} \mid d_{L/K}$.
    \vspace{1mm}
     
    In particular, only finitely many ideals ramify in $L$.
\end{cor}
\begin{defn}
    The \textbf{inverse different} is $$D^{-1}_{L/K} = \{y \in L \mid \text{Tr}_{L/K}(xy) \in \mathcal{O}_K ~\forall x \in \mathcal{O}_L\}$$
    which is an $\mathcal{O}_L$--submodule of $\mathcal{O}_L$.
\end{defn}
\begin{lemma}
    $D^{-1}_{L/K}$ is a fractional ideal in $L$ containing $\mathcal{O}_L$.
\end{lemma}
\begin{proof}
    Let $x_1,\ldots,x_n \in \mathcal{O}_L$ be a $K$--basis for $L/K$. Set $d = \Delta(x_1,\ldots,x_n) = \det(\text{Tr}(x_ix_j)) \neq 0$ (as an extension of number fields is separable). For $x \in D^{-1}_{L/K}$, write $x = \sum_{j=1}^{n} \lambda_j x_j$ for $\lambda_j \in K$. Then $\text{Tr}(xx_j)=\sum_{j=1}^{n} \lambda_j \text{Tr}(x_i,x_j) \in\mathcal{O}_K$. Set $A_{ij} = \text{Tr}_{L/K}(x_i x_j)$. Multiplying by $\text{Adj}(A) \in M_{n}(\mathcal{O}_K)$ gives \[
    d \begin{pmatrix} \lambda_1 \\ \vdots \\ \lambda_n \end{pmatrix} = \text{Adj}(A) \begin{pmatrix} \text{Tr}_{L/K}(x x_1) \\ \vdots \\ \text{Tr}_{L/K}(x x_n) \end{pmatrix}.
    \]
    Hence $\lambda_i \in \frac{1}{d}\mathcal{O}_K$, so $x \in \frac{1}{_L} \mathcal{O}_L$, so $D^{-1}_{L/K} \subset \frac{1}{d}\mathcal{O}_L$, so $D^{-1}_{L/K}$ is a fractional ideal.
    \vspace{1mm}
     
    Finally, $\text{Tr}(x) \in \mathcal{O}_K ~\forall x \in \mathcal{O}_L$, so $\mathcal{O}_L \subset D^{-1}_{L/K}$.
\end{proof}
\begin{defn}
    The inverse $D_{L/K} \subset \mathcal{O}_L$ of $D^{-1}_{L/K}$ is the \textbf{different ideal}.
\end{defn}

\marginpar{09 Nov 2022, Lecture 15}

Let $L/K$ be a degree $n$ extension of number fields, and let $I_L, I_K$ be the groups of fractional ideals. Then Theorem \ref{theorem9.5} gives us that 
\begin{align*}
    &I_L \cong \bigoplus_{\substack{0 \neq \mathfrak{p}\\\mathfrak{p} \text{ prime ideal in }\mathcal{O}_L}} \mathfrak{p} \\
    &I_K \cong \bigoplus_{\substack{0 \neq \mathfrak{p}\\\mathfrak{p} \text{ prime ideal in }\mathcal{O}_K}} \mathfrak{p}.
\end{align*}

Define $N_{L/K}: I_L \to I_K$ induced by $\mathcal{P} \mapsto \mathfrak{p}^f$ for $\mathfrak{p} = \mathcal{P} \cap \mathcal{O}_K$ and $f = f_{\mathcal{P}/\mathfrak{p}}$. Then it is a fact that the following diagram commutes: $\begin{tikzcd}[]
    L^{\times} & I_L\\
    K^{\times} & I_K
    \arrow["", from=1-1, to=1-2]
    \arrow["N_{L/K}", from=1-2, to=2-2]
    \arrow["N_{L/K}", from=1-1, to=2-1]
    \arrow["", from=2-1, to=2-2]
\end{tikzcd}$.
One way to see this is to use Corollary \ref{cor10.10} and $v_{\mathfrak{p}}(N_{L_\mathcal{P}/K_{\mathfrak{p}}}(x)) = f_{\mathcal{P}/\mathfrak{p}} v_{\mathcal{P}}(x)$ for $x \in L_{\mathcal{P}}$ and $v_\mathfrak{p}, v_\mathcal{P}$ the normalized valuations on $L_{\mathcal{P}}$, $K_{\mathfrak{p}}$. (Remember that here $f = [\mathcal{O}_L/\mathcal{P} : \mathcal{O}_K/\mathfrak{p}]$).

\begin{theorem}
    $N_{L/K}(D_{L/K}) = d_{L/K}$.
\end{theorem}
\begin{proof}
    First assume that $\mathcal{O}_K, \mathcal{O}_L$ are PIDs. Let $x_1,\ldots,x_n$ be a $\mathcal{O}_K$--basis for $\mathcal{O}_L$ and $y_1,\ldots,y_n$ the dual basis with respect to the trace form. Then $y_1,\ldots,y_n$ gives a basis for $D^{-1}_{L/K}$. Let $\sigma_1,\ldots,\sigma_n : L \to \overline{K}$ be the embeddings and consider $\sum_{i=1}^{n} \sigma_i(x_j)\sigma_i(y_k) = \text{Tr}_{L/K}(x_j y_k) = \delta_{jk}$.
    \vspace{1mm}
     
    But $\Delta(x_1,\ldots,x_n) = \det(\sigma_i(x_j))^2$. Thus $\Delta(x_1,\ldots,x_n)\Delta(y_1,\ldots,y_n) = 1$. Write $D^{-1}_{L/K} = \beta \mathcal{O}_L$ for some $\beta \in \mathcal{O}_L$ (as $\mathcal{O}_L$ is assumed to be a PID). Then the change of basis matrix between $y_1,\ldots,y_n$ and $\beta x_1, \ldots, \beta x_n$ is invertible in $\mathcal{O}_K$, so
    \begin{align*}
        d^{-1}_{L/K} =& \Delta(x_1,\ldots,x_n)^{-1} = \Delta(y_1,\ldots,y_n) \\
        =& \Delta(\beta x_1, \ldots, \beta x_n) = N_{L/K}(\beta)^2 \Delta(x_1,\ldots,x_n).
    \end{align*}
    Thus $d_{L/K}^{-1} = N_{L/K}(D_{L/K}^{-1})^2 d_{L/K}$, so $N_{L/K}(D_{L/K})=d_{L/K}$.
    \vspace{1mm}
     
    In general, localize at $S = \mathcal{O}_K/\mathfrak{p}$ and note that localizing $\mathcal{O}_K$ gives a DVR (hence a PID), localizing $\mathcal{O}_L$ gives a Dedekind domain with finitely many prime ideals (hence a PID by example sheet 2) and that $S^{-1}D_{L/K} = D_{S^{-1}\mathcal{O}_L/S^{-1}\mathcal{O}_K}$ and $S^{-1} d_{L/K} = d_{S^{-1} \mathcal{O}_L/S^{-1} \mathcal{O}_K}$ (details left as exercise).
\end{proof}

\begin{theorem}\label{theorem12.6}
    If $\mathcal{O}_{L} = \mathcal{O}_K[\alpha]$ and $\alpha$ has minimal polynomial $g(X) \in \mathcal{O}_K[X]$, then $D_{L/K} = (g'(\alpha))$.
\end{theorem}
\begin{proof}
    Write $\frac{g(X)}{X-\alpha}=\beta_{n-1}X^{n-1}+ \ldots + \beta_1 X + \beta_0$ with $\beta_i \in \mathcal{O}_L$ and $\beta_{n-1} = 1$ and let $\alpha=\alpha_1,\ldots,\alpha_n$ be the roots of $g$. We claim that for $0\le r\le n-1$, 
    \begin{align*}
        \sum_{i=1}^{n} \frac{g(X)}{X-\alpha_i}\frac{\alpha_i^r}{g'(\alpha_i)} = X^r.
    \end{align*}
    Indeed, the difference is a polynomial in $X$ of degree $<n$ which vanishes for $X=\alpha_1,\ldots,\alpha_n$. Equating coefficients of $X^s$ on both sides gives $\delta_{rs} = \text{Tr}_{L/K}\left(\frac{\alpha^r \beta_s}{g'(\alpha)}\right)$. Hence $\mathcal{O}_L$ has $\mathcal{O}_K$--basis $1,\alpha,\ldots,\alpha^{n-1}$, $D^{-1}_{L/K}$ has $\mathcal{O}_K$--basis $\frac{\beta_0}{g'(\alpha)}, \frac{\beta_1}{g'(\alpha)}, \ldots, \frac{\beta_{n-1}}{g'(\alpha)}=\frac{1}{g'(\alpha)}$. Hence the last element generates all the others, so $D_{L/K}^{-1} = \left(\frac{1}{g'(\alpha)}\right)$, so $D_{L/K} = (g'(\alpha))$.
\end{proof}

Take $\mathcal{P}$ a nonzero prime ideal of $\mathcal{O}_L$ and $\mathfrak{p} = \mathcal{P} \cap \mathcal{O}_K$. We can define $D_{L_\mathcal{P}/K_\mathfrak{p}}$ using $\mathcal{O}_{L_\mathcal{P}}$ and $\mathcal{O}_{K_{\mathfrak{p}}}$. We can identify $D_{L_\mathcal{P}/K_\mathfrak{p}}$ with a power of $\mathcal{P}$. 

\begin{theorem}\label{theorem12.7}
    $D_{L/K} = \prod_{\mathcal{P}\mid \mathfrak{p}}^{} D_{L_\mathcal{P}/K_\mathfrak{p}}$. 
\end{theorem}
Note that we will later verify that the product is finite.
\begin{proof}
    Let $x \in L$ and $\mathfrak{p} \subset \mathcal{O}_K$ a prime ideal. Then $\text{Tr}_{L/K}(x) \stackrel{(\star)}{=} \sum_{\mathcal{P}\mid  \mathfrak{p}}^{} \text{Tr}_{L_\mathcal{P}/K_\mathfrak{p}}(x)$ (compare with Corollary \ref{cor10.10}). Let $r(\mathcal{P}) = v_{\mathfrak{p}}(D_{L/K})$ and $s(\mathcal{P}) = v_{\mathcal{P}}(D_{L_\mathcal{P}/K_\mathfrak{p}})$. We want to show that $r(\mathcal{P}) = s(\mathcal{P})$. 
    \vspace{1mm}
     
    For $\subset$ (i.e. $r(\mathcal{P})\ge s(\mathcal{P})$), take $x \in L$ with $v_{\mathcal{P}}(x) \ge -s(\mathcal{P}) ~\forall \mathcal{P}$. Then $\text{Tr}_{L_\mathcal{P}/K_\mathfrak{p}}(xy) \in \mathcal{O}_{K_\mathfrak{p}} ~\forall y \in \mathcal{O}_L$ and $\forall \mathcal{P}$. Then $(\star)$ gives $\text{Tr}_{L/K}(xy) \in \mathcal{O}_{K_\mathfrak{p}} ~\forall y \in \mathcal{O}_L$ and $\forall \mathfrak{p}$. Hence $\text{Tr}_{L/K}(xy) \in \mathcal{O}_K ~\forall y \in \mathcal{O}_L$, i.e. $x \in D_{L/K}$. Thus $D_{L/K} \subset \prod_{\mathcal{P}}^{} D_{L_\mathcal{P}/K_\mathfrak{p}}$.
    \vspace{1mm}
     
    For $\supset$, i.e. $r(\mathcal{P})\le s(\mathcal{P})$, fix $\mathcal{P}$ and let $x \in \mathcal{P}^{-r(\mathcal{P})}\setminus \mathcal{P}^{-r(\mathcal{P})+1}$. Then $v_{\mathcal{P}}(x) = -r(\mathcal{P})$, so $v_{\mathcal{P}'}(x) \ge 0 ~\forall \mathcal{P}' \neq \mathcal{P}$. By $(\star)$, $$\text{Tr}_{L_\mathcal{P}/K_{\mathfrak{p}}}(xy) = \underbrace{\text{Tr}_{L/K}(xy)}_{\mathcal{O}_K} - \sum_{\substack{\mathcal{P}' \mid \mathfrak{p}\\\mathcal{P}'\neq\mathcal{P}}}^{} \underbrace{\text{Tr}_{L_{\mathcal{P}'/K_\mathfrak{p}}}(xy)}_{\in \mathcal{O}_{K_\mathfrak{p}}} ~\forall y \in \mathcal{O}_L.$$
    By continuity, $\text{Tr}_{L_\mathcal{P}/K_\mathfrak{p}}(xy) \in \mathcal{O}_{K_\mathfrak{p}} ~\forall y \in \mathcal{O}_{L_\mathcal{P}}$, so $x \in D^{-1}_{L_\mathcal{P}/K_\mathfrak{p}}$, i.e. $-v_{\mathcal{P}}(x) = r(\mathcal{P}) \le s(\mathcal{P})$. Hence $D_{L/K} \supset \prod_{\mathcal{P}}^{} D_{L_\mathcal{P}/K_\mathfrak{p}}$. 
\end{proof}
\begin{cor}
    $d_{L/K} = \prod_{\mathcal{P}\mid \mathfrak{p}}^{} d_{L_\mathcal{P}/K_\mathfrak{p}}$.
\end{cor}
\begin{proof}
    Apply $N_{L/K}$ to $D_{L/K} = \prod_{\mathcal{P}}^{} D_{L_\mathcal{P}/K_{\mathfrak{p}}}$.
\end{proof}
\section{Unramified and totally ramified extensions of local fields}
In this section, let $L/K$ be a finite separable extension of non--archimedean local fields. By Corollary \ref{cor11.4}, $[L:K] = e_{L/K}f_{L/K}$.
\begin{lemma}
    Let $M/L/K$ be finite separable extensions of local fields. Then $f_{M/K} = f_{M/L}f_{L/K}$ and $e_{M/K} = e_{M/L}e_{L/K}$.
\end{lemma}
\begin{proof}
    $f_{M/K} = [k_M:k] = [k_M : k_L][k_L : k]= f_{M/L}f_{L/K}$. The other result follows from this one and $\text{Tr}_{L/K}(x) = \sum_{\mathcal{P}\mid  \mathfrak{p}}^{} \text{Tr}_{L_\mathcal{P}/K_\mathfrak{p}}(x)$.
\end{proof}
\begin{defn}
    The extension $L/K$ is said to be 
    \begin{itemize}
        \item \textbf{unramified} if $e_{L/K} = 1$, i.e. $f_{L/K}=[L:K]$.
        \item \textbf{ramified} if $e_{L/K}>1$, i.e. $f_{L/K}< [L:K]$.
        \item \textbf{totally ramified} if $e_{L/K} = [L:K]$, i.e. if $f_{L/K}=1$.
    \end{itemize}
\end{defn}

\marginpar{11 Nov 2022, Lecture 16}

\begin{theorem}\label{theorem13.2}
    Let $L/K$ be a finite separable extension of local fields. Then there exists a field $K_0$ with $K \subset K_0 \subset L$ such that 
    \begin{enumerate}[(i)]
        \item $K_0/K$ is unramified.
        \item $L/K_0$ is totally ramified.
    \end{enumerate}
    Moreover, $[K_0 : K] = f_{L/K}$, $[L:K_0] = e_{L/K}$ and $K_0/K$ is Galois.
\end{theorem}
\begin{proof}
    Let $k = \mathbb{F}_q$, so that $k_L = \mathbb{F}_{q^f}$ for $f = f_L/K$ the residue class degree. Set $m = q^f - 1$ and $[~] : \mathbb{F}_{q^f} \to L$ the Teichmüller map for $L$. Let $\zeta_m = [\alpha]$ for $\alpha$ a generator for $\mathbb{F}_{q^f}^\times$. Then $\zeta_m$ is a primitive $m^{\text{th}}$ root of unity. Set $K_0 = K(\zeta_m) \subset L$, then $K_0/K$ is Galois and $K_0$ has residue field $k_0 = \mathbb{F}_q(\alpha) = k_L$. Hence $f_{L/K_0} = 1$, so $L/K_0$ is totally ramified.
    \vspace{1mm}
     
    Let $\text{res}: \text{Gal}(K_0/K) \to \text{Gal}(k_0/k)$ be the natural map. For $\sigma \in \text{Gal}(K_0/K)$, we have $\sigma(\zeta_m) = \zeta_m \iff \sigma(\zeta_m) \equiv \zeta_m \pmod{\mathfrak{m}_0}$, since by Hensel's lemma, we have a bijection between $\mu_m(K_0)$ and $\mu_m(k_0)$. Hence $\text{res}$ is injective. Thus $|\text{Gal}(K_0/K)|\le |\text{Gal}(k_0/k)| = f_{K_0/K}$. Hence $[K_0:K] = f_{K_0/K}$, so $\text{res}$ is an isomorphism and $K_0/K$ is unramified.
\end{proof}

\begin{theorem}
    Let $K$ be a local field and let $k = \mathbb{F}_q$. For each $n\ge 1$, there exists a unique unramified extension $L/K$ of degree $n$. Moreover, $L/K$ is Galois and the natural map $\text{Gal}(L/K) \to \text{Gal}(k_L/k)$ is an isomorphism. In particular, $\text{Gal}(L/K) \cong \langle \text{Frob}_{L/K} \rangle$, where $\text{Frob}_{L/K}(x) \equiv x^q \pmod{\mathfrak{m}_L} ~\forall x \in \mathcal{O}_L$.
\end{theorem}
\begin{proof}
    For $n\ge 1$, take $L = K(\zeta_m)$ where $m = q^n-1$. As in Theorem \ref{theorem13.2}, $\text{Gal}(L/K) \cong \text{Gal}(k_L/k) \cong \text{Gal}(\mathbb{F}_{q^n}/\mathbb{F}_q)$. Hence $\text{Gal}(L/K)$ is cyclic and generated by a lift of $x \mapsto x^q$. 
    \vspace{1mm}
     
    For uniqueness, we have a degree $n$ unramified extension $L/K$. By the Teichmüller lift, $\zeta_m \in L$, so $L = K(\zeta_m)$.
\end{proof}
\begin{cor}
    For $L/K$ a finite Galois extension, the map $\text{res} : \text{Gal}(L/K) \to \text{Gal}(k_L/k)$ is surjective.
\end{cor}
\begin{proof}
    The map $\text{res}$ factors as $\text{Gal}(L/K) \twoheadrightarrow \text{Gal}(K_0/K) \cong \text{Gal}(k_L/k)$.
\end{proof}
\begin{defn}
    For $L/K$ a Galois extension, the \textbf{inertia subgroup} is \[
    I_{L/K} = \text{ker}(\text{Gal}(L/K) \twoheadrightarrow \text{Gal}(k_L/k)).
    \]
\end{defn}
Note that:
\begin{itemize}
    \item Since $e_{L/K}f_{L/K} = [L:K]$, we have $|I_{L/K}| = e_{L/K}$.
    \item $I_{L/K} = \text{Gal}(L/K_0)$, where $K_0$ is as in Theorem \ref{theorem13.2}.
\end{itemize}

\begin{defn}
    Let $f(X)=X^n + a_{n-1}X^{n-1} + \ldots + a_0 \in \mathcal{O}_K[X]$. We say $f$ is \textbf{Eisenstein} if $v_{K}(a_i)\ge 1 ~\forall 0<i<n$ and $v_K(a_0) = 1$ for $v_K$ the normalized evaluation on $K$.
\end{defn}
\textbf{Fact.} $f(x)$ Eisenstein $\implies f(x)$ irreducible.
\begin{theorem}\label{theorem13.5}
    \begin{enumerate}[(i)]
        \item Let $L/K$ be a finite totally ramified extension and let $\pi_L \in \mathcal{O}_L$ be a uniformizer. Then the minimal polynomial of $\pi_L$ is Eisenstein and $\mathcal{O}_L = \mathcal{O}_K[\pi_L]$ (so $L = K(\pi_L)$).
        \item Conversely, if $f(X) \in \mathcal{O}_K[X]$ is Eisenstein and $\alpha$ is a root of $f$, then $K(\alpha)/K$ is totally ramified and $\alpha$ is a uniformizer in $\mathcal{O}_L$ (for $L = K(\alpha)$).
    \end{enumerate}
\end{theorem}
\begin{proof}
    \begin{enumerate}[(i)]
        \item Let $[L:K] = e = e_{L/K}$ and let $f(X) = X^m + a_{m-1}X^{m-1} + \ldots + a_0 \in \mathcal{O}_K[X]$ be the minimal polynomial for $\pi_L$. Then $m\le e$. Since $v_L(K^\times) = e \mathbb{Z}$, we have $v_L(a_i \pi_L^i) \equiv  i \pmod{e}$ for $i\le m$. Hence $a_i \pi_L^i$ have distinct valuations. As $\pi_L^m = - \sum_{i=0}^{m-1} a_i \pi_L^i$, we have $$m = v_L(\pi_L^m) = \min_{0\le i\le m-1}(i + e v_K(a_i)).$$
        Hence $v_K(a_i)\ge 1 ~\forall i$ and hence $v_K(a_0) = 1$ and $m = e$. Thus $f(X)$ is Eisenstein and $L = K(\pi_L)$.
        \vspace{1mm}
         
        For $y \in L$, write $ y = \sum_{i=0}^{e-1} \pi_L^i b_i$ for $b_i \in K$. Then $$v_L(y) = \min_{0\le i\le e-1}(i + e v_K(b_i)).$$
        Thus $y \in \mathcal{O}_L \iff v_L(y) \ge 0 \iff v_K(b_i)\ge 0~\forall i \iff y \in \mathcal{O}_K[\pi_L]$.
        \item Let $f(X) = X^n + a_{n-1}X^{n-1} + \ldots + a_0 \in \mathcal{O}_K[X]$ be Eisenstein and $e = e_{L/K}$ (for $L = K(\alpha)$). Thus $v_L(a_i)\ge e$ and $v_L(a_0) = e$. If $v_L(\alpha)\le 0$, then $v_L(\alpha^n)\le v_L \left( \sum_{i=0}^{n-1} a_i \alpha^i \right)$, a contradiction and hence $v_L(\alpha)>0$. For $i\neq 0$, $v_L(a_i \alpha^i)>e = v_L(a_0)$. Therefore $v_L(\alpha^n) = v_L\left(-\sum_{i=0}^{n-1} a_i \alpha^i \right) = v_L(a_0)$. Hence $n v_L(\alpha) = v_L(a_0) = e$, but $n = [L:K]\ge e$, so $n=e$ and $v_L(\alpha) = 1$.
    \end{enumerate}
\end{proof}
\section{Structure of units}

\begin{prop}
    Let $K$ be a finite extension of $\mathbb{Q}_p$, write $e = e_{K/\mathbb{Q}_p}$ and let $\pi$ be a uniformizer in $\mathcal{O}_K$. If $r>\frac{e}{p-1}$, then $\text{exp}(x)= \sum_{n=0}^{\infty} \frac{x^n}{n!}$ converges on $\pi^r \mathcal{O}_K$ and induces an isomorphism $(\pi^r \mathcal{O}_K, +) \cong (1 + \pi^r \mathcal{O}_K, \times)$.
\end{prop}
\begin{proof}
    $v_K(n!) = e v_p(n!) \stackrel{(\star)}{=} \frac{e(n-s_p(n))}{p-1} \le e \left(\frac{n-1}{p-1}\right)$ for $(\star)$ from example sheet 1. For $x \in \pi^r \mathcal{O}_K$ and $n\ge 1$, compute $v_K(\frac{x^n}{n!})\ge nr - e \frac{n-1}{p-1} = r + (n-1)\left(r-\frac{e}{p-1}\right)$, so $v_K(\frac{x^n}{n!}) \to \infty$ as $n\to \infty$. Hence $\text{exp}(x)$ converges on $\pi^r \mathcal{O}_K$. Since $v_K(\frac{x^n}{n!})\ge r ~\forall n\ge 1$, $\text{exp}(x) \in 1 + \pi^r \mathcal{O}_K$.
    \vspace{1mm}
     
    Similarly consider $\log(x) : 1 + \pi^r \mathcal{O}_K \to \pi^r \mathcal{O}_K$ by $\log(1+x) = \sum_{n=1}^{\infty} \frac{(-1)^{n-1}}{n}x^n$. We can check the convergence as before (exercise, this is easier than the previous case). Recall the identities in $\mathbb{Q}[[X,Y]]: \exp(X+Y) = \exp(X)\exp(Y)$ and $\exp(\log(1+X))=1+X$, so $\exp(\log(X))=X$. These identities together give the isomorphism.
\end{proof}
\marginpar{14 Nov 2022, Lecture 17}
When $K$ is a local field, possibly of equal characteristic, write $U_K = \mathcal{O}_K^\times$ and let $\pi \in \mathcal{O}_K$ be a uniformizer.
\begin{defn}
    For $s \in \mathbb{Z}_{\ge 1}$, the $s^{\text{th}}$ unit group $U_K^{(s)}$ is defined by $$U_K^{(s)} = (1 + \pi^s \mathcal{O}_K, \times)$$ and $U_K^{(0)} = U_K$. Then we have $\ldots \subset U_{K}^{(s)} \subset \ldots \subset U_K^{(1)} \subset U_K^{(0)}$.
\end{defn}
\begin{prop}\label{prop14.2}
    \begin{enumerate}[(i)]
        \item $U_K^{(0)}/U_{K}^{(1)} \cong (k^{\times}, \times)$ (for $k = \mathcal{O}_K/\mathfrak{m}$).
        \item $U_K^{(s)}/U_{K}^{(s+1)} \cong (k,+)$ for $s\ge 1$.
    \end{enumerate}
\end{prop}
\begin{proof}
    \begin{enumerate}[(i)]
        \item The natural map $\mathcal{O}_K^\times \to k^{\times}$ given by reduction mod $\pi$ is surjective with kernel $1 + \pi\mathcal{O}_K = U_{K}^{(1)}$.
        \item Define $f : U_K^{(s)} \to k$ by $1 + \pi^s x \mapsto x \pmod{\pi}$. This is a homomorphism, as $(1 + \pi^s x)(1 + \pi^s y)= 1 + \pi^s(x+y+\pi^s xy)$ and $x+y+\pi^s xy \equiv x+y \pmod{\pi}$ as desired. $f$ is surjective with $\text{ker}(f) = U_K^{(s+1)}$.
    \end{enumerate}
\end{proof}
\begin{cor}
    Let $[K : \mathbb{Q}_p]<\infty$. Then there exists a finite index subgroup of $\mathcal{O}_K^\times$ isomorphic to $(\mathcal{O}_K,+)$.
\end{cor}
\begin{proof}
    Take $r>\frac{e}{p-1}$, so $U_K^{(r)} \cong (\mathcal{O}_K,+)$, and $U_k^{(r)} \subset U_k$ has finite index by the previous proposition.
\end{proof}
\textbf{Remark.} This is not true for $K$ of equal characteristic, since $\exp$ is not defined.
\begin{example}
    Consider $\mathbb{Z}_p^\times$ from $K=\mathbb{Q}_p$ for $p>2$, so $e=1$. Then we can take $r=1$, so $\mathbb{Z}_p^\times \cong (\mathbb{Z}/p \mathbb{Z})^\times \times (1 + p \mathbb{Z}_p)$ by $x \mapsto (x \pmod{p}, \frac{x}{[x \pmod{p}]})$. We can rewrite $\mathbb{Z}_p^\times \cong (\mathbb{Z}/p \mathbb{Z})^\times \times (1 + p \mathbb{Z}_p) \cong \mathbb{F}_p^\times \times \mathbb{Z}_p \cong \mathbb{Z}/(p-1)\mathbb{Z} \times \mathbb{Z}_p$.
    \vspace{1mm}
     
    If $p=2$, take $r=2$ to get $\mathbb{Z}_2^{\times} \cong (\mathbb{Z}/4\mathbb{Z})^\times \times (1 + 4\mathbb{Z}_2)$ by $x \mapsto (x \pmod{4}, \frac{x}{\epsilon(x)})$ where $\epsilon(x) = \begin{cases}
        1 & x \equiv 1\pmod{4}\\
        -1 & x \equiv 3\pmod{4}
    \end{cases}$. Again note $\mathbb{Z}_2^{\times} \cong (\mathbb{Z}/4\mathbb{Z})^\times \times (1 + 4\mathbb{Z}_p) \cong \mathbb{Z}/2\mathbb{Z} \times \mathbb{Z}_2$.
\end{example}
This gives another proof of $\mathbb{Z}_p^\times/(\mathbb{Z}_p^\times)^2 \equiv \begin{cases}
    \mathbb{Z}/2\mathbb{Z} & p>2\\
    (\mathbb{Z}/2\mathbb{Z})^2 & p=2
\end{cases}$.
\section{Higher ramification groups}
Fix a finite Galois extension $L/K$ of local fields. From now on, assume a local field is non--archimedean. Fix $\pi_L \in \mathcal{O}_L$ a uniformizer.
\begin{defn}
    Let $v_L$ be the normalized valuation on $L$. For $s \in \mathbb{R}_{\ge -1}$, the $s^{\text{th}}$ \textbf{ramification group} is 
    \begin{align*}
        G_s(L/K) = \{\sigma \in \text{Gal}(L/K) \mid v_L(\sigma(x)-x)\ge s+1 ~\forall x \in \mathcal{O}_L\}.
    \end{align*}
\end{defn}
\textbf{Remark.} $G_s$ only changes at the integers. $s \in \mathbb{R}_{\ge -1}$ is used to define upper numbering (to be done later).
\begin{example}
    $G_{-1}(L/K) = \text{Gal}(L/K)$ and 
    \begin{align*}
        &G_0(L/K) = \{\sigma \in \text{Gal}(L/K) \mid \sigma(x) \equiv x \pmod{\pi} ~\forall x \in \mathcal{O}_L\} \\
        =& \text{ker}(\text{Gal}(L/K) \to \text{Gal}(k_L/k)) = I_{L/K}
    \end{align*}
\end{example}
\textbf{Note.} For $s \geq 0$, $G_s(L/K) = \text{ker}(\text{Gal}(L/K) \to \text{Aut}(\mathcal{O}_L/\pi_L^{s+1}\mathcal{O}_L))$, hence $G_s(L/K)$ are normal in $\text{Gal}(L/K)$. We have $$\ldots \subset G_s(L/K) \subset G_{s-1}(L/K) \subset \ldots \subset G_1(L/K) \subset \text{Gal}(L/K).$$
\begin{theorem}\label{theorem15.1}
    \begin{enumerate}[(i)]
        \item For $s\ge 1$, $G_s = \{\sigma \in G_0 \mid v_L(\sigma(\pi_L)-\pi_L)\ge s+1\}$.
        \item $\bigcap_{n=0}^{\infty} G_n= \{1\}$.
        \item Let $s \in \mathbb{Z}_{\ge 0}$. Then there exists an injective group homomorphism 
        \begin{align*}
            G_s/G_{s+1} &\hookrightarrow U_L^{(s)}/U_L^{(s+1)}\\
            \sigma &\mapsto \frac{\sigma(\pi_L)}{\pi_L}.
        \end{align*}
        This map is independent of choice of $\pi_L$.
    \end{enumerate}
\end{theorem}
\begin{proof}
    Let $K_0 \subset L$ be the maximal unramified extension of $K$ in $L$. By replacing $K$ by $K_0$, we may assume that $L/K$ is totally ramified.
    \begin{enumerate}[(i)]
        \item By Theorem \ref{theorem13.5}, $\mathcal{O}_L = \mathcal{O}_K[\pi_L]$. Suppose $v_L(\sigma(\pi_L)-\pi_L)\ge s+1$. Let $x \in \mathcal{O}_L$, then $x = f(\pi_L)$, where $f(X) \in \mathcal{O}_K[X]$. Then compute 
        \begin{align*}
            \sigma(x)-x = \sigma(f(\pi_L)) - f(\pi_L) = f(\sigma(\pi_L)) - f(\pi_L) = (\sigma(\pi_L)-\pi_L)g(\pi_L)
        \end{align*}
        for $g(X) \in \mathcal{O}_K[X]$, since $X - Y \mid f(X) - f(Y)$. Thus 
        \begin{align*}
            v_L(\sigma(x)-x) = v_L(\sigma(\pi_L)-\pi_L) + v_L(g(\pi_L)) \ge (s+1) + 0 = s+1.
        \end{align*}
        \item Suppose $\sigma \in \text{Gal}(L/K)$ with $\sigma \neq 1$. Then $\sigma(\pi_L) \neq \pi_L$ as $L = K(\pi_L)$. Hence $v_L(\sigma(\pi_L)-\pi_L)<\infty$. Then $\sigma \not\in G_s$ for $s$ large enough by (i). 
        \item Note that for $\sigma \in G_s$, $s \in \mathbb{Z}_{\ge 0}$, we have $\sigma(\pi_L) \in \pi_L + \pi_L^{s+1}\mathcal{O}_L$. This gives us that $\frac{\sigma(\pi_L)}{\pi_L} \in 1 + \pi_L^s \mathcal{O}_L = U_L^{(s)}$. We claim that $\phi : G_s \to U_L^{(s)}/U_L^{(s+1)}$ by $\sigma \mapsto \frac{\sigma(\pi_L)}{\pi_L}$ is a group homomorphism with kernel $G_{s+1}$.
        \vspace{1mm}
         
        For $\sigma, \tau \in G_s$, let $\tau(\pi_L) = u \pi_L$ for $u \in \mathcal{O}_L^\times$. Then \begin{align*}
            \frac{\sigma \tau(\pi_L)}{\pi_L} = \frac{\sigma(\tau(\pi_L))}{\tau(\pi_L)} \frac{\tau(\pi_L)}{\pi_L} = \frac{\sigma(u)}{u} \frac{\sigma(\pi_L)}{\pi_L}\frac{\tau(\pi_L)}{\pi_L}.
        \end{align*}
        But $\sigma(u) \in u + \pi_L^{s+1}\mathcal{O}_L$ since $\sigma \in G_s$, so $\frac{\sigma(u)}{u} \in 1 + \pi_L^{s+1}\mathcal{O}_L$ as $u$ is a unit. Hence $\frac{\sigma \tau (\pi_L)}{\pi_L} = \frac{\sigma(\pi_L)}{\pi_L} \frac{\tau(\pi_L)}{\pi_L} \pmod{U_L^{(s+1)}}$ is a group homomorphism.
        \vspace{1mm}
         
        Moreover, the kernel is $\{\sigma \in G_s \mid \sigma(\pi_L) \equiv \pi_L \pmod{\pi_L^{s+2}}\} = G_{s+1}$.
        \vspace{1mm}
         
        If $\pi_L' = a \pi_L$ is another uniformizer for $a \in \mathcal{O}_L^\times$, then \begin{align*}
            \frac{\sigma(\pi_L')}{\pi_L'} = \frac{\sigma(a)}{a}\frac{\sigma(\pi_L)}{\pi_L} \equiv \frac{\sigma(\pi_L)}{\pi_L} \pmod{U_L^{(s+1)}}.
        \end{align*}
    \end{enumerate}
\end{proof}
\begin{cor}
    $\text{Gal}(L/K)$ is solvable.
\end{cor}
\begin{proof}
    By Proposition \ref{prop14.2} and Theorem \ref{theorem15.1}, for $s \in \mathbb{Z}_{s \ge -1}$, we have 
    \begin{align*}
        G_s/G_{s+1} \cong \text{a subgroup of } \begin{cases}
            \text{Gal}(k_L/k) &\text{ if }s = -1\\
            (k_L^\times,x) & \text{ if }s=0\\
            (k_L, +) &\text{ if }s\ge 1
        \end{cases}.
    \end{align*}
    Thus $G_s/G_{s+1}$ is solvable as all of the groups on the right are, as they are abelian. We conclude using Theorem \ref{theorem15.1} (ii), which says that the intersection of all of these groups is trivial.
\end{proof}
Suppose $\text{char }k = p$. Then $|G_0/G_1|$ is coprime to $p$ and $|G_1| = p^n$ for some $n\ge 0$. Thus $G_1$ is the unique (since normal) Sylow $p$--subgroup of $I_{L/K}$.
\begin{defn}
    The group $G_1$ is called the \textbf{wild inertia group}.
\end{defn}
\begin{defn}
    The quotient $G_0/G_1$ is called the \textbf{tame inertia group}.
\end{defn}
\marginpar{16 Nov 2022, Lecture 18}
If $L/K$ is a finite separable extension of local fields, then we say $L/K$ is \textbf{tamely ramified} if $\text{char }k \nmid e_{L/K}$. Otherwise, we say $L/K$ is \textbf{wildly ramified}.
\begin{theorem}\label{theorem15.3}
    Assume $[K: \mathbb{Q}_p]<\infty$, so $L/K$ is finite, hence separable. Write $D_{L/K} = (\pi_L^{\delta(L/K)})$. Then $\delta(L/K)\ge e_{L/K}-1$ with equality if and only if $L/K$ is tamely ramified.
    \vspace{1mm}
     
    In particular, $L/K$ is unramified $\iff D_{L/K} = \mathcal{O}_L$. 
\end{theorem}
\begin{proof}
    In example sheet 3 we show $D_{L/K} = D_{L/K_0} D_{K_0/K}$, where $K_0$ is the subextension of $L$ such that $K_0/K$ is unramified and $L/K$ is totally ramified. Hence it suffices to consider the unramified and totally ramified case separately.
    \begin{enumerate}[(i)]
        \item $L/K$ is unramified. Then by Proposition \ref{prop6.8}, $\mathcal{O}_L = \mathcal{O}_K[\alpha]$ for some $\alpha \in \mathcal{O}_L$ with $k_L = k(\overline{\alpha})$. Let $g(X) \in \mathcal{O}_K[X]$ be the minimal polynomial of $\alpha$. $[L:K] = [k_L:k] \implies \overline{g}(X) \in k[X]$ is the minimal polynomial of $\overline{\alpha} \implies \overline{g}(X)$ is separable and hence $g'(\alpha) \not\equiv 0 \pmod{\pi_L}$. Now Theorem \ref{theorem12.6} says that $D_{L/K} = (g'(\alpha)) = \mathcal{O}_L$.
        \item $L/K$ is totally ramified. Let $[L:K] = e$. Then $\mathcal{O}_L = \mathcal{O}_K[\pi_L]$ with $\pi_L$ a root of $g(X) = X^e +\sum_{i=1}^{e-1} a_i X^i \in \mathcal{O}_K[X]$, which is Eisenstein. Now compute $g'(\pi_L) = \underbrace{e \pi_L^{e-1}}_{v_L\ge e-1} + \underbrace{\sum_{i=0}^{e-1} i a_i \pi_L^{i-1}}_{v_L\ge e}$. Thus $v_L(g'(\pi_L))\ge e-1$ with equality if and only if $\text{char }k = p \nmid e$.
    \end{enumerate}
\end{proof}
\begin{cor}
    Let $L/K$ be an extension of number fields and $0 \neq \mathcal{P} \subset \mathcal{O}_L$ is a prime ideal with $\mathfrak{p} = \mathcal{P} \cap \mathcal{O}_K$. Then $e(\mathcal{P}/\mathfrak{p})>1$ if and only if $\mathcal{P} \mid  D_{L/K}$.
\end{cor}
\begin{proof}
    Theorem \ref{theorem12.7} tells us that $D_{L/K} = \prod_{\mathcal{P}}^{} D_{L_\mathcal{P}/K_\mathfrak{p}}$. Then use $e(\mathcal{P}/\mathfrak{p}) = e_{L_\mathcal{P}/K_\mathfrak{p}}$ and Theorem \ref{theorem15.3}.
\end{proof}
\begin{example}
    Take $K = \mathbb{Q}_p$, $\zeta_{p^n}$ a primitive $p^{n^{\text{th}}}$ root of unity and $L = \mathbb{Q}_p(\zeta_{p^n})$. The $p^{n^{\text{th}}}$ cyclotomic polynomial is \begin{align*}
        \Phi_{p^n}(X) = \frac{X^{p^n}-1}{X^{p^{n-1}}-1} = X^{p^{n-1}(p-1)} + X^{p^{n-1}(p-2)} + \ldots + 1 \in \mathbb{Z}_p[X].
    \end{align*}
    An exercise on example sheet 3 will show that:
    \begin{itemize}
        \item $\Phi_{p^n}(X)$ is irreducible and hence the minimal polynomial of $\zeta_{p^n}$.
        \item $L/\mathbb{Q}_p$ is Galois and totally ramified of degree $p^{n-1}(p-1)$.
        \item If we define $\pi = \zeta_{p^n}-1$, then $\pi$ is a uniformizer of $\mathcal{O}_L$. Hence $\mathcal{O}_L = \mathbb{Z}_p[\zeta_{p^n}-1] = \mathbb{Z}_p[\zeta_{p^n}]$.
        \item $\text{Gal}(L/\mathbb{Q}_p) \cong (\mathbb{Z}/p^n \mathbb{Z})^\times$ (abelian) by $\sigma_m \leftrightarrow m$ where $\sigma_m(\zeta_{p^n})=\zeta_{p^n}^m$.
    \end{itemize}
    To find the higher ramification group, we compute
    \begin{align*}
        v_L(\sigma_m(\pi)-\pi) = v_L(\zeta_{p^n}^m - \zeta_{p^n}) = v_L(\zeta_{p^n}^{m-1}-1).
    \end{align*}
    Let $k$ be maximal such that $p^k \mid m-1$. Then $\zeta_{p^n}^{m-1}$ is a primitive $p^{(n-k)^{\text{th}}}$ root of unity and hence $(\zeta_{p^n}^{m-1}-1)$ is a uniformizer in $L' = \mathbb{Q}_p(\zeta_{p^n}^{m-1})$. Hence $$v_L(\zeta_{p^n}^{m-1}-1) = e_{L/L'} = \frac{e_{L/\mathbb{Q}_p}}{e_{L'/\mathbb{Q}_p}} = \frac{[L:\mathbb{Q}_p]}{[L':\mathbb{Q}_p]} = \frac{p^{n-1}(p-1)}{p^{n-k-1}(p-1)}=p^k.$$
    Now Theorem \ref{theorem15.1} (i) tells us that $\sigma \in G_i \iff p^k\ge i+1$. Thus $$G_i \cong \begin{cases}
        (\mathbb{Z}/p^n \mathbb{Z})^\times & i\le 0\\
        (1+p^k \mathbb{Z}/p^n\mathbb{Z}) & p^{k-1}-1\le i \le p^k-1 (\text{here }1\le k\le n-1)\\
        \{1\} & i > p^{n-1}-1
    \end{cases}.$$
\end{example}
\section{Local class field theory}
\subsection{Infinite Galois theory}
Let $L/K$ be an algebraic extension of fields.
\begin{defn}
    \begin{itemize}
        \item $L/K$ is separable if $\forall \alpha \in L$, its minimal polynomial $f_\alpha(X) \in K[X]$ is separable.
        \item $L/K$ is normal if $f_{\alpha}(X)$ splits in $L ~\forall \alpha \in L$.
        \item $L/K$ is Galois if it is normal and separable.
    \end{itemize}
\end{defn}
We write $\text{Gal}(L/K) = \text{Aut}_K(L)$ in this case.
\vspace{1mm}
 
If $L/K$ is finite and Galois, we have the Galois correspondence \begin{align*}
    \{\text{subextensions }K\subset K' \subset L\} &\leftrightarrow \{\text{subgroups of }\text{Gal}(L/K)\}\\
    K' &\mapsto \text{Gal}(L/K')
\end{align*}
In the infinite case, we have the following definition.
\begin{defn}
    Let $(I,\le )$ be a poset. We say $I$ is a \textbf{directed set} if for all $i, j \in I, \exists k \in I$ such that $i\le k$ and $j\le k$.
\end{defn}
\begin{example}
    Any total order is a direct set, for example $(\mathbb{N},\le)$ or $(\mathbb{Z}_{\ge 1})$ ordered by divisibility.
\end{example}
\begin{defn}
    Let $(I, \le )$ be a directed set and $(G_i)_{i \in I}$ a collection of groups together with group homomorphisms (transition maps) $\phi_{ij} : G_j \to G_i$ such that $\phi_{ik} = \phi_{ij} \circ \phi_{jk} ~\forall i\le j\le k$ and $\phi_{ii} = \text{id} ~\forall i$. We say $((G_i)_{i \in I}, \phi_{ij})$ is an \textbf{inverse system}. The \textbf{inverse limit} of this inverse system $((G_i)_{i \in I}, \phi_{ij})$ is 
    \begin{align*}
        \underset{i \in I}{\varprojlim}~ G_i = \{(g_i)_{i \in I} \in \prod_{i \in I}^{} G_i \mid \phi_{ij}(g_j) = g_i \}.
    \end{align*}
\end{defn}
\textbf{Remarks.} 
\begin{itemize}
    \item Taking $(\mathbb{N}, \le )$ recovers our previous definition of an inverse limit.
    \item There exist projection maps $\Psi_j : \underset{i \in I}{\varprojlim}~ G_i \to G_j$.
    \item $\underset{i \in I}{\varprojlim} G_i$ satifies a universal property (defining left as exercise).
\end{itemize}
Assume $G_i$ is finite. Then the \textbf{profinite topology} on $\underset{i \in I}{\varprojlim}~ G_i$ is the weakest topology making these projection maps continuous, i.e. such that the maps $\Psi_j$ are continuous $\forall j \in I$.
\begin{prop}\label{prop16.1}
    Let $L/K$ be a (possibly infinite) Galois extension.
    \begin{enumerate}[(i)]
        \item The set $I = \{F/K \mid F \subset L, F/K \text{ is finite and Galois}\}$ is a directed set under inclusion.
        \item For $F,F' \in I$ with $F \subset F'$, there is a restriction map $\text{res}_{F,F'}: \text{Gal}(F'/K) \twoheadrightarrow \text{Gal}(F/K)$ and the natural map $\text{Gal}(L/K) \to \underset{F \in I}{\varprojlim} ~ \text{Gal}(F/K)$ is an isomorphism.
    \end{enumerate}
\end{prop}
\begin{proof}
    Exercise on example sheet 4.
\end{proof}
\begin{theorem}[Fundamental theorem of Galois theory]\label{theorem16.2}
    Let $L/K$ be Galois. By Proposition \ref{prop16.1}, we can endow $\text{Gal}(L/K)$ with the profinite topology (equal to the discrete topology is $L/K$ is finite). Then there exists a bijection 
    \begin{align*}
        \{F/K \text{ subextension of }L/K\} &\leftrightarrow \{\text{closed subgroups of }\text{Gal}(L/K)\}\\
        F &\mapsto \text{Gal}(L/F)\\
        L^H &\mapsfrom H
    \end{align*}
    Moreover:
    \begin{itemize}
        \item $F/K$ is finite $\iff$ $\text{Gal}(L/F)$ is open.
        \item $F/K$ is Galois $\iff$ $\text{Gal}(L/F)$ is normal in $\text{Gal}(L/K)$.
    \end{itemize}
\end{theorem}
\begin{proof}
    Exercise on example sheet 4. The idea is to reduce everything to finite extensions.
\end{proof}
\marginpar{18 Nov 2022, Lecture 19}

In other words, for $L/K$ Galois, $\text{Gal}(L/K) \cong \underset{\substack{K \subset F \subset L\\F/K \text{ finite Galois}}}{\varprojlim} \text{Gal}(F/K)$.

\begin{example}
    Let $K = \mathbb{F}_q$ and $L = \overline{\mathbb{F}_q}$ the algebraic closure. Then $L/K$ is Galois (for example since $\mathbb{F}_q$ is perfect). Then 
    \begin{align*}
        \{F/K \text{ finite Galois}\} &\leftrightarrow \mathbb{Z}_{\ge 1}\\
        \mathbb{F}_{q^n} &\leftrightarrow n
    \end{align*}
    where on the left we have ordering by containment and on the right by divisibility, i.e. $\mathbb{F}_{q^m} \subset \mathbb{F}_{q^n} \iff m \mid n$.
    
    We can show there exists a commutative diagram $\text{Frob}_q \in \text{Gal}(\mathbb{F}_{q^n}/\mathbb{F}_q) \twoheadrightarrow \text{Gal}(\mathbb{F}_{q^m}/\mathbb{F}_q) \ni \text{Frob}(q)$ corresponding to $1 \in\mathbb{Z}/n\mathbb{Z} \twoheadrightarrow \mathbb{Z}/m\mathbb{Z} \ni 1$.
    This gives us \begin{align*}
        \text{Gal}(\overline{\mathbb{F}_q}/\mathbb{F}_q) \cong \underset{n \in (\mathbb{N}, \mid )}{\varprojlim} \mathbb{Z}/n\mathbb{Z} = \hat{\mathbb{Z}},
    \end{align*}
    where $\hat{\mathbb{Z}}$ is the \textbf{profinite completion} of $\mathbb{Z}$. On example sheet 3 we will see $\hat{\mathbb{Z}} \cong \prod_{p}^{} \mathbb{Z}_p$. Note further that $\text{Frob}_q \in \text{Gal}(\overline{\mathbb{F}_q}/\mathbb{F}_q)$ corresponds to $1 \in \hat{\mathbb{Z}}$. Let $\langle \text{Frob}_q \rangle \subset \text{Gal}(\overline{\mathbb{F}_q}/\mathbb{F}_q)$ generated by $\text{Frob}_q$. Then $ \text{Frob}_q \subset \text{Gal}(\overline{\mathbb{F}_q}/\mathbb{F}_q)$ corresponds to $\mathbb{Z} \subset \hat{\mathbb{Z}}$.
\end{example}

\section{Weil groups}
Let $K$ be a local field and $L/K$ a separable algebraic extension.
\begin{defn}
    We say that
    \begin{enumerate}[(i)]
        \item $L/K$ is \textbf{unramified} if $F/K$ is unramified for all $F/K$ finite subextensions.
        \item $L/K$ is \textbf{totally ramified} if $F/K$ is totally ramified for all $F/K$ finite subextensions.
    \end{enumerate}
\end{defn}
\begin{prop}
    If $L/K$ is unramified, then $L/K$ is Galois and $\text{Gal}(L/K) \cong \text{Gal}(k_L/k)$.
\end{prop}
\begin{proof}
    We know that every finite subextension $F/K$ is unramified and hence Galois. Hence $L/K$ is normal and separable and hence Galois. Moreover, there exists a commutative diagram $\text{Gal}(L/K) \stackrel{\text{res}}{\to} \text{Gal}(k_L/k)$ which by Theorem \ref{theorem16.2} are correspondingly isomorphic to $\underset{\substack{F \subset L\\F/K \text{ finite}}}{\varprojlim} \text{Gal}(F/K) \stackrel{\text{res}}{\to} \underset{\substack{k' \subset k_L \\ k'/k \text{ finite}}}{\varprojlim} \text{Gal}(k'/k)$, but the LHS here is isomorphic to $\underset{\substack{F \subset L\\F/K \text{ finite}}}{\varprojlim} \text{Gal}(k_F/k)$. But $k_F = k$ for all $F/K$ finite subextensions, so we get the desired isomorphism. But we have a bijection between $\{F/K \text{ finite}, F \subset K\}$ and $\{k'/k \text{ finite}, k' \subset k_L\}$, which means our restriction map is an isomorphism.
\end{proof}
If $L_1/K$ and $L_2/K$ are finite unramified extensions, then example sheet 3 shows that $L_1L_2/K$ is unramified. Thus for any $L/K$, we can define a maximal unramified subextension $K_0/K$. 
\vspace{1mm}
 
Now let $L/K$ be any Galois extension. We can show that there exists a surjective map 
\begin{align*}
    \text{res}: \text{Gal}(L/K) \twoheadrightarrow \text{Gal}(K_0/K) \cong \text{Gal}(k_L/k).
\end{align*}
Here we define $I_{L/K} = \text{ker}(\text{res})$ to be the \textbf{inertia subgroup}.
\vspace{1mm}
 
Let $\text{Fr}_{k_L/k} \in \text{Gal}(k_L/k)$ be the Frobenius map $x \mapsto x^q$ and let $\langle \text{Fr}_{k_L/k} \rangle$ denote the subgroup generated by $\text{Fr}_{k_L/k}$. 
\begin{defn}
    Let $L/K$ be a Galois extension. The \textbf{Weil group} $W(L/K) \subset \text{Gal}(L/K)$ is \begin{align*}
        \text{res}^{-1}(\langle \text{Fr}_{k_L/k} \rangle).
    \end{align*}
\end{defn}
\textbf{Remark.} If $k_L/k$ is finite, then $W(L/K) = \text{Gal}(L/K)$ (since then the subgroup is cyclic and generated by $\text{Fr}_{k_L/k}$).
\vspace{1mm}
 
Another way to think of this is through a commutative diagram with exact rows:
$\begin{tikzcd}[row sep=2em, column sep=2em]
0\rar & I_{L/K} \dar[draw=none, "\LARGE\cong" {sloped, anchor=south}] \rar[""] & W(L/K) \dar[leftrightarrow] \rar[""]& \langle \text{Fr}_{k_L/k} \rangle \dar[leftrightarrow] \rar & 0 \\
0\rar & I_{L/K} \rar[""]& \text{Gal}(L/K) \rar[""]& \text{Gal}(k_L/k) \rar & 0
\end{tikzcd}$.
\vspace{1mm}
 
However, the most important property of $W(L/K)$ is its topology. For this, endow $W(L/K)$ with the weakest topology such that:
\begin{enumerate}[(i)]
    \item $W(L/K)$ is a topological group, i.e. a group with a topology such that the multiplication and inversion maps are continuous.
    \item $I_{L/K}$ is an open subgroup of $W(L/K)$, where $I_{L/K} \subset \text{Gal}(L/K)$ is equipped with the subspace topology (which here in fact is equivalent to the profinite topology).
\end{enumerate}
In other words, a basis of open sets of $W(L/K)$ is just the translates of open sets in $I_{L/K}$ by elements in $W(L/K)$.
\vspace{1mm}
 
\textbf{Warning.} If $k_L/k$ is infinite, then this is not the subspace topology on $W(L/K) \subset \text{Gal}(L/K)$. For example, $I_{L/K} \subset W(L/K)$ is open by definition, but $I_{L/K} \subset \text{Gal}(L/K)$ is not open in the subspace topology. 
\begin{prop}
    Let $L/K$ be Galois. Then 
    \begin{enumerate}[(i)]
        \item $W(L/K)$ is dense in $\text{Gal}(L/K)$.
        \item If $F/K$ is a finite subextension of $L/K$, then $W(L/F) = W(L/K) \cap \text{Gal}(L/F)$.
        \item If $F/K$ is a finite Galois subextension, then $\frac{W(L/K)}{W(L/F)} \cong \text{Gal}(F/K)$.
    \end{enumerate}
\end{prop}
\begin{proof}
    \begin{enumerate}[(i)]
        \item $W(L/K)$ is dense in $\text{Gal}(L/K) \iff ~\forall F/K$ a finite Galois subextension, $W(L/K)$ will intersect every coset of $\text{Gal}(L/F) \iff ~\forall F/K$ finite Galois, $W(L/K)$ surjects onto $\text{Gal}(F/K)$. For this last result, consider the following commutative diagram with exact rows:
        $$\begin{tikzcd}[row sep=2em, column sep=2em]
            0\rar & I_{L/K} \dar["(a)"] \rar[""] & W(L/K) \dar["(b)"] \rar[""]& \langle \text{Fr}_{k_L/k} \rangle \dar["(c)"] \rar & 0 \\
            0\rar & I_{F/K} \rar[""]& \text{Gal}(F/K) \rar[""]& \text{Gal}(k_F/k) \rar & 0
        \end{tikzcd}$$
        We show $(a)$ and $(c)$ are surjective, which will imply that $(b)$ is surjective through a diagram chase.
        \begin{itemize}
            \item For $(a)$, if $K_0/K$ is the maximal unramified extension contained in $L$, then $K_0 \cap F$ is the maximal unramified extension contained in $F$. This is because $I_{L/K} = \text{Gal}(L/K_0) \twoheadrightarrow \text{Gal}(FK_0/K_0) \cong \text{Gal}(F/K_0 \cap F)$, but $\text{Gal}(L/K_0)$ also surjects onto $\text{Gal}(F/K_0 \cap F)$, so $\text{Gal}(F/K_0 \cap F) = I_{F/K}$.
            \item For $(c)$, $\text{Gal}(k_F/k)$ is generated by $\text{Fr}_{k_F/k}$, which implies $(c)$ is surjective.
        \end{itemize}
        \item Let $F/K$ be a finite subextension and consider the surjective maps 
        \begin{align*}
            \begin{matrix} \text{Gal}(L/K) & \twoheadrightarrow & \text{Gal}(k_L/k) & \supset & \langle \text{Fr}_{k_L/k} \rangle \\ \rotatebox[origin=c]{90}{$\hookrightarrow$} & & \rotatebox[origin=c]{90}{$\hookrightarrow$}& & \rotatebox[origin=c]{90}{$\hookrightarrow$} \\ \text{Gal}(L/F) & \twoheadrightarrow & \text{Gal}(k_L/k_F) & \supset & \langle \text{Fr}_{k_L/k_F} \rangle \end{matrix}.
        \end{align*}
        For $\sigma \in \text{Gal}(L/F)$, $\sigma \in W(L/F) \iff \sigma|_{k_L} \in \langle \text{Fr}_{k_L/k_F} \rangle \stackrel{(\star)}{\iff} \sigma|_{k_L} \in \langle \text{Fr}_{k_L/k} \rangle \iff \sigma \in W(L/K)$, where the second equivalence $(\star)$ follows as $\text{Gal}(k_L/k_F) \cap \langle \text{Fr}_{k_L/k} \rangle = \langle \text{Fr}_{k_L/k_F} \rangle$.
        \item As quotients, \begin{align*}
            \frac{W(L/F)}{W(L/K)} \stackrel{\text{(ii)}}{=} \frac{W(L/K)}{W(L/K) \cap \text{Gal}(L/F)} = \frac{W(L/K)\text{Gal}(L/F)}{\text{Gal(L/F)}} \stackrel{\text{(i)}}{=} \frac{\text{Gal}(L/K)}{\text{Gal}(L/F)} \cong \text{Gal}(F/K).
        \end{align*}
    \end{enumerate}
\end{proof}
\marginpar{20 Nov 2022, Lecture 20}
Let $K$ be a local field.
\begin{defn}
    Let $K$ be a local field. An extension $L/K$ is \textbf{abelian} if it is Galois and $\text{Gal}(L/K)$ is an abelian group.
\end{defn}
\textbf{Facts.} Let $L_1/K$ and $L_2/K$ be abelian. Then:
\begin{enumerate}[(i)]
    \item $L_1L_2/K$ is abelian.
    \item If $L_1 \cap L_2 = K$, then there exists a canonical isomorphism $\text{Gal}(L_1L_2/K) \cong \text{Gal}(L_1/K)\times \text{Gal}(L_2/K)$.
\end{enumerate} 
Note that (i) implies that there exists a maximal abelian extension $K^{\text{ab}}$ of $K$.
\begin{example}
    Let $K^{\text{ur}}$ denote the maximal unramified extension of $K$ inside $K^{\text{sep}}$ (the separable closure of $K$). Then $K^{\text{ur}} = \bigcup_{m=1}^{\infty} K(\zeta_{q^m-1})$ (for $q = |k|$). Also note that $k_{K^{\text{ur}}} = \overline{\mathbb{F}_q}$.
    \vspace{1mm}
     
    We know that for unramified extensions, $\text{Gal}(K^{\text{ur}}/K) \cong \text{Gal}(\overline{\mathbb{F}_q}/\mathbb{F}_q) \cong \hat{\mathbb{Z}}$, so this extension is abelian. Also note that $\text{Fr}_{\overline{\mathbb{F}_q}/\mathbb{F}_q} \in \text{Gal}(\overline{\mathbb{F}_q}/\mathbb{F}_q)$, which corresponds to $\text{Fr}_{K^{\text{ur}}/K}$. Hence $K^{\text{ur}}$ is abelian and $K^{\text{ur}} \subset K^{\text{ab}}$. This gives us an exact sequence
    \[
    0 \to I_{K^{\text{ab}}/K} \to W(K^{\text{ab}}/K) \twoheadrightarrow \mathbb{Z} = \langle \text{Fr}_{K^{\text{ab}}/K} \rangle \to 0.
    \]
\end{example}
\begin{theorem}\label{theorem17.3}
    \begin{enumerate}[(i)]
        \item (Local Artin reciprocity). For $K$ a local field, there exists a unique topological isomorphism (i.e. a group isomorphism and a homomorphism) $\text{Art}_K : K^\times \to W(K^{\text{ab}}/K)$ satisfying:
        \begin{itemize}
            \item $\text{Art}_K(\pi)|_{K^{\text{ur}}} = \text{Fr}_{K^{\text{ur}}/K}$ for any uniformizer $\pi$ of $K$.
            \item For each finite subextension $L/K$ inside $K^{\text{ab}}/K$, $\text{Art}_{K}(N_{L/K}(L^\times))|_{L} = \{1\}$. 
        \end{itemize}
        $\text{Art}_K$ is called the \textbf{Artin map} or the \textbf{Artin reciprocity map}.
        \item Let $L/K$ be a finite abelian extension. Then $\text{Art}_K$ induces an isomorphism $K^\times/N_{L/K}(L^{\times}) \cong \frac{W(K^{\text{ab}}/K)}{W(K^{\text{ab}}/L)} \cong \text{Gal}(L/K)$ (as quotients).
    \end{enumerate}
\end{theorem}
Properties of the Artin map:
\begin{itemize}
    \item (Existence theorem). For $H \subset K^\times$ an open finite index subgroup, there exists a finite abelian extension $L/K$ such that $N_{L/K}(L^\times) = H$. In particular, $\text{Art}_K$ induces an (inclusing reversing) isomorphism of posets $\{\text{open finite index subgroups of }K^\times\} \leftrightarrow \{\text{finite abelian extensions }L/K\}$, given by $H \mapsto (K^{\text{ab}})^{\text{Art}_K(H)}$ (the fixed field) and $N_{L/K}(L^\times) \mapsfrom L/K$.
    \item (Norm functoriality). Let $L/K$ be a finite separable extension. Then we have the commutative diagram 
    $\begin{tikzcd}
        L^\times & W(L^{\text{ab}}/L)\\
        K^\times & W(K^{\text{ab}}/K)
        \arrow["N_{L/K}", from=1-1, to=2-1]
        \arrow["\text{Art}_L", from=1-1, to=1-2]
        \arrow["\text{Art}_K", from=2-1, to=2-2]
        \arrow["\text{res}", from=1-2, to=2-2]
    \end{tikzcd}$.
\end{itemize}
The rest of the course is now focused the construction of the Artin map.
\begin{prop}
    If $L/K$ is finite abelian extension of degree $n$, then $e_{L/K} = [\mathcal{O}_K^\times : N_{L/K}(\mathcal{O}_L^\times)]$.
\end{prop}
\begin{proof}
    Given $x \in L^\times$, we have $v_K(N_{L/K}(x)) = f_{L/K}v_L(x)$. This implies that we have a surjection $K^\times/N_{L/K}(L^\times) \stackrel{v_K}{\to} \mathbb{Z}/f_{L/K}\mathbb{Z}$ with kernel $\frac{\mathcal{O}_K^\times N_{L/K}(L^\times)}{N_{L/K}(L^\times)} \cong \frac{\mathcal{O}_K^\times}{\mathcal{O}_K^\times \cap N_{L/K}(L^\times)} \cong \frac{\mathcal{O}_K^\times}{N_{L/K}(\mathcal{O}_L^\times)}$. By Theorem \ref{theorem17.3} (ii), $n = [K^\times : N_{L/K}(L^\times)] = f_{L/K} [\mathcal{O}_K^\times : N_{L/K}(\mathcal{O}_L^\times)]$. To finish, use the fact that $n = e_{L/K}f_{L/K}$.
\end{proof}
\begin{cor}
    Let $L/K$ be finite abelian. Then $L/K$ is unramified $\iff N_{L/K}(\mathcal{O}_L^\times) = \mathcal{O}_K^\times$. 
\end{cor}
\subsection{Construction of $\text{Art}_{\mathbb{Q}_p}$}
Recall that $\mathbb{Q}_p^{\text{ur}} = \bigcup_{m=1}^{\infty} \mathbb{Q}_p(\zeta_{p^m-1}) = \bigcup_{p \nmid m} \mathbb{Q}_{p}(\zeta_{m})$. On the other hand, $\mathbb{Q}_p(\zeta_{p^n})/\mathbb{Q}_p$ is totally ramified of degree $p^{n-1}(p-1)$ with $\theta_n : \text{Gal}(\mathbb{Q}_p(\zeta_{p^n})/\mathbb{Q}_p) \stackrel{\sim}{\to} (\mathbb{Z}/p^n\mathbb{Z})^\times$.
\vspace{1mm}
 
For $n\ge m\ge 1$, there exists a commutative diagram $$\begin{tikzcd}
    \text{Gal}(\mathbb{Q}_p(\zeta_{p^n})/\mathbb{Q}_p) & \text{Gal}(\mathbb{Q}_p    (\zeta_{p^m})/\mathbb{Q}_p)\\
    (\mathbb{Z}/p^n \mathbb{Z})^\times & (\mathbb{Z}/p^m \mathbb{Z})^\times
    \arrow["\theta_n", from=1-1, to=2-1]
    \arrow["", from=1-1, to=1-2, two heads]
    \arrow["\substack{\text{canonical}\\\text{projection}}", from=2-1, to=2-2]
    \arrow["\theta_m", from=1-2, to=2-2]
\end{tikzcd}$$
with the vertical arrows being isomorphisms. Set $\mathbb{Q}_p(\zeta_{p^\infty}) = \bigcup_{n=1}^\infty \mathbb{Q}_p(\zeta_{p^n})$. Then $\mathbb{Q}_p(\zeta_{p^{\infty}})/\mathbb{Q}_p$ is Galois and we have 
\begin{align*}
    \theta : \text{Gal}(\mathbb{Q}_p(\zeta_{p^{\infty}})/\mathbb{Q}_p) \stackrel{(\star)}{\cong} \underset{n}{\varprojlim} (\mathbb{Z}/p^n \mathbb{Z})^\times \cong \mathbb{Z}_p^\times
\end{align*}
with $(\star)$ following from example sheet 4.
\vspace{1mm}
 
We have $\mathbb{Q}_p(\zeta_{p^{\infty}}) \cap \mathbb{Q}_p^{\text{ur}} = \mathbb{Q}_p$, since the first extension is totally ramified and the second is unramified. Hence there exists an isomorphism $\text{Gal}(\mathbb{Q}_p(\zeta_{p^{\infty}})/\mathbb{Q}_p) \cong \hat{\mathbb{Z}} \times \mathbb{Z}_p^\times$.
\begin{theorem}[Local Kronecker--Weber]
    $\mathbb{Q}_p^{\text{ab}} = \mathbb{Q}_p^{\text{ur}} \mathbb{Q}_p(\zeta_{p^{\infty}})$.
\end{theorem}
\begin{proof}
    Omitted in this course.
\end{proof}
Assuming this theorem, we can construct the Artin map as follows: We have $\mathbb{Q}_p^\times \cong \mathbb{Z} \times \mathbb{Z}_p^\times$ by $(n,u) \mapsto p^n u$. Then let $$\text{Art}_{\mathbb{Q}_p}(p^n u) = (\text{Fr}_{K^{\text{ur}}/K}^n, \theta^{-1}(u^{-1}))$$ with the first element lying inside $\text{Gal}(\mathbb{Q}_p^{\text{ur}}/\mathbb{Q}_p) \times \text{Gal}(\mathbb{Q}_p(\zeta_{p^{\infty}})/\mathbb{Q}_p) \cong \mathbb{Q}_p^{\text{ab}}/\mathbb{Q}_p$ and we can check that the image of this map lies inside $W(\mathbb{Q}_p^{\text{ab}}/\mathbb{Q}_p)$.

\subsection{Construction of $\text{Art}_K$}
Let $K$ be a local field and $\pi$ a uniformizer of $K$. For $n \ge 1$, construct $K_{\pi,n}$ totally ramified and Galois over $K$ such that:
\begin{enumerate}[(i)]
    \item $K \subset \ldots \subset K_{\pi, n} \subset K_{\pi, n+1} \subset \ldots$.
    \item For $n \ge m\ge 1$, there exists a commutative diagram $$\begin{tikzcd}
        \text{Gal}(K_{\pi, n}/K) & \text{Gal}(K_{\pi, m}/K) \\
        \mathcal{O}_K^\times/U_k^{(n)} & \mathcal{O}_K^\times/U_k^{(m)}
        \arrow["\Psi_n", from=1-1, to=2-1]
        \arrow["", from=1-1, to=1-2, two heads]
        \arrow["", from=2-1, to=2-2, two heads]
        \arrow["\Psi_m", from=1-2, to=2-2]
    \end{tikzcd}$$
    with the vertical arrows being isomorphisms.
    \item Setting $K_{\pi, \infty} = \bigcup_{n=1}^{\infty} K_{\pi,n}$, we have $K^{\text{ab}} = K^{\text{ur}}K_{\pi, \infty}$.
\end{enumerate}
Once we have this, we can define the Artin map: (ii) implies that there exists an isomorphism $\Psi : \text{Gal}(K_{\pi, \infty}/K) \stackrel{\sim}{\to} \mathcal{O}_K^\times$, and now define $\text{Art}_K$ by 
\begin{align*}
    K^{\times} &\cong \mathbb{Z} \times \mathcal{O}_K^\times \to \text{Gal}(K^{\text{ur}}/K)\times\text{Gal}(K_{\pi, \infty}/K) \cong \text{Gal}(K^{\text{ab}}/K).\\
    p^n u &\mapsfrom (n,u) \mapsto (\text{Fr}_{K^{\text{ur}}/K}^n, \Psi^{-1}(u^{-1})).
\end{align*}
Our goal now is to construct these $K_{\pi, n}$.
\marginpar{23 Nov 2022, Lecture 21}

\textbf{Remark.} There is no maximal totally ramified extension of $K$. Hence $\text{Art}_K$ depends on the choice $K_{\pi, \infty}$, and the isomorphism $K^{\times} \cong \mathbb{Z} \times \mathcal{O}_K^\times$. These choices are both dependent on the choice of $\pi$ -- they ''cancel out'' and $\text{Art}_K$ is canonical.

\section{Lubin--Tate theory}
\subsection{Formal group laws}
Let $R$ be a ring. We denote by $R[[X_1,\ldots,X_n]] = \{\sum_{k_1,\ldots,k_n\ge 0}^{} a_{k_1\ldots,k_n}X^{k_1}\ldots X^{k_n} \mid a_{k_1\ldots k_n} \in R\}$ the ring of formal power series in variables over $R$.
\begin{defn}
    A (1--dimensional) \textbf{formal group law} over $R$ is a power series $F[[X,Y]] \in R[[X,Y]]$ satisfying
    \begin{enumerate}
        \item $F(X,Y) \equiv X + Y \pmod{\text{deg }2}$ (i.e. modulo $X^2,XY,Y^2$).
        \item $F(X,F(Y,Z)) = F(F(X,Y), Z)$ -- the associativity axiom.
        \item $F(X,Y) = F(Y,X)$.
    \end{enumerate}
\end{defn}
\begin{example}
    \begin{itemize}
        \item $\hat{\mathcal{G}}_a = X + Y$. This is called the formal additive group.
        \item $\hat{\mathcal{G}}_m = X + Y + XY$. This is the formal multiplicative group.
    \end{itemize}
\end{example}
\begin{lemma}
    For $F$ a formal group law over $R$,
    \begin{enumerate}[(i)]
        \item $F(X,0) = X$ and $F(0,Y) = Y$.
        \item There exists a unique power series $i(X) \in X R[[X]]$ with   $F(X,i(X))=0$.
    \end{enumerate}
\end{lemma}
\begin{proof}
    Exercise on example sheet 4.
\end{proof}
If $K$ is a non--archimedean valued field and $F$ is a formal group law over $\mathcal{O}_K$, then $F(x,y)$ converges $\forall x,y \in \mathfrak{m}_K$ to an element in $\mathfrak{m}_K$. This allows us to define $x \cdot_F y = F(x,y)$, which turns $(\mathfrak{m}_K, \cdot_F)$ into a commutative group.

\begin{example}
    $\hat{\mathcal{G}}_m/\mathbb{Z}_p$, so $x \cdot_{\hat{\mathcal{G}}_m} y = x + y + xy$. Then $(p \mathbb{Z}_p, \hat{\mathcal{G}}_m) \cong (1 + p \mathbb{Z}_p, \times)$ by $x \mapsto 1 + x$.
\end{example}
\begin{defn}
    Let $F,G$ be formal group laws over a ring $R$. A \textbf{homomorphism} $f : F \to G$ is an element $f(X) \in XR[[X]]$ such that $f(F(X,Y))=G(f(X),f(Y))$.
    \vspace{1mm}
     
    A homomorphism $f : F \to G$ is an \textbf{isomorphism} if there exists a homomorphism $g : G \to F$ such that $f(g(X))=g(f(X))=X$.
\end{defn}
\begin{defn}
    We define $\text{End}_R(F)$ to be the set of homomorphisms $f: F \to F$.
\end{defn}
\begin{prop}
    Let $R$ be a $\mathbb{Q}$--algebra. Then there is an isomorphism of formal group laws $\exp : \hat{\mathcal{G}}_a \to \hat{\mathcal{G}}_m$ given by $\exp(X) = \sum_{n=1}^{\infty} \frac{X^n}{n!}$ (notice that the constant term is missing).
\end{prop}
\begin{proof}
    Define $\log(X) = \sum_{n=1}^{\infty} (-1)^{n-1}\frac{X^n}{n}$. Then there exists an equality (of formal power series) $\log(\exp(X))=\exp(\log(X))=X$ and $\exp(\hat{\mathcal{G}}_a(X,Y))=\hat{G}_m(X,Y)$. 
\end{proof}
Note that we need $R$ to be a $\mathbb{Q}$--algebra for $\frac{1}{n!}$ to make sense. An exercise on example sheet 4 shows that if we replace $R$ with a field of positive characteristic then there are no nontrivial homomorphisms between the additive group and the multiplicative group.
\begin{lemma}
    $\text{End}_R(F)$ is a ring with addition $(f +_{F} g) = F(f(X),g(X))$ and multiplication given by composition.
\end{lemma}
\begin{proof}
    For $f, g \in \text{End}_R(F)$, we have 
    \begin{align*}
        &(f +_F g) \circ F(X,Y) = F(f(F(X,Y)), g(F(X,Y))) \\
        =& F(F(f(X),f(Y)),F(g(X),g(Y)))\\
        \stackrel{(\star)}{=} & F(F(f(X),g(X)),F(f(Y),g(Y)))\\
        =& F(f +_F g(X), f +_F g(Y))\\
        \implies & f +_F g \in \text{End}_R(F),
    \end{align*}
    where $(\star)$ follows by associativity and commutativity axioms.
    \vspace{1mm}
     
    Also $f \circ g \circ F  = f \circ F \circ g = F \circ f \circ g$, so $f \circ g \in \text{End}_R(F)$. The ring axioms are left as an exercise.
\end{proof}
\subsection{Lubin--Tate formal groups}
Let $K$ be a non--archimedean local field with residue field $k$ of size $q$.
\begin{defn}
    A \textbf{formal $\mathcal{O}_K$--module over $\mathcal{O}_K$} is a formal group law $F(X,Y) \in \mathcal{O}_K[[X,Y]]$ together with a ring homomorphism $[ ~]_F : \mathcal{O}_K \to \text{End}_{\mathcal{O}_K}(F)$ such that $\forall a \in \mathcal{O}_K, [a]_F(X) \equiv aX \pmod{X^2}$ (i.e. modulo degree 2 terms).
    \vspace{1mm}
     
    A \textbf{homomorphism/isomorphism} of $f: F \to G$ of formal $\mathcal{O}_K$--modules s a homomorphism/isomorphism of formal group laws such that $f \circ [a]_F  = [a]_G \circ f ~\forall a \in \mathcal{O}_K$. 
\end{defn}
\begin{defn}
    Let $\pi \in \mathcal{O}_K$ be a uniformizer. A \textbf{Lubin--Tate series} for $\pi$ is a power series $f(X) \in \mathcal{O}_K[[X]]$ such that
    \begin{enumerate}[(i)]
        \item $f(X) \equiv \pi X \pmod{X^2}$.
        \item $f(X) \equiv X^q \pmod{\pi}$.
    \end{enumerate}
\end{defn}
\begin{example}
    $\pi X + X^q$ is a Lubin--Tate series.
\end{example}
\begin{theorem}\label{theorem18.4}
    Let $f(X)$ be a Lubin--Tate series for $\pi$. Then 
    \begin{enumerate}[(i)]
        \item There exists a unique formal group law $F_f$ over $\mathcal{O}_K$ such that $f \in \text{End}_{\mathcal{O}_K}(F_f)$.
        \item There exists a ring homomorphism $[~]_{F_f}: \mathcal{O}_K \to \text{End}_{\mathcal{O}_K}(F_f)$ satisfying $[\pi]_{F_f}(X)=f(X)$ which makes $F_f$ a formal $\mathcal{O}_K$--module over $\mathcal{O}_K$.
        \item If $g(X)$ is another Lubin--Tate series for $\pi$, then $F_f \cong F_g$ as formal $\mathcal{O}_K$--modules.
    \end{enumerate}
    $F_f$ is called the \textbf{Lubin--Tate formal group law for $\pi$} (which by (iii) only depends on $\pi$ up to isomorphism).
\end{theorem}
\marginpar{26 Nov 2022, Lecture 22}
\begin{example}
    If $K= \mathbb{Q}_p$, then $f(X) = (X+1)^p-1$ is a Lubin--Tate series for (the uniformizer) $p$. The Lubin--Tate formal group law in this case is $\hat{\mathcal{G}}_m$. For this, it suffices to show that $f \circ \hat{\mathcal{G}}_m = \hat{\mathcal{G}}_m \circ f$, but we compute 
    \begin{align*}
        f \circ \hat{\mathcal{G}}_m(X,Y) &= (1+X)^p(1+Y)^p-1 = \hat{\mathcal{G}}_m(f(X),f(Y))
    \end{align*}
    using $\hat{\mathcal{G}}_m(X,Y) = (1+X)(1+Y)-1$.
\end{example}
The key to the proof of Theorem \ref{theorem18.4} is the following lemma:
\begin{lemma}\label{lemma18.5}
    Let $f(X),g(X)$ be Lubin--Tate series for $\pi$ and $L(X_1,\ldots,X_n) = \sum_{i=1}^{n} a_i X_i$ with $a_i \in \mathcal{O}_K$. Then there exists a unique power series $F(X_1,\ldots,X_n) \in \mathcal{O}_K[[X_1,\ldots,X_n]]$ such that
    \begin{enumerate}[(i)]
        \item $F(X_1,\ldots,X_n) \equiv L(X_1,\ldots,X_n) \pmod{\text{deg }2}$
        \item $f(F(X_1,\ldots,X_n))  = F(g(X_1),\ldots, g(X_n))$.
    \end{enumerate}
\end{lemma}
\begin{proof}
    We show by induction that there exist unique polynomials $F_m \in \mathcal{O}_K[X_1,\ldots,X_m]$ of total degree $\le m$ satisfying:
    \begin{enumerate}[(a)]
        \item $f(F_m(X_1,\ldots,X_n)) \equiv F_m(g(X_1),\ldots,g(X_n)) \pmod{\text{deg }m+1}$.
        \item $F_m(X_1,\ldots,X_n) \equiv L(X_1,\ldots,X_n) \pmod{\text{deg }2}$.
        \item $F_m \equiv F_{m+1} \pmod{\text{deg }m+1}$.
    \end{enumerate}
    For $m=1$, take $F_1=L$. It satisfies property (b), and (a) follows from the fact that $f(F(X_1,\ldots,X_n)) \stackrel{(\star)}{\equiv } \pi L(X_1,\ldots,X_n)\equiv F(g(X_1),\ldots,g(X_n)) \pmod{\text{deg }2}$, where $(\star)$ follows as $f(X) \equiv \pi X \pmod{X^2}$. 
    \vspace{1mm}
     
    Now suppose we've constructed $F_m$ for some $m\ge 1$. We set $F_{m+1} = F_m + h$ for $h \in \mathcal{O}_K[X_1,\ldots,X_n]$ homogeneous of degree $m+1$. Since $f(X+Y) = f(X) + f'(X)Y + Y^2(\ldots)$ and $f'(X) \equiv \pi \pmod{X}$, we have $$f \circ (F_m + h) = f \circ F_m + \pi h \pmod{\text{deg }m+2}.$$ Similarly $(F_m + h) \circ g \stackrel{(\star)}{\equiv}  F_m \circ g + h(\pi X_1,\ldots,\pi X_n) \equiv F_m \circ g + \pi^{m+1}h(X_1,\ldots,X_n) \pmod{\text{deg }m+2}$ where $(\star)$ follows from $g(X) \equiv \pi X \pmod{X^2}$. Thus (a), (b), (c) will be satisfied if and only if $$f \circ F_m - F_m \circ g \equiv (\pi - \pi^{m+1})h \pmod{\text{deg }m+2}.$$ But using the fact that $f(X) \equiv g(X) \equiv X^q \pmod{\pi}$, we find $$f \circ F_m - F_m \circ g \equiv F_m(X_1,\ldots,X_n)^q - F_m(X_1^q,\ldots,X_n^q) \equiv 0 \pmod{\pi}.$$
    Let $r(X_1,\ldots,X_n)$ be the degree $m+1$ terms in $f \circ F_m - F_m \circ g$. Then set $h =\frac{1}{\pi(1-\pi^m)}r \in \mathcal{O}_K[X_1,\ldots,X_n]$, so our computations show $F_{m+1}$ satifies (a), (b), (c). This is unique, since $h$ is determined by (a).
    \vspace{1mm}
     
    Set $F = \lim_{m \to \infty}F_m$, so $F(X_1,\ldots,X_n) \in \mathcal{O}_K[[X_1,\ldots,X_n]]$ and $F$ satisfies properties (i) and (ii). The uniqueness of $F$ follows from the uniqueness of the $F_m$, as truncating the power series $F$ needs to give $F_m$.
\end{proof}
\begin{proof}[Proof of Theorem \ref{theorem18.4}]
    \begin{enumerate}[(i)]
        \item By Lemma \ref{lemma18.5}, there exists a unique $F_f(X,Y) \in \mathcal{O}_K[[X,Y]]$ such that 
        \begin{itemize}
            \item $F(X,Y) \equiv X + Y \pmod{\text{deg }2}$.
            \item $f(F_f(X,Y)) = F_f(f(X),f(Y))$.
        \end{itemize}
        We need to check that $F_f$ is a formal group law. 
        \begin{itemize}
            \item Associativity: $F_f(X,F_f(Y,Z)) \equiv X + Y + Z \equiv F_f(F_f(X,Y),Z)\pmod{\text{deg }2}$ and $f \circ F_f(X,F_f(Y,Z)) = F_f(f(X),f(F_f(Y,Z))) = F_f(f(X),F_f(f(Y),F(Z)))$. Similarly we find $f \circ F_f(F_f(X,Y),Z) = F_f(F_f(f(X),f(Y)),f(Z))$. By uniqueness in Lemma \ref{lemma18.5}, we have $F_f(X,F_f(Y,Z)) = F_f(F_f(X,Y),Z)$.
            \item Commutativity: analogously check that the two power series are equal, have the same linear terms and are both compatible with the Lubin--Tate series.
        \end{itemize}
        \item By Lemma \ref{lemma18.5}, for $a \in \mathcal{O}_K$ there exists a unique $[a]_{F_f} \in \mathcal{O}_K[[X]]$ such that $[a]_{F_f} \equiv a X \pmod{X^2}$ and $f \circ [a]_{F_f} = [a]_{F_f} \circ f$. Then $[a]_{F_f} \circ F_f = F_f \circ [a]_{F_f}$ by uniqueness, i.e. $[a]_{F_f} \in \text{End}_{\mathcal{O}_K}(F_f)$. The map $[~]_{F_f} : \mathcal{O}_K \to \text{End}_{\mathcal{O}_K}(F_f)$ is a ring homomorphism (again by uniqueness). This implies that $F_f$ is a formal $\mathcal{O}_K-$-module over $\mathcal{O}_K$. The fact that $[\pi]_{F_f} = f$ follows by uniqueness again.
        \item If $g(X)$ is another Lubin--Tate series for $\pi$, then let $\theta(X) \in \mathcal{O}_K[[X]]$ such that
        \begin{itemize}
            \item $\theta(X) \equiv X \pmod{X^2}$
            \item $\theta \circ f(X) = g \circ \theta(X)$.
        \end{itemize}
        Then $\theta \circ F_f = F_g \circ \theta$ (by uniqueness again), so $\theta \in \text{Hom}(F_f,F_g)$. Reversing the roles of $f$ and $g$ we obtain $\theta^{-1} \in \mathcal{O}_K[[X]]$ such that $\theta^{-1} \in \text{Hom}(F_g,F_f)$. Then $\theta \circ \theta^{-1}(X) = X = \theta^{-1} \circ \theta(X)$ (by uniqueness in Lemma \ref{lemma18.5}). Finally, uniqueness implies that $\theta \circ [a]_{F_f} = [a]_{F_g} \circ \theta ~\forall a \in \mathcal{O}_K$, so $\theta$ is an isomorphism of $\mathcal{O}_K$--modules.
    \end{enumerate}
\end{proof}
\subsection{Lubin--Tate extensions}
Let $\overline{K}$ be a separable closure of $K$ and $\overline{\mathfrak{m}} \subset \mathcal{O}_{\overline{K}}$ the maximal ideal.
\begin{lemma}
    Let $F$ be a formal $\mathcal{O}_K$--module over $\mathcal{O}_K$. Then $\overline{\mathfrak{m}}$ becomes a (genuine) $\mathcal{O}_K$--module with operations $x+_F y = F(x,y)$ for $x,y \in \overline{\mathfrak{m}}$ and $a \cdot _F x = [a]_F(x)$ for $a \in \mathcal{O}_K, x \in \overline{\mathfrak{m}}$. 
\end{lemma}
\begin{proof}
    We can't apply the argument directly to $\overline{K}$, as it is not complete. Instead, $x \in \overline{\mathfrak{m}} \implies x \in \mathfrak{m}_L$ for some $L/K$ finite. Then $[a]_F \in \mathcal{O}_K[[X]] \implies [a]_F(x)$ converges in $L$ and since $\mathfrak{m}_L$ is closed, $[a]_F(x) \in \mathfrak{m}_L \subset \overline{\mathfrak{m}}$. 
    \vspace{1mm}
     
    Similarly $x +_F y  \in \overline{\mathfrak{m}}$.
    \vspace{1mm}
     
    The module structure follows from definitions.
\end{proof}
\end{document}