\documentclass{article}
%build with recipe latexmk
\usepackage[utf8]{inputenc}
\usepackage[T1]{fontenc}
\usepackage{textcomp}
\usepackage{fancyhdr}
\pagestyle{fancy}

\usepackage{tcolorbox}
\tcbuselibrary{theorems}
\usepackage{babel}
\usepackage{enumerate}
\usepackage{amsmath, amssymb, amsthm}
%\usepackage{a4wide}
\usepackage{float}
\usepackage{tikz-cd}
\usepackage{tikz}
\usepackage{graphicx}
\usepackage{caption}
\usepackage{wrapfig}
\usepackage{setspace}
\setstretch{1.1}
\usepackage{color}
\usepackage{hyperref}
\hypersetup{
    colorlinks=true, %set true if you want colored links
    linktoc=all,     %set to all if you want both sections and subsections linked
    linkcolor=black,  %choose some color if you want links to stand out
}

\theoremstyle{definition}
\newtheorem{theorem}{Theorem}[section]
\newtheorem{lemma}[theorem]{Lemma}
\newtheorem{cor}[theorem]{Corollary}
\newtheorem{prop}[theorem]{Proposition}
\newtheorem{example}{Example}[section]
\newtheorem{defn}{Definition}[section]

\title{Part III - Local Fields
    \\ \large
    Lectured by Rong Zhou 
}
 
\author{Artur Avameri}
\date{}
 
\setcounter{section}{-1}
 
\begin{document}
\maketitle
\tableofcontents
\newpage
 
\section{Introduction}

This is a first class in graduate algebraic number theory. Something we'd like to do is solve diophantine equations, e.g. $f(x_1,\ldots,x_r) \in \mathbb{Z}[x_1,\ldots,x_r]$. In general, solving $f(x_1,\ldots,x_r) = 0$ is very difficult. A simpler question we might consider is solving $f(x_1,\ldots,x_r) \equiv 0 \pmod{p}$, or $\pmod{p^2}, \pmod{p^3}$, etc. Local fields package all of this information together. 

\section{Basic Theory}
\subsection{Absolute values}
\begin{defn}
    Let $K$ be a field. An \textbf{absolute value} on $K$ is a function $|\cdot| : K \to \mathbb{R}_{\ge 0}$ satisfying:
    \begin{enumerate}[(1)]
        \item $|x| = 0 \iff x = 0$.
        \item $|xy| = |x||y| ~\forall x,y \in K$.
        \item $|x+y|\le |x|+|y| ~\forall x,y \in K$ (triangle inequality).
    \end{enumerate}
\end{defn}
We say that $(K, |\cdot|)$ is a \textbf{value field}.
Examples:
\begin{itemize}
    \item Take $K=\mathbb{Q},\mathbb{R},\mathbb{C}$ with the usual absolute value $|a+ib| = \sqrt{a^2+b^2}$. We call this $|\cdot |_{\infty}$.
    \item For $K$ any field, we have the trivial absolute value $|x| = \begin{cases}
        0 & \text{ if } x = 0\\
        1 & \text{ else.}
    \end{cases}$
    We will ignore this in this course.
    \item Take $K= \mathbb{Q}$ and $p$ a prime. For $0 \neq x \in \mathbb{Q}$, write $x = p^n \frac{a}{b}$ where $(a,p)=(b,p)=1$. Then the \textbf{$p$--adic absolute value} is defined to be \[
    |x|_p = \begin{cases}
        0 & x=0\\
        p^{-n} & x = p^n\frac{a}{b}.
    \end{cases}
    \] 
    We can check the axioms:
    \begin{enumerate}[(1)]
        \item The first axiom is clear.
        \item $$|xy|_p = \left|p^{n+m}\frac{ac}{bd}\right|_p = p^{-(n+m)} = |x|_p|y|_p.$$
        \item WLOG let $m\ge n$. Then \[
        |x+y|_p = \left|p^n\left(\frac{ad+p^{m-n}bc}{bd}\right)\right|_p \le p^{-n} = \max(|x|_p, |y|_p).
        \]
    \end{enumerate}
\end{itemize}

Any absolute value $|\cdot |$ on $K$ induces a metric $d(x,y) = |x-y|$ on $K$, hence induces a topology on $K$.

\begin{defn}
    Suppose we have two absolute values $|\cdot |, |\cdot |'$ on $K$. We say these absolute values are \textbf{equivalent} if they induce the same topology. An equivalence class is called a \textbf{place}.
\end{defn}
\begin{prop}
    Let $|\cdot|, |\cdot|'$ be (nontrivial) absolute values on $K$. Then the following are equivalent:
    \begin{enumerate}[(i)]
        \item $|\cdot|$ and $|\cdot|'$ are equivalent.
        \item $|x| < 1 \iff |x|' < 1 ~\forall x \in K$.
        \item $\exists c \in \mathbb{R}_{>0}$ such that $|x|^c = |x'| ~\forall x \in K$.
    \end{enumerate}
\end{prop}
\begin{proof}
    (i) $\implies $(ii): $|x|<1 \iff x^n \to 0$ with respect to $|\cdot| \iff x^n \to 0$ with respect to $|\cdot|'$ (since the topologies are the same) $\iff |x|'<1$. 
    \vspace{1mm}
     
    (ii) $\implies $(iii): Note that $|x|^c = |x|' \iff c \log |x| = \log |x|'$. Take $a \in K^\times$ such that $|a| > 1$. This exists since $|\cdot|$ is nontrivial. We need to show that $\forall x \in K^\times,$\[
     \frac{\log |x|}{\log |a|} = \frac{\log|x|'}{\log|a|'}.
    \]
    Assume $\frac{\log |x|}{\log |a|} < \frac{\log|x|'}{\log|a|'}.$ Choose $m, n \in \mathbb{Z}$ such that $\frac{\log |x|}{\log |a|} < \frac{m}{n} < \frac{\log|x|'}{\log|a|'}.$ We then have 
    \begin{align*}
        &\begin{cases}
            &n \log |x| < m \log |a|\\
            &n \log |x|' > m \log|a|'
        \end{cases}\\
        \implies & \left|\frac{x^n}{a^m}\right|<1, \left|\frac{x^n}{a^m}\right|' > 1,
    \end{align*}
    a contradiction. The other inequality is analogous.
    \vspace{1mm}
     
    (iii) $\implies$ (i): Clear.
\end{proof}

\textbf{Remark.} $|\cdot|_{\infty}^2$ on $\mathbb{C}$ is not an absolute value by our definition (doesn't satisfy the triangle inequality). Some authors replace this by the condition $|x+y|^{\beta} \le |x|^{\beta} + |y|^{\beta}$ for some fixed $\beta \in \mathbb{R}_{>0}$. The induced topologies are the same in either case.

In this course, we will mainly be interested in the following:

\begin{defn}
    An absolute value $|\cdot|$ on $K$ is said to be \textbf{non-archimedean} if it satisfies the \textbf{ultrametric inequality} \[
    |x+y|\le \max(|x|,|y|).
    \]
    If $|\cdot|$ is not non-archimedean, we say it is \textbf{archimedean}. 
\end{defn}
\begin{example}
    \begin{itemize}
        \item $|\cdot|_{\infty}$ on $\mathbb{R}$ is archimedean.
        \item $|\cdot|_{p}$ on $\mathbb{Q}$ is non--archimedean.
    \end{itemize}
\end{example}

\begin{lemma}
    Let $(K, |\cdot|)$ be non--archimedean and $x,y \in K$. If $|x|<|y|$, then $|x-y|= |y|$.
\end{lemma}
\begin{proof}
    On the one hand, $|x-y|\le \max(|x|,|y|) = |y|$ (using $|x|=|-x|$).
    \vspace{1mm}
     
    On the other, $|y| \le \max(|x|, |x-y|) = |x-y|$.
\end{proof}

Convergence is easier in non--archimedean fields:
\begin{prop}
    Let $(K,|\cdot|)$ be non--archimedean and $(x_n)_{n=1}^{\infty}$ a sequence on $K$. If $|x_n-x_{n+1}| \to01$, then $(x_n)_{n=1}^{\infty}$ is Cauchy. In particular, if $K$ is complete, then the sequence converges.
\end{prop}
\begin{proof}
    For $\epsilon > 0 $, choose $N$ such that $|x_n-x_{n+1}| < \epsilon$ for $n \ge N$. Then for $N<n<m$, \[
    |x_n - x_m| = |(x_n - x_{n+1}) + (x_{n+1} - x_{n+2}) + \ldots + (x_{m-1}-x_m)| < \epsilon,
    \]
    so $(x_n)$ is Cauchy.
\end{proof}
\begin{example}
    For $p=5$, we can construct a sequence in $\mathbb{Q}$ satisfying:
    \begin{enumerate}[(i)]
        \item $x_n^2+1 \equiv  0 \pmod{5^n}$,
        \item $x_n \equiv x_{n+1} \pmod{5^n}$.
    \end{enumerate}
    We construct it by induction. Take $x_1=2$. Now suppose we've constructed $x_n$ and write $x_n^2+1 = a\cdot 5^n$ and set $x_{n+1} = x_n + b\cdot 5^n$. We compute \[
    x_{n+1}^2 + 1 = x_n^2 + 2b x_n 5^n + b^2 5^{2n} + 1 = a5^n + 2bx_n 5^n + \stackrel{\equiv 0 \pmod{5^{n+1}}}{b^2 5^{2n}}. 
    \]
    Hence we choose $b$ such that $a+2bx_n \equiv 0 \pmod{5}$ and we're done. 
    \vspace{1mm}
     
    Now (ii) tells us that $(x_n)$ is Cauchy, but we claim it doesn't converge. Suppose it does, $x_n \to l \in \mathbb{Q}$. Then $x_n^2 \to l^2 \in \mathbb{Q}$. But by (i), $x_n^2 \to -1$, so $l^2 = -1$, a contradiction.
\end{example}
This tells us that $(\mathbb{Q}, |\cdot|_{5})$ is not complete.
\begin{defn}
    The $p$--adic numbers $\mathbb{Q}_p$ are the completion of $\mathbb{Q}$ with respect to $|\cdot|_p$.
\end{defn}

\end{document}
 